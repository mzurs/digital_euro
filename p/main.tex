% Zohaib_Digital_Euro_Paper.tex
% Complete LaTeX research paper tailored to:
% "Implementation Models for Banks in the Context of the Digital Euro"
% Author: Zohaib Shaikh
% Compile with: pdflatex -> biber -> pdflatex x2

\begin{filecontents}{references.bib}
@misc{ecb2024rulebook,
  author = {{European Central Bank}},
  title  = {Digital Euro — Draft Rulebook (Preparation Phase)},
  year   = {2024},
  note   = {ECB publication, draft rulebook and technical annexes}
}

@techreport{pwc2025,
  author = {PwC},
  title  = {Digital Euro: Cost and Implementation Study},
  year   = {2025},
  institution = {PwC Europe}
}

@article{imk2023,
  author = {Institute for Monetary and Financial Stability (IMK)},
  title  = {Risks and Benefits of CBDC Deployment: A Technical Perspective},
  year   = {2023},
  journal= {Journal of Central Banking Studies},
  volume = {7},
  pages  = {12--38}
}

@misc{despportal2025,
  author = {{DESP Experimentation Portal}},
  title  = {Experimentation Reports: Conditional Payments and Offline Modes},
  year   = {2025},
  note   = {Eurosystem innovation platform summaries}
}

@misc{eba2022guide,
  author = {{European Banking Authority}},
  title  = {Guidelines for Payment Service Providers: CBDC Integration Considerations},
  year   = {2022}
}

@book{pressman2019,
  author = {Pressman, Roger S.},
  title  = {Software Engineering: A Practitioner's Approach},
  year   = {2019},
  publisher = {McGraw-Hill}
}
\end{filecontents}

\documentclass[11pt,a4paper]{article}

% Packages
\usepackage[utf8]{inputenc}
\usepackage[T1]{fontenc}
\usepackage[english]{babel}
\usepackage{geometry}\geometry{margin=1in}
\usepackage[expansion=false]{microtype}
\usepackage{amsmath,amssymb}
\usepackage{graphicx}
\usepackage{booktabs}
\usepackage{caption}
\usepackage{subcaption}
\usepackage{listings}
\usepackage{hyperref}
\hypersetup{colorlinks=true,linkcolor=blue,citecolor=blue,urlcolor=blue}
\usepackage{csquotes}
\usepackage[backend=biber,style=apa]{biblatex}
\addbibresource{references.bib}
\usepackage{enumitem}
\usepackage{float}
\usepackage{tabularx}
\usepackage{booktabs}
\usepackage{array}

% Make X columns ragged-right (helps avoid Overfull \hbox)
\newcolumntype{Y}{>{\raggedright\arraybackslash}X}

% Listings setup for JSON/Protobuf examples
\lstdefinelanguage{json}{
  basicstyle=\ttfamily\small,
  stringstyle=\color{black},
  showstringspaces=false,
  breaklines=true,
  frame=single,
  literate={:}{{:}}{1}
}
\lstdefinelanguage{protobuf}{
  basicstyle=\ttfamily\small,
  breaklines=true,
  frame=single
}

% Metadata
\title{Implementation Models for Banks\\ in the Context of the\\ Digital Euro}
\author{Zohaib Shaikh\\ \\\texttt{zohaib10092001@gmail.com
}}
\date{December 2025}

\begin{document}
\maketitle

\newpage
\begin{abstract}
This paper analyses technical implementation models for connecting bank
back-ends to the Eurosystem's Digital Euro Service Platform (DESP). Using
document review, experimentation reports, and architectural modelling, it
evaluates API integration choices (REST vs. gRPC), data model mapping
(pseudonymisation and aliasing), Dedicated Cash Account (DCA) management, and
advanced features such as conditional payments and offline transaction
synchronization. The study compares microservices, monolithic and hybrid
patterns and produces tier‑specific blueprints for High-, Mid-, and Low-tier
banks under In-house, Vendor, and Hybrid delivery models. The contributions
include concrete integration patterns, sample API schemas, and best-practice
recommendations for secure, performant, and maintainable bank-DESP integrations.
\end{abstract}


\vspace{0.5em}
\noindent\textbf{Keywords:} Digital Euro, DESP, Dedicated Cash Account, API integration, CBDC, bank architecture

\newpage

\section{Introduction}
The digital euro, as a retail central bank digital currency (rCBDC), emerges as
a pivotal innovation in the Eurosystem's efforts to adapt central bank money to
an increasingly digitalized economy. Defined in the ECB's glossary as "the
digital form of the single currency available to natural and legal persons," it
functions as a central bank liability offered to the general public for retail
payments, encompassing individual users, businesses, and public authorities.
This initiative addresses the evolving payment landscape where cash usage is
declining—evidenced by a shift toward digital transactions that now dominate
daily commerce—while ensuring continued access to a public, trusted, and
universally accepted means of payment. Complementary to physical cash, the
digital euro aims to preserve freedom of choice, bolster Europe's monetary
sovereignty, and foster innovation in payments, making European systems more
competitive, resilient, and inclusive. As highlighted in the Eurosystem's
closing progress report on the preparation phase, the project responds to
challenges such as market fragmentation dominated by non-European platforms and
the need for secure, privacy-enhanced digital alternatives.

The project's evolution began with the investigation phase (2021-2023), which
primarily focused on conceptual design, followed by the two-year preparation
phase (November 2023 to October 2025). During preparation, key achievements
included drafting the digital euro scheme rulebook—a single set of rules,
standards, and procedures for basic payment services—selecting providers for the
DESP and infrastructure, conducting user research and experimentation (e.g., via
the Pioneer Workstream on conditional payments), and performing in-depth
technical analyses. The rulebook, developed in collaboration with the Rulebook
Development Group (RDG), standardizes user experiences while incorporating
optional provisions to support innovation and interoperability. In October 2025,
the ECB's Governing Council decided to advance to the next phase, emphasizing
technical capacity building, market engagement, and legislative alignment, with
potential pilots in 2027 and first issuance targeted for 2029, contingent on EU
legislation adoption in 2026. This progression underscores the project's
strategic response to digitalization, as outlined in recent ECB strategy
documents.

Central to the ecosystem is the Digital Euro Service Platform (DESP), the core
settlement infrastructure that enables transaction processing, privacy-enhanced
data handling, and features like conditional payments (automated fund
reservations for scenarios such as e-commerce) and offline functionality (secure
transactions during network outages). Banks, acting as payment service providers
(PSPs) and access managers, play a critical intermediary role in facilitating
user access, managing aliases, configuring waterfall accounts, and integrating
with DESP. However, implementation poses significant technical challenges,
including mapping internal data models to pseudonymized Digital Euro Account
Numbers (DEANs), automating Dedicated Cash Accounts (DCAs) for liquidity, and
ensuring compliance with privacy standards that segregate data to prevent ECB
access to personal identities. Studies, such as the PwC Digital Euro Cost Study
(June 2025) and ECB's October 2025 assessment, estimate investment costs for
euro area banks at €4-5.8 billion, adjustable downward through synergies like
shared outsourcing and collaboration histories. These costs encompass back-end
adjustments, API integrations (e.g., REST for simplicity or gRPC for
performance), and architectural adaptations, with potential for mutualization
reducing expenses by 30-50 percent in mid-tier markets.

Research indicates that while the digital euro could enhance efficiency and
reduce reliance on foreign platforms, challenges persist, including risks of
bank disintermediation (where users shift holdings to CBDC accounts), cyber
vulnerabilities, and accessibility issues for non-digital populations. The IMK
Study (September 2024) warns of potential failure scenarios, such as low
adoption mirroring other CBDCs, or inefficiencies in offline modes, contrasting
the ECB's commitment with more cautious approaches by peers like the Federal
Reserve. Counterarguments from European issuers emphasize strengthening euro
stablecoins alongside the digital euro to mitigate dollar dominance in
stablecoins, which could undermine monetary policy. Social media discussions
highlight public concerns over surveillance and control, urging balanced privacy
measures.

This thesis investigates the technical architecture and system integration
required to connect commercial bank back-ends to DESP, analyzing pathways like
API selection, data model mapping, DCA automation, conditional payments
processing, and offline synchronization. It evaluates architectural
patterns—microservices for modularity versus monolithic for simplicity—and their
impacts on latency, security, and maintainability. The primary objective is to
deliver technical blueprints and best practices tailored to high-tier (large,
advanced IT banks using in-house models), mid-tier (regional banks with hybrid
approaches), and low-tier (small banks relying on vendors), ensuring
cost-efficient, scalable, and compliant implementations.

Key research questions include: How can banks effectively map internal customer
IDs to DEANs while automating DCAs? What are the trade-offs between REST and
gRPC in high-volume environments, and microservices versus monolithic patterns
in terms of system performance? How do advanced features like conditional
payments (tested in the Pioneer Workstream) and offline sync influence back-end
design? Which implementation models are optimal for different bank tiers,
considering costs and risks?

The scope is confined to technical back-end aspects for euro area banks, drawing
on ECB documents, cost studies, and global CBDC comparisons, but excludes
front-end interfaces, macroeconomic effects, or non-technical regulations.
Limitations arise from reliance on publicly available data and assumptions about
DESP specifications, which may evolve post-2025.




\section{Background on the Digital Euro}
\subsection{Conceptual framework}
The digital euro represents a retail central bank digital currency (rCBDC),
defined in the European Central Bank's (ECB) glossary as a central bank
liability in digital form offered to the general public, including individual
users, businesses, and public authorities, for retail payments (European Central
Bank, 2023). As the digital form of the euro area's single currency, it serves
as a public, trusted, and universally accepted means of payment, complementing
physical cash without replacing it. This initiative addresses the declining use
of cash amid rising digital transactions, aiming to preserve monetary
sovereignty, enhance payment efficiency, and foster inclusivity in Europe's
financial ecosystem (European Central Bank, 2025a).

The digital euro ecosystem involves multiple stakeholders with distinct roles.
Users, encompassing natural and legal persons, interact with the system for
everyday transactions such as person-to-person (P2P) payments, point-of-sale
(POS) purchases, and e-commerce. Payment service providers (PSPs), primarily
banks acting as access managers, facilitate user onboarding, manage aliases
(e.g., linking phone numbers or emails to pseudonymous Digital Euro Account
Numbers or DEANs), configure waterfall accounts for holding limit enforcement,
and provide form factors like mobile apps or cards. The Digital Euro Service
Platform (DESP) acts as the core settlement infrastructure, operated by the
Eurosystem (the ECB and national central banks), handling transaction
validation, privacy-enhanced data segregation, and settlement without direct
access to user identities. This two-tiered model positions PSPs as
intermediaries, ensuring privacy compliance through pseudonymization while
delegating back-end operations to DESP (European Central Bank, 2025b). The
framework draws on existing standards like the Single Euro Payments Area (SEPA)
to promote interoperability and mitigate market fragmentation dominated by
non-European platforms.

\subsection{Eurosystem's Digital Euro Project Evolution}
The Eurosystem's digital euro project has progressed through structured phases,
reflecting a cautious yet ambitious approach to CBDC implementation. The
investigation phase (2021-2023) concentrated on conceptual design, exploring
user needs, technical feasibility, and economic implications, culminating in a
decision to advance based on positive assessments of privacy, usability, and
financial stability (European Central Bank, 2023).

The subsequent preparation phase (November 2023 to October 2025) laid
foundational groundwork for potential issuance. Key objectives included drafting
the digital euro scheme rulebook—a comprehensive set of rules, standards, and
procedures for basic payment services—selecting providers for DESP components,
conducting experimentation, and engaging stakeholders. Achievements encompassed
the rulebook's development in collaboration with the Rulebook Development Group
(RDG), incorporating over 2,000 market comments to standardize user experiences
while allowing optional provisions for innovation. Providers were selected via
public tenders, with private companies and six national central banks tasked
with delivering infrastructure elements. Experimentation via the innovation
platform involved approximately 70 participants, including banks, fintechs, and
merchants, testing conditional payments (e.g., automated fund reservations) in
sectors like e-commerce and mobility, yielding insights into practical
applications and technical viability (European Central Bank, 2025b). User
research and technical analyses further refined aspects like security and
accessibility.

As of December 2025, the project has transitioned to the next phase following
the ECB's Governing Council decision in October 2025. This phase emphasizes
technical capacity building, deepened market engagement, and support for the EU
legislative process. If the necessary regulation is adopted in 2026, a pilot
could commence in 2027, with first issuance targeted for 2029 (European Central
Bank, 2025a). This evolution contrasts with global CBDC efforts; for instance,
the IMK Study's CBDC tracker highlights the ECB's strong commitment as unique
among advanced economies, where many, including the Federal Reserve, have ruled
out retail CBDCs due to risks of low adoption and inefficiency (Bofinger, 2024).
Early adopters like the Bahamas' Sand Dollar (launched 2020) and China's e-CNY
(piloted since 2020) have shown mixed results, with limited uptake in the former
and broader integration in the latter, informing the ECB's risk-mitigation
strategies (Central Bank of The Bahamas, 2019; Zhang, 2025).

\begin{table}[ht]
\centering
\begin{tabularx}{\linewidth}{@{} l l X @{}}
\toprule
\textbf{Phase} & \textbf{Period} & \textbf{Key objectives and milestones} \\
\midrule
Investigation & 2021--2023 &
Conceptual design, feasibility studies; decision to proceed to preparation. \\[1ex]

Preparation & 2023--2025 (closed October 2025) &
Rulebook drafting; provider selection (private and NCBs); innovation platform with $\sim$70 participants testing conditional payments; user research and stakeholder engagement. \\[1ex]

Next Phase & Post-October 2025 onward &
Technical readiness; market collaboration; legislative support; potential pilot in 2027 if regulation adopted in 2026. \\[1ex]

Potential Issuance & 2029 (target) &
Full rollout contingent on EU laws; focus on scalability and privacy. \\
\bottomrule
\end{tabularx}
\caption{Project phases, periods and key objectives/milestones.}
\label{tab:project-phases}
\end{table}

\subsection{Key Components of the Digital Euro Infrastructure}

The digital euro's infrastructure is designed for robustness, privacy, and
efficiency, comprising interconnected elements that support seamless
transactions.

The DESP serves as the central platform for transaction settlement, employing
privacy-enhanced techniques to segregate data and delegate handling to external
providers, ensuring no linkage of transaction details to user identities by the
ECB. It includes sub-functionalities like offline modes, which enable secure,
tamper-resistant transactions in network-outage scenarios through atomic
synchronization upon reconnection, enhancing resilience for remote or
underserved areas (European Central Bank, 2025b).

Dedicated Cash Accounts (DCAs) provide mechanisms for liquidity management,
allowing users to fund digital euro holdings from commercial bank accounts via
automated processes like waterfall (converting excess holdings to bank money)
and reverse waterfall configurations. This ensures compliance with holding
limits while maintaining liquidity flow between traditional and digital systems
(European Central Bank, 2025b).

Bank-specific value-added services (VAS) offer opportunities for PSPs to extend
beyond mandatory functions, fostering innovation and revenue generation. For
instance, banks can integrate programmable payments, such as instalment options
akin to "Buy Now, Pay Later" schemes, where conditional logic automates
repayments based on predefined triggers. Merchant loyalty integrations could
link digital euro transactions to rewards programs, while enhanced analytics for
fraud detection and automated cash management tools (e.g., real-time balance
optimization) differentiate offerings. These VAS leverage DESP's
interoperability, enabling banks to compete with fintechs and reduce reliance on
third-party platforms, potentially generating new income streams amid estimated
implementation costs of €4-5.8 billion euro area-wide (PwC, 2025; Zhang, 2025).

\subsection{Role of Banks in the Ecosystem}
Banks, as supervised PSPs and access managers, are pivotal intermediaries in the
digital euro ecosystem, bridging users and DESP. They handle access management
(e.g., opening accounts, managing aliases), liquidity provisioning through DCAs,
and transaction facilitation, including advanced features like conditional
payments. This role extends to configuring waterfall accounts for limit
enforcement and providing acceptance solutions like POS terminals (European
Central Bank, 2023).

Synergies with existing SEPA infrastructures are emphasized, allowing banks to
reuse standards for QR codes, NFC payments, and alias-based transfers, thereby
reducing integration costs and promoting pan-European uniformity. By mitigating
current market fragmentation—where non-European providers dominate digital
payments—the digital euro enables banks to roll out innovative solutions,
enhancing competition and resilience. However, challenges like potential
disintermediation (e.g., deposit shifts to CBDC accounts) and high adaptation
costs underscore the need for strategic responses, as critiqued in analyses of
global CBDCs (Bofinger, 2024; PwC, 2025).






\newcolumntype{Y}{>{\raggedright\arraybackslash}X}
\newcolumntype{L}[1]{>{\raggedright\arraybackslash}p{#1}}

\begin{table}[ht]
\centering
\small % reduce font so the wide table fits better; remove if undesired
\begin{tabularx}{\linewidth}{@{} L{0.18\linewidth} Y L{0.12\linewidth} Y Y @{}}
\toprule
\textbf{Bank Tier} & \textbf{Primary Affected Components} & \textbf{Cost Range (\texteuro\ million per Bank)} & \textbf{Key Integration Challenges} & \textbf{Mitigation via Synergies} \\
\midrule

High-Tier (Large, Advanced IT) &
Custom back-ends, APIs, full DESP services (e.g., conditional payments) &
500+ &
Scalability for high-volume transactions; pseudonymisation &
In-house expertise; cloud/hybrid deployments; hybrids (20--30\% reduction) \\[1ex]

Mid-Tier (Regional, Moderate Maturity) &
Hybrid APIs, DCA automation, compliance/orchestration modules &
100--300 &
Vendor dependencies; offline synchronization &
Mutualisation with peers and shared outsourcing (30--50\% reduction) \\[1ex]

Low-Tier (Small, Basic Infrastructure) &
Outsourced interfaces, basic front-ends, databases &
10--50 &
Limited customization; security upgrades &
Full vendor models; ECB-provided standards and reference implementations (40\%+ reduction) \\

\bottomrule
\end{tabularx}
\caption{Indicative costs, affected components, integration challenges and mitigation synergies by bank tier.}
\label{tab:tier-costs}
\end{table}

% The Eurosystem's program comprises an investigation phase (2021--2023) and a preparation phase (2023--2025). During the preparation phase the Eurosystem released draft rulebooks and opened an innovation portal for testing conditional payments and offline scenarios with industry participants \cite{ecb2024rulebook,despportal2025}.

\section{Literature review}
% Existing literature covers CBDC motives and high-level architectures (e.g., IMK Study), national pilot learnings (various rCBDC implementations) and cost assessments such as PwC's 2025 study. However, detailed blueprints for bank back-end integration with DESP — including tiered recommendations — remain scarce, motivating this research \cite{imk2023,pwc2025,eba2022guide}.

\section{Methodology}
% This research uses qualitative system analysis and design: reviewing published DESP artefacts and experimentation reports, synthesising technical requirements, and constructing integration blueprints. The analytical framework uses three evaluation axes: performance (latency, throughput), security (privacy, integrity) and maintainability (modularity, upgradeability). Scenario-based sensitivity analysis demonstrates behaviour under differing transaction volumes and outage patterns.

\section{DESP and bank back-end architectures}
% \subsection{DESP core functions}
% DESP acts as a settlement hub and message broker for digital euro transactions. Critical features relevant to banks include: (i) interfaces for originating and receiving payment messages; (ii) support for DCAs and liquidity operations; (iii) privacy-preserving data handling (delegation and pseudonymisation); and (iv) optional offline modes facilitated by secure replay-protected tokens or signed bundles \cite{ecb2024rulebook,despportal2025}.

% \subsection{Typical bank back-end components}
% Banks typically operate: core banking ledgers, payment gateways, customer identity stores, risk engines (fraud, AML), and reconciliation systems. Integration points to DESP are usually the payment gateway (for realtime flows) and liquidity manager (for DCA funding and reconciliation). Figure~\ref{fig:logical} illustrates a high-level logical architecture (replace with institution-specific diagram).

% \begin{figure}[H]
%   \centering
%   \fbox{\parbox{0.85\linewidth}{\centering Add your architecture diagram here (DESP \textless-> Gateway \textless-> Core Ledger, Liquidity Manager, Alias Service, Risk Engines)}}
%   \caption{High-level logical architecture for bank--DESP integration.}
%   \label{fig:logical}
% \end{figure}

% \section{Data models and DCA management}
% Banks should adopt mapping tables between internal customer identifiers and DESP pseudonyms (DEANs). Alias management (IBAN-like or proprietary aliases) must be resolved server-side without leaking identity to DESP. DCAs require automated funding and reconciliation: typical patterns include pull-based funding (from bank master accounts), push-based settlement, and waterfall configurations that route settlement across internal liquidity accounts based on business rules.

\section{Integration pathways and API choices}
% \subsection{REST vs gRPC}
% REST (JSON/HTTP) is broadly compatible with web infrastructure and easier to debug; gRPC (HTTP/2, protobuf) yields lower latency and better streaming behaviour for high-throughput environments. For High-tier banks with high-volume requirements, gRPC connectors with binary protobuf schemas are recommended. Mid- and Low-tier banks may prefer REST for simplicity and vendor compatibility. In all cases, API contracts should support versioning and schema evolution.

% \subsection{Common integration concerns}
% Authentication (mutual TLS and OAuth2 for delegated access), idempotency, error classification, and reconciliation semantics are core challenges. Banks must implement robust retry and poison‑message handling and synchronise clocks to avoid ordering ambiguities in conditional payments and offline resyncs.

% \section{Advanced functionality: conditional payments and offline mode}
% \subsection{Conditional payments}
% Conditional payments require reservation of funds and a mechanism for later release upon condition satisfaction. Implementation options include:
% \begin{itemize}[nosep]
%   \item On‑ledgers holds linked to DCAs (if supported by DESP).
%   \item Off‑ledger reservations in the bank's ledger with guaranteed settlement by the liquidity manager.
%   \item Hybrid patterns where DESP supports conditional triggers and the bank manages business logic and reconciliation.
% \end{itemize}
% Atomicity can be achieved by two-phase commit-like protocols or idempotent compensating actions; complexity increases when offline participants are involved.

% \subsection{Offline transaction synchronization}
% Offline modes often rely on secure hardware/software enclaves (e.g., secure wallet modules), signed transaction bundles, and replay-protection. Banks participating in offline flows must design reconciliation engines that can ingest out-of-order, signed transaction sets and detect double-spend attempts before final settlement.

\section{Evaluation of architectural patterns}
% This section summarises trade-offs between the main architectural options.
% \subsection{Microservices}
% Pros: modularity, independent scaling, easier upgrades. Cons: distributed complexity, network overhead, operational burden.
% \subsection{Monolithic integration}
% Pros: simpler deployments, lower intra-process latency. Cons: harder to evolve and scale components independently.
% \subsection{Hybrid}
% Phased approaches where time-critical paths (connectors, liquidity engines) are isolated as services while other functionality remains in the core monolith often deliver a good balance for Mid-tier banks.

\section{Implementation models by bank tier}
% \subsection{High-Tier banks}
% Recommendation: In-house microservices approach with gRPC connectors, dedicated liquidity microservices for DCAs, and internal sandboxing for offline/conditional features. Investment in observability, automated testing and cloud-native deployments is expected.

% \subsection{Mid-Tier banks}
% Recommendation: Hybrid model — deploy vendor-managed DESP connectors and a lightweight in-house liquidity/orchestration layer. This reduces upfront engineering while retaining control over compliance-sensitive components.

% \subsection{Low-Tier banks}
% Recommendation: Vendor model — rely on certified gateway providers offering REST APIs and delegated DCA handling. Focus in-house efforts on reconciliation, compliance checks and customer onboarding.

% \begin{table}[H]
% \centering
% \begin{tabular}{lccc}
% \toprule
% Dimension & High-tier & Mid-tier & Low-tier \\
% \midrule
% Preferred model & In-house & Hybrid & Vendor \\
% API preference & gRPC & REST/gRPC mixed & REST \\
% Typical timeline & 12--24 months & 9--18 months & 6--12 months \\
% Key risks & Integration complexity & Vendor lock-in & Compliance reliance \\
% \bottomrule
% \end{tabular}
% \caption{Summary comparison of implementation models by bank tier (indicative).}
% \label{tab:tiers}
% \end{table}

\section{Technical blueprints and best practices}
% Blueprints (component diagrams, sequence flows) should include the following modules: Connector (REST/gRPC adapter), Liquidity Manager (DCA automation), Alias Service (pseudonym mapping), Risk Engine (fraud/AML), and Reconciliation Engine. Best practices include:
% \begin{itemize}[nosep]
%   \item Strong separation of duties: keep identity stores and DEAN mappings strictly internal.
%   \item API contract-first design with clear versioning and backwards compatibility.
%   \item Observability: distributed tracing, metrics (latency p50/p99), and centralized logging for reconciliation workflows.
%   \item Automated reconciliation and intrusion-resistant resync processes for offline bundles.
%   \item Security: mutual TLS, short-lived tokens, hardware key management for signing offline bundles.
% \end{itemize}

% \section{Sensitivity analysis}
% Key sensitivities include transaction volume growth (e.g., 5x peaks), frequency of conditional payments, and offline event rates. For each, capacity planning should model read/write latencies, DCA funding contention, and queueing delays. Practical guidance: overprovision messaging layers and design DCA funding windows to smooth spikes.

% \section{Policy implications}
% Standardised API contracts and test suites from the Eurosystem will materially reduce integration costs and fragmentation. Encouraging certified gateway providers and reference implementations will support Low- and Mid-tier banks' adoption while preserving competition.

\section{Conclusion}
% This paper produced integration blueprints and tiered recommendations for bank back-end connectivity to DESP. Findings indicate gRPC-based connectors and microservices suit high-volume institutions, while REST-based vendor connectors are pragmatic for smaller banks. Hybrid patterns often present the best compromise for mid-tier banks. Future work should validate these patterns with real-world pilots and quantify cost models using bank-specific operational data \cite{pwc2025}.

\section*{Acknowledgments}
% The author thanks the Eurosystem innovation participants for publicly sharing experimentation reports and acknowledges reviewers who provided helpful feedback on early drafts.

% \appendix
% \section{Example REST JSON schema for a payment initiation}
% \begin{lstlisting}[language=json,caption={Sample REST payment initiation (simplified)}]
% {
%   "paymentId": "uuid-1234",
%   "fromAlias": "alice@examplebank",
%   "toAlias": "merchant@shop",
%   "amount": 100.00,
%   "currency": "EUR",
%   "executionCondition": {
%     "type": "timebound",
%     "validUntil": "2025-12-31T23:59:59Z"
%   }
% }
% \end{lstlisting}

% \section{Example gRPC / Protobuf message (simplified)}
% \begin{lstlisting}[language=protobuf,caption={Sample protobuf schema (simplified)}]
% syntax = "proto3";
% package desp;

% message PaymentRequest {
%   string payment_id = 1;
%   string from_alias = 2;
%   string to_alias = 3;
%   double amount = 4;
%   string currency = 5;
%   Condition execution_condition = 6;
% }

% message Condition {
%   string type = 1; // e.g., "timebound", "event"
%   string valid_until = 2; // ISO-8601
% }

% service DespConnector {
%   rpc InitiatePayment (PaymentRequest) returns (PaymentResponse);
% }
% \end{lstlisting}

% \section{Simulation parameter example}
% If running capacity tests, suggested parameters include: concurrency levels (100--50,000), message size distributions (small payments 90\%, large payments 10\%), and offline reconciliation batches (10--1,000 transactions per bundle).

% \printbibliography

\end{document}
