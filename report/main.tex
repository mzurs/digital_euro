\documentclass[fontsize=12pt]{scrartcl}
\usepackage{setspace}
\onehalfspacing
\usepackage{amsmath,amssymb,amsfonts,amsthm,mathtools}
\usepackage[english]{babel}
\usepackage[T1]{fontenc}
\usepackage[utf8x]{inputenc}
\usepackage{lmodern}
\usepackage{dsfont}
\usepackage{bbm}
\usepackage[round]{natbib}
\usepackage{color} 
\usepackage[defaultlines=2,all]{nowidow}
\usepackage{caption}
\usepackage[labelformat=simple]{subcaption}
\usepackage{graphicx}
\usepackage{booktabs}
\usepackage[backend=biber,style=numeric]{biblatex}
\usepackage{csquotes}
\usepackage[margin=1in]{geometry}
\usepackage{array}
\usepackage{graphicx}
\usepackage{ragged2e}
\usepackage{float}
\usepackage[T1]{fontenc}
\usepackage{tabularx}
\usepackage[table]{xcolor}
\usepackage{enumitem}
\usepackage{ragged2e}

\usepackage[T1]{fontenc}
\usepackage[utf8]{inputenc}   % keep for pdfLaTeX
\usepackage{textcomp}         % provides \texteuro
\usepackage{geometry}
\geometry{margin=1in}

\usepackage{ragged2e}
\usepackage{array}
\usepackage{booktabs}
\usepackage{tabularx}
\usepackage{xltabular}        % multi-page + X columns
\usepackage[table]{xcolor}
\usepackage[hidelinks]{hyperref}
\usepackage[table]{xcolor} % for row colors
\usepackage[most]{tcolorbox}
\usepackage{enumitem}
\usepackage{tikz}
\usetikzlibrary{arrows.meta,positioning}
\usepackage{float}     % for [H] if you want fixed placement
\usepackage{placeins}  % for \FloatBarrier (optional)
\documentclass{article}
\usepackage[edges]{forest}
% Preamble:
\usepackage{tikz}
\usetikzlibrary{positioning, fit, arrows.meta}

\tcbset{
  reqbox/.style={
    enhanced,
    breakable,
    width=\linewidth,
    colback=white,
    colframe=black!35,
    boxrule=0.6pt,
    arc=2mm,
    left=2mm,right=2mm,top=1mm,bottom=1mm,
    fonttitle=\bfseries,
    colbacktitle=black!12,
    coltitle=black,
    before=\par\medskip,
    after=\par\medskip
  }
}

\setlength{\emergencystretch}{2em}
\raggedbottom


% Convenience macro (optional)
\newcommand{\EUR}{\texteuro}

% ============================

\addbibresource{references.bib}

\renewcommand\thesubfigure{(\alph{subfigure})}

% Set margin if needed, otherwise scrartcl default is used. 
% Standard academic reports often use geometry, but kept minimal here to match your original style.

\setlength\parindent{0pt}
\setlength{\parskip}{6pt plus 1pt minus 1pt}

\newcommand{\red}{\textcolor{red}}

\begin{document}

\begin{titlepage}
      \centering
      {\scshape\LARGE Roland Berger \par}
      \vspace{5cm}

      {\huge\bfseries Implementation Models for Banks in the Context of the Digital Euro\par}
      \vspace{3cm}
      {\Large Author: Zohaib Shaikh \par}
      \vspace{0.5 cm}
      {\large Email: zohaib10092001@gmail.co \par}
      \vspace{2cm}
      {\large Instructor: \par}
      {\large Christian Hartmann \par}

      \vspace{1cm}
      {\large \today\par}
\end{titlepage}

\newpage

% --- SECTION 0: ABSTRACT ---
\begin{abstract}
      This research thesis examines the technical architecture, implementation pathways, and strategic models required for banks to integrate the Digital Euro Service Platform (DESP) into their existing infrastructure. The study synthesizes findings from the European Central Bank's preparation phase (2023-2025), industry cost analyses, and technical specifications to provide a comprehensive framework for understanding how different bank tiers—High- tier (large, international), Mid-tier (regional), and Low-tier (small, community)—can adopt the Digital Euro through In-house, Hybrid, or Outsourced implementation models.

      The research demonstrates that successful Digital Euro integration depends on
      technical alignment with the Rulebook Development Group standards, careful
      cost-benefit analysis of implementation models, and strategic leverage of
      shared infrastructure and mutualization opportunities. Key findings indicate
      that costs can be substantially reduced through effective synergy mechanisms.
      The thesis provides technical blueprints, implementation frameworks, and policy
      recommendations to guide banks through this critical transition.

\end{abstract}

\newpage

\tableofcontents
\thispagestyle{empty}

\cleardoublepage

\setcounter{page}{1}

% --- SECTION 1: INTRODUCTION ---
\section{Introduction}
\subsection{Background and Motivation}
The Eurosystem's Digital Euro initiative represents a fundamental evolution in
European monetary infrastructure. As payment behavior shifts toward digital
channels and cash usage declines, the European Central Bank (ECB) has initiated
a comprehensive project to provide a retail central bank digital currency
(CBDC) that complements physical cash while ensuring Europe's monetary
sovereignty in an increasingly digitalized economy. The investigation phase
(2021-2023) established the conceptual framework for the Digital Euro,
exploring design options and distribution models. The subsequent preparation
phase (2023-2025) focused on transforming these concepts into operational
reality: developing the Digital Euro Scheme Rulebook, selecting technology
providers, conducting experimentation through innovation platforms, and
validating technical feasibility across diverse use cases including conditional
payments and offline functionality.

Europe's payment landscape remains fragmented and vulnerable to external
dependencies. Approximately two-thirds of euro area card-based transactions are
processed by non-European entities, while 13 euro area countries depend
entirely on international card schemes or mobile solutions for in-store
payments\cite{ecbPrep}. The Digital Euro addresses this strategic vulnerability
by establishing a pan-European, public digital payment infrastructure that:
preserves consumer freedom of choice in payment methods, strengthens European
financial autonomy and competitiveness, enables seamless cross-border payments
throughout the euro area, provides a foundation for innovation in payment
services, and maintains financial inclusion across diverse user segments.

\subsection{Research Problem and Objectives}
Despite the ECB's comprehensive preparation work, significant uncertainties
persist regarding practical implementation for banks:

\textbf{Primary Research Challenge:} How can banks effectively integrate the Digital Euro into
their technical infrastructure while managing implementation costs, compliance
requirements, and business model adaptations?

\textbf{Research Objectives:}
\begin{enumerate}
      \item \textbf{Technical Analysis:} Examine the technical architecture of the DESP and required
            back-end integration patterns for different bank categories
      \item \textbf{Implementation Modeling:} Evaluate three distinct implementation approaches (In-
            house, Vendor/Outsourced, Hybrid) with respect to cost efficiency, scalability, and
            compliance
      \item \textbf{Bank Tier Stratification:} Develop tier-specific implementation strategies addressing
            the distinct capabilities and constraints of High-tier, Mid-tier, and Low-tier
            institutions
      \item \textbf{Shared Infrastructure Assessment:} Analyze opportunities for cost mutualization
            through shared services, collaborative platforms, and vendor consolidation
      \item \textbf{Cost-Benefit Analysis:} Synthesize findings from multiple cost studies and develop
            realistic financial projections for different implementation scenarios
      \item \textbf{Policy Implications:} Formulate recommendations for banks, regulators, and the
            ECB to optimize implementation outcomes.

\end{enumerate}

\subsection{Research Questions}
This thesis addresses the following core research questions:
\begin{itemize}
      \item \textbf{RQ1: Technical Integration}
            \begin{enumerate}
                  \item How should banks map internal data models and systems to Digital Euro Service
                        Platform requirements?
                  \item What are the technical implications of different API protocols and
                        architectural patterns (microservices vs. monolithic)?
                  \item How do conditional payments and offline synchronization affect back-end design
                        decisions?
            \end{enumerate}
      \item \textbf{RQ2: Implementation Models}
            \begin{enumerate}
                  \item What are the comparative advantages and disadvantages of In-house, Hybrid, and
                        Outsourced implementation approaches?
                  \item How do implementation costs, timelines, and risk profiles differ across these
                        models?
                  \item Which implementation model is optimal for each bank tier?
            \end{enumerate}
      \item \textbf{RQ3: Shared Infrastructure and Mutualization}
            \begin{enumerate}
                  \item What cost synergies can be achieved through shared infrastructure and
                        collaborative vendor engagement?
                  \item How do market-specific factors (vendor concentration, outsourcing prevalence,
                        collaboration history) influence synergy potential?
                  \item What organizational and contractual arrangements facilitate effective cost
                        mutualization?
            \end{enumerate}
      \item \textbf{RQ4: Risk and Feasibility}
            \begin{enumerate}
                  \item What are the primary technical, operational, and financial risks in Digital
                        Euro integration?
                  \item How can banks effectively manage the complex interplay between mandatory
                        compliance and optional innovation?
                  \item What governance structures and expertise requirements are necessary for
                        successful implementation?
            \end{enumerate}
\end{itemize}

\subsection{Research Scope and Methodology}
The research focuses on the European Central Bank’s (ECB) digital euro
initiative, with particular attention to the preparation phase from 2023 to
2025 and the anticipated implementation period between 2025 and 2029. Within
this temporal frame, the analysis covers banking systems across the 20 euro
area countries, concentrating on retail banks with significant customer bases
and differentiating them by asset size and market position. The technical
perspective is limited to back-end integration, core system modifications,
application programming interface (API) implementation, and compliance-related
infrastructure, while front-end user interfaces and broader macroeconomic
impact assessments are deliberately excluded from the scope.

Methodologically, the study adopts a mixed-methods design that integrates
several complementary approaches to ensure both analytical depth and practical
relevance. First, it undertakes a structured document analysis of ECB
rulebooks, technical specifications, progress reports, and relevant regulatory
frameworks. Second, it synthesizes existing cost studies by incorporating
findings from the PwC Digital Euro Cost Study, ECB cost assessment exercises,
and estimates from banking associations. Third, it includes technical modeling
of functional architectures, API specifications, and data flow diagrams to
capture the operational implications of different integration choices. Fourth,
the research applies comparative case analysis to examine implementation
approaches across diverse banking models and geographic contexts within the
euro area. Fifth, it conducts a synergy assessment through quantitative
evaluation of mutualization and outsourcing opportunities, relying on
structured vendor and partnership analyses. Finally, it develops scenario-based
cost and complexity projections across three distinct implementation model
scenarios to explore potential outcome ranges under varying strategic and
technical assumptions.

% --- SECTION 2: THEORETICAL BACKGROUND ---
\section{Background on the Digital Euro: Conceptual and
  Infrastructural Foundations}
\subsection{Conceptual Framework and Definitions}
\subsubsection{Digital Euro: Definition and Functional Characteristics}
The Digital Euro, or CBDC, is a digital form of central bank
money—specifically, a direct liability of the Eurosystem—available to the
general public for electronic payments. It differs fundamentally from
commercial bank money, e-money, and private cryptocurrencies:

\begin{table}[ht]
      \centering
      \caption{Comparison of Digital Currencies and Money Types}
      \label{tab:digital-currency-comparison}

      % --- light colors (each row different, header separate) ---
      \definecolor{TableHeader}{RGB}{128,128,128} % header (light blue)
      \definecolor{RowA}{RGB}{211,211,211}        % very light mint
      \definecolor{RowB}{RGB}{211,211,211}       % very light lavender
      \definecolor{RowC}{RGB}{211,211,211}        % very light peach
      \definecolor{RowD}{RGB}{211,211,211}        % very light sky
      \definecolor{RowE}{RGB}{211,211,211}        % very light gray
      \definecolor{RowF}{RGB}{211,211,211}        % very light pink

      \resizebox{\textwidth}{!}{%
            \begin{tabular}{lcccc}
                  \hline
                  \rowcolor{TableHeader}
                  Characteristic       & Digital Euro               & Commercial Bank Money      & E-Money              & Cryptocurrency             \\
                  \hline
                  \rowcolor{RowD}
                  Issuer               & ECB/Eurosystem             & Commercial banks           & E-money institutions & Decentralized/Private      \\
                  \rowcolor{RowD}
                  Legal Status         & Central bank liability     & Bank liability             & Prepaid value        & Varies (often unregulated) \\
                  \rowcolor{RowD}
                  Settlement           & Real-time, final           & Interbank clearing         & Custodian-based      & Blockchain-based           \\
                  \rowcolor{RowD}
                  Privacy              & High (pseudonymous)        & Low                        & Medium               & Variable                   \\
                  \rowcolor{RowD}
                  Universal Access     & Yes (within the euro area) & Conditional                & Conditional          & Open                       \\
                  \rowcolor{RowD}
                  Regulatory Oversight & Full (ECB)                 & Full (Banking Supervision) & Moderate             & Limited                    \\
                  \hline
            \end{tabular}%
      }
\end{table}

The digital euro is designed to fulfil several complementary roles within the
European payment ecosystem. As a store of value, users can hold digital euro
balances, subject to calibrated holding limits intended to safeguard financial
stability. As a medium of exchange, it supports seamless transactions in
peer-to-peer contexts, at the point of sale, and in e-commerce environments,
thereby integrating into everyday payment use cases. Denominated in euros and
maintaining one-to-one parity with physical cash, it also functions as a unit
of account, ensuring consistency with existing monetary denominations. In
addition, the provision of offline payment capabilities is intended to enhance
payment system resilience by enabling transactions during temporary network
outages. Finally, the digital euro is conceived as a tool for promoting
financial inclusion, as it would be accessible to all residents of the euro
area, including those without traditional banking relationships.

\subsubsection{Key Digital Euro Ecosystem Actors}
The Digital Euro ecosystem comprises several interconnected participant
categories:
\begin{itemize}
      \item \textbf{Eurosystem (ECB):}
            The Eurosystem (ECB) develops and maintains the Digital Euro Service Platform (DESP), establishes regulatory standards through the Rulebook Development Group, manages settlement and core clearing functions, ensures system resilience and cybersecurity, and does not see end-user identities due to its privacy-preserving architecture.
      \item \textbf{Payment Service Providers (PSPs) - Banks and Non-Bank Operators:}
            Payment Service Providers (PSPs), including banks and non-bank operators, distribute Digital Euro services to end users, manage customer onboarding and Know-Your-Customer (KYC) compliance, perform pre-authorization and fraud prevention, handle funding and defunding operations for end users, and provide customer support and dispute resolution.
      \item \textbf{End Users (Natural and Legal Persons):}
            It includes individual consumers who utilize the Digital Euro for daily transactions, as well as merchants and businesses that accept Digital Euro payments.
      \item \textbf{External Service Providers:}
            External Service Providers include technology vendors specializing in alias lookup, fraud detection, and app development; platform developers focused on mobile wallets and payment apps; security and encryption service providers; and payment terminal manufacturers.
\end{itemize}

% ==
\subsection{The Eurosystem's Digital Euro Project Evolution}

\subsubsection{Project Phases and Timeline}
The project phases and timeline consist of the \textbf{Investigation Phase}
from October 2021 to October 2023, the \textbf{Preparation Phase} from November
2023 to October 2025, and the \textbf{Implementation Phase} from November 2025
to 2029. Planned activities include technical capacity building and
development, pilot testing and validation with a potential start in mid-2027,
market readiness programs and compliance certification, phased functionality
roll-out, and a possible first issuance targeted for 2029.\cite{ecbPrep}

% ==
\subsection{Structural Components of the Digital Euro Infrastructure}
\subsubsection{Digital Euro Service Platform (DESP) Architecture}
The DESP represents the technical core of the Digital Euro infrastructure,
providing centralized settlement and clearing functions while enabling
distributed processing across PSPs and Eurosystem components. The architecture
embodies several key design principles:

\begin{itemize}
      \item \textbf{Multi-Region Resilience:}
            The system achieves multi-region resilience via a centralized ledger that maintains authoritative transaction records, with deployment across three geographic regions featuring multiple servers and data centers per region. It incorporates automatic failover and regional disaster recovery capabilities, ensuring operational continuity even in the event of a complete regional infrastructure failure.\cite{ecbPrep}

      \item \textbf{Privacy-Preserving Design:}
            The privacy-preserving design ensures that end-user identities remain unknown to the Eurosystem, with Payment Service Providers (PSPs) handling KYC/AML functions and user onboarding. Transactions are processed using pseudonymous identifiers, employing segregated and distributed processing across Digital Euro Settlement Platform (DESP) components, while preventing the central bank from linking any transactions to individuals.\cite{ecbPrep}

      \item{\textbf{Core Components:}\cite{ecbRuleBook}}
            \begin{enumerate}
                  \item Access Management Service: Manages onboarding, offboarding, wallet
                        provisioning, and user authentication
                  \item Liquidity Management Service: Handles funding/defunding operations via
                        Dedicated Cash Accounts (DCAs), waterfall mechanisms
                  \item Transaction Management Service: Processes payment instruction, clearing, and
                        settlement
                  \item Settlement Layer: Maintains ledger, executes final settlement, ensures
                        atomicity
                  \item Offline Management: Handles secure element provisioning and offline transaction
                        reconciliation
            \end{enumerate}
            \vspace{5pt}
      \item \textbf{Functional Domains (three-layer architecture):}
            % \begin{figure}[H]
            %       \centering
            %       \fbox{%
            %             \begin{minipage}{0.95\textwidth}
            %                   \centering\textbf{User Domain (Front-End)}\\[6pt]
            %                   \begin{tabular}{@{} >{\centering\arraybackslash}p{0.30\textwidth}
            %                         >{\centering\arraybackslash}p{0.30\textwidth}
            %                         >{\centering\arraybackslash}p{0.30\textwidth} @{}}
            %                         \fbox{\parbox{0.25\textwidth}{\centering\textbf{Payment}     \\Instruments\\(cards, wearables, devices)}} &
            %                         \fbox{\parbox{0.25\textwidth}{\centering\textbf{User-to-App} \\Interfaces\\(mobile and web apps, banking apps)}} &
            %                         \fbox{\parbox{0.25\textwidth}{\centering\textbf{Acceptance}  \\Solutions\\(NFC, QR code, payment links)}} \\
            %                   \end{tabular}
            %             \end{minipage}%
            %       }

            %       \vspace{8pt}
            %       \textbf{$\updownarrow$ REST API Interface $\updownarrow$}\\[10pt]

            %       \fbox{%
            %             \begin{minipage}{0.75\textwidth}
            %                   \centering\textbf{PSP Domain (Front-End)}\\[6pt]
            %                   \begin{tabular}{@{} >{\centering\arraybackslash}p{0.45\textwidth}
            %                         >{\centering\arraybackslash}p{0.45\textwidth} @{}}
            %                         \fbox{\parbox{0.43\textwidth}{\centering\textbf{Distributing} \\PSP Services\\(access mgmt, onboarding)}} &
            %                         \fbox{\parbox{0.43\textwidth}{\centering\textbf{Acquiring}    \\PSP Services\\(merchant acquiring, authorization, settlement)}} \\
            %                   \end{tabular}
            %             \end{minipage}%
            %       }

            %       \vspace{8pt}
            %       \textbf{$\updownarrow$ REST API Interface $\updownarrow$}\\[10pt]

            %       \fbox{%
            %             \begin{minipage}{0.95\textwidth}
            %                   \centering\textbf{DESP Domain (Back-End)}\\[6pt]
            %                   \begin{tabular}{@{} >{\centering\arraybackslash}p{0.32\textwidth}
            %                         >{\centering\arraybackslash}p{0.32\textwidth}
            %                         >{\centering\arraybackslash}p{0.32\textwidth} @{}}
            %                         \fbox{\parbox{0.25\textwidth}{\centering\textbf{Access}      \\Management\\Service}} &
            %                         \fbox{\parbox{0.25\textwidth}{\centering\textbf{Liquidity}   \\Management\\Service}} &
            %                         \fbox{\parbox{0.25\textwidth}{\centering\textbf{Transaction} \\Management\\Service}} \\
            %                   \end{tabular}
            %             \end{minipage}%
            %       }

            %       \caption{Functional domains (three-layer architecture)}
            % \end{figure}
            \begin{figure}[!htbp]
                  \centering
                  \includegraphics[width=0.8\textwidth]{./assests/2.3.1_Functional_Domains_Functional-Domains .png}
                  \caption{Functional domains (three-layer architecture)}
                  \label{fig:functional-domains}
            \end{figure}
\end{itemize}

\subsubsection{Dedicated Cash Accounts (DCAs) and Liquidity Management}
Dedicated Cash Accounts (DCAs) serve as a core component in the Digital Euro
ecosystem, linking Payment Service Providers' (PSPs) liquidity management with
the Eurosystem’s settlement, issuance, and redemption processes within the
Digital Euro Service Platform (DESP). Each DCA is a central bank money account
held by a PSP (as the DCA holder) and accessible through the Eurosystem or
National Central Bank (NCB) as part of the DESP infrastructure. DCAs function
as the primary liquidity source for funding and defunding end users' digital
euro holdings, while supporting the issuance, redemption, and settlement of
digital euro by providing the essential liquidity bridge between PSPs and the
Eurosystem. Additionally, DCAs enable waterfall and reverse-waterfall
mechanisms, which facilitate automatic conversions between digital euro and
linked private-money sources (such as commercial bank accounts) in accordance
with holding limits and payment requirements.

\textbf{Operational Mechanics}
\begin{figure}[!htbp]
      \centering
      \includegraphics[width=0.8\textwidth]{./assests/2.3.2_Operational Mechanics_Container.png}
      \caption{Operational Mechanics (High-Level Flow)}
      \label{fig:operational-mechanics}
\end{figure}

\textbf{Reverse Waterfall (Customer Preference):}
\begin{itemize}
      \item Users can link Digital Euro wallets directly to commercial bank accounts
      \item Enables large payments without pre-funding
      \item Maintains user convenience while managing holding limits
      \item Reduces liquidity burden on PSPs
\end{itemize}

\subsubsection{Advanced Digital Euro Features and Functionalities}
\textbf{Conditional Payments:}
Conditional payments in the digital euro allow funds to be reserved and released based on predefined conditions. While the Eurosystem executes settlement and manages the reservation of funds, condition evaluation and business logic are implemented by market participants outside the core settlement layer. Conditional payments support a range of use cases such as e-commerce, subscriptions, and service delivery scenarios, with expiry mechanisms ensuring that unreleased funds are returned to the payer if conditions are not fulfilled.

\textbf{Offline Payments:}
Offline functionality is a key feature of the digital euro, designed to ensure payment resilience during network disruptions or unavailability, mimicking cash-like characteristics for enhanced accessibility and privacy. It incorporates bearer-like functionality, where digital euro value is stored locally in a secure hardware element on the user's device (such as smartphones or compatible cards), enabling proximity-based payments without online connectivity—typically via technologies like NFC for direct device-to-device transactions between payer and payee. This provides enhanced privacy, as no real-time transaction data is transmitted to Payment Service Providers (PSPs) or the Eurosystem during offline payments; details remain stored locally on devices and are synchronized for settlement and reconciliation only once connectivity is restored. Offline holdings are securely stored in tamper-resistant components, including eSIMs (software-updatable with strong mobile integration), embedded Secure Elements (eSE) (high-security hardware chips), or integrated secure enclaves (processor-embedded environments balancing security and device integration—excluding lower-security Trusted Execution Environments). Key use cases include payments in rural/remote areas with limited connectivity, emergency scenarios like network failures or power outages, bolstering overall payment system resilience, and offering greater privacy than standard online payments (though not full anonymity). Upon reconnection, transaction data is uploaded to the Eurosystem for final settlement, reconciliation, and enforcement of rules such as holding limits.

\subsection{Rulebook Development: Standards and Governance}
The digital euro Rulebook is being developed through a collaborative, iterative
process led by the Rulebook Development Group (RDG), engaging market
participants and stakeholders, including a market consultation on the draft
that generated over 2,000 comments which were reviewed and incorporated where
appropriate. Development is organised across multiple RDG workstreams and
follows a phased approach aligned with ECB design decisions and the evolving EU
legislative framework. The Rulebook covers the functional and operational model
(core services, use cases, participant roles, user journeys, and end-to-end
flows), technical requirements (APIs, integration standards, common data models
and message formats, security and cryptographic requirements, and selected
non-functional requirements to be further specified), user experience standards
(minimum consistent UX across PSPs, authentication/notifications/information
presentation, accessibility, and flexibility to innovate beyond the baseline),
compliance and risk management (fraud prevention, dispute handling, AML/KYC
alignment, resilience), an adherence and onboarding framework (conformance
expectations plus testing/onboarding processes to be further specified), and
brand rules and user protections (brand usage principles and user rights and
protections, including dispute management).

Implementation specifications convert the Rulebook’s requirements into
practical, operational guidance for PSP developers. They include front-end
specifications covering end-user device interfaces (mobile, wearable, card),
acceptance solutions (NFC, QR codes, payment links), user-journey process flows
and decision trees, integration with existing PSP apps, and support for
standards such as CPACE, EPC standards, nexo, and Berlin Group protocols. They
also include back-end specifications detailing PSP-to-DESP API interfaces and
protocols, settlement service requirements, alias lookup and account
identification, liquidity management and DCA operations, transaction-processing
workflows, and offline reconciliation procedures.\cite{ecbRuleBook}

% ========================================================================
\section{Literature Review: Integration of Global CBDC
  Experience and Technical Standards}

\subsection{Global CBDC Implementation Experiences}
While the Digital Euro represents a unique initiative focused on retail CBDC
with sophisticated integration requirements, examining comparable CBDC projects
provides valuable insights into technical and organizational challenges.

\subsubsection{Comparative Analysis of Retail CBDC Projects}
\begin{itemize}
      \item  \textbf{e-CNY (China Digital Currency Electronic Payment)}

            This project work began around 2014; pilots began in April 2020 and later
            expanded to more locations. It is often described as the world’s largest CBDC
            pilot by scale/value.

            \textbf{Key Technical Features:\cite{BIS-E-CNY}}
            \begin{itemize}
                  \item Two-tier architecture: PBoC issues/manages core elements; authorised operators
                        (commercial banks) provide e-CNY exchange/circulation services.
                  \item Offline payments supported; e-CNY can be used via apps and hardware/card-like
                        wallets in some scenarios.
                  \item Programmability: can be enabled via smart contracts to support
                        conditional/\allowbreak guaranteed payments and other use cases.
                  \item Settlement model: described as “settled upon payment”
            \end{itemize}

            \textbf{Lessons for Digital Euro:}
            \begin{itemize}
                  \item Two-tier PSP engagement model validates distribution approach
                  \item Offline functionality complexity requires long development and testing cycles
                  \item Hardware integration challenges (device manufacturers, security requirements)
                  \item User adoption highly dependent on merchant acceptance incentives
            \end{itemize}

      \item \textbf{Sand Dollar (Bahamas)}

            National rollout began 20 Oct 2020; often cited as the first nationwide retail
            CBDC launch.\cite{centralbankbahamas}

            \textbf{Key technical features: \cite{SandDollar}}
            \begin{itemize}
                  \item Inclusion-oriented wallet access: tiered wallets allow lower-KYC entry (e.g.,
                        Tier I) to broaden access, not mobile-only: available via mobile app; also
                        supports physical card access.
                  \item Two-tier distribution via authorised financial institutions; higher-tier
                        wallets can be linked to bank accounts.
            \end{itemize}

            \textbf{Lessons for Digital Euro:}

            \begin{itemize}
                  \item Trust/security are adoption drivers; identity/KYC tiering illustrates practical
                        trade-offs between access and safeguards.
            \end{itemize}

      \item \textbf{Sweden e-Krona Pilot}

            Pilot began in 2020 and produced multi-phase outputs through Phase 4 (2024);
            Riksbank notes the technical pilot project completed in 2023. Offline payments
            explored extensively in Phase 4 for situations where telecom/electricity are
            disrupted. Pilot network is described as parallel to existing card/credit
            transfer infrastructures.\cite{E-krona-riksbank}

            \textbf{Lessons for Digital Euro:\cite{E-krona-riksbank}}
            \begin{itemize}
                  \item  Offline design requires broad scenario testing (coverage gaps, emergencies,
                        outages) and is a key digital euro workstream.
                  \item Merchant integration is a critical requirement.
                  \item Iterative delivery matters - multi-phase pilots and phased rollout approaches
                        enable learning and adjustment.
            \end{itemize}

\end{itemize}

\subsection{Technical Standards and Best Practices}
\subsubsection{Payment Industry Standards Applicable to Digital Euro}

The digital euro design builds on existing payment industry standards and
market experience to reduce integration complexity, ensure interoperability,
and support PSP onboarding. The Eurosystem aims to align with widely adopted
payment messaging and technical standards where appropriate, using structured
and machine-readable formats common in European payment systems, with specific
standards and message formats to be further detailed in the Rulebook and
technical documentation. It also reflects PSP experience with EU regulation
such as PSD2 by promoting familiar integration approaches, leveraging existing
technical infrastructure and operational processes where possible. The digital
euro is intended to complement, not replace, existing instruments like SEPA
credit transfers and instant payments; design choices are informed by SEPA
Instant Credit Transfer experience, but without reusing SEPA payment rails. For
the front end, market participants are expected to develop user interfaces and
acceptance technologies, potentially reusing industry standards where
appropriate, while the Eurosystem does not currently mandate specific front-end
scheme standards.\cite{ecbRuleBook}

\subsubsection{Cybersecurity and Privacy Standards}
The digital euro adopts a GDPR-aligned privacy approach based on data
minimisation and segregated information, using distributed and segregated
processing across DESP components so end-user identities are not visible to the
Eurosystem. PSPs onboard users under existing AML/KYC obligations while
collecting only the personal data necessary under GDPR, and they submit
transactions to the DESP for settlement using pseudonymous identifiers so the
ECB cannot link a transaction to a specific individual. For offline payments,
privacy is reinforced through near-instant, device-to-device transfers of
cryptographically secure tokens, with transaction details remaining on the
devices and not shared with PSPs or the Eurosystem during or after the offline
process. Cybersecurity is designed around state-of-the-art controls—secure
development practices, cryptographic agility, reduced attack surface, and
regular planned testing against simulated cyberattacks backed by strong
governance—and secure-element form factors for offline use (eSE, integrated
secure elements/secure enclaves, and eSIM) are assessed as capable of achieving
security equivalent to Common Criteria AVA-VAN.5.

\subsection{Cost and Feasibility Studies: Synthesis and Analysis}
\paragraph{Primary Cost Research}

\textbf{PwC Digital Euro Cost Study (2025)}\cite{PwC}

\textbf{Scope:} 19 participating retail banks across the euro area, with bank sizes captured in four total-asset brackets:
€30bn and below, €30–€100bn, €100bn–€1tr, and above €1tr.

\textbf{Key Findings:}
\begin{itemize}
      \item Average implementation (change) cost per participating bank: €110 million
      \item Euro area extrapolation of baseline change costs: €18 billion.
      \item Indicative additional scenario: change costs could be as high as €30 billion
            when other functionalities and multiple intermediaries are included and lower
            synergies are assumed.
      \item Technical requirements are the main cost driver: around 75\% of total estimated
            costs (reported as above €1.5 billion for the study panel).
\end{itemize}

\textbf{Cost Distribution by Service Bundle (PwC Payment Layer Model):}\\
PwC provides an aggregated view of average costs per service bundle, summing to an average-bank bundle total of €124 million.

\begin{table}[ht]
      \centering
      \caption{Cost Distribution by Service Bundle (PwC Payment Layer Model)}

      \begin{center}

            % --- light colors ---
            \definecolor{TableHeader}{RGB}{128,128,128} % light blue header
            \definecolor{RowA}{RGB}{222,222,222}        % very light mint
            \definecolor{RowB}{RGB}{211,211,211}        % very light lavender
            \definecolor{RowC}{RGB}{222,222,222}       % very light peach
            \definecolor{RowD}{RGB}{211,211,211}        % very light sky
            \definecolor{RowE}{RGB}{222,222,222}        % very light gray

            \begin{tabular}{@{} >{\raggedright\arraybackslash}p{0.52\textwidth} >{\raggedright\arraybackslash}p{0.20\textwidth} >{\raggedright\arraybackslash}p{0.20\textwidth} @{}}
                  \hline
                  \rowcolor{TableHeader}
                  \textbf{Service bundle}                                   & \textbf{Average cost (mn€)} & \textbf{Share of €124mn} \\
                  \hline
                  \rowcolor{RowA} Customer contracts                        & 5                           & 4.0\%                    \\
                  \rowcolor{RowB} Legal                                     & 2                           & 1.6\%                    \\
                  \rowcolor{RowC} Marketing \& sales                        & 7                           & 5.6\%                    \\
                  \rowcolor{RowD} Market launch                             & 4                           & 3.2\%                    \\
                  \rowcolor{RowE} Payment channels                          & 23                          & 18.5\%                   \\
                  \rowcolor{RowA} Branch \& ATM network                     & 11                          & 8.9\%                    \\
                  \rowcolor{RowB} POS terminal \& e-commerce infrastructure & 17                          & 13.7\%                   \\
                  \rowcolor{RowC} Accounts                                  & 15                          & 12.1\%                   \\
                  \rowcolor{RowD} Liquidity                                 & 9                           & 7.3\%                    \\
                  \rowcolor{RowE} Risk \& compliance                        & 7                           & 5.6\%                    \\
                  \rowcolor{RowA} Interfaces                                & 12                          & 9.7\%                    \\
                  \rowcolor{RowB} Fee calculation                           & 2                           & 1.6\%                    \\
                  \rowcolor{RowC} Reporting \& payment statistics           & 2                           & 1.6\%                    \\
                  \rowcolor{RowD} Data management                           & 2                           & 1.6\%                    \\
                  \rowcolor{RowE} Processes                                 & 6                           & 4.8\%                    \\
                  \hline
            \end{tabular}

      \end{center}
\end{table}

\textbf{Key Caveats:}
\begin{itemize}
      \item Baseline estimates exclude offline functionality, multiple accounts, and
            merchant acquiring.
      \item Estimates reflect the rulebook version used in the study (v0.8a); requirements
            may evolve in later rulebook iterations.
      \item Resource constraint: respondents estimate that about 46\% of relevant skilled
            resources would be tied up per year for a period of four years.
\end{itemize}

\bigskip
\textbf{ECB assessment / synthesis of digital euro investment costs (October 2025 note)\cite{ecbCost}}

\textbf{Approach:} The note builds on banking-sector cost studies (including PwC) and incorporates (i) design-consistent adjustments and (ii) quantified synergy/cost-mutualisation assumptions (banking-group and market synergies).

\textbf{Adjusted baseline costs (design-consistent adjustments):}
\begin{itemize}
      \item Physical cards: -€6.0 million (card form factor uses existing infrastructure;
            PSPs already have card-issuing capabilities and rely on specialised providers). \item Point-of-sale terminals: -€7.0 million (natural refresh cycles and shift to
            smart/soft POS reduce incremental replacement needs).
      \item ATMs: -€5.1 million (existing NFC/QR support, gradual upgrades, and
            outsourcing/utilities/IAD trends reduce incremental costs).
      \item Fee calculation component: -€2.0 million (handled by the Eurosystem).
      \item Overall: average investment costs reduced by about 16\% (about €20.1 million
            out of €124 million).
\end{itemize}

\textbf{Adjusted Average Costs by Bank Size (downward-adjusted PwC estimates):}
\begin{itemize}
      \item Top banks (above €1 trillion assets): €152 million
      \item Large banks (€100–€1,000 billion assets): €89 million
      \item Medium banks (€30–€100 billion assets): €24 million
      \item Small banks (below €30 billion assets): €8 million
\end{itemize}

\textbf{Euro Area Total with Synergies:}
\begin{itemize}
      \item Base synergies case (banking-group synergies 90--98\%, weighted average market
            synergy factor 30\%): adjusted PwC estimates extrapolate to €5.77 billion;
            other available but undisclosed banking studies imply €4.0--€4.2 billion.
      \item High synergies case (market synergy factor 40\%, banking-group synergies
            maintained at 90--98\%): adjusted PwC estimates extrapolate to €5.07 billion.
            % \item Comparison to European Commission range: the note cites the Commission estimate
            %       of €2.8 to €5.4 billion for euro area banks; €5.77 billion is described as
            %       close to the upper bound, while €5.07 billion falls within the
            %       range.
\end{itemize}

% \textbf{Supplementary Evidence on Bank IT Cost Structures (CIPA IT Survey, Italy)}

% CIPA -- Rilevazione sull’IT nel settore bancario italiano (Profili economici e
% organizzativi, Esercizio 2024)

% Scope: 23 banking groups and 32 banks participating in the CIPA/ABI survey
% (with participating groups representing 92 percent of Italian banking groups by
% consolidated total assets).

% \textbf{Key Findings (banking groups sample):}

% Total IT \emph{TCO} (operating expenses + depreciation) reported for 2024:
% €6,334 million. Total IT \emph{cash out} (operating expenses + investments)
% reported for 2024: €6,618 million. Split of cash out into operating expenses
% vs. investments: €4,478 million operating expenses and €2,140 million
% investments; depreciation reported at €1,857 million. Across the groups that
% provided sufficient granularity, the aggregate cash-out mix is reported as 67.7% operating expenses and 32.3% investments. 

% Indicative Cost Allocation by Thematic Area (cash out, group sample with
% sufficient granularity):
% \begin{center}
%       \begin{tabular}{@{} >{\raggedright\arraybackslash}p{0.70\textwidth} r @{}}
%             \hline
%             \textbf{Thematic area (CIPA)}                       & \textbf{Share of IT cash out} \\
%             \hline
%             Applications                                        & 51.6\%                        \\
%             Data center                                         & 21.7\%                        \\
%             Peripheral systems (end-user / distributed systems) & 11.9\%                        \\
%             Transmission systems (networks/telecom)             & 4.9\%                         \\
%             IT Security                                         & 4.3\%                         \\
%             Unclassified IT costs                               & 5.6\%                         \\
%             \hline
%       \end{tabular}
% \end{center}
% \noindent (Reported as average shares for the subset of groups providing sufficient detail.)

% \textbf{Sourcing / Outsourcing Signal (bank component; small-bank sub-sample):}

% In the CIPA appendix breakdown for a small-bank sub-sample (“10 banche Piccole
% A e B”), outsourcing-related cost lines are material, with:

% \emph{Outsourcing IT} provided by banks or group instrumental companies: 24.04 percent of total IT costs (TCO) for the sub-sample; and
% \emph{Outsourcing IT} provided by external suppliers: 32.03 percent of total IT costs (TCO) for the sub-sample.
% (Interpretation: in this sub-sample, the two outsourcing lines sum to 56.07 percent of TCO, indicating strong reliance on third-party/group IT service provision.)

% \textbf{IT Capacity Indicator:}

% The survey reports a total of 12,887 IT employees across the 23 participating
% groups as of 31 December 2024.

% \textbf{Relevance to Digital Euro Cost/Feasibility (contextual, not\linebreak
%       digital-euro-specific):}

% While CIPA measures overall bank IT cost structures (not digital euro
% implementation costs), the dominance of \emph{applications} and \emph{data
%       centre/infrastructure} within cash out provides an empirical baseline
% suggesting that major payment-rail changes are likely to load primarily onto
% application change and infrastructure operations rather than purely “front-end
% only” adaptations. The outsourcing intensity observed (especially for smaller
% banks) is directionally consistent with the view that higher outsourcing
% concentration can enable greater cost mutualisation/synergies for shared
% payment initiatives (e.g., reuse of common vendor platforms and shared
% infrastructures).

% \textbf{Key Caveats:}

% The CIPA survey is Italy-specific and covers bank/group-wide IT (not a digital
% euro programme budget), so figures should be used as a benchmark for
% \emph{baseline IT cost structure and sourcing patterns}, not as direct digital
% euro cost estimates. The “small banks” outsourcing shares are drawn from a
% defined sub-sample table and may not generalise to all Italian banks or to euro
% area banks.

% \paragraph{Bridge mapping: CIPA ``aree tematiche'' to PwC service bundles / layers (heuristic).}
% The CIPA survey reports bank-group-wide IT cost structures by \emph{thematic
%       area} (e.g., Applications, Data center, Transmission systems), while PwC
% estimates \emph{incremental change costs} for implementing the digital euro
% across a structured set of \emph{service bundles} grouped into three layers
% (commercial, technical, operational). Accordingly, the mapping below is a
% \emph{conceptual bridge} (not a conversion): it indicates where incremental
% digital-euro work would most likely land within typical bank IT cost buckets.%

% \begin{center}
%       \begin{tabular}{@{} >{\raggedright\arraybackslash}p{0.17\textwidth} >{\raggedright\arraybackslash}p{0.17\textwidth} >{\raggedright\arraybackslash}p{0.24\textwidth} >{\raggedright\arraybackslash}p{0.30\textwidth} @{}}
%             \hline
%             \textbf{CIPA thematic area}                                                                                                                                                                                  & \textbf{Closest PwC layer(s)}                                          & \textbf{Closest PwC service bundle(s) / ECB grouping} & \textbf{Rationale for the bridge (why this is the closest match)}                                                                                     \\
%             \hline
%             \textbf{Applicazioni}                                                                                                                                                                                        & Mainly \textbf{Technical} + \textbf{Operational} (and some Commercial) &
%             \emph{Payment channels; POS \& e-commerce; Interfaces; Accounts; Liquidity; Risk \& compliance; Invoicing/reporting bundles (fee calculation, reporting \& payment statistics, data management, processes).} &
%             CIPA ``Applications'' captures software build/change and application maintenance. PwC places the bulk of digital euro work in the technical layer (payment channels, POS/e-com, interfaces, branch/ATM, accounts/liquidity, risk/compliance), and the operational layer (fee calculation, reporting, data management, processes). Thus, most incremental digital-euro change costs are application-heavy and map most directly here.                                                                  \\
%             \hline
%             \textbf{Data center}                                                                                                                                                                                         & \textbf{Technical} + \textbf{Operational}                              &
%             \emph{Interfaces; Accounts; Liquidity; POS \& e-commerce; Branch \& ATM} (supporting infrastructure)                                                                                                         &
%             CIPA ``Data center'' (mainframe/server farm) represents compute/storage hosting the core and payments stack. PwC technical layer includes backends and connectivity components (e.g., access gateway/issuer hub connections), plus account/liquidity functions; operational functions (reporting/fee calc) also run on these platforms.                                                                                                                                                               \\
%             \hline
%             \textbf{Sistemi trasmissivi} (networks/telecom)                                                                                                                                                              & \textbf{Technical}                                                     &
%             \emph{Interfaces} (connectivity), plus \emph{Payment channels} and \emph{POS \& e-commerce}, \emph{Branch \& ATM}                                                                                            &
%             CIPA ``Transmission systems'' underpins end-to-end connectivity (bank front-ends to back-ends; bank systems to external rails/platforms). PwC highlights interfaces (authorisation backend, access gateway connectivity, issuer hub connectivity) and the channel/acceptance estate (POS/ATM) as key technical-layer elements requiring adaptation.                                                                                                                                                   \\
%             \hline
%             \textbf{Sistemi periferici} (distributed/end-user systems)                                                                                                                                                   & \textbf{Technical} (edge/channel estate)                               &
%             \emph{Payment channels; Branch \& ATM; POS \& e-commerce}                                                                                                                                                    &
%             CIPA ``Peripheral systems'' covers end-user/distributed systems and (in CIPA drilldowns) includes items such as distributed equipment and channel devices; these correspond most closely to PwC’s bank-side channel and acceptance touchpoints (apps/frontends, branch/ATM, POS-related components).                                                                                                                                                                                                  \\
%             \hline
%             \textbf{Sicurezza IT}                                                                                                                                                                                        & Cross-cutting; closest to \textbf{Technical} + \textbf{Operational}    &
%             \emph{Risk \& compliance} (fraud/dispute/risk), plus security controls across interfaces/channels                                                                                                            &
%             CIPA reports IT security as a distinct cost area. In PwC, security-driven work concentrates in ``Risk \& compliance'' (e.g., fraud prevention, dispute management, authorisation controls) and also affects interface hardening and operational processes (monitoring, incident handling, reporting).                                                                                                                                                                                                 \\
%             \hline
%             \textbf{Costi IT non classificabili}                                                                                                                                                                         & Not directly mappable                                                  & N/A                                                   & Residual/unclassified costs cannot be reliably mapped to PwC bundles; they may include cross-cutting overheads and accounting allocation differences. \\
%             \hline
%       \end{tabular}
% \end{center}

% \paragraph{Empirical anchor from CIPA (context, not conversion).}
% For 2024, CIPA reports that the \emph{cash out IT} distribution by thematic
% area (average shares for the subset of groups with sufficient granularity) is
% dominated by \emph{Applications} (51.6\%) and \emph{Data center} (21.7\%),
% followed by \emph{Peripheral systems} (11.9\%), \emph{Transmission systems}
% (4.9\%), and \emph{IT Security} (4.3\%), with \emph{unclassified} at 5.6\%.
% This pattern is directionally consistent with PwC’s finding that
% technical-layer (system/channel/interface) work is the dominant component of
% digital-euro change costs.%

% \paragraph{Caveats (critical).}
% This bridge is \emph{heuristic}: CIPA measures \emph{overall bank/group IT
%       spending}, while PwC estimates \emph{incremental digital-euro implementation
%       change costs}. Therefore, the mapping should be read as indicating \emph{where}
% digital-euro change effort is likely to fall within typical bank IT cost
% buckets, not as a method to translate CIPA percentages into euro-denominated
% digital-euro budgets.

\subsection{Bank Integration Case Studies: Implementation Approaches}

\begingroup
\small
\setlength{\tabcolsep}{3pt}
\renewcommand{\arraystretch}{1.25}

% --- light colors ---
\definecolor{HeaderBlue}{RGB}{211,211,211}
\definecolor{BodyTint}{RGB}{240,240,240}        % very light sky

\begin{xltabular}{\textwidth}{>{\bfseries}p{3.1cm} >{\RaggedRight\arraybackslash}X >{\RaggedRight\arraybackslash}X >{\RaggedRight\arraybackslash}X}
      \caption{Digital Euro Integration Approaches Across Bank Tiers: Profiles, Integration, Cost, and Risk}
      \label{tab:digital-euro-integration-approaches}\\

      \toprule
      \rowcolor{HeaderBlue}
      & \textbf{In-House (High-Tier Banks)} & \textbf{Outsourced (Low-Tier Banks)} & \textbf{Hybrid (Mid-Tier Banks)} \\
      \midrule
      \endfirsthead

      \caption[]{Digital Euro Integration Approaches Across Bank Tiers (continued)}\\
      \toprule
      \rowcolor{HeaderBlue}
      & \textbf{In-House (High-Tier Banks)} & \textbf{Outsourced (Low-Tier Banks)} & \textbf{Hybrid (Mid-Tier Banks)} \\
      \midrule
      \endhead

      \midrule
      \multicolumn{4}{r}{\small\itshape Continued on next page}\\
      \endfoot

      \bottomrule
      \endlastfoot

      \rowcolors{1}{HeaderBlue}{HeaderBlue}

      Typical Profile & Banks with more than \EUR{}500 billion in assets have
      advanced IT capabilities, operate in a decentralized way across multiple
      jurisdictions, and serve a large retail base that requires sophisticated
      features. & Smaller regional and community banks with \EUR{}10--100 billion in
      assets have limited IT development capacity, depend on third parties for core
      systems, and focus on traditional relationships and local markets. & Mid-sized
      banks with \EUR{}50--300 billion in assets seek a balance of efficiency and
      differentiation, maintain partial internal IT capabilities with selective
      outsourcing, and focus strategically on specific value-added services and
      differentiated offerings. \\ \addlinespace

      Integration Characteristics & Fully develop interfaces and middleware in-house
      using custom microservices, integrate fraud detection and risk management,
      apply advanced real-time analytics, and run dedicated Digital Euro units with
      specialized teams. & Integrate established vendor Digital Euro platforms with
      minimal in-house development, rely on vendor compliance and fraud tools, accept
      limited customization and standard features, and usually use licensing or SaaS
      models. & Outsource core integration to vendors while building proprietary
      value-added services in-house, customize integration with core banking systems,
      make selective build-versus-buy decisions, and sometimes collaborate with peers
      on shared infrastructure. \\ \addlinespace

      Cost Implications & High upfront costs in the \EUR{}150--200 million range,
      intensive use of IT staff over 3--4 years (roughly 50--60\% of senior IT
      capacity), but lower long-term operating costs and the ability to gain
      competitive advantages through differentiated features. & Lower development
      costs (\EUR{}20--50 million) with ongoing licensing or SaaS fees, reduced
      internal IT burden (10--20\%), and shared infrastructure costs, but less
      synergy and increased lock-in due to limited vendor choice. & Moderate costs of
      \EUR{}60--120 million, blended vendor and internal spending, significant IT
      resource use (30--40\%), and phased implementation (vendor core integration
      first, enhancements over time). \\ \addlinespace

      Risk Profile & High execution risk in large, complex programs, significant
      resource diversion from other innovation, a maintenance burden for the entire
      stack, and full regulatory accountability for security compliance. & Vendor
      dependency and high migration costs, feature constraints tied to vendor
      roadmaps, exposure to vendor stability issues, and reduced competitive
      differentiation when many banks share similar feature sets. & Balanced risk
      posture with reduced vendor dependency but added integration complexity,
      governance challenges in coordinating internal and vendor work, and a need for
      strong organizational alignment between business and technical strategy. \\

\end{xltabular}
\endgroup

% ========================================================================
\section{Technical Architecture of DESP and Bank Back-End Integration}
\subsection{DESP Architecture Overview and Core Components}
\subsubsection{Structural Design Principles}
The Eurosystem’s envisaged Digital Euro Service Platform (DESP) design
emphasises (i) \emph{distributed and segregated processing} with
\emph{minimised data and segregated information} such that end-user identities
are not visible to the Eurosystem and transactions can be sent using
pseudonymous identifiers, alongside a \emph{multi-region} central-ledger set-up
across three regions with rerouting to maintain continuity in adverse
scenarios; (ii) a \emph{stateful} back-end processing approach that retains
transaction state throughout the pre-settlement process to reduce complexity,
repetitive steps and processing times; (iii) a \emph{two-layer} arrangement for
conditional payments comprising a Eurosystem-provided \emph{settlement layer}
and a market-participant \emph{conditionality layer} to define/trigger
conditions while preserving integrity and privacy; and (iv) a \emph{synchronous
      REST interface} between PSPs and the back-end infrastructure, developed with
industry involvement and aligned with standards familiar from PSD2
implementation, to support low-latency transactions and cost-efficient
integration with existing systems.

\subsubsection{Core DESP Services and Functions}

\begin{table}[htbp]
      \centering
      \scriptsize
      \setlength{\tabcolsep}{4pt}
      \renewcommand{\arraystretch}{1.1}

      % --- more light colors ---
      \definecolor{TableHeader}{RGB}{180,180,180} % stronger light blue header
      \definecolor{RowA}{RGB}{222,222,222}        % very light blue
      \definecolor{RowB}{RGB}{211,211,211}        % very light mint
      \definecolor{RowC}{RGB}{222,222,222}        % very light lavender
      \definecolor{RowD}{RGB}{211,211,211}        % very light peach
      \definecolor{RowE}{RGB}{222,222,222}        % very light gray

      \begin{tabularx}{\textwidth}{>{\raggedright\arraybackslash}p{2.8cm} X >{\raggedright\arraybackslash}p{3.0cm} >{\raggedright\arraybackslash}p{3.6cm}}
            \toprule
            \rowcolor{TableHeader}
            \textbf{Service bundle}                                                                                                                                                         & \textbf{Key functions (what)}                                & \textbf{Responsibility (who)} & \textbf{Integration touchpoints (how)} \\
            \midrule

            \rowcolor{RowA}
            Access Management                                                                                                                                                               &
            Onboarding/offboarding, wallet lifecycle, alias registration/lookup, authentication/credentials, form factors (incl. offline-capable devices), waterfall account configuration. &
            PSP                                                                                                                                                                             &
            KYC/AML/CFT handled locally; exchange of scheme identifiers (DEAN, aliases, PSP IDs); integrate with alias look-up and onboarding-related scheme processes.                                                                                                                                                             \\

            \rowcolor{RowB}
            Liquidity Management                                                                                                                                                            & DCA monitoring and funding/defunding; waterfall and
            reverse-waterfall logic; limit checks; reconciliation and reporting.                                                                                                            & PSP
            treasury + DESP                                                                                                                                                                 & Treasury integration for DCA operations in
            central bank money; automated funding/defunding execution; reconciliation with
            platform settlement outputs.                                                                                                                                                                                                                                                                                            \\

            \rowcolor{RowC}
            Transaction Management                                                                                                                                                          & Channel payment initiation (POS/e-com/P2P), validation
            and limits; reservation/conditional flows; settlement verification/recording at
            platform layer; auditability.                                                                                                                                                   & PSP + Eurosystem (settlement)                                & Integrate bank
            authorisation/switching to scheme connectivity; support reservation/payment
            lifecycles; align internal ledgers and audit trails with settlement
            confirmations.                                                                                                                                                                                                                                                                                                          \\

            \rowcolor{RowD}
            Offline Service                                                                                                                                                                 & Offline holdings on device; secure local
            verification/recording; device-to-device offline payments; online
            reconciliation after reconnection; loss/theft recovery handling.                                                                                                                & PSP + scheme
            offline model(DESP)                                                                                                                                                             & Device/form-factor enablement; offline controls and recovery
            workflows; support reconciliation processes and customer support operations.                                                                                                                                                                                                                                            \\

            \rowcolor{RowE}
            Risk, Fraud, Compliance \& Disputes                                                                                                                                             & PSP AML/CFT and fraud controls; dispute
            intake and case handling; consumption of scheme/platform risk/fraud signals (if
            provided).                                                                                                                                                                      & PSP                                                          & Integrate fraud/risk
            engines with digital-euro payment flows; dispute workflows aligned to scheme
            processes; reporting interfaces and controls.                                                                                                                                                                                                                                                                           \\

            \bottomrule
      \end{tabularx}

      \caption{Digital euro service bundles, responsibilities, and bank integration touchpoints.}
      \label{tab:deuro_services_integration}
\end{table}

\subsection{Bank Back-End System Integration Pathways}
\subsubsection{Core System Integration Architecture}

Banks must integrate the DESP with existing back-end systems across multiple
dimensions:

\definecolor{HeaderBlue}{RGB}{210,228,255}
\definecolor{BodyTint}{RGB}{236,246,255}
\begin{xltabular}{\textwidth}{>{\raggedright\arraybackslash}p{3.2cm} X X}
      \caption{Bank integration dimensions for DESP connectivity and internal system impacts.}
      \label{tab:bank_desp_integration_matrix}\\
      \toprule
      \rowcolor{HeaderBlue}
      \textbf{Dimension} & \textbf{What changes (capabilities)} & \textbf{Primary bank systems impacted} \\
      \midrule
      \endfirsthead

      \toprule
      \textbf{Dimension} & \textbf{What changes (capabilities)} & \textbf{Primary bank systems impacted} \\
      \midrule
      \endhead

      \midrule
      \multicolumn{3}{r}{\emph{Continued on next page}}\\
      \endfoot

      \bottomrule
      \endlastfoot
      \rowcolor{BodyTint}
      Core banking / ledger integration & Customer reference mapping (Customer ID
      $\leftrightarrow$ DEAN), alias linkage, holding-limit controls
      (waterfall/reverse-waterfall), statement views, sub-ledger postings for
      funding/defunding and DCA movements; fee/billing policy (if applicable). &
      Customer master/reference data, sub-ledger/GL, statement engine, limits engine,
      billing/invoicing (optional). \\
      \rowcolor{BodyTint}

      Integration layer & Connector to digital euro access gateway; security/auth;
      workflow orchestration (funding/defunding, reservations/payments);
      mapping/transformation; eventing and monitoring. & API management/connector
      services, ESB/iPaaS, message bus/streaming, orchestration engine,
      observability. \\ \rowcolor{BodyTint}

      Channels & Mobile/web UX; optional branch/ATM distribution workflows;
      acquiring/acceptance enablement where the bank provides merchant services;
      notifications and receipts. & Mobile/web apps, branch teller platform
      (optional), ATM switch (optional), acquiring gateway/merchant portal (if
      applicable), notification services. \\ \rowcolor{BodyTint}

      Back-office operations & Treasury/DCA liquidity monitoring and replenishment;
      AML/CFT and fraud controls; dispute/case management; customer support;
      reconciliation and reporting/statistics. & Treasury systems, compliance
      screening, fraud engine, dispute/case tool, CRM/call centre, reconciliation \&
      reporting stack. \\

\end{xltabular}

\subsubsection{Data Model Mapping and Transformation \cite{ecbPrep} \cite{ecbRuleBook}}

The DESP operates with specific data models that banks must map to internal
representations:

% \textbf{Digital Euro Account Number (DEAN):}

% A DEAN is a unique identifier for a digital euro account. During service
% opening, the PSP requests a DEAN from the DESP; the DESP generates the DEAN and
% returns it to the PSP, which assigns serve as scheme identifiers in payment
% flows and support inter-PSP exchange via PSP mapping (linking DEAN to a PSP
% identifier). Portability requirements imply that a user may change PSP while
% keeping the same DEAN, so PSPs must manage lifecycle and mapping changes
% accordingly. 

% \textbf{Alias look-up and alias management:}

% Users may register an alias (e.g., account number, phone number, email
% address). The alias look-up service stores aliases and links them to scheme
% identifiers (including DEAN). PSPs must provide capabilities to
% register/change/disable aliases, apply strong customer authentication, obtain
% user consent, verify alias ownership, and keep alias records updated; they are
% liable for incorrect alias–account association. When a payment instruction
% contains an alias, PSPs must resolve it via the alias look-up service as part
% of transaction processing.

% \textbf{Message semantics and interface style:}

% The PSP interfaces reuse the ISO 20022 data dictionary where applicable, while
% allowing digital-euro-specific elements. The scheme specifications are
% documented in a market-standard RESTful API style. Accordingly, PSPs should map
% internal payment objects to the scheme’s canonical data elements (ISO 20022
% dictionary where applicable) and exchange them through the specified APIs,
% ensuring consistent ability across systems.

% \textbf{Privacy and pseudonymisation:}

% The scheme aims to preserve user privacy by ensuring the Eurosystem/DESP domain
% processes pseudonymised identifiers rather than directly identifying
% individuals. PSPs retain the mapping between real-world identity and scheme
% identifiers (e.g., DEAN/aliases) within their regulated compliance perimeter,
% enabling PSP-side AML/CFT controls and legally grounded reporting while
% limiting unnecessary data exposure.

\begin{table}[htbp]
      \centering
      \small
      \setlength{\tabcolsep}{5pt}
      \renewcommand{\arraystretch}{1.15}
      \definecolor{TableHeader}{RGB}{180,180,180} % stronger light blue header
      \definecolor{RowA}{RGB}{222,222,222}
      \begin{tabularx}{\textwidth}{
            >{\raggedright\arraybackslash}p{2.7cm}
            X
            >{\raggedright\arraybackslash}p{4.2cm}}
            \toprule
            \rowcolor{TableHeader}
            \textbf{Scheme artefact}                                                                                                                           & \textbf{Meaning in the scheme} & \textbf{PSP internal mapping (recommended)} \\
            \midrule
            \rowcolor{RowA}
            DEAN                                                                                                                                               &
            Unique identifier of the digital euro account; generated by DESP upon PSP request and assigned to the user by the PSP.                             &
            Maintain \emph{Customer/Wallet-ID $\leftrightarrow$ DEAN} in a reference store; support lifecycle events (provisioning, offboarding, portability).                                                                                \\
            \rowcolor{RowA}
            Alias                                                                                                                                              &
            User-friendly identifier (e.g., account number / phone / email) stored in the alias look-up service and linked to scheme identifiers (incl. DEAN). &
            Maintain \emph{Alias $\leftrightarrow$ DEAN} locally for performance; enforce consent + SCA + ownership verification; propagate updates/deactivation promptly.                                                                    \\
            \rowcolor{RowA}
            Transaction data elements                                                                                                                          &
            Settlement interfaces reuse ISO 20022 data dictionary where applicable; exchanged via RESTful APIs with digital-euro-specific elements as needed.  &
            Implement canonical mapping layer: internal payment object $\rightarrow$ scheme data elements; versioned transformations + validation + audit logging.                                                                            \\
            \bottomrule
      \end{tabularx}
      \caption{Core mapping objects required for PSP integration with the digital euro scheme.}
\end{table}

\textbf{Bank integration requirements (structured).}\par

\begin{tcolorbox}[reqbox,title=DEAN integration requirements]
      \begin{itemize}[leftmargin=*,nosep]
            \item Maintain a controlled mapping: \texttt{Customer/Wallet-ID} $\leftrightarrow$
                  \texttt{DEAN}.
            \item Treat DEAN as portable: support reassignment of \texttt{DEAN} $\rightarrow$
                  \texttt{PSP context} during PSP switching while keeping audit history.
            \item Segregate identity/KYC data from scheme identifiers; expose only pseudonymised
                  identifiers to scheme-facing services.
      \end{itemize}
\end{tcolorbox}

\begin{tcolorbox}[reqbox,title=Alias integration requirements]
      \begin{itemize}[leftmargin=*,nosep]
            \item Provide register/change/disable flows; apply SCA and capture explicit user
                  consent.
            \item Verify alias ownership before registration; keep alias state updated and
                  deactivate promptly upon changes.
            \item Resolve aliases in payment processing by calling the alias look-up service when
                  an instruction contains an alias.
      \end{itemize}
\end{tcolorbox}

\begin{tcolorbox}[reqbox,title=Message mapping requirements]
      \begin{itemize}[leftmargin=*,nosep]
            \item Map internal payment objects to scheme canonical elements (ISO 20022 data
                  dictionary where applicable).
            \item Implement REST API adapters; apply schema validation, idempotency, and audit
                  logging end-to-end.
            \item Version mappings to handle evolving rulebook/specification revisions.
      \end{itemize}
\end{tcolorbox}

\subsubsection{Liquidity Management Integration: DCA Operations}
The DCA represents the critical bridge between bank liquidity management and
Digital Euro distribution

\begingroup
\small
\setlength{\tabcolsep}{5pt}
\renewcommand{\arraystretch}{1.2}

% Optional colors (you can remove these if you don't want shading)
\definecolor{HeaderGray}{RGB}{211,211,211}

\begin{xltabular}{\textwidth}{
      >{\raggedright\arraybackslash}p{2.7cm}
      >{\RaggedRight\arraybackslash}X
      >{\raggedright\arraybackslash}p{3.2cm}
      }
      \caption{DCA-linked liquidity flows in the digital euro scheme: funding, defunding, waterfall and reverse waterfall.}
      \label{tab:dca_flows_matrix}\\

      \toprule
      \rowcolor{HeaderGray}
      \textbf{Flow} & \textbf{Trigger \& business meaning} & \textbf{Key DCA implication} \\
      \midrule
      \endfirsthead

      \caption[]{DCA-linked liquidity flows in the digital euro scheme (continued).}\\
      \toprule
      \rowcolor{HeaderGray}
      \textbf{Flow} & \textbf{Trigger \& business meaning} & \textbf{Key DCA implication} \\
      \midrule
      \endhead

      \midrule
      \multicolumn{3}{r}{\small\itshape Continued on next page}\\
      \endfoot

      \bottomrule
      \endlastfoot

      Funding (top-up) & Customer initiates conversion from private money to digital
      euro (wallet funding). & Platform credits user digital euro; intermediary
      liquidity is adjusted via DCA (with liquidity transfers via TARGET CLM as
      needed). \\ \addlinespace

      Defunding (withdrawal) & Customer initiates conversion from digital euro to
      linked private-money account. & Platform debits user digital euro; intermediary
      DCA is credited; PSP credits linked account and reconciles postings. \\
      \addlinespace

      Waterfall (defunding on overflow) & Automatic conversion of excess digital euro
      into private money when holding limit would be exceeded. & Inbound amount above
      limit is defunded; linked account credited; DCA adjusted accordingly. \\
      \addlinespace

      Reverse waterfall (auto-funding for payment) & Online payment exceeds user’s
      current digital euro holdings; missing amount is auto-funded from linked
      private-money account. & DCA supports the issuance/funding leg while payment
      settles; requires strong controls and user settings. \\

\end{xltabular}
\endgroup

\subsubsection{Multi-Channel Integration: Enabling Diverse Payment Methods}
PSPs must integrate Digital Euro across multiple customer interaction channels
\begingroup \small \renewcommand{\arraystretch}{1.2}
\setlength{\tabcolsep}{5pt}

% Optional header color (remove if you don't want shading)
\definecolor{HeaderGray}{RGB}{211,211,211}

\begin{xltabular}{\textwidth}{
      >{\raggedright\arraybackslash}p{2.3cm}
      >{\raggedright\arraybackslash}p{2.0cm}
      >{\RaggedRight\arraybackslash}X
      >{\raggedright\arraybackslash}p{2.5cm}
      >{\raggedright\arraybackslash}p{2.0cm}
      }
      \caption{Multi-channel integration view for digital euro capabilities (conceptual).}
      \label{tab:multi_channel_integration}\\

      \toprule
      \rowcolor{HeaderGray}
      \textbf{Channel} & \textbf{Mode} & \textbf{Primary capabilities} & \textbf{Key dependencies} & \textbf{Complexity} \\
      \midrule
      \endfirsthead

      \caption[]{Multi-channel integration view for digital euro capabilities (continued).}\\
      \toprule
      \rowcolor{HeaderGray}
      \textbf{Channel} & \textbf{Mode} & \textbf{Primary capabilities} & \textbf{Key dependencies} & \textbf{Complexity} \\
      \midrule
      \endhead

      \midrule
      \multicolumn{5}{r}{\small\itshape Continued on next page}\\
      \endfoot

      \bottomrule
      \endlastfoot

      POS & Online + Offline & Contactless acceptance (NFC priority), QR fallback;
      merchant confirmation/receipt; online authorisation vs offline device-to-device
      confirmation with later reconciliation. & Common acceptance layer (QR),
      terminal protocol, reconciliation. & High \\ \addlinespace

      E-commerce & Online & Checkout integration (button/redirect/API); wallet-based
      authorisation; order-session correlation; pay-by-link support. & Gateway/plugin
      integration, PSP back-end services, session correlation, monitoring. & Medium
      \\ \addlinespace

      P2P & Online + Offline (proximity) & Alias/contact-based recipient selection;
      wallet authorisation; confirmation/receipts; QR optional. & Alias resolution,
      PSP fraud controls, customer support flows, reconciliation. & Low--Medium \\
      \addlinespace

      ATM & Online & Optional distribution/support channel: funding/defunding or
      wallet servicing; NFC priority with QR fallback on legacy. & ATM
      software/hardware rollout, certification, secure channel to PSP back-end,
      reconciliation. & Medium--High \\

\end{xltabular}
\endgroup

% % ========================================================================

% \section{Implementation Models: Technical and Strategic Analysis}
% \subsection{In-House Implementation Model: Architecture and Requirements}
% The in-house implementation model is suited for large or technologically mature
% banks and payment service providers (PSPs) with multi-squad delivery
% capabilities, robust change governance, and regulated payments maturity.
% Essential attributes include deep security engineering expertise in PKI/key
% management, HSM operations, secure SDLC, threat modeling, and cryptographic
% integration; 24/7 operational resilience with incident response, observability,
% and disaster recovery compliant with EU standards like DORA; and a complex
% product/channel landscape encompassing mobile/web, branch, merchant flows, and
% ATM/POS touchpoints requiring tailored integration. This model entails full
% proprietary responsibility for design, build, operation, security, and
% auditing, offering maximum control over feature delivery, integration, and
% customer experience within scheme constraints, while assuming direct
% accountability for security, privacy, and compliance. It demands the highest
% upfront investment, complexity, and delivery/operational risks, ideal for
% entities seeking strategic differentiation and roadmap control beyond basic
% scheme requirements.

% \newpage
% \subsubsection{Technical Architecture for In-House Implementation}
% \textbf{Microservices Architecture Approach (logical view, scheme-aligned boundary)}
% \begin{figure}[!htbp]
%       \centering
%       \includegraphics[width=\textwidth, height=0.8\textheight, keepaspectratio]{./assests/5.1.1_In-House Implementation Microservices Architecture Approach_InHouse_Microservices_Logical.png}
%       \caption{In-House Implementation (Logical View): Microservices Architecture Approach}
%       \label{fig:microservices-logical}
% \end{figure}

% \textbf{Microservices Components}
% \begin{enumerate}
% \item \textbf{Access Management Service}
% \begin{itemize}
%       \item Functions: onboarding, wallet lifecycle/provisioning, alias management,
%             customer support tooling
%       \item API endpoints: user registration, verification status, wallet
%             activation/lock/unlock
%       \item Dependencies: identity/KYC, CRM, consent/privacy registers (GDPR), core
%             customer master data

% \end{itemize}
% \item \textbf{Liquidity Management Service}
% \begin{itemize}
%       \item Functions: holding-limit enforcement support, funding/unfunding orchestration,
%             (reverse) waterfall triggers where applicable, user notifications
%       \item API endpoints: limit status inquiry, funding/unfunding request,
%             waterfall/reverse-waterfall orchestration

%       \item Dependencies: treasury, core banking accounts, reconciliation/GL, scheme
%             interface adapter

% \end{itemize}
% \item \textbf{Transaction Management Service}
% \begin{itemize}
%       \item Functions: payment initiation, authorization decisioning, idempotency,
%             exception handling, reconciliation and dispute/return workflows (as defined by
%             scheme rules)

%       \item API endpoints: payment initiation, status inquiry, cancellation/exception
%             handling, reconciliation events

%       \item Dependencies: channel adapters, fraud signals (where applicable), scheme
%             interface adapter
% \end{itemize}
% \item \textbf{Risk and Compliance Service}
% \begin{itemize}
%       \item Functions: AML/sanctions screening where legally required, fraud/risk scoring
%             for online flows, monitoring and reporting

%       \item API endpoints: screening request, risk score request, reporting export

%       \item Dependencies: sanctions lists, AML engines, regulatory reporting tooling

%       \item Offline constraint: offline transaction details should not be visible to
%             PSPs/central bank; for offline, controls pivot to limits, device integrity, and
%             post-sync anomaly detection rather than transaction-level AML visibility.
% \end{itemize}
% \item \textbf{Offline Management Service}
% \begin{itemize}
%       \item Functions: credential provisioning, secure storage enablement (e.g., device
%             secure element/TEE where used), offline wallet lifecycle, post-offline sync and
%             conflict handling

%       \item API endpoints: credential provisioning, offline wallet setup,
%             sync/reconciliation events

%       \item Dependencies: device security capabilities, key management, testing lab
%             processes

% \end{itemize}

% \textbf{Integration with Existing Systems}

% The figure below presents a layered integration view of how a bank or PSP can
% embed digital euro capabilities into its existing enterprise landscape. The
% core banking and enterprise systems (systems of record) provide foundational
% functions such as customer master data and identity/KYC, linked funding
% accounts, treasury and limit governance, general ledger and reconciliation,
% regulatory reporting, privacy/consent management (GDPR), and security
% monitoring/incident response aligned with operational resilience expectations
% (e.g., DORA). The digital euro microservices layer, which implements the
% bank/PSP-owned distribution capabilities—wallet onboarding and lifecycle
% management, liquidity and holding-limit orchestration (including
% funding/unfunding logic), payment initiation and exception handling, compliance
% controls for applicable online flows, offline enablement and post-sync
% reconciliation controls, and security/key management for cryptographic
% protection and audit evidence. The external/scheme boundary encapsulates a
% controlled scheme interface adapter (DESP connector) that translates internal
% service calls into scheme-compliant interactions with the external platform,
% while also interfacing with device-security ecosystems (e.g., secure
% element/TEE provisioning where relevant) and third-party security
% testing/certification activities. Together, the layers emphasize clear
% separation of responsibilities, traceable data/control flows, and a robust
% integration perimeter between bank systems and the Eurosystem digital euro
% infrastructure.
% \begin{figure}[!htbp]
%       \centering
%       \includegraphics[width=\textwidth, height=0.9\textheight, keepaspectratio]{./assests/5.1.1_Integration with Existing SystemsDigitalEuroLayer_Containers.png}
%       \caption{Integration with Existing Systems}
%       \label{fig:integration-existing-systems}
% \end{figure}

% \subsubsection{Development and Deployment Considerations}
% \textbf{Development and Deployment Considerations}

% \begin{table}[!htbp]
%       \centering
%       \small
%       \renewcommand{\arraystretch}{1.15}
%       \setlength{\tabcolsep}{6pt}

%       \begin{tabularx}{\textwidth}{
%             >{\RaggedRight\arraybackslash}p{5.0cm}
%             >{\raggedleft\arraybackslash}p{2.2cm}
%             >{\RaggedRight\arraybackslash}X
%             }
%             \toprule
%             \textbf{Role}                    & \textbf{Required FTEs} & \textbf{Key Expertise}                                                          \\
%             \midrule
%             Platform / Enterprise Architects & 2--4                   & Cloud + integration architecture, domain decomposition, security architecture   \\
%             Backend Developers               & 18--28                 & Payment-grade APIs, event-driven design, data consistency, performance          \\
%             Channel / Integration Engineers  & 6--10                  & Core banking integration, POS/ATM/merchant integration, adapters                \\
%             DevOps / SRE                     & 6--10                  & Kubernetes, CI/CD, observability, incident management, DR testing               \\
%             QA / Test Engineers              & 10--16                 & Automation, performance/load testing, integration testing, test data management \\
%             Security Engineers               & 4--7                   & Cryptography, HSM/PKI, threat modelling, secure SDLC                            \\
%             Privacy / Data Protection        & 1--2                   & GDPR                                                                            \\
%             Compliance (AML liaison/SME)     & 1--3                   & AML controls, reporting, controls testing                                       \\
%             Product Managers                 & 2--4                   & Rulebook alignment, scope management, channel rollout strategy                  \\
%             Program / Project Manager        & 1--2                   & Governance, stakeholder management, delivery tracking                           \\
%             \midrule
%             \textbf{Total}                   & \textbf{51--88}        & \textbf{Full-time commitment $\sim$3--4 years (scope-dependent)}                \\
%             \bottomrule
%       \end{tabularx}

%       \caption{Indicative delivery team composition for a digital euro integration programme.}
%       \label{tab:fte_roles_matrix}
% \end{table}

% \textbf{Development Timeline}

% \begin{table}[H]
%       \centering
%       \small
%       \renewcommand{\arraystretch}{1.2}
%       \setlength{\tabcolsep}{6pt}

%       \begin{tabularx}{\textwidth}{
%             >{\RaggedRight\arraybackslash}p{3.3cm}
%             >{\raggedright\arraybackslash}p{2.6cm}
%             >{\RaggedRight\arraybackslash}X
%             }
%             \toprule
%             \textbf{Phase}                      & \textbf{Timeline} & \textbf{Key deliverables / workstreams} \\
%             \midrule
%             Phase 1: Foundation                 &
%             Months 1--6                         &
%             \begin{itemize}\setlength\itemsep{0.2em}
%                   \item Architecture + operating model design (incl.\ DORA controls)
%                   \item Security baseline (HSM/PKI, audit logging, secure SDLC)
%                   \item Interface strategy aligned to scheme rulebook evolution
%                   \item DevOps/SRE platform + environments
%             \end{itemize}                                 \\

%             Phase 2: Core Services              & Months 7--18      &
%             \begin{itemize}\setlength\itemsep{0.2em}
%                   \item Access \& wallet services
%                   \item Payments initiation/status + exception workflows
%                   \item Liquidity/limits orchestration
%                   \item Scheme interface adapter / DESP connector baseline (per evolving
%                         rulebook/technical specs)
%             \end{itemize}                             \\

%             Phase 3: Enhancement \& Integration & Months 19--30     &
%             \begin{itemize}\setlength\itemsep{0.2em}
%                   \item Offline enablement (provisioning + limits + sync controls)
%                   \item Broader channel integration (merchant, ATM/POS touchpoints)
%                   \item Observability hardening + DR maturity uplift (DORA)
%                   \item Security hardening + privacy-by-design refinements
%             \end{itemize}                                  \\

%             Phase 4: Testing \& Readiness       & Months 31--36     &
%             \begin{itemize}\setlength\itemsep{0.2em}
%                   \item End-to-end integration + performance testing
%                   \item Pen testing / red teaming + security sign-off
%                   \item Compliance validation (controls testing + audit evidence)
%                   \item Operational runbooks + incident exercises (DORA)
%             \end{itemize}                                    \\

%             Phase 5: Pilot \& Production        & Months 37--48     &
%             \begin{itemize}\setlength\itemsep{0.2em}
%                   \item Pilot rollout (scope-limited)
%                   \item Monitoring + defect burn-down
%                   \item Scale-out to full production (by channel/tier)
%             \end{itemize}                                               \\
%             \bottomrule
%       \end{tabularx}

%       \caption{Indicative phased implementation roadmap for a bank/PSP digital euro integration programme.}
%       \label{tab:implementation_roadmap_phases}
% \end{table}

% \subsubsection{Cost and Resource Implications}

% \paragraph{Scope definitions}
% \begin{itemize}
%       \item \textbf{Core scope:} online wallet and payment flows, DESP connectivity layer, limits/funding orchestration, baseline monitoring and security assurance,
%             and a limited offline/device test capability.
%       \item \textbf{Extended scope:} broader channel rollout (bank-owned estate pilots for POS/ATM), scaled offline enablement/device testing,
%             increased non-functional requirements (resilience, logging/audit volumes), and expanded assurance/testing.
% \end{itemize}

% \paragraph{Unit-cost inputs (examples used in the model).}
% Representative public unit prices are used for reproducibility: EKS
% control-plane fees, storage, log ingestion and cryptographic modules (HSM),
% together with monitoring and identity list
% prices.\cite{aws_eks_pricing,aws_cloudwatch_logs_pricing,aws_ebs_gp3_pricing,aws_s3_pricing,aws_cloudhsm_pricing,datadog_pricing,entra_p1_pricing}
% Security-assurance effort is modelled using a publicly available UK rate card
% example for penetration testing.\cite{uk_gcloud_pentest_rate} Offline and
% channel pilot costs use publicly visible device and kit
% prices.\cite{aws_device_farm_pricing,cdw_verifone_p400_price,dollar_atm_emv_kit_price}
% Training uses an external benchmark for training spend per
% learner.\cite{trainingmag_2025_report}

% \paragraph{Four-year development investment estimates (EUR, ranges).}
% Below table summarise four-year build costs. Ranges reflect uncertainty in
% scale drivers (e.g., number of environments/hosts, storage/log volumes, size of
% pilot device estate, and number of trained staff). Contingency is applied at
% 15--25\% due to specification evolution risk and integration uncertainties.

% \begin{table}[H]
%       \centering
%       \caption{In-house implementation: four-year cost estimate (EU-based delivery), EUR millions}
%       \label{tab:costs_euuk}
%       \small
%       \setlength{\tabcolsep}{6pt}
%       \renewcommand{\arraystretch}{1.15}

%       \begin{tabularx}{\textwidth}{
%             >{\RaggedRight\arraybackslash}X
%             >{\raggedleft\arraybackslash}p{3.2cm}
%             >{\raggedleft\arraybackslash}p{3.4cm}
%             }
%             \toprule
%             \textbf{Cost category (4-year)}                                      & \textbf{Core (low--high)} & \textbf{Extended (low--high)} \\
%             \midrule
%             Personnel                                                            & 10.39--15.00              & 14.72--25.39                  \\
%             Infrastructure (cloud compute, storage, logs, HSM; incl.\ test envs) & 0.45--1.28                & 1.39--4.09                    \\
%             Security assurance (pentests, audits/red-team effort proxy)          & 0.22--0.45                & 0.45--0.90                    \\
%             Third-party software/licenses (monitoring + IAM list prices)         & 0.15--0.54                & 0.56--1.36                    \\
%             Offline enablement tooling + device testing labs                     & 0.11--0.24                & 0.26--0.56                    \\
%             Channel integration (bank-owned POS/ATM \emph{pilot} estate)         & 0.05--0.12                & 0.53--1.18                    \\
%             Training \& change management                                        & 0.37--1.48                & 1.48--5.91                    \\
%             Contingency (15--25\% applied to subtotal)                           & 1.76--4.77                & 2.91--9.85                    \\
%             \midrule
%             \textbf{Total}                                                       & \textbf{13.51--23.87}     & \textbf{22.28--49.24}         \\
%             \bottomrule
%       \end{tabularx}
% \end{table}

% \paragraph{Notes}
% \begin{itemize}
%       \item \textbf{Personnel cost construction (EU):} role benchmarks (e.g., software engineer, DevOps, security, QA, solution architect, PM)
%             are taken from Glassdoor pages.\cite{glassdoor_uk_sw_eng,glassdoor_uk_devops,glassdoor_uk_seceng,glassdoor_uk_arch,glassdoor_uk_qa,glassdoor_uk_pm,glassdoor_uk_projpm,govuk_secondary_class1_2025_26,pensions_regulator_min_contrib}
%       \item \textbf{Infrastructure and tooling:} totals depend primarily on host counts, log/audit volume and HA/DR posture. Unit prices are taken from vendor pricing pages or published price snapshots.\cite{aws_eks_pricing,aws_cloudwatch_logs_pricing,aws_cloudhsm_pricing,aws_ebs_gp3_pricing,aws_s3_pricing,datadog_pricing,entra_p1_pricing}
%       \item \textbf{Offline and channel pilots:} the tables represent \emph{bank-owned estate pilots} (not economy-wide merchant replacement). Hardware unit examples are sourced from public listings.\cite{aws_device_farm_pricing,cdw_verifone_p400_price,dollar_atm_emv_kit_price}
%       \item \textbf{Contingency rationale:} the ECB highlights that investment cost estimates are sensitive to scope and assumptions, while RDG outputs evolve iteratively.\cite{ecb_view_cost_assessments_2025,ecb_rdg_progress_2025}
% \end{itemize}

% \subsubsection{Risk Profile and Mitigation Strategies}
% \textbf{Key Risks in In-House Implementation}

% \begin{table}[H]
%       \centering
%       \small
%       \renewcommand{\arraystretch}{1.15}
%       \setlength{\tabcolsep}{6pt}

%       \begin{tabularx}{\textwidth}{
%             >{\RaggedRight\arraybackslash}p{5.0cm}
%             >{\raggedright\arraybackslash}p{2.1cm}
%             >{\raggedright\arraybackslash}p{2.1cm}
%             >{\RaggedRight\arraybackslash}X
%             }
%             \toprule
%             \textbf{Risk}                                          & \textbf{Probability} & \textbf{Impact} & \textbf{Mitigation}            \\
%             \midrule
%             Development delays and overruns                        &
%             HIGH                                                   &
%             HIGH                                                   &
%             Agile delivery + stage gates, scope discipline, contingency budget.                                                              \\

%             Skills gaps in security/offline enablement             & MEDIUM               & HIGH            & Targeted hiring,
%             specialist vendors, internal security guild.                                                                                     \\

%             Legacy integration complexity                          & HIGH                 & HIGH            & Strangler pattern, integration
%             test harness, dedicated adapter squads.                                                                                          \\

%             Security vulnerabilities                               & MEDIUM               & CRITICAL        & Secure SDLC, third-party
%             testing, red teaming, SOC integration.                                                                                           \\

%             Regulatory compliance gaps                             & MEDIUM               & HIGH            & Early compliance engagement,
%             controls testing, audit evidence pipeline.                                                                                       \\

%             Privacy/GDPR non-compliance                            & MEDIUM               & HIGH            & Data minimisation, consent
%             management, DPIAs, privacy-by-design reviews (European Central Bank).                                                            \\

%             Offline reconciliation / double-spend integrity issues & MEDIUM               & HIGH            &
%             Offline limits, device integrity controls, post-sync reconciliation checks
%             (European Central Bank).                                                                                                         \\

%             Specification volatility (rulebook/spec changes)       & HIGH                 & HIGH            & Versioned
%             adapters, contract testing, change buffer; track RDG updates (European Central
%             Bank).                                                                                                                           \\

%             Operational resilience gaps (DORA)                     & MEDIUM               & HIGH            & Resilience testing,
%             incident exercises, DR drills, third-party ICT governance (EUR-Lex).                                                             \\
%             \bottomrule
%       \end{tabularx}

%       \caption{Risk register for digital euro integration programme (illustrative).}
%       \label{tab:risk_register_digital_euro}
% \end{table}

% % ========================================================================

% \section{Implementation Models by Bank Tier: Tailored Strategies}
% % ========================================================================

% ======================================================================================================
% =========================================================
% SECTION 5–7 (REVISED) — Copy/paste into your thesis .tex
% =========================================================
% Assumptions:
% - You use hyperref + url packages for \url{} in footnotes.
% - You can keep numbering in titles to preserve your outline format.
% =========================================================
\section{Implementation Models: Technical and Strategic Analysis}
\subsection{In-House Implementation Model: Architecture and Requirements}
The in-house implementation model is suited for large or technologically mature
banks and payment service providers (PSPs) with multi-squad delivery
capabilities, robust change governance, and regulated payments maturity.
Essential attributes include deep security engineering expertise in PKI/key
management, HSM operations, secure SDLC, threat modeling, and cryptographic
integration; 24/7 operational resilience with incident response, observability,
and disaster recovery compliant with EU standards like DORA; and a complex
product/channel landscape encompassing mobile/web, branch, merchant flows, and
ATM/POS touchpoints requiring tailored integration. This model entails full
proprietary responsibility for design, build, operation, security, and
auditing, offering maximum control over feature delivery, integration, and
customer experience within scheme constraints, while assuming direct
accountability for security, privacy, and compliance. It demands the highest
upfront investment, complexity, and delivery/operational risks, ideal for
entities seeking strategic differentiation and roadmap control beyond basic
scheme requirements.

\newpage
\subsubsection{Technical Architecture for In-House Implementation}
\textbf{Microservices Architecture Approach (logical view, scheme-aligned boundary)}
\begin{figure}[!htbp]
      \centering
      \includegraphics[width=\textwidth, height=0.8\textheight, keepaspectratio]{./assests/5.1.1_In-House Implementation Microservices Architecture Approach_InHouse_Microservices_Logical.png}
      \caption{In-House Implementation (Logical View): Microservices Architecture Approach}
      \label{fig:microservices-logical}
\end{figure}

\textbf{Microservices Components}
\begin{enumerate}
\item \textbf{Access Management Service}
\begin{itemize}
      \item Functions: onboarding, wallet lifecycle/provisioning, alias management,
            customer support tooling
      \item API endpoints: user registration, verification status, wallet
            activation/lock/unlock
      \item Dependencies: identity/KYC, CRM, consent/privacy registers (GDPR), core
            customer master data

\end{itemize}
\item \textbf{Liquidity Management Service}
\begin{itemize}
      \item Functions: holding-limit enforcement support, funding/unfunding orchestration,
            (reverse) waterfall triggers where applicable, user notifications
      \item API endpoints: limit status inquiry, funding/unfunding request,
            waterfall/reverse-waterfall orchestration

      \item Dependencies: treasury, core banking accounts, reconciliation/GL, scheme
            interface adapter

\end{itemize}
\item \textbf{Transaction Management Service}
\begin{itemize}
      \item Functions: payment initiation, authorization decisioning, idempotency,
            exception handling, reconciliation and dispute/return workflows (as defined by
            scheme rules)

      \item API endpoints: payment initiation, status inquiry, cancellation/exception
            handling, reconciliation events

      \item Dependencies: channel adapters, fraud signals (where applicable), scheme
            interface adapter
\end{itemize}
\item \textbf{Risk and Compliance Service}
\begin{itemize}
      \item Functions: AML/sanctions screening where legally required, fraud/risk scoring
            for online flows, monitoring and reporting

      \item API endpoints: screening request, risk score request, reporting export

      \item Dependencies: sanctions lists, AML engines, regulatory reporting tooling

      \item Offline constraint: offline transaction details should not be visible to
            PSPs/central bank; for offline, controls pivot to limits, device integrity, and
            post-sync anomaly detection rather than transaction-level AML visibility.
\end{itemize}
\item \textbf{Offline Management Service}
\begin{itemize}
      \item Functions: credential provisioning, secure storage enablement (e.g., device
            secure element/TEE where used), offline wallet lifecycle, post-offline sync and
            conflict handling

      \item API endpoints: credential provisioning, offline wallet setup,
            sync/reconciliation events

      \item Dependencies: device security capabilities, key management, testing lab
            processes

\end{itemize}

\textbf{Integration with Existing Systems}

The figure below presents a layered integration view of how a bank or PSP can
embed digital euro capabilities into its existing enterprise landscape. The
core banking and enterprise systems (systems of record) provide foundational
functions such as customer master data and identity/KYC, linked funding
accounts, treasury and limit governance, general ledger and reconciliation,
regulatory reporting, privacy/consent management (GDPR), and security
monitoring/incident response aligned with operational resilience expectations
(e.g., DORA). The digital euro microservices layer, which implements the
bank/PSP-owned distribution capabilities—wallet onboarding and lifecycle
management, liquidity and holding-limit orchestration (including
funding/unfunding logic), payment initiation and exception handling, compliance
controls for applicable online flows, offline enablement and post-sync
reconciliation controls, and security/key management for cryptographic
protection and audit evidence. The external/scheme boundary encapsulates a
controlled scheme interface adapter (DESP connector) that translates internal
service calls into scheme-compliant interactions with the external platform,
while also interfacing with device-security ecosystems (e.g., secure
element/TEE provisioning where relevant) and third-party security
testing/certification activities. Together, the layers emphasize clear
separation of responsibilities, traceable data/control flows, and a robust
integration perimeter between bank systems and the Eurosystem digital euro
infrastructure.
\begin{figure}[!htbp]
      \centering
      \includegraphics[width=\textwidth, height=0.9\textheight, keepaspectratio]{./assests/5.1.1_Integration with Existing SystemsDigitalEuroLayer_Containers.png}
      \caption{Integration with Existing Systems}
      \label{fig:integration-existing-systems}
\end{figure}

\subsubsection{Development and Deployment Considerations}

\paragraph{Team Structure and Expertise Requirements (illustrative):}
\begin{table}[H]
      \centering
      \begin{tabular}{l c p{8.2cm}}
            \hline
            \textbf{Role}             & \textbf{Required FTEs} & \textbf{Key Expertise}                                            \\
            \hline
            Platform Architects       & 2--3                   & Cloud architecture, microservices, payments, integration patterns \\
            Backend Developers        & 15--20                 & APIs, event-driven systems, data modelling, payments domain       \\
            DevOps/SRE Engineers      & 5--8                   & Kubernetes, CI/CD, observability, SLOs, capacity planning         \\
            QA/Testing Engineers      & 8--12                  & Automation, performance testing, security testing, test data mgmt \\
            Security Engineers        & 3--5                   & Cryptography, HSM/KMS, threat modelling, secure coding            \\
            Product/Business Analysts & 2--4                   & Scheme requirements interpretation, rulebook traceability         \\
            Program/Project Manager   & 1                      & Governance, dependencies, delivery tracking                       \\
            \hline
            \textbf{Total}            & \textbf{36--52}        & \textbf{Full-time commitment for multi-year build}                \\
            \hline
      \end{tabular}
\end{table}

\paragraph{Development Timeline (example):}
\begin{itemize}
      \item Phase 1 (Months 1--6): Architecture, stack selection, DevSecOps baseline,
            rulebook traceability model
      \item Phase 2 (Months 7--18): Core services (access/liquidity/transaction/risk) and
            connectivity layer
      \item Phase 3 (Months 19--30): Offline and advanced features, full channel
            integration, hardening
      \item Phase 4 (Months 31--36): Certification-grade testing, security assessments,
            operational readiness
      \item Phase 5 (Months 37--48): Pilot, production rollout, continuous compliance and
            change management
\end{itemize}

\subsubsection{Cost and Resource Implications}

\paragraph{Development cost estimate (48-month in-house build; EU market baselines).}
Table below provides an illustrative bottom-up estimate for a large PSP/bank
building a rulebook-aligned digital euro stack in-house (core services,
connectivity layer, security/compliance controls, channel integration). Unit
costs are anchored to publicly available European sources and converted to EUR
where necessary.

\begin{table}[htbp]
      \centering
      \begin{tabular}{p{5.6cm} r r}
            \hline
            \textbf{Cost Category (48 months)}                                        & \textbf{Low Estimate} & \textbf{High Estimate} \\
            \hline
            Personnel (core delivery + security + QA/SRE)                             & \EUR~17.3M            & \EUR~35.4M             \\
            External consulting (architecture, security, regulatory testing support)  & \EUR~0.8M             & \EUR~3.5M              \\
            Cloud + security infrastructure (K8s, compute, HSM/KMS, logs/monitoring)  & \EUR~2.0M             & \EUR~5.5M              \\
            Tooling \& software licenses (DevSecOps, collaboration, security tooling) & \EUR~0.8M             & \EUR~2.5M              \\
            Independent testing \& assurance (pentest/red team, performance, audits)  & \EUR~0.7M             & \EUR~1.8M              \\
            Training \& professional development (security + platform certifications) & \EUR~0.4M             & \EUR~1.0M              \\
            \hline
            Contingency (10--15\%)                                                    & \EUR~2.2M             & \EUR~7.5M              \\
            \hline
            \textbf{Total (48 months)}                                                & \textbf{\EUR~24.2M}   & \textbf{\EUR~57.2M}    \\
            \hline
      \end{tabular}
      \caption{Illustrative in-house implementation cost range for a large EU bank/PSP (48 months).}
      \label{tab:inhouse_costs}
\end{table}

\noindent\textit{Estimation notes and unit-cost references (how each line is calculated).}
\begin{itemize}

      \item \textbf{Personnel.}
            We model \textbf{36--52 FTEs} (Table 5.1.3) across architecture, backend, SRE/DevOps, QA, security engineering, and product/business analysis.
            As an EU salary anchor, WSI Lohnspiegel reports a typical \emph{Software Engineer} monthly gross salary around \EUR~5{,}908 (\emph{Nov 2025, Germany}).\footnote{\url{https://www.lohnspiegel.de/it-berufe/software-engineer-gehalt-14440.htm}}
            Employer social security contributions in Germany are commonly \textbf{20--23\% of gross salary} (range guidance).\footnote{\url{https://fmcgroup.com/employment-cost-germany/}}
            For conversion of USD-denominated price references used elsewhere in this table, we apply the ECB reference rate (EUR/USD), e.g., \emph{03 Feb 2026: 1 EUR = 1.1801 USD}.\footnote{\url{https://www.ecb.europa.eu/stats/shared/pdf/eurofxref.pdf}}
            \newline
            \textbf{Formula (illustrative):}
            \[
                  \text{Personnel Cost} = (\text{FTE}) \times (48/12)\times (\text{loaded cost per FTE-year})
            \]
            Loaded cost per FTE-year is modeled at \textbf{\EUR~120k--\EUR~170k} to reflect
            (i) market salaries (anchored above), (ii) employer contributions (20--23\%),
            and (iii) typical enterprise overheads (equipment, internal platform costs,
            bonus/retention, internal chargebacks), which vary materially by bank.

      \item \textbf{External consulting.}
            We include external support for: independent architecture review, cryptography/HSM advisory, security assurance, and specialist integration testing.
            BDU reports average consulting \textbf{daily rates around \EUR~1{,}300} (German consulting market context).\footnote{\url{https://www.bdu.de/media/118665/220601\_bdu\_tagessaetze\_2022\_komplett.pdf}}
            \newline
            \textbf{Formula (illustrative):}
            \[
                  \text{Consulting Cost} = (\text{consultants}) \times (\text{billable days}) \times (\text{day rate})
            \]
            Low case: 2 consultants $\times$ 300 days $\times$ \EUR~1{,}300 $\approx$
            \EUR~0.78M. High case: 4 consultants $\times$ 500 days $\times$ \EUR~1{,}700
            $\approx$ \EUR~3.4M (upper day-rate bound aligned with EU pentest/assurance day
            rates and senior specialist consulting).

      \item \textbf{Cloud + security infrastructure.}
            This line covers non-personnel runtime foundations for DEV/UAT/PROD (compute, Kubernetes control-plane fees, cryptographic services, monitoring/logging, networking).
            Examples of publicly-listed unit prices:
            \begin{itemize}
                  \item Managed Kubernetes control plane: Amazon EKS is priced at \textbf{\$0.10 per
                              cluster-hour}.\footnote{\url{https://aws.amazon.com/eks/pricing/}}
                  \item Key management: AWS KMS includes \textbf{\$1 per key/month} plus request fees
                        (public list pricing).\footnote{\url{https://aws.amazon.com/kms/pricing/}}
                  \item Cloud HSM (EU region example): AWS CloudHSM hourly rates vary by region; EU
                        Central (Frankfurt) hourly pricing can be sourced from AWS pricing
                        indices.\footnote{\url{https://awsprice.khuong.dev/service/AWSCloudHSM/region/eu-central-1}}
                  \item Compute (EU region example): EC2 on-demand hourly prices vary by instance
                        class; EU Central (Frankfurt) examples can be sourced from EU-region price
                        tables.\footnote{\url{https://www.cloudoptimo.com/aws-pricing-calculator/ec2-pricing-calculator}}
                  \item Log ingestion (as a common driver of observability cost): CloudWatch Logs price
                        lists provide per-GB ingestion/storage
                        pricing.\footnote{\url{https://aws.amazon.com/cloudwatch/pricing/}}
            \end{itemize}
            \textbf{Formula (illustrative):}
            \[
                  \text{Infra Cost} \approx \sum_{s \in services} (\text{unit price}) \times (\text{quantity}) \times (\text{hours or months})
            \]
            Low case assumes single-region non-prod + dual-zone prod with moderate
            throughput; high case assumes active-active (multi-region) production, heavier
            performance testing environments, and higher log/monitoring retention. USD
            prices are converted using the ECB EUR/USD reference
            rate.\footnote{\url{https://www.ecb.europa.eu/stats/shared/pdf/eurofxref.pdf}}

      \item \textbf{Tooling \& software licenses.}
            This covers DevSecOps platform subscriptions (code hosting, CI/CD), collaboration tooling, and security add-ons.
            Example publicly-listed unit prices:
            \begin{itemize}
                  \item GitHub Enterprise: \textbf{\$21 per user/month} (list
                        pricing).\footnote{\url{https://github.com/pricing}}
                  \item Atlassian Jira Software Data Center: published tier pricing (used here as an
                        anchor for enterprise issue-tracking/ITSM
                        licensing).\footnote{\url{https://www.atlassian.com/licensing/jira-software}}
            \end{itemize}
            \textbf{Formula (illustrative):}
            \[
                  \text{License Cost} = \sum (\text{users} \times \text{price/user/month} \times 48) + \text{enterprise tooling add-ons}
            \]
            Low case assumes 150--200 named users for core engineering + security; high
            case assumes wider adoption across delivery, risk, and operations plus security
            add-ons and enterprise support.

      \item \textbf{Independent testing \& assurance.}
            This line covers external penetration testing, red teaming, retesting cycles, and (where applicable) independent security assessment support.
            EU market anchors:
            \begin{itemize}
                  \item Binsec (Germany) states daily pentest rates typically
                        \textbf{\EUR~1{,}470--\EUR~1{,}960/day} and total engagements often in the
                        \textbf{\EUR~3{,}000--\EUR~25{,}000}
                        range.\footnote{\url{https://binsec.com/en/pentest/cost/}}
                  \item A concrete German example (microCAT) shows \textbf{\EUR~14{,}000} for a scoped
                        engagement including test + social engineering +
                        retest.\footnote{\url{https://www.microcat.de/blog/pentest/kosten-pentest/}}
                  \item A UK/EU-oriented guide (Secforce) gives a typical \textbf{EU day-rate reference
                        around \EUR~1{,}400/day} for thorough manual
                        testing.\footnote{\url{https://www.secforce.com/the-blog/pen-testing-price-list-uk-and-eu-guide-2025/}}
            \end{itemize}
            \textbf{Formula (illustrative):}
            \[
                  \text{Assurance Cost} = (\text{test days per cycle}) \times (\text{cycles}) \times (\text{day rate}) + \text{retests}
            \]
            Low case assumes 3--4 major cycles (API, mobile, infra, offline/security
            review) with 8--10 days each plus retest; high case assumes broader scope and
            more cycles (e.g., additional channels, multi-region, more red-team exercises).

      \item \textbf{Training \& professional development.}
            We include advanced security/platform training for security engineers and SRE/DevOps plus selected certifications.
            EU-priced anchors:
            \begin{itemize}
                  \item SANS SEC401 (Amsterdam) lists \textbf{\EUR~8{,}230} course
                        price.\footnote{\url{https://www.sans.org/cyber-security-courses/security-essentials-network-endpoint-cloud}}
                  \item ISC2 exam pricing lists CISSP in EMEA at
                        \textbf{EUR~719.04}.\footnote{\url{https://www.isc2.org/register-for-exam/isc2-exam-pricing}}
            \end{itemize}
            \textbf{Formula (illustrative):}
            \[
                  \text{Training Cost} = \sum (\text{seats in courses} \times \text{course price}) + \sum (\text{cert exams} \times \text{exam fee})
            \]
            Low case assumes 30--40 seats in advanced courses across 4 years; high case
            assumes 60--80 seats plus multiple certification tracks.

\end{itemize}

\paragraph{Ongoing operating costs (steady-state, annual).}
After go-live, the cost profile shifts to operations, resilience, continuous
compliance, and incremental feature work (including rulebook evolution).
Table~\ref{tab:inhouse_opex} provides a steady-state annual estimate.

\begin{table}[htbp]
      \centering
      \begin{tabular}{p{6.6cm} r r}
            \hline
            \textbf{Cost Category (annual)}                                & \textbf{Low}       & \textbf{High}      \\
            \hline
            Run \& change personnel (10--16 FTEs)                          & \EUR~1.3M          & \EUR~2.7M          \\
            Cloud/platform operations (compute, logs, KMS/HSM, networking) & \EUR~0.6M          & \EUR~1.6M          \\
            Tooling/licenses (DevSecOps + enterprise support)              & \EUR~0.3M          & \EUR~0.9M          \\
            Assurance refresh (annual pentest/red-team + retests)          & \EUR~0.2M          & \EUR~0.6M          \\
            \hline
            \textbf{Total annual operating cost}                           & \textbf{\EUR~2.4M} & \textbf{\EUR~5.8M} \\
            \hline
      \end{tabular}
      \caption{Illustrative steady-state annual operating cost for the in-house model.}
      \label{tab:inhouse_opex}
\end{table}

\noindent\textit{Opex calculation notes.}
Personnel uses the same loaded-cost logic as above (salary anchor + employer contributions + enterprise overhead).\footnote{\url{https://www.lohnspiegel.de/it-berufe/software-engineer-gehalt-14440.htm}; \url{https://fmcgroup.com/employment-cost-germany/}}
Platform operations are driven by always-on environments and observability/log retention; unit prices can be traced to managed service pricing (e.g., EKS, KMS, logging/monitoring) and converted with ECB reference rates where needed.\footnote{\url{https://aws.amazon.com/eks/pricing/}; \url{https://aws.amazon.com/kms/pricing/}; \url{https://aws.amazon.com/cloudwatch/pricing/}; \url{https://www.ecb.europa.eu/stats/shared/pdf/eurofxref.pdf}}
Assurance refresh is anchored to EU pentest day-rate ranges and engagement examples.\footnote{\url{https://binsec.com/en/pentest/cost/}; \url{https://www.microcat.de/blog/pentest/kosten-pentest/}}

\subsubsection{Risk Profile and Mitigation Strategies}

\paragraph{Key Risks in In-House Implementation:}
\begin{table}[H]
      \centering
      \begin{tabular}{p{3.4cm} p{2.1cm} p{2.1cm} p{7.2cm}}
            \hline
            \textbf{Risk}                 & \textbf{Probability} & \textbf{Impact} & \textbf{Mitigation}                                                        \\
            \hline
            Delivery overruns             & High                 & High            & Incremental delivery, architecture reviews, staged certification readiness \\
            Legacy integration complexity & High                 & Medium          & Strangler patterns, integration test harnesses, data contract governance   \\
            Security vulnerabilities      & Medium               & Critical        & Secure SDLC, independent testing, red teaming, key mgmt hardening          \\
            Compliance drift              & Medium               & High            & Rulebook traceability matrix, change mgmt with release gates               \\
            Operational resilience gaps   & Medium               & High            & DORA-aligned resilience testing, RTO/RPO targets, incident playbooks       \\
            \hline
      \end{tabular}
\end{table}

\paragraph{Mitigation Strategies (summary):}
\begin{enumerate}
      \item Independent architecture/security reviews at milestones
      \item Pilot first; production rollout via controlled waves
      \item Formal rulebook traceability + change control board
      \item Security testing (pentest, red team) and continuous monitoring
\end{enumerate}

% =========================================================
\subsection{ Vendor/Outsourced Implementation Model}
% =========================================================

\subsubsection{Model Characteristics and Applicability}
The vendor/outsourced implementation model is ideal for smaller to mid-sized
banks (e.g., €10–150B assets) with limited in-house development capacity,
prioritizing rapid compliance-grade delivery, predictable costs, and leveraging
existing outsourcing in areas like payment processing, AML tooling, and IAM,
alongside mature vendor governance. In this approach, the vendor delivers a
managed platform handling core PSP stack elements such as connectivity,
operations, and updates, while the bank focuses on integration, configuration,
channel distribution, customer support, and oversight. It offers reduced
customization potential, with differentiation achieved through UX enhancements,
product overlays, analytics, or ecosystem partnerships. Regulatory and
contractual priorities emphasize auditability, GDPR-compliant data protection,
DORA-aligned operational resilience, and strong authentication for relevant
channels.
\subsubsection{5.2.2 Vendor Ecosystem and Service Models}

\paragraph{Vendor categories:}
\begin{itemize}
      \item Pan-European payment processors / infrastructure providers (e.g., Worldline,
            Nexi, equensWorldline)
      \item Domestic payment infrastructure utilities (e.g., SIBS, Redsys, CBI)
\end{itemize}

% \paragraph{}
Vendor capability for digital euro must be evaluated against the \emph{digital
      euro scheme rulebook}, RDG outputs and future certification/testing artifacts
as published.\footnote{Draft scheme rulebook v0.9 and RDG progress materials
      are ECB-hosted publications; reference them in your bibliography/appendix.}

\paragraph{Service Model Options:}
\begin{itemize}
      \item \textbf{Full-service platform model (``Digital Euro as a Service'')}: vendor runs the PSP operational stack and connectivity; bank focuses on distribution/servicing.
      \item \textbf{Component outsourcing model}: bank keeps some services (e.g., risk scoring) and outsources others (e.g., connectivity + liquidity ops).
      \item \textbf{Connectivity gateway outsourcing model}: vendor provides scheme interface gateway; bank builds downstream business services.
\end{itemize}

\subsubsection{Vendor Selection and Evaluation Framework}

\paragraph{Critical Selection Criteria:}
\begin{enumerate}
      \item \textbf{Rulebook compatibility and roadmap}: evidence of rulebook mapping, change process, and test/cert readiness.
      \item \textbf{Security and resilience}: key management, audit logging, incident response SLAs, resilience testing (incl. DORA expectations).
      \item \textbf{Integration design}: API completeness, eventing model, sandbox and test tooling, data portability.
      \item \textbf{Commercials}: one-time onboarding, recurring subscription, per-transaction (if used), change request pricing.
      \item \textbf{Governance}: subcontractor transparency, exit plans, escrow (where applicable), audit rights.
\end{enumerate}

\paragraph{Vendor Selection Process (recommended):}
\begin{itemize}
      \item Stage 1 (2--3 weeks): market scan + shortlist
      \item Stage 2 (4--6 weeks): deep technical/compliance assessment + reference checks
      \item Stage 3 (4--8 weeks): PoC in bank sandbox (connectivity + one channel +
            reconciliation)
      \item Stage 4 (2--4 weeks): negotiation (SLA, exit, audit, change control)
      \item Stage 5 (4--6 weeks): implementation planning and certification-grade test plan
\end{itemize}

\subsubsection{ Implementation Timeline and Phases}

\paragraph{Typical Outsourced Implementation Timeline: 18--24 Months (baseline)}
\begin{itemize}
      \item Phase 1 (Months 1--4): platform provisioning/config, environments, rulebook
            mapping, initial controls
      \item Phase 2 (Months 5--12): integration build, channel enablement, reconciliation
            and reporting, test automation
      \item Phase 3 (Months 13--18): pilot (limited rollout), operational readiness,
            hardening
      \item Phase 4 (Months 19--24): staged production rollout, SLA monitoring,
            post-go-live stabilization
\end{itemize}

\subsubsection{ Cost and Resource Implications}

\paragraph{Implementation Cost Estimate (own bottom-up estimate; EU-market anchors; excludes consumer/merchant device rollouts):}
\begin{table}[H]
      \centering
      \begin{tabular}{l r r}
            \hline
            \textbf{Cost Category}                                        & \textbf{Low}       & \textbf{High}      \\
            \hline
            (1) Vendor onboarding \& platform setup (one-time)            & \EUR 1.5M          & \EUR 3.5M          \\
            (2) Bank integration delivery (8--12 FTE, 18 months)          & \EUR 1.3M          & \EUR 2.0M          \\
            (3) External integration/architecture support (250--500 days) & \EUR 0.3M          & \EUR 0.7M          \\
            (4) Security \& compliance testing (pentest/red-team, audits) & \EUR 0.2M          & \EUR 0.6M          \\
            (5) Non-prod infra + test tooling (sandbox, CI runners, logs) & \EUR 0.15M         & \EUR 0.45M         \\
            (6) Training \& change mgmt (ops, support, compliance, IT)    & \EUR 0.05M         & \EUR 0.20M         \\
            (7) Contingency (15\%)                                        & \EUR 0.53M         & \EUR 1.12M         \\
            \hline
            \textbf{Total Implementation Cost}                            & \textbf{\EUR 4.1M} & \textbf{\EUR 8.5M} \\
            \hline
      \end{tabular}
\end{table}

\paragraph{Calculation notes and EU-market sources:}
\begin{itemize}
      \item (2) Bank integration labor uses German/EU salary anchors (Bundesagentur f\"ur Arbeit ``Entgeltatlas'' IT roles) and an employer-cost uplift factor for Germany; adjust with your local market.\footnote{Entgeltatlas salary benchmarks: \url{https://web.arbeitsagentur.de/entgeltatlas/}. Employer cost factor discussion: \url{https://fmcgroup.com/germany/employer-costs-in-germany/}.}
      \item (3) Consulting day rates anchored on German consulting benchmarks (BDU) and can be triangulated with EU freelancer benchmarks.\footnote{BDU benchmark overview (day rates): \url{https://www.bdu.de/}. Freelancer benchmark triangulation: \url{https://www.freelancermap.de/blog/}.}
      \item (4) Penetration test / red team price anchors from EU providers; range reflects multiple test cycles (pre-pilot and pre-go-live) plus remediation retests.\footnote{Example EU pentest price anchors: \url{https://www.binsec.com/pentest/prix}.}
      \item (5) Non-prod infra includes container control-plane fees and crypto/key services as an anchor (e.g., managed Kubernetes + KMS/HSM where used).\footnote{EKS control-plane fee anchor: \url{https://aws.amazon.com/blogs/containers/cost-optimization-for-kubernetes-on-aws/}. AWS KMS pricing anchor: \url{https://aws.amazon.com/kms/pricing/}. AWS CloudHSM pricing page (hourly fee concept + listed rates): \url{https://aws.amazon.com/de/cloudhsm/pricing/}.}
      \item (6) Training anchors from EU-delivered security training price lists; you can scale down if most staff take internal training.\footnote{Example EU training price anchor (SANS): \url{https://www.sans.org/}.}
      \item (1) Vendor onboarding fee is modeled as (vendor professional services + environment provisioning + certification readiness pack) with a commercial margin consistent with EU payment tech providers.\footnote{Illustrative margin anchors: Worldline FY results/EBITDA margin disclosures and Nexi FY results (as paytech benchmark): see investor relations pages.}
\end{itemize}

\paragraph{Ongoing Costs (Annual; own estimate; EU-market anchors):}
\begin{itemize}
      \item Managed platform subscription + run (incl. connectivity, updates, SLA): \EUR
            0.9M--1.8M
      \item Bank operations oversight (2--4 FTE) and vendor mgmt: \EUR 0.22M--0.44M
      \item Security/compliance recurring (annual pentest slice, audits, tooling): \EUR
            0.10M--0.25M
      \item Residual cloud/crypto services (if bank-operated components exist): \EUR
            0.15M--0.35M
      \item Enhancements/change requests: \EUR 0.20M--0.40M
\end{itemize}

\paragraph{Resource Requirements (bank side):}
\begin{itemize}
      \item Implementation: 8--12 FTE (PM, integration, QA/test automation, compliance
            liaison, ops readiness)
      \item Ongoing: 3--6 FTE (vendor management, ops oversight, incident mgmt, compliance)
\end{itemize}

\subsubsection{Risk Profile and Mitigation Strategies}

\paragraph{Key Risks in Vendor/Outsourced Model:}
\begin{table}[H]
      \centering
      \begin{tabular}{p{3.3cm} p{2.1cm} p{2.1cm} p{7.3cm}}
            \hline
            \textbf{Risk}               & \textbf{Probability} & \textbf{Impact} & \textbf{Mitigation}                                                           \\
            \hline
            Vendor lock-in              & Medium               & Medium          & Exit clauses, data portability, documented APIs, escrow where feasible        \\
            Regulatory/compliance drift & Low--Med             & High            & Rulebook mapping in contract, change-control gates, audit rights              \\
            Operational disruption      & Low--Med             & High            & SLAs + penalties, resilience testing, incident playbooks, DR evidence         \\
            Limited customization       & Medium               & Medium          & Clear differentiation strategy above platform; API extensibility requirements \\
            Third-party chain risk      & Medium               & High            & Subcontractor transparency, DORA-aligned oversight, continuous monitoring     \\
            \hline
      \end{tabular}
\end{table}

\paragraph{Mitigation Strategies:}
\begin{enumerate}
      \item Contractualize rulebook traceability + update SLA (time-to-compliance)
      \item Enforce exit plan (data migration test annually)
      \item Require independent security testing rights and reporting
      \item Design internal ``thin layer'' to keep bank control of customer experience and
            critical analytics
\end{enumerate}

% =========================================================
\subsection{ Hybrid Implementation Model: Balanced Approach}
% =========================================================

\subsubsection{Model Characteristics and Applicability}

The hybrid implementation model provides a balanced approach suitable for
mid-sized to large banks with assets between €100–500 billion and
moderate-to-advanced IT capabilities. It is ideal for institutions that require
targeted differentiation in selected domains such as treasury, SME/merchant
services, fraud management, and analytics, while seeking an optimal balance
between time-to-market, cost predictability, and strategic control. In this
model, the vendor manages commodity components—including connectivity, baseline
operations, and standard reporting—while the bank develops and maintains
value-added modules such as risk analytics, conditional-payment overlays, and
corporate features. Clear modular boundaries, defined through well-specified
APIs and event contracts, ensure separation of responsibilities, prevent
duplicated logic, and maintain audit clarity.

\subsubsection{Hybrid Model Architecture}

\paragraph{Tiered Integration Approach:}
\begin{itemize}
      \item \textbf{Tier 1 (Vendor-managed core)}: connectivity, baseline transaction processing, baseline compliance controls, standard reports
      \item \textbf{Tier 2 (Bank-managed enhancements)}: advanced fraud scoring, analytics, customer segmentation, treasury dashboards
      \item \textbf{Tier 3 (Bank-developed products)}: merchant loyalty, SME working-capital flows, corporate treasury integration, specialized conditional logic
\end{itemize}

\paragraph{Implementation Architecture:}
% \begin{verbatim}
% ┌──────────────────────────────────────────────────────┐
% │           Bank Customer Interfaces                    │
% │   (Mobile, Web, ATM, POS, Corporate Treasury)       │
% └────┬──────────────────────────┬───────────────────────┘
%      │                          │
%      ▼                          ▼
% ┌──────────────────────┐  ┌─────────────────────┐
% │  Bank-Developed      │  │  Vendor Platform    │
% │  Value-Added         │  │  Core Integration   │
% │  Services            │  │                     │
% │ ├─Advanced Fraud     │  │ ├─Connectivity      │
% │ ├─Conditional Logic  │  │ ├─Baseline Txn Mgmt │
% │ ├─Cash Mgmt/Treasury │  │ ├─Baseline Liquidity│
% │ └─Analytics          │  │ └─Std Compliance     │
% └────────┬─────────────┘  └──────┬──────────────┘
%          │                       │
%          └───────────┬───────────┘
%                      │
%               ┌──────▼──────┐
%               │ Integration │
%               │ Layer/APIs  │
%               └──────┬──────┘
%                      │
%               ┌──────▼──────────┐
%               │ Eurosystem/Scheme│
%               └─────────────────┘
% \end{verbatim}

\begin{figure}[!htbp]
      \centering
      \includegraphics[width=\textwidth, height=0.9\textheight, keepaspectratio]{./assests/5.3.2 _Hybrid Implementation ArchitectureHybrid.png}
      \caption{Hybrid Implementation Architecture}
      \label{fig:hybrid_architecture}
\end{figure}
\subsubsection{Value-Added Service Examples}

\paragraph{Advanced conditional-payment (B2B):}
\begin{itemize}
      \item Scheduling/orchestration, escrow-like workflows, installment/recurring
            management (product-layer)
      \item Bank-owned condition evaluation logic; vendor platform remains the
            execution/settlement substrate
\end{itemize}

\paragraph{Merchant loyalty and incentive management:}
\begin{itemize}
      \item Offer engine and cashback/points triggered by transaction metadata
      \item Merchant onboarding and reconciliation integration (bank/acquirer partnerships)
\end{itemize}

\paragraph{Liquidity and treasury enhancements:}
\begin{itemize}
      \item Forecasting dashboards, automated funding heuristics, intraday monitoring
      \item Can be anchored on existing Eurosystem liquidity concepts (e.g., DCA-style
            accounts) for operations patterns.\footnote{Example fee/ops anchor for
                  Eurosystem DCA operations (TIPS):
                  \url{https://www.ecb.europa.eu/paym/target/tips/professional/fees/html/index.en.html}.}
\end{itemize}

\subsubsection{ Development and Integration Approach}

\paragraph{Phased Implementation Strategy:}
\begin{itemize}
      \item Phase 1 (Months 1--12): vendor core integration + channel enablement + pilot
            baseline
      \item Phase 2 (Months 9--20): parallel build of enhancements; integration test
            harness
      \item Phase 3 (Months 18--30): deploy first value-added modules; iterate via metrics
            and feedback
      \item Phase 4 (Months 25--36): scale and optimize; extend corporate/merchant features
\end{itemize}

\paragraph{Team Structure:}
\begin{table}[H]
      \centering
      \begin{tabular}{l l c l}
            \hline
            \textbf{Component}       & \textbf{Team}                  & \textbf{Size} \\
            \hline
            Core Integration         & Vendor + bank integration team & 3--6 bank FTE \\
            Advanced Fraud/Analytics & Bank risk/data team            & 4--8 FTE      \\
            Product overlays         & Bank product + engineering     & 6--8 FTE      \\
            Ops/Compliance oversight & Bank ops/compliance            & 2--3 FTE      \\
            \hline
            Total                    &                                & 15-25 FTE     \\
            \hline
      \end{tabular}
\end{table}

\subsubsection{ Cost and Resource Implications}

\paragraph{Implementation Cost Estimate (own bottom-up estimate; EU-market anchors; excludes consumer/merchant device rollouts):}
\begin{table}[H]
      \centering
      \begin{tabular}{l r r}
            \hline
            \textbf{Cost Category}                                         & \textbf{Low}        & \textbf{High}       \\
            \hline
            (1) Vendor core platform onboarding/implementation             & \EUR 4.0M           & \EUR 7.0M           \\
            (2) Bank build of value-added services (15--25 FTE, 24 months) & \EUR 4.5M           & \EUR 9.0M           \\
            (3) External integration/consulting (500--900 days)            & \EUR 0.7M           & \EUR 1.2M           \\
            (4) Testing, security, compliance readiness (multi-cycle)      & \EUR 1.5M           & \EUR 3.0M           \\
            (5) Infra/tooling (analytics, logging, CI/CD, key mgmt)        & \EUR 0.6M           & \EUR 1.2M           \\
            (6) Training and change management                             & \EUR 0.3M           & \EUR 0.8M           \\
            (7) Contingency (15\%)                                         & \EUR 1.76M          & \EUR 3.30M          \\
            \hline
            \textbf{Total Implementation Cost}                             & \textbf{\EUR 13.3M} & \textbf{\EUR 26.1M} \\
            \hline
      \end{tabular}
\end{table}

\paragraph{Calculation notes and EU-market sources (per row):}
\begin{itemize}
      \item (2) Bank engineering cost uses Entgeltatlas salary anchors and employer-cost uplift factor; for senior specialists you should apply a premium vs. median.\footnote{Entgeltatlas: \url{https://web.arbeitsagentur.de/entgeltatlas/}. Employer-cost uplift discussion: \url{https://fmcgroup.com/germany/employer-costs-in-germany/}.}
      \item (3) Consulting day-rate anchor from German consulting benchmarks (BDU).\footnote{BDU benchmarks: \url{https://www.bdu.de/}.}
      \item (4) Security testing anchors from EU pentest providers; hybrid adds complexity because both vendor and bank-owned modules must be tested end-to-end.\footnote{Example EU pentest price anchors: \url{https://www.binsec.com/pentest/prix}.}
      \item (5) Infra anchors use managed Kubernetes and key services pricing as \emph{anchor points}; compute/storage vary by throughput and retention.\footnote{EKS fee anchor: \url{https://aws.amazon.com/blogs/containers/cost-optimization-for-kubernetes-on-aws/}. AWS CloudHSM hourly fee concept and listed rates: \url{https://aws.amazon.com/de/cloudhsm/pricing/}. ECB EUR/USD reference rates (if you convert USD inputs to EUR): \url{https://www.ecb.europa.eu/stats/eurofxref/}.}
      \item (1) Vendor onboarding in hybrid is higher than fully-outsourced because it typically includes: integration boundary design, joint observability, and certification-grade test assets; vendor commercial margin can be triangulated against EU paytech disclosures.\footnote{Investor relations margin disclosures can be used as benchmark (e.g., Worldline, Nexi).}
      \item (6) Training anchors from EU-delivered security training price lists; scale based on number of staff trained.\footnote{Example: \url{https://www.sans.org/}.}
\end{itemize}

\paragraph{Ongoing Operating Costs (Annual; own estimate):}
\begin{itemize}
      \item Vendor subscription + SLA run: \EUR 1.2M--2.2M
      \item Bank enhancement team (5--8 FTE): \EUR 0.55M--0.9M
      \item Infra/tooling: \EUR 0.2M--0.5M
      \item Security/compliance recurring: \EUR 0.15M--0.30M
      \item \textbf{Total annual: \EUR 2.1M--3.9M}
\end{itemize}

\subsubsection{ Risk Profile and Mitigation Strategies}

\paragraph{Key Risks in Hybrid Model:}
\begin{table}[H]
      \centering
      \begin{tabular}{p{3.4cm} p{2.1cm} p{2.1cm} p{7.2cm}}
            \hline
            \textbf{Risk}                  & \textbf{Probability} & \textbf{Impact} & \textbf{Mitigation}                                                       \\
            \hline
            Integration boundary ambiguity & Medium               & Medium          & Contracted API/event contracts; architecture review; golden-path journeys \\
            Duplication/conflicts          & Medium               & Low--Med        & Clear separation of concerns; ``one owner per control'' principle         \\
            Delivery delay in enhancements & Medium               & Medium          & Agile governance; feature flags; phased go-lives                          \\
            Security gaps across layers    & Medium               & High            & Unified threat model; joint testing; end-to-end monitoring                \\
            \hline
      \end{tabular}
\end{table}

\paragraph{Mitigation Strategies:}
\begin{enumerate}
      \item Formal interface governance (APIs + events + SLAs) and versioning policy
      \item Joint steering committee with documented decision rights
      \item End-to-end test harness and certification-grade regression suite
\end{enumerate}

% =========================================================
\section{Implementation Models by Bank Tier: Tailored Strategies}
% =========================================================

\subsection{High-Tier Banks (Large, Internationally Active)}

\subsubsection{ Bank Profile and Strategic Context}
High-tier banks, classified as large and internationally active, typically
feature total assets from €300 billion to over €3 trillion, multi-country
operations with complex legal-entity structures and broad customer bases, and
advanced yet heterogeneous IT estates blending legacy and modern systems.
Strategic imperatives for the Digital Euro encompass compliance readiness with
minimal disruption, differentiation via corporate/merchant features and
advanced analytics, and robust governance and operational resilience across
jurisdictions.
\subsubsection{Recommended Implementation Approach: In-House with Selective Partnerships}
(Keep your in-house architecture as in Section 5.1; selectively outsource niche components such as device secure elements, specialized fraud engines, or certification support.)

\textbf{Core Strategy:}
\begin{itemize}
      \item Develop comprehensive in-house Digital Euro platform
      \item Utilize selective partnerships for specialized components (offline, fraud
            detection)
      \item Create innovation center for next-generation features
      \item Establish Digital Euro as competitive differentiator
\end{itemize}
\textbf{Detailed Implementation Architecture:}

\textbf{Tier 1 Core Development (In-House)}

\begin{forest}
      for tree={
      grow'=0, % grow to the right
      parent anchor=east,
      child anchor=west,
      anchor=west,
      edge path={
                  \noexpand\path[\forestoption{edge}]
                  (!u.parent anchor) -- +(6pt,0) |- (.child anchor)\forestoption{edge label};
            },
      l sep=10pt, % vertical separation
      s sep=2pt   % horizontal separation
      }
      [Core Platform Components
            [Access Management Service]
            [Liquidity Management Service]
            [Transaction Management Service]
            [Risk and Compliance Service]

      ]
\end{forest}

\textbf{Technical Approach:}

The technical approach employs a microservices architecture to enable
independent scaling and updates, a distributed systems design for resilience
across geographies, cloud-native deployment for enhanced flexibility and
scalability, and an event-driven architecture for real-time processing.

\textbf{Tier 2 Channel Integration (In-House + Partnerships)}

\begin{forest}
      for tree={
      grow'=0, % grow to the right
      parent anchor=east,
      child anchor=west,
      anchor=west,
      edge path={
                  \noexpand\path[\forestoption{edge}]
                  (!u.parent anchor) -- +(6pt,0) |- (.child anchor)\forestoption{edge label};
            },
      l sep=10pt,
      s sep=2pt
      }
      [Distribution Channels
            [Mobile Banking
                        [Native iOS/Android apps]
                        [Digital Euro wallet UI]
                        [Offline capability support]
                        [Biometric authentication]
            ]
            [Web Banking
                        [Enhanced web interfaces]
                        [Corporate treasury portal]
                        [Merchant acceptance tools]
            ]
            [POS \& Merchant
                        [Terminal integration framework]
                        [Merchant onboarding]
                        [Acceptance network development]
            ]
            [ATM Network
                        [NFC upgrade support]
                        [QR code capability]
                        [Funding/defunding operations]
            ]
      ]
\end{forest}

\textbf{Tier 3 Advanced Services (In-House Development)}

\begin{forest}
      for tree={
      grow'=0, % grow to the right
      parent anchor=east,
      child anchor=west,
      anchor=west,
      edge path={
                  \noexpand\path[\forestoption{edge}]
                  (!u.parent anchor) -- +(6pt,0) |- (.child anchor)\forestoption{edge label};
            },
      l sep=10pt,
      s sep=2pt
      }
      [Advanced Services
            [Conditional Payments Engine
                        [Advanced escrow capabilities]
                        [Supply chain financing]
                        [Treasury payment orchestration]
            ]
            [Loyalty \& Rewards Integration
                        [Automatic point allocation]
                        [Merchant incentive programs]
                        [Consumer engagement features]
            ]
            [Advanced Analytics
                        [Real-time transaction analytics]
                        [Fraud pattern detection]
                        [Customer behavior insights]
                        [Competitive intelligence]
            ]
            [API Economy for Partners
                        [Developer ecosystem]
                        [Third-party service integration]
                        [Fintech partnership framework]
            ]
            [Blockchain Integration (Future)
                  [Smart contract compatibility]
                        [Supply chain transparency]
                        [Advanced programmability]
            ]
      ]
\end{forest}
\subsubsection{ Governance and Organizational Structure}
(Keep your governance structure; ensure vendor oversight aligns with DORA third-party risk management and audit rights.)

\textbf{Organizational Design:}

\begin{center}
      \resizebox{\textwidth}{!}{%
            \begin{forest}
                  for tree={
                  grow'=0,
                  parent anchor=east,
                  child anchor=west,
                  anchor=west,
                  edge path={
                              \noexpand\path[\forestoption{edge}]
                              (!u.parent anchor) -- +(6pt,0) |- (.child anchor)\forestoption{edge label};
                        },
                  l sep=10pt,
                  s sep=2pt
                  }
                  [Organization
                        [Chief Technology Officer (CTO)
                              [Head, Digital Euro Platform
                                                [Platform Architecture Team  ]
                                                [Core Services Development
                                                            [Access \& Onboarding  ]
                                                            [Liquidity \& Settlement  ]
                                                            [Transaction Processing ]
                                                ]
                                                [DevOps \& Infrastructure  ]
                                                [Security \& Privacy  ]
                                                [QA \& Testing ]
                                    ]
                                    [Head, Digital Euro Channels
                                                [Mobile Banking Enhancement  ]
                                                [Web \& Corporate ]
                                                [POS \& Merchant Integration ]
                                                [ATM Integration ]
                                                [Quality Assurance ]
                                    ]
                                    [Head, Digital Euro Innovation
                                                [Advanced Payments Team]
                                                [Analytics \& Data Science]
                                                [Partner Ecosystem ]
                                                [Research \& Development]
                                    ]
                        ]
                        [Chief Risk Officer (CRO)
                              [Head, Digital Euro Compliance
                                                [Regulatory Affairs]
                                                [AML/KYC Compliance]
                                                [Risk Management]
                                                [Internal Audit]
                                    ]
                        ]
                  ]
            \end{forest}%
      }
\end{center}

\subsubsection{ Cost and Timeline for High-Tier Banks}
(Keep your table, but state that salary assumptions should be localized and that regulatory testing cycles are a major schedule driver.)

\textbf{Timeline}

\textbf{Q1--Q4 Year 1: Architecture \& Foundation}
\begin{itemize}[noitemsep]
      \item Technology selection and architecture design
      \item DevOps infrastructure setup
      \item Core framework development
      \item Initial API design
      \item Development team recruitment (staffing)
\end{itemize}

\textbf{Q1--Q4 Year 2: Core Services Development}
\begin{itemize}[noitemsep]
      \item Access Management Service development
      \item Liquidity Management Service development
      \item Transaction Management Service development
      \item DESP integration framework
      \item Beta testing with select customers (internal)
\end{itemize}

\textbf{Q1--Q4 Year 3: Enhancement \& Channel Integration}
\begin{itemize}[noitemsep]
      \item Risk \& Compliance Service development
      \item Offline Management Service development
      \item Mobile, web, and POS channel integration
      \item Pilot with 5{,}000--10{,}000 external users
      \item Performance optimization and hardening
\end{itemize}

\textbf{Q1--Q4 Year 4: Advanced Features \& Production}
\begin{itemize}[noitemsep]
      \item Advanced services (conditional payments, analytics)
      \item Full channel rollout (POS, ATM, corporate)
      \item Advanced fraud detection deployment
      \item Production readiness and go-live preparation
      \item Gradual customer activation and monitoring
\end{itemize}

\subsection{ Mid-Tier Banks ( Moderate Complexity)}

\subsubsection{ Bank Profile and Strategic Context}

Mid-tier banks, characterized as regional with moderate complexity, typically
possess total assets ranging from €50–300 billion, operate primarily in a
single market with limited cross-border reach, and exhibit moderate IT maturity
combining legacy systems and API-first components. Strategic imperatives for
the Digital Euro include meeting scheme requirements while managing costs and
risks, and leveraging vendor acceleration to expedite implementation while
retaining control over selected differentiators.

\subsubsection{ Recommended Implementation Approach: Hybrid Model}
(Use Section 5.3 architecture; keep 60/25/15 split as a planning heuristic, but enforce strict API boundaries.)

\textbf{Core Strategy:}
\begin{itemize}
      \item Outsource core integration via vendor platform
      \item Develop selective proprietary services for regional differentiation
      \item Leverage vendor expertise while controlling costs
      \item Achieve faster time-to-market than full in-house development
\end{itemize}

\textbf{Detailed Implementation Architecture:}
\textbf{Tier 1: Vendor-Managed Core}
\begin{center}

      \begin{forest}
            for tree={
            grow'=0, % grow to the right
            parent anchor=east,
            child anchor=west,
            anchor=west,
            edge path={
                        \noexpand\path[\forestoption{edge}]
                        (!u.parent anchor) -- +(6pt,0) |- (.child anchor)\forestoption{edge label};
                  },
            l sep=10pt,
            s sep=2pt
            }
            [Components (Vendor Responsibility)
            [Access Management (onboarding, KYC, wallet provisioning)]
                  [Liquidity Management (DCA operations, waterfall)]
                  [Basic Transaction Processing]
                  [Compliance Framework (rulebook requirements)]
                  [Standard Reporting and Statement Engine]
            ]
      \end{forest}
\end{center}

\textbf{Tier 2: Shared Infrastructure}
\begin{center}
      \begin{forest}
            for tree={
            grow'=0, % grow to the right
            parent anchor=east,
            child anchor=west,
            anchor=west,
            edge path={
                        \noexpand\path[\forestoption{edge}]
                        (!u.parent anchor) -- +(6pt,0) |- (.child anchor)\forestoption{edge label};
                  },
            l sep=10pt,
            s sep=2pt
            }
            [Market-Based Collaboration
                  [Shared infrastructure (POS/terminal)]
                  [Shared Fraud Detection Consortium]
                  [Shared ATM Network Operations]
                  [ Shared Testing Infrastructure]
            ]
      \end{forest}
\end{center}

\textbf{Tier 3: Bank-Developed Differentiation}
\begin{center}
      \begin{forest}
            for tree={
            grow'=0, % grow to the right
            parent anchor=east,
            child anchor=west,
            anchor=west,
            edge path={
                        \noexpand\path[\forestoption{edge}]
                        (!u.parent anchor) -- +(6pt,0) |- (.child anchor)\forestoption{edge label};
                  },
            l sep=10pt,
            s sep=2pt
            }
            [ Services
                  % [Regional Payment Integration
                  %   [Local payment method integration]
                  %   [Regional merchant ecosystem]
                  %   [Regional customer behavior analysis]
                  % ]
                  % [SME/Corporate Offerings
                  %   [Supply chain financing for regional suppliers]
                  %   [Working capital solutions]
                  %   [Cash management tools]
                  % ]
                  [Enhanced Customer Experience
                              [Personalized merchant offers]
                              [Regional promotions]
                              [Community engagement features]
                  ]
                  [Data \& Analytics
                              [Transaction analytics for customers]
                              [Business intelligence dashboards]
                  ]
            ]
      \end{forest}
\end{center}

\subsubsection{6.2.3 Governance and Organizational Structure}
(Keep your structure; add an explicit \emph{Vendor Risk Owner} and \emph{Rulebook Change Owner}.)

\textbf{Organizational Design:}
\begin{center}
      \begin{forest}
            for tree={
            grow'=0, % grow to the right
            parent anchor=east,
            child anchor=west,
            anchor=west,
            edge path={
                        \noexpand\path[\forestoption{edge}]
                        (!u.parent anchor) -- +(6pt,0) |- (.child anchor)\forestoption{edge label};
                  },
            l sep=10pt,
            s sep=2pt
            }
            [Organization
                  [Chief Information Officer (CIO)
                        [Head, Digital Euro Program
                                          [Program Manager]
                                          [Systems Integration Manager]
                                          [QA Lead]
                                          [Operations Manager]
                                          [Vendor Relationship Manager]
                              ]
                              [Head, Digital Euro Innovation
                                          [Product Manager]
                                          [Developers]
                                          [Analysts]
                              ]
                  ]
                  [Chief Risk Officer (CRO)
                        [Head
                                          [Regulatory Compliance Officer]
                                          [AML/KYC Manager]
                              ]
                  ]

            ]
      \end{forest}
\end{center}

\subsubsection{6.2.4 Cost and Timeline for Mid-Tier Banks}
\paragraph{Revised Mid-Tier Hybrid Cost Envelope (aligned to Section 5.3.5):}
Use \textbf{\EUR 13.3M--26.1M} as baseline (excluding device rollouts); tune
based on scope (offline, ATM/POS depth, corporate modules).

\textbf{Timeline:}

\textbf{Months 1--4: Vendor Selection \& Planning}\\[-2mm]
\begin{itemize}[leftmargin=*, itemsep=1pt, topsep=2pt]
      \item Vendor evaluation and selection
      \item Implementation planning
      \item Resource recruitment
      \item Integration architecture design
\end{itemize}

\textbf{Months 5--12: Core Integration Phase}\\[-2mm]
\begin{itemize}[leftmargin=*, itemsep=1pt, topsep=2pt]
      \item Vendor platform setup and configuration
      \item Integration with bank core systems
      \item Compliance and testing framework
      \item Limited pilot (500--1,000 users)
\end{itemize}

\textbf{Months 13--18: Enhancement Development}\\[-2mm]
\begin{itemize}[leftmargin=*, itemsep=1pt, topsep=2pt]
      \item Parallel development of proprietary services
      \item Regional feature integration
      \item SME/Corporate product development
      \item Enhanced pilot expansion (2,000--5,000 users)
\end{itemize}

\textbf{Months 19--24: Production Preparation}\\[-2mm]
\begin{itemize}[leftmargin=*, itemsep=1pt, topsep=2pt]
      \item Full regulatory testing and certification
      \item Production deployment planning
      \item Training and documentation
      \item Go-live readiness assessment
\end{itemize}

\textbf{Months 25--30: Production Launch \& Scaling}\\[-2mm]
\begin{itemize}[leftmargin=*, itemsep=1pt, topsep=2pt]
      \item Staged customer activation
      \item Monitoring and support
      \item Feature rollout to all customers
      \item Optimization and enhancement
\end{itemize}

\subsection{Low-Tier Banks (Small, Community-Focused)}

\subsubsection{ Bank Profile and Strategic Context}
Low-tier banks, classified as small and community-focused, typically exhibit
total assets ranging from €5–50 billion, operate with limited IT staff, and
rely heavily on outsourced core banking and payments.

\subsubsection{6.3.2 Recommended Implementation Approach: Vendor/Outsourced Model}
(Use Section 5.2; prioritize turnkey delivery, compliance evidence, and exit planning.)
\textbf{Core Strategy:}
\begin{itemize}
      \item Engage established vendor for end-to-end Digital Euro platform
      \item Minimize internal development and complexity
      \item Rely on vendor expertise and support
      \item Focus bank resources on core banking operations
\end{itemize}

\textbf{Detailed Implementation Architecture:}

\textbf{Vendor Platform}

\begin{center}
      \begin{forest}
            for tree={
            grow'=0, % grow to the right
            parent anchor=east,
            child anchor=west,
            anchor=west,
            edge path={
                        \noexpand\path[\forestoption{edge}]
                        (!u.parent anchor) -- +(6pt,0) |- (.child anchor)\forestoption{edge label};
                  },
            l sep=10pt,
            s sep=2pt
            }
            [ Vendor-Provided Services
                  [Access Management (full onboarding, KYC, wallet management)]
                  [Liquidity Management (DCA operations, waterfall, automation)]
                  [Transaction Processing (full transaction lifecycle)]
                  [Risk \& Compliance (fraud detection, AML screening)]
                  [Channel Integration (mobile, web, ATM support)]
                  [Reporting and Reconciliation]
                  [Customer Support Framework]
                  [Operational Monitoring]
            ]
      \end{forest}
\end{center}

\textbf{Bank-Specific Configuration}

\begin{center}
      \begin{forest}
            for tree={
            grow'=0, % grow to the right
            parent anchor=east,
            child anchor=west,
            anchor=west,
            edge path={
                        \noexpand\path[\forestoption{edge}]
                        (!u.parent anchor) -- +(6pt,0) |- (.child anchor)\forestoption{edge label};
                  },
            l sep=10pt,
            s sep=2pt
            }
            [Bank Customization
                  [Brand integration (logo, colors, bank messaging)]
                  [Customer communication templates]
                  [Regulatory compliance documentation]
                  [Internal process documentation]
                  [Staff training materials]
                  [Customer support scripts]
            ]
      \end{forest}
\end{center}

Option 1 involves a single vendor relationship, where one provider delivers a
complete platform and support, offering the simplest approach with minimal
complexity but entailing full vendor dependency risk and the lowest total cost
of ownership; typical vendors include Temenos, SAP, and Oracle. Option 2 adopts
a cooperative/consortium model, enabling multiple banks to share a vendor
platform via a cooperative structure, which reduces individual costs through
common infrastructure, fosters shared governance and decision-making, and
secures better collective terms; examples include Spanish Redsys.

\subsubsection{Governance and Organizational Structure}
(Keep streamlined structure; ensure audit rights, incident reporting, and exit plan are operationalized.)

\textbf{Streamlined Organization}
\begin{center}
      \begin{forest}
            for tree={
            grow'=0, % grow to the right
            parent anchor=east,
            child anchor=west,
            anchor=west,
            edge path={
                        \noexpand\path[\forestoption{edge}]
                        (!u.parent anchor) -- +(6pt,0) |- (.child anchor)\forestoption{edge label};
                  },
            l sep=10pt,
            s sep=2pt
            }
            [Organization
                  [Chief Information Officer (CIO)
                        [Head
                                          [Vendor Relationship Manager]
                                          [Systems Administrator]
                              ]
                  ]
                  [Chief Risk Officer (CRO)
                        [Compliance Officer]
                              [Regulation Monitoring Officer]
                  ]

            ]
      \end{forest}
\end{center}

\subsubsection{ Cost and Timeline for Low-Tier Banks}
\paragraph{Revised Low-Tier Vendor Cost Envelope (aligned to Section 5.2.5):}
Use \textbf{\EUR 4.1M--8.5M} baseline (excluding device rollouts). For
cooperative/consortium procurement, expect lower onboarding and better
subscription terms.

\textbf{Timeline:}

\textbf{Months 1--3: Vendor Selection \& Planning}\\[-2mm]
\begin{itemize}[leftmargin=*, itemsep=1pt, topsep=2pt]
      \item Vendor evaluation
      \item Contract negotiation
      \item Project planning
      \item Resource allocation
\end{itemize}

\textbf{Months 4--9: Implementation}\\[-2mm]
\begin{itemize}[leftmargin=*, itemsep=1pt, topsep=2pt]
      \item Vendor platform setup
      \item Bank configuration
      \item Testing (vendor-supported)
      \item Pilot with 1,000--2,000 users
\end{itemize}

\textbf{Months 10--18: Integration \& Testing}\\[-2mm]
\begin{itemize}[leftmargin=*, itemsep=1pt, topsep=2pt]
      \item Full regulatory testing
      \item Production preparation
      \item Staff training
      \item Customer communication
\end{itemize}

\textbf{Months 19--24: Production Launch}\\[-2mm]
\begin{itemize}[leftmargin=*, itemsep=1pt, topsep=2pt]
      \item Gradual customer activation
      \item Ongoing monitoring
      \item Vendor support
      \item Stable production operations
\end{itemize}

% =========================================================
\section{ Shared Infrastructure, Synergies, and Cost Mutualization}
% =========================================================

\subsection{Synergy   Mechanisms}

Savings when independent banks use shared infrastructure, common vendors, or
joint service providers (e.g., shared ATM entities, shared processing, shared
certification).

\begin{itemize}
      \item \textbf{One build, many deployments}: shared codebase and common compliance controls reduce duplicated engineering.
      \item \textbf{Shared operations}: SOC/SIEM patterns, fraud monitoring, incident response and audit evidence can be pooled.
      \item \textbf{Joint procurement}: reduced unit costs for platform subscription, testing labs, and certification services.
\end{itemize}

Model mutualization as a reduction in the \emph{fixed cost component} (platform
build, certification harness) while leaving a per-bank variable component
(configuration, local training, channel rollout).

\paragraph{Evidence anchors for shared cash/ATM models:}
European retail cash infrastructure discussions include examples of ATM
pooling/joint ventures (e.g., Batopin, Geldmaat) as mechanisms to reduce
duplicated infrastructure.\footnote{Example anchor: ERPB reports on cash/ATM
      infrastructure and pooling initiatives:
      \url{https://www.ecb.europa.eu/paym/groups/erpb/html/index.en.html}.}

\subsubsection{ Cost Mutualization Opportunities}

\begin{enumerate}
      \item { Shared Testing and Certification Infrastructure}
            \begin{itemize}
                  \item Central test harness aligned to scheme rulebook test cases
                  \item Shared investment in device labs, integration stubs, regression suites
            \end{itemize}

      \item{Shared Fraud Detection and AML Services}
            \begin{itemize}
                  \item Consortium analytics models; pooled typologies; shared investigation playbooks
                  \item Must be designed with GDPR constraints and clear controller/processor roles
            \end{itemize}

      \item{ Shared ATM / cash-like access operations (where digital euro offers cash-like features)}
            \begin{itemize}
                  \item Pooling terminal estate and maintenance contracts (where applicable)
                  \item Harmonized device software update management
            \end{itemize}

      \item{ Shared Digital Euro as a Service (DaaS) Platform}
            \begin{itemize}
                  \item Multi-tenant managed platform serving multiple PSPs with standardized
                        compliance and operations
                  \item Best suited for low-tier and parts of mid-tier banks
            \end{itemize}

      \item{Shared Merchant Acceptance Enablement}
            \begin{itemize}
                  \item Joint merchant onboarding toolkits and technical support
                  \item Shared promotion campaigns to bootstrap acceptance (subject to competition law
                        constraints)
            \end{itemize}
\end{enumerate}
\subsection{ Scenarios and Sensitivity Analysis}

\subsubsection{Cost Scenarios by Implementation Model}

\paragraph{Scenario parameters (recommendation):}

\textbf{Define:}

\begin{itemize}
      \item Fixed cost shareability factor (portion of platform build/cert that is
            mutualized)
      \item Variable per-bank integration factor (local channels, data migrations,
            training)
      \item Scope factor (offline, ATM/POS depth, corporate modules)
      \item Legacy factor (integration complexity uplift)
\end{itemize}

\subsubsection{ Bank-Specific Cost Analysis by Tier and Model}

\paragraph{High-Tier Banks (\EUR 300B+ assets) — In-House:}
Maintain \EUR 24.2M--57.2M range (scope dependent);

\paragraph{Mid-Tier Banks (\EUR 100--300B assets) — Hybrid:}
Use \EUR 13.3--26.1M baseline (from Section 5.3.5),

\paragraph{Low-Tier Banks (\EUR 5--50B assets) — Vendor:}
Use \EUR 4.1--8.5M baseline (from Section 5.2.5)

\subsubsection{ Cost Drivers and Sensitivity Analysis}

\paragraph{Key cost drivers:}
\begin{enumerate}
      \item \textbf{Scope}: offline + advanced programmability overlays increase cost (security + device lifecycle).
      \item \textbf{Channel depth}: adding ATM/POS/branch workflows increases testing matrix and operational controls.
      \item \textbf{Legacy complexity}: uplift for non-API cores, fragmented customer master data, multiple ledgers.
      \item \textbf{Regulatory testing cycles}: additional retest rounds and remediation rework can add 10--25\%.
      \item \textbf{Mutualization}: reduces fixed platform/cert costs; does not eliminate local change costs.
\end{enumerate}
% =========================================================

\section{Technical Blueprints and Best Practices}

\subsection{Implementation Architecture}
\textbf{Core Infrastructure Stack:}

\label{subsubsec:ref_impl_arch}

Figure~\ref{fig:digital_euro_impl_arch} presents a scheme-aligned reference
architecture for a supervised bank (or PSP) integrating digital euro services
into its existing enterprise landscape. The architecture is structured around a
clear boundary between (i) the distribution layer operated by the bank/PSP and
(ii) the Eurosystem-operated Digital Euro Service Platform (DESP). Consistent
with the scheme approach, the bank retains responsibility for customer
onboarding and wallet lifecycle, channel delivery (mobile/web and optional
POS/ATM enablement), internal risk and compliance controls, and operational
processes such as reconciliation and customer support, while interacting with
DESP via standardised PSP-to-platform interfaces.

The design emphasises low-latency, resilient end-to-end payment processing. In
particular, the bank-side processing path incorporates a dedicated
workflow/orchestration function to manage end-to-end transaction state,
retries, compensation steps, and auditable traceability. This is motivated by
scheme non-functional requirements that include strict reliability and
recoverability expectations, as well as processing-latency targets for online
transactions at both payer and payee PSPs. The architecture therefore treats
observability (metrics, logs, traces) and cryptographic controls (key
management/HSM) as first-class cross-cutting capabilities rather than auxiliary
tooling.

At the functional level, the bank distribution layer is decomposed into
services that correspond to the scheme’s service bundles: (i) Access \& Wallet
Management, (ii) Liquidity \& Limits Management (including holding-limit and
“waterfall” checks), (iii) Transaction Processing (payment initiation,
authorisation decisioning, exception handling), (iv) Risk \& Compliance (fraud
and AML/CFT hooks under privacy constraints), and (v) Offline Wallet \& Device
Security (secure provisioning, offline controls, and post-reconnection
synchronisation). A dedicated DESP Connectivity / Scheme Adapter isolates the
remainder of the bank domain from DESP interface evolution and encapsulates
mTLS, signing, idempotency, versioning, and correlation identifiers. Finally, a
reconciliation and ledger-posting function aligns internal records, statements,
and audit trails with platform confirmations and settlement-related outputs,
ensuring end-to-end integrity across the scheme boundary.
\begin{figure}[H]
      \centering

      \includegraphics[width=\linewidth]{./assests/8.1_Implementation ArchitectureContainerView_Impl_Architecture.png}
      \caption{Reference Implementation Architecture for Bank/PSP Integration with the Digital Euro (scheme-aligned).}
      \label{fig:digital_euro_impl_arch}
\end{figure}
\textbf{Explanation:}
\paragraph{Component-by-component explanation (Figure~\ref{fig:digital_euro_impl_arch}).}
The architecture can be read top-down as a set of layers that progressively translate user actions into scheme-compliant DESP interactions, while preserving internal controls and auditability.

\begin{enumerate}
      \item \textbf{Customer-facing channels (Mobile, Web, POS/Merchant, ATM).}
            These components implement the end-user journeys (P2P, e-commerce checkout, POS proximity payments, and optional ATM-based servicing/funding/defunding). They handle user interaction, authentication UX, receipts/notifications, and (where applicable) merchant/terminal interaction. Channel diversity is a major driver of complexity because each journey has different session-correlation, certification, and operational requirements.

      \item \textbf{API Gateway.}
            The gateway is the controlled entry point for all channels. It enforces authentication and authorisation, rate-limiting, routing, and security protections (e.g., WAF integration), and ensures consistent observability headers/correlation IDs. This layer is critical for controlling attack surface and stabilising downstream services under high load.

      \item \textbf{Channel API / BFF (Back-end for Front-end).}
            The BFF aggregates domain APIs into channel-optimised endpoints, reducing coupling between channels and core domain services. This prevents channel-specific UI changes from forcing frequent changes to transaction and wallet services, and simplifies version management across multiple channels.

      \item \textbf{Payment Orchestration \& Workflow.}
            This is the system’s process engine for end-to-end digital euro journeys. It coordinates multi-step flows (wallet onboarding, funding/defunding, payment initiation, exception handling, dispute triggers, reconciliation initiation) using a stateful workflow pattern (Saga/workflows). It also implements robust retry/compensation logic and produces an auditable event trail, which is essential when scheme requirements enforce strict availability, recoverability, and latency constraints.

      \item \textbf{Access \& Wallet Management Service.}
            This service manages onboarding/offboarding, wallet provisioning and lifecycle, alias binding (e.g., phone/email), and user/device association. It integrates with the bank’s identity/KYC systems and customer master data to ensure correct mapping between internal customer identifiers and scheme identifiers.

      \item \textbf{Liquidity \& Limits Service.}
            This service enforces and monitors holding/usage limits and orchestrates funding/defunding flows, including “waterfall” checks where a linked funding account is used to ensure sufficient balance. It integrates with treasury/liquidity tooling and produces events for reconciliation and operational monitoring.

      \item \textbf{Transaction Processing Service.}
            This service is the payment state machine. It accepts payment initiation requests, performs decisioning steps (including idempotency handling), coordinates fraud/limits checks, submits scheme-bound messages via the Scheme Adapter, and manages exceptions (cancellations, errors, returns/dispute hooks). It persists transaction state and exposes status queries for channels and back-office.

      \item \textbf{Risk \& Compliance Service.}
            This service applies bank-side fraud and AML/CFT controls and interfaces to sanctions screening and case-management tooling. Importantly, for offline payments the architecture assumes limited transaction-level visibility and therefore pivots controls toward device integrity, limits governance, and post-sync anomaly detection rather than relying solely on conventional online transaction monitoring.

      \item \textbf{Offline Wallet \& Device Security Service.}
            This service handles secure provisioning of offline-capable wallets (e.g., secure element/TEE), offline limit enforcement logic at the device boundary, anti-replay and device lifecycle controls, and the reconciliation/synchronisation process once devices reconnect. It relies heavily on cryptographic controls and operational procedures for loss/theft recovery.

      \item \textbf{Data stores (Transactional DB, Document DB, Search Index) and Cache.}
            The transactional store persists authoritative wallet/payment state and reconciliation artefacts; the document store supports operational data such as device metadata and investigation/case documents; the search index supports operational search with strict data minimisation; and the cache accelerates read-heavy operations and may hold session and idempotency metadata (implementation-dependent).

      \item \textbf{Event Bus.}
            The event bus decouples services through asynchronous events (domain events, audit events, reconciliation events, alerts). It reduces tight coupling and enables scalable audit logging and analytics without placing additional latency on the synchronous payment path.

      \item \textbf{Reconciliation \& Ledger Posting.}
            This function aligns scheme-side confirmations/settlement-related outputs with internal accounting: sub-ledger/GL postings, statement generation, operational reporting, and exception investigation. It is essential for end-to-end integrity because the bank must produce auditable records consistent with the scheme lifecycle.

      \item \textbf{DESP Connectivity / Scheme Adapter.}
            The adapter encapsulates the PSP-to-DESP boundary. It manages mTLS connectivity, message signing where required, idempotent request handling, retries/backoff, version compatibility, and correlation identifiers. It isolates the bank domain from DESP interface evolution and provides a controlled integration point for certification and monitoring.

      \item \textbf{Cross-cutting controls: Observability and Key Management / HSM.}
            Observability centralises metrics/logs/traces and supports SLOs, incident response, and audit retention. Key management/HSM services protect private keys and secrets for channel authentication, DESP connectivity (mTLS/signing), and secure device provisioning. These are fundamental to operational resilience and compliance evidence.
\end{enumerate}

\paragraph{Illustrative end-to-end flow (online payment).}
A typical online payment begins when the user initiates a transaction via a
channel. The API Gateway and BFF route the request to the Workflow Engine,
which coordinates (i) balance/limit checks (including waterfall checks where
relevant), (ii) fraud/compliance decisioning, (iii) scheme message submission
via the Scheme Adapter to DESP, and (iv) state persistence and user-facing
status updates. The transaction is then reconciled and posted to the bank’s
ledgers and statements through the Reconciliation \& Ledger Posting function,
with domain and audit events emitted to the Event Bus for monitoring and
investigations.

\paragraph{Illustrative end-to-end flow (offline payment).}
Offline journeys rely on device security and offline controls enforced by the
Offline Wallet \& Device Security Service. Offline transactions are later
synchronised after reconnection, with anomaly checks and reconciliation
ensuring consistency between device-held states, bank records, and scheme
requirements.

\subsubsection{Access Management Service: Bank/PSP Distribution Layer for Digital Euro Onboarding and Wallet Identity}
% \label{subsubsec:access_mgmt_service}

Figure~\ref{fig:access_mgmt_service} illustrates a reference architecture for
an \textit{Access Management Service} within a bank/PSP digital euro
distribution layer. The Access Management Service is responsible for customer
onboarding and lifecycle management, binding the bank’s internal customer
identity to a scheme-facing digital euro wallet/account identifier, and
managing alias registration (e.g., phone/email/IBAN addressing) under explicit
consent and verification controls. The service acts as the authoritative source
for \textit{access state} and \textit{identity-to-wallet mappings}; it does not
serve as the authoritative ledger for balances or limits, which are typically
owned by dedicated liquidity/ledger components.

The architecture separates concerns across (i) secure ingress and policy
enforcement via an API Gateway, (ii) domain logic implemented by the Access
Management Service, and (iii) external dependencies such as KYC/identity
verification, customer master/core banking, wallet provisioning, and
liquidity/limits services. To support operational resilience and
compliance-grade traceability, the service persists lifecycle state and
idempotency keys in a transactional database, publishes auditable lifecycle
events to an event bus, and integrates with observability tooling for metrics,
logs, and traces. Cryptographic controls (e.g., key management/HSM) are treated
as a first-class dependency to support secure communication (mTLS), signing
operations where required, and secure handling of secrets and provisioning
keys. Overall, this microservice design provides a modular, auditable, and
scalable foundation for digital euro customer access and wallet lifecycle
management.

\begin{figure}[H]
      \centering
      \includegraphics[width=\linewidth]{./assests/8.1_Access Management Service (PSP Distribution Layer)AccessMgmt_Component.png}
      \caption{Access Management Service reference architecture (container and component views) for bank/PSP digital euro onboarding, wallet binding, and alias lifecycle.}
      \label{fig:access_mgmt_service}
\end{figure}

\textbf{Explanation}


Container view: the service and its major dependencies (internal + external).

Component view: the internal building blocks inside the Access Management
Service.

A. Container view (top-level building blocks)

1) Customer (Person)

Represents retail/corporate users who initiate onboarding, check status, and
manage aliases via bank channels (mobile/web/etc.).

They never call internal services directly; all access is mediated by gateway
and security controls.

2) API Gateway (Bank container)

The controlled entry point into the bank/PSP distribution layer.

Typical responsibilities:

Authentication/authorisation policy enforcement

Rate limiting and DDoS/WAF integration

Request routing to the appropriate domain service

Injecting correlation IDs for traceability

Producing edge metrics/logs for security and ops

3) Access Management Service (Bank container)

The domain microservice that implements:

Onboarding: create a “digital euro access profile” and initiate wallet
provisioning

Offboarding: deactivate access, revoke bindings, coordinate closure/defunding
steps

Alias management: register and verify phone/email/IBAN aliases

Status: return the customer’s lifecycle state and wallet binding information

Key principle: it is the source of truth for access state and
identity-to-wallet binding, not for balances.

4) Access DB (PostgreSQL)

Stores:

user/wallet mappings

alias references (and verification status)

lifecycle state (active/suspended/offboarded, timestamps)

idempotency keys (to prevent duplicate onboarding or alias duplication)

Explicitly does not store raw KYC documents; those belong in a regulated
document vault.

5) Event Bus (Kafka)

Publishes lifecycle and audit events such as:

UserOnboarded, WalletProvisioned, AliasRegistered, UserOffboarded

Benefits:

decouples downstream consumers (risk, customer support, reporting)

supports asynchronous workflows (notifications, monitoring, analytics)

provides an additional audit-friendly event trail

6) Observability (OTel + SIEM/Monitoring)

Centralised logs, metrics, and traces:

helps prove “who did what and when” (auditability)

supports incident response (MTTR reduction)

enables SLO/SLA reporting and capacity planning

7) Key Management / HSM

Provides cryptographic primitives and secrets custody:

mTLS client certs / private keys for secure service-to-service calls

signing keys where applicable

secrets management and rotation

provisioning keys for sensitive flows (e.g., wallet/device provisioning)

External dependencies (outside the bank domain boundary):

8) KYC/AML & Identity Verification

Performs identity checks and returns verification outcomes:

document validation

customer due diligence (CDD)

sanction screening hooks (depending on bank architecture)

The Access service consumes results/status, not raw documents.

9) Customer Master / Core Banking

Authoritative customer record system:

internal customer identity

account linkage (where funding accounts are relevant)

master data needed to tie digital euro access to an existing customer

10) Wallet Provisioning Service

Creates/initialises a digital euro wallet/account reference and returns a
scheme-facing identifier.

The Access service stores and manages the binding between internal user ID and
this wallet identifier.

11) Liquidity / Limits Service

Holds authoritative information on:

balance, holding limits, transaction limits

funding/defunding orchestration (if applicable)

The Access service may call this only to enrich status responses (optional).

12) Notification Service

Sends lifecycle notifications (email/SMS/push):

onboarding complete

alias verified

offboarding confirmed

Often triggered asynchronously via events.

B. Component view (inside the Access Management Service)

These are the internal “modules” that implement clean separation of concerns:

1) Access API

The HTTP handlers/controllers for:

/users/onboard

/users/offboard

/aliases/register

/users/{id}/status

Responsible for validation, auth context extraction, and calling orchestrators.

2) Onboarding Orchestrator

The core onboarding workflow logic:

checks idempotency

calls KYC adapter

calls provisioning adapter

persists lifecycle state and mapping

emits audit + lifecycle events

Ensures onboarding is safe under retries (very important in distributed
systems).

3) Offboarding Orchestrator

The closure/deactivation workflow logic:

updates state (deactivate access)

coordinates defunding/closure steps with liquidity/provisioning (as supported)

emits audit + offboarding events

Handles edge cases: pending transactions, disputes, reconciliation
dependencies.

4) Alias Manager

Owns alias registration lifecycle:

alias format validation

ownership verification triggers (e.g., OTP)

uniqueness rules (prevent alias collisions)

state transitions: pending → verified → active/revoked

emits alias-related events

5) Identity/KYC Adapter

Encapsulates integration with KYC systems:

retries/circuit breaker

normalises results into a consistent internal format

avoids leaking KYC vendor peculiarities into core onboarding logic

6) Provisioning Adapter

Calls wallet provisioning dependency:

requests wallet creation

handles retries safely

persists the external scheme identifier mapping

7) Status Aggregator

Builds status responses:

reads lifecycle + mapping from DB

optionally calls Liquidity service to attach balance/limits view

ensures response is consistent and privacy-safe

8) Audit Logger

Writes immutable audit records/events for key lifecycle actions:

who/what/when, request IDs, correlation IDs

critical for compliance evidence and post-incident reconstruction

9) Idempotency Manager

Prevents duplicates and supports safe replays:

stores idempotency keys and prior results

ensures repeated requests do not create multiple wallets or aliases

This is essential for onboarding and alias flows under network
retries/timeouts.

Typical end-to-end flows (nice to include as explanation) Flow 1: Onboarding
(POST /users/onboard)

Customer initiates onboarding via mobile/web → API Gateway

Gateway routes to Access API

Onboarding Orchestrator checks Idempotency Manager

Calls Identity/KYC Adapter → KYC system returns status

Calls Provisioning Adapter → wallet provisioning returns wallet identifier

Persists mapping + lifecycle state in Access DB

Audit Logger records evidence, event published to Event Bus

Notification Service sends confirmation (often asynchronously)

Flow 2: Alias registration (POST /aliases/register)

Customer submits alias request → gateway → Alias Manager

Alias Manager verifies ownership/consent (OTP flow or verification procedure)

Stores alias state in DB, emits AliasRegistered/AliasVerified events

Downstream services can subscribe to alias events (e.g., directory/resolution)

Flow 3: Offboarding (POST /users/offboard)

Request enters via gateway → Offboarding Orchestrator

Orchestrator coordinates any defunding/closure steps (liquidity/provisioning)

Updates access state and revokes alias/device bindings

Emits audit + offboarding events and triggers customer notifications

Flow 4: Status query (GET /users/{id}/status)

Reads lifecycle state and wallet binding from DB

Optionally enriches with balance/limits view via Liquidity service

Returns a stable, channel-friendly status response

\subsubsection{Liquidity Management Service: DCA-Backed Funding/Defunding, Waterfall and Reverse Waterfall}
\label{subsubsec:liquidity_mgmt_service}

Figure~\ref{fig:liquidity_mgmt_service} presents a reference architecture for
the \textit{Liquidity Management Service} within a distributing bank/PSP
digital euro stack. The service operationalises the scheme’s liquidity
management requirements by orchestrating (i) user funding and defunding between
a user’s online digital euro holdings and a linked non-digital euro payment
account, (ii) \textit{reverse waterfall} on the payer side when digital euro
holdings are insufficient, and (iii) \textit{waterfall} on the payee side when
an incoming transaction would breach the online holding limit. In the digital
euro scheme, reverse waterfall and waterfall are not “post-processing add-ons”;
they are embedded in the pre-settlement validation (“balance pre-check”) and
are integrated into the end-to-end settlement flow, meaning that failure (or
user non-activation where allowed) results in rejection of both the liquidity
operation and the related payment transaction \cite{ecbRulebook2025}.

\paragraph{Reverse waterfall (payer-side).}
Reverse waterfall enables automatic conversion from the user’s linked
non-digital euro payment account to digital euro when holdings are insufficient
to complete an online digital euro payment. The scheme specifies that the
reverse-waterfall requirement is evaluated during the payer PSP’s
pre-settlement balance pre-check and that its settlement is integrated into the
payment settlement. If reverse waterfall is required but not activated, or if
it fails (e.g., insufficient funds on the linked account), the PSP must reject
the payment; if a transaction including reverse waterfall fails, the PSP must
immediately reverse any debit/reservation on the linked non-digital euro
payment account \cite{ecbRulebook2025}.

\paragraph{Waterfall (payee-side).}
Waterfall enables automatic conversion from digital euro to the user’s linked
non-digital euro payment account when an incoming online transaction would
exceed the holding limit. The excess amount is calculated as:
\[
      \text{Excess} = (\text{current digital euro balance} + \text{incoming amount}) - \text{holding limit}.
\]
The waterfall requirement is evaluated during the payee PSP’s pre-settlement
balance pre-check, and the settlement of the waterfall is integrated into the
payment settlement. If waterfall is required but not activated, or if it fails,
the PSP must not process the payment. The PSP must credit the user’s linked
non-digital euro payment account immediately after receiving confirmation from
the DESP that the waterfall instruction has been settled. The rulebook also
anticipates exceptional scenarios requiring an additional post-settlement
holding-limit check to handle concurrency effects \cite{ecbRulebook2025}.

\paragraph{Architecture interpretation.}
At container level, the architecture separates (a) channel ingress and policy
enforcement (API Gateway), (b) payment orchestration (Transaction Management
Service), (c) liquidity orchestration (Liquidity Management Service), (d)
durable state (Liquidity DB) and low-latency hot state (Cache), (e)
asynchronous audit/event propagation (Event Bus and Outbox), (f) operational
automation (Scheduler/Rules Engine), and (g) secure scheme connectivity (DESP
Connectivity Adapter with KMS/HSM support). The Liquidity Management Service
coordinates internal ledger actions against the bank’s core banking system
(reservation/debit/credit of linked non-digital euro accounts) while sending
scheme-facing liquidity instructions via the DESP connector. Observability and
SIEM integration provide traceability and operational control for SLA/SLO
monitoring and incident response. The overall decomposition supports the
scheme’s operational constraints (e.g., continuous availability of waterfall
and reverse waterfall) and performance expectations for PSP processing latency
where waterfall checks are explicitly considered in timing definitions
\cite{ecbRulebook2025}.

\paragraph{Standardisation context.}
The Eurosystem’s Rulebook Development Group (RDG) has explicitly treated
waterfall/reverse-waterfall and liquidity management as key areas for PSP
implementation specifications and end-to-end flow refinement, reinforcing the
need for a clear separation between scheme connectivity, PSP orchestration, and
bank internal ledger operations \cite{ecbRdgProgress2025}.

\begin{figure}[H]
      \centering
      \includegraphics[width=\linewidth]{./assests/8.1_Liquidity Management ServiceLiquidityComponents.png}
      \caption{Reference architecture for the Liquidity Management Service in a bank/PSP digital euro distribution layer, supporting DCA-backed liquidity operations and settlement-integrated waterfall/reverse-waterfall flows.}
      \label{fig:liquidity_mgmt_service}
\end{figure}

\textbf{Explanations}

Container View (how systems talk to each other), and

Component View (how the Liquidity Management Service is internally decomposed).

1) Container view (bank/PSP boundary vs. Eurosystem boundary)

Actors

Digital Euro User: initiates manual funding/defunding and sets preferences
(e.g., reverse-waterfall opt-in).

Treasury / Operations: monitors DCA position, liquidity buffers, exceptions,
and operational KPIs.

Bank / PSP containers

API Gateway: single ingress point for all channels; enforces
authentication/authorisation, rate limits, routing, and request correlation
IDs.

Transaction Management Service: owns the payment transaction lifecycle and
calls liquidity pre-checks/executions:

Calls /precheck/payer (reverse waterfall decision) and /precheck/payee
(waterfall decision),

Calls /execute/reverse-waterfall and /execute/waterfall when required.

This matches the rulebook requirement that reverse waterfall / waterfall are
integrated in balance pre-check and settlement flows.

Liquidity Management Service: orchestrates funding/defunding + DCA position
management + settlement-integrated waterfall and reverse waterfall.

Liquidity DB (PostgreSQL): durable source of truth for:

user funding agreements (opt-in flags),

idempotency keys,

liquidity transaction states,

reconciliation checkpoints.

Liquidity Cache (Redis): accelerates frequent reads (DCA available headroom,
user opt-in flags, thresholds) to meet latency requirements.

Event Bus (Kafka): publishes liquidity lifecycle events
(created/settled/failed, reconciliation exceptions) so downstream services can
react without tight coupling.

Scheduler / Rules Engine: runs periodic jobs:

DCA monitoring,

trigger evaluation (minimum/target reserves),

reconciliation cycles,

forecast refresh.

DESP Connectivity Adapter: standardised “scheme connector”:

mTLS, signing, retries, correlation IDs,

hides scheme connectivity complexity from the business logic.

Key Management / HSM: protects secrets and crypto operations used by the
connector (mTLS keys/certs, signing keys).

Core Banking / Payments Ledger: performs reservation/debit/credit on the user’s
linked non-digital euro payment account:

Reverse waterfall needs debit/reservation + rollback on failure.

ecb.derdgp250731_Draft_digital_…

Waterfall requires crediting the linked account immediately after DESP
settlement confirmation.

ecb.derdgp250731_Draft_digital_…

Treasury System: consumes DCA metrics/alerts and supports reserve strategy and
reporting.

Observability & SIEM: logs/metrics/traces + security monitoring for
auditability and SLA reporting.

External system

DESP (Eurosystem): receives scheme-facing liquidity instructions and returns
confirmations; PSP settlement integration depends on these confirmations.

2) Component view (inside the Liquidity Management Service)

These components exist to enforce scheme rules reliably under retries,
concurrency, and failures.

Liquidity API: implements the service endpoints (manual fund/defund; precheck;
execute waterfall/reverse-waterfall).

Pre-check Evaluator: performs the rulebook-aligned decision logic:

payer-side “insufficient holdings → require reverse waterfall”,

payee-side “holding-limit breach → require waterfall”.

Holding Limit Engine: calculates excess amount and supports the post-settlement
holding-limit check scenario (concurrent incoming transactions).

ecb.derdgp250731_Draft_digital_…

Reverse Waterfall Orchestrator: coordinates:

reserve/debit linked account (core banking),

send funding instruction via DESP client,

rollback reservation immediately if the payment fails.

ecb.derdgp250731_Draft_digital_…

Waterfall Orchestrator: coordinates:

compute excess,

send defunding instruction via DESP client,

credit linked non-digital euro account immediately after DESP confirms
settlement.

ecb.derdgp250731_Draft_digital_…

DCA Position Manager: maintains the PSP’s DCA balance snapshots and available
headroom for settlement-related operations.

Trigger Manager: applies policies (minimumReserve/targetReserve) and initiates
replenishment actions when DCA headroom drops.

Forecast Engine: generates horizon-based liquidity forecasts (ops/treasury
reporting).

Reconciliation Worker: reconciles internal state vs DESP
confirmations/statements and raises exceptions.

Core Banking Adapter: encapsulates idempotent reserve/debit/credit posting into
core banking and supports reversals.

DESP Client: formats and sends scheme-facing instructions through the DESP
Connectivity Adapter and processes confirmations.

Outbox/Event Publisher: ensures reliable event publication (avoid “DB commit
succeeded but event publish failed”).

Audit Logger: writes immutable audit evidence with correlation IDs.

Idempotency Manager: prevents double execution across retries/timeouts
(critical for debit/credit safety).

3) Why these pieces matter (one paragraph you can reuse)

Scheme alignment: Reverse waterfall and waterfall are settlement-integrated and
depend on strict pre-check decisions, rejection behaviour, and corrective
actions such as immediate reversal and immediate crediting after confirmation.

Operational constraints: The rulebook defines service availability expectations
and PSP-side processing performance targets where waterfall checks are
explicitly considered in latency definitions.

ecb.derdgp250731_Draft_digital_…

RDG emphasis: Waterfall/reverse-waterfall and liquidity management are
explicitly highlighted as key topics refined through RDG feedback and
implementation specification workstreams.

\subsubsection{Transaction Management Service: Scheme-Aligned Payment Orchestration and State Control}
\label{subsubsec:tx_mgmt_service}

Figure~\ref{fig:tx_mgmt_service} presents a reference architecture for the
\textit{Transaction Management Service (TMS)} within a bank/PSP digital euro
distribution layer. The service is the primary bank-side component responsible
for orchestrating the end-to-end lifecycle of online digital euro payments,
including initiation, validation of payer and payee conditions, integration
with liquidity and holding-limit mechanisms (e.g., reverse-waterfall and
waterfall where applicable), scheme-facing message submission through the DESP
connectivity boundary, and finalisation based on settlement confirmations and
rejection notifications. Architecturally, the design enforces a clear
separation of responsibilities: customer-facing channels access transaction
capabilities through a controlled ingress (API Gateway), while scheme
connectivity is encapsulated by a dedicated \textit{DESP Connectivity Adapter}
that standardises secure transport (e.g., mTLS), correlation identifiers,
retries, and version handling.

The TMS persists a deterministic transaction state machine in a transactional
database to ensure correctness under retries, concurrency, and partial
failures. To support operational resilience and compliance-grade traceability,
the service includes explicit idempotency controls, audit logging with reason
codes and correlation identifiers, and an event publication pattern (outbox
plus event bus) that enables downstream consumers (notifications, analytics,
operations tooling) without tight coupling. Finally, the architecture
integrates bank-internal systems for identity/addressing (Access Management and
Alias/Directory resolution), liquidity and limit enforcement (Liquidity
Management), and internal accounting hooks (core banking/ledger), ensuring that
both scheme-level settlement outcomes and bank-side customer/account
representations remain consistent.

\begin{figure}[H]
      \centering
      \includegraphics[width=\linewidth]{./assests/8.1.3_Transaction Management_Service_Component.png}
      \caption{Transaction Management Service reference architecture (container and component views) for a bank/PSP integrating digital euro payments with DESP connectivity, liquidity checks, reconciliation, and auditability.}
      \label{fig:tx_mgmt_service}
\end{figure}

\textbf{Explanation}

Container view: how systems/services interact across boundaries

Component view: how the Transaction Management Service is decomposed internally

A) Container view explanation Actors

Digital Euro User Initiates payments (P2P/e-commerce/POS via bank channels) and
queries status/history.

Merchant / Payee Represents a merchant or payee channel context (particularly
relevant to POS/e-commerce flows where the payee side may initiate parts of the
process).

Operations / Support Monitors system health, exceptions, disputes, and
operational KPIs; uses observability outputs for incident response and
reporting.

Bank/PSP Distribution Layer (containers)

API Gateway What it does: single ingress point for payment APIs.
Responsibilities: authentication/authorisation enforcement, routing,
throttling/rate-limiting, correlation ID propagation, and edge logging. Why it
exists: reduces attack surface and stabilises downstream services during
spikes.

Access Management Service What it does: provides wallet/account lifecycle state
and the binding between internal customer identity and scheme-facing
identifiers. How TMS uses it: checks whether payer/payee wallet is
active/suspended, retrieves wallet metadata required for transaction
processing.

Alias/Directory Resolver What it does: resolves a human-friendly identifier
(e.g., phone/email/IBAN alias) into a scheme-facing identifier (e.g.,
DEAN/wallet ID). How TMS uses it: turns “Payee Alias” into a routable target
reference before DESP submission.

Liquidity Management Service What it does: performs balance/limit pre-check
logic and executes liquidity operations needed for a payment (e.g., reverse
waterfall for payer, waterfall for payee, where required). How TMS uses it:
calls pre-check and execution endpoints to ensure a transaction meets
balance/limit rules and to trigger required funding/defunding steps.

Transaction Management Service (TMS) What it does: orchestrates the payment’s
full lifecycle. Core responsibilities:

initiate transaction request and validate payload

coordinate payer and payee validation steps

orchestrate liquidity checks and risk checks

submit scheme-facing messages through the DESP adapter

handle callbacks/confirmations and finalise state

persist full state transitions and evidence

Transaction DB (PostgreSQL) What it does: stores transaction state machine,
idempotency records, correlation IDs, timestamps, reason codes, and audit
metadata. Why it matters: transaction correctness and recoverability depend on
durable state under retries and failures.

Outbox (DB-backed outbox pattern) What it does: ensures events are published
reliably after DB commits. Why it matters: prevents the classic failure mode
where the DB commit succeeds but the event publish fails (or vice versa).

Event Bus (Kafka) What it does: distributes transaction lifecycle events to
downstream services (notifications, analytics, monitoring tools)
asynchronously. Why it matters: decouples downstream consumers from the hot
payment path and supports scalability.

Core Banking / Internal Ledger What it does: internal accounting hooks,
postings (where applicable), customer account status checks, and potentially
internal “reservation” semantics depending on how the bank models digital euro
balances. Why it matters: ensures internal statements and reporting remain
consistent with settlement outcomes.

Notification Service What it does: sends user receipts and updates
(settled/rejected, etc.) based on preferences and channel policies. Why it
matters: completes the customer journey and supports operational transparency.

DESP Connectivity Adapter What it does: encapsulates the scheme boundary.
Responsibilities: secure connectivity (mTLS), retries/backoff, correlation IDs,
request signing (if required), interface version handling, and a stable
internal API for TMS. Why it matters: isolates business logic from
scheme-interface evolution and centralises security controls.

Key Management / HSM What it does: protects keys/certificates/secrets used for
mTLS and signing. Why it matters: prevents key leakage, supports rotation, and
centralises cryptographic compliance.

Observability & SIEM What it does: central logs, metrics, and traces; alerting;
security monitoring. Why it matters: required for SLA/SLO management, incident
response, audit evidence, and forensic analysis.

External system

DESP (Eurosystem) What it does: scheme platform routing validation requests,
performing settlement, and issuing confirmations or rejections (and scheme-side
risk signals where applicable). How it interacts: via the DESP Connectivity
Adapter only; TMS never “talks” directly to DESP.

B) Component view explanation (inside Transaction Management Service)

These components exist to keep the payment path correct, auditable, and
resilient:

Transaction API (txApi) HTTP handlers for initiate, status, history; validates
input structure and triggers orchestration.

Idempotency Manager (idempotency) Stores idempotency keys and outcomes so
retries do not create duplicate payments or inconsistent states.

Payment Orchestrator (orchestrator) The main coordination engine: resolves
payee, calls liquidity checks, triggers fraud checks, submits to DESP, applies
compensations on failure, triggers notifications.

Alias Resolver Adapter (aliasAdapter) Integrates to the alias/directory system;
handles caching and fallback behaviours so alias resolution remains stable.

Liquidity Adapter (liquidityAdapter) Encapsulates calls to Liquidity Management
Service for pre-check and execution steps (waterfall/reverse-waterfall).

Fraud & Risk Checks (fraudEngine) Runs PSP-side fraud rules/ML checks and
consumes risk signals as required; can be asynchronous where scheme/time budget
allows.

DESP Client (despClient) Builds scheme-aligned messages, manages correlation
IDs, and exchanges messages through the DESP Connectivity Adapter.

Transaction State Machine (stateMachine) Enforces deterministic state
transitions (e.g., RECEIVED → SUBMITTED → VALIDATED → SETTLED/REJECTED) and
ensures invalid transitions cannot occur.

Reconciliation Worker (reconWorker) Processes settlement
confirmations/rejections (including late callbacks), resolves mismatches, and
finalises or repairs state.

Audit Logger (audit) Writes immutable evidence: who/what/when, reason codes,
correlation IDs, and key state transitions.

Event Publisher (eventPublisher) Emits lifecycle events through the outbox to
Kafka (e.g., PaymentInitiated, PaymentValidated, PaymentSettled,
PaymentRejected).

Notification Dispatcher (notifier) Triggers customer notifications after
settlement/rejection and optionally intermediate updates (e.g., “pending”).

3) Typical end-to-end flow (you can include as narrative)
Flow: Online payment initiation → settlement

User initiates payment via channel → API Gateway → Transaction API

Idempotency Manager checks the request is not a duplicate

Payment Orchestrator resolves payee alias via Alias Resolver Adapter

Orchestrator invokes Liquidity Adapter for payer/payee pre-checks and executes
any required liquidity steps

Orchestrator triggers Fraud & Risk Checks (sync/async based on policy)

Orchestrator submits scheme message via DESP Client → DESP Connectivity Adapter
→ DESP

TMS persists transitions in Transaction DB and emits events via Outbox → Kafka

On DESP confirmation/rejection, Reconciliation Worker finalises state in DB,
Audit Logger records evidence, and Notification Dispatcher informs the user

\subsubsection{Offline Management Service (OMS): Intermediary-Side Offline Distribution and Reconciliation}
\label{subsubsec:offline_mgmt_service}

Figure~\ref{fig:offline_mgmt_service} presents a reference architecture for the
Offline Management Service (OMS) in a bank/PSP distribution layer. The OMS
governs the lifecycle of offline-capable digital euro wallets and provides the
intermediary-side control plane required to support proximity payments when
end-user devices are temporarily disconnected from the network. The
architecture explicitly separates (i) the offline payment execution path, which
occurs locally between devices in physical proximity, from (ii) the online
control and reconciliation path, through which wallets are provisioned,
funded/defunded, policy-controlled (e.g., transaction and holding limits), and
periodically reconciled with Eurosystem verification services.

At a system level, end users and merchants conduct offline payments using
secure devices (e.g., secure-element backed wallets on mobile phones or
card-like form factors) that can locally authenticate counterpart devices and
maintain tamper-resistant value state. The OMS interacts with these devices
through a secured API gateway and provides provisioning and integrity controls
(secure element provisioning and device attestation), policy distribution
(signed and versioned limit parameters), and operational workflows for
funding/defunding (via core banking accounts and optional ATM channels).
Offline transaction records are uploaded during reconnection sessions and
forwarded for central verification to detect inconsistencies (including
potential double-spending) and to reset local wallet state. Reliability and
auditability are ensured through durable storage of wallet bindings and
reconciliation sessions, event-driven monitoring and alerting, and
cryptographic protections via key management/HSM services for mutual TLS,
signing, and key custody.

\begin{figure}[H]
      \centering
      \includegraphics[width=\linewidth]{./assests/8.1.4_Offline Management Service_ComponentOfflineComponents.png}
      \caption{Offline Management Service (OMS) reference architecture (container view) for secure element provisioning, policy-controlled offline payments, funding/defunding, and periodic online reconciliation with Eurosystem verification.}
      \label{fig:offline_mgmt_service}
\end{figure}

\textbf{Explanation}

A. Actors and offline endpoints

1) Digital Euro User

Holds and uses an offline-capable wallet (mobile or card form factor).

Initiates: provisioning, funding/defunding requests, reconciliation sessions.

Executes offline payments locally with merchants or other users.

2) Merchant / Payee

Uses a merchant acceptance device (offline-capable POS or merchant wallet
device).

Receives offline payments and stores them until reconnection/reconciliation.

3) Operations / Support

Monitors provisioning success rates, reconciliation exceptions, device risk
signals, policy rollouts, and incident response.

Consumes events/alerts and observability outputs to resolve failures (e.g.,
attestation failures, reconciliation mismatches).

B. Offline execution path (device-to-device proximity payments)

4) Offline Digital Euro Device (End-user device)

This is the user’s offline wallet environment (mobile or card-like).

Internally, it includes a secure execution environment (typically a Secure
Element (SE) or equivalent trusted hardware/TEE) that:

holds cryptographic keys,

enforces offline limits,

updates local value state atomically (decrement payer / increment payee),

stores offline transaction records for later upload.

5) Offline-capable Merchant Device / POS

Merchant acceptance endpoint for offline transactions.

Verifies the payer device’s authenticity and the transaction’s signature.

Updates merchant-side offline balance and stores transaction record.

Relationship: device -> pos (NFC/QR)

Represents offline proximity payment transfer:

executed locally and synchronously,

no third-party call at the moment of payment,

communication via NFC (most typical) or QR.

C. Online control plane (provisioning, policy, and reconciliation)

6) API Gateway

The secure ingress for device/app calls once the device is online.

Enforces:

authentication and authorization,

throttling/rate limiting,

correlation IDs for traceability,

security controls to prevent abuse (e.g., bot-like provisioning attempts).

Relationship: device -> apiGateway (HTTPS)

Device calls are used for:

secure element provisioning workflow coordination,

retrieving signed policies,

starting reconciliation sessions,

initiating funding/defunding (if supported over-the-air).

D. The core service and its supporting infrastructure

7) Offline Management Service (OMS)

The intermediary-side “brain” for offline distribution.

Responsibilities:

Provisioning orchestration

coordinates SE applet loading/activation via provisioning adapter,

binds wallet to user identity,

records metadata and lifecycle state.

Device integrity

requests attestation results before enabling offline mode.

Policy distribution

fetches signed/immutable offline policies (limits) and ensures the device
enforces the latest version.

Funding/defunding orchestration

debits/credits linked commercial bank accounts via core banking,

optionally supports ATM channel workflows.

Reconciliation packaging and forwarding

collects offline transaction records and device status on reconnection,

forwards to Eurosystem verification for validation/consistency checks,

applies outcomes (e.g., wallet reset state, lockout, exception workflow).

Auditability

produces traceable logs/events for compliance and operational forensics.

8) Offline Wallet Registry & Reconciliation Store (PostgreSQL)

Durable system of record for:

wallet bindings (user ↔ device wallet),

device metadata and status,

policy versions applied,

reconciliation sessions and upload receipts,

idempotency keys (to avoid double provisioning/duplicate reconciliations).

9) Offline Policy/Session Cache (Redis)

Speeds up:

fetching policy bundles (hot reads),

session tokens for reconciliation workflows,

caching attestation outcomes for short periods.

Reduces load on the DB and keeps device interactions responsive.

10) Event Bus (Kafka)

Publishes asynchronous events such as:

WalletProvisioned, PolicyUpdated, ReconciliationStarted,

ReconciliationSucceeded, ReconciliationFailed,

DeviceAttestationFailed, OfflineLimitBreached.

Enables:

ops dashboards and alerting,

fraud analytics and anomaly detection pipelines,

decoupling from the OMS hot path.

11) Observability & SIEM

Collects logs/metrics/traces for:

SLA/SLO monitoring (availability of provisioning/reconciliation),

security monitoring (suspicious device patterns),

audit evidence and incident investigations.

12) Key Management / HSM

Cryptographic trust anchor for OMS:

holds keys used to sign policy bundles (or verify signatures),

supports mutual TLS for Eurosystem connectivity,

protects secrets used for secure provisioning workflows.

E. Provisioning and trust services (security-critical dependencies)

13) SE Provisioning Adapter

Handles the mechanics of secure element applet delivery and activation:

downloads/installs the wallet applet,

collects the SE public keys/certificates (keys generated and retained inside
SE),

ensures correct versioning and rollback safety.

14) Device Attestation Service

Determines whether a device is trustworthy to hold offline value:

verifies device integrity (e.g., not rooted/jailbroken),

validates application authenticity and certificate chain,

issues an allow/deny decision and risk attributes to OMS.

OMS uses this decision to:

permit offline enablement,

require re-provisioning,

block or lock devices that fail integrity checks.

15) Offline Policy Distribution

Provides signed, versioned policy parameters such as:

per-transaction offline caps,

cumulative offline balance cap,

max number of offline transactions before reconnection,

maximum time since last successful reconciliation,

fraud/risk thresholds that can trigger lockout.

OMS delivers the policies to the device and records which version is active.

F. Funding/defunding integration (bridging offline and bank money)

16) Core Banking / Customer Accounts

Used to debit/credit the customer’s commercial bank account when:

funding the offline wallet (top-up),

defunding (cash-out / revert to bank account).

Ensures the offline value is anchored to bank-side accounting and customer
statements.

17) ATM / Funding Device Channel

Optional channel to allow funding/defunding via physical devices:

important for inclusion and offline-heavy scenarios,

provides an alternative when mobile connectivity is unavailable.

G. Eurosystem verification / reconciliation boundary

18) Eurosystem Offline Issuance / Verification (External System)

Receives reconciliation packages (offline transaction records + device state).

Performs central verification such as:

detecting inconsistencies and potential double spending,

validating transaction chains and device legitimacy,

returning reconciliation outcomes (accept/reset/lock/exception).

OMS applies the outcome and updates wallet/device lifecycle state.

19) DESP (optional contextual link)

Included as an optional “context reference”:

the OMS can reference online identifiers for lifecycle consistency,

but offline verification is modelled separately via the Eurosystem verification
service.

3) Typical end-to-end operational flows (concise, thesis-friendly)
Flow 1: Secure element provisioning

Device contacts API Gateway → OMS

OMS calls Device Attestation to verify integrity

OMS calls SE Provisioning Adapter to install/activate wallet applet

OMS stores binding and metadata in Offline DB

OMS fetches signed policy from Offline Policy Distribution, pushes to device

OMS emits event to Kafka; ops sees provisioning success via observability

Flow 2: Offline proximity payment

Payer device and merchant POS exchange data via NFC/QR

Both secure environments validate authenticity and update local balances

Both store transaction records locally for later reconciliation

Flow 3: Online reconciliation

Device reconnects and uploads offline records via API Gateway → OMS

OMS stores session + records in Offline DB, caches session state

OMS forwards reconciliation package to Eurosystem Verification

Eurosystem returns outcome; OMS updates local state and records result

Events are published; exceptions generate alerts for ops/support

% =========================================================
%  Conclusion and Recommendations
% =========================================================

\section{Conclusion and Recommendations}
\label{sec:conclusion}

\subsection{Key Findings}
\label{subsec:key_findings}

This thesis investigated how banks and other payment service providers (PSPs)
can integrate the digital euro into existing banking landscapes, focusing on
(i) scheme-aligned technical integration patterns, (ii) implementation models
(in-house, outsourced, hybrid), (iii) tier-based bank strategies, and (iv)
opportunities for shared infrastructure and cost mutualization.

\paragraph{Finding 1: The Eurosystem design enables bank integration if the scheme boundary is respected.}
Across the analysed ECB technical deliverables, the digital euro architecture
consistently positions banks/PSPs as the distribution layer responsible for
onboarding, customer-facing channels, compliance controls, and operational
processes, while the Eurosystem provides the settlement-layer services through
the Digital Euro Service Platform (DESP). This supports a clear ``scheme
boundary'' in which banks integrate via standardised interfaces, rather than
re-implement central settlement logic.\footnote{European Central Bank (ECB).
      Digital euro overview and project information (accessed 2026-02-04).
      \url{https://www.ecb.europa.eu/euro/digital_euro/html/index.en.html}} In your
report, this boundary is operationalised through DESP-aligned service bundles
(Access Management, Liquidity Management, Transaction Management, Offline
Service, and Risk/Fraud/Disputes) with explicit integration touchpoints into
bank core systems, channels, and back-office operations.\footnote{See Section~4
      of this thesis for the service-bundle mapping and bank back-end integration
      dimensions.}

\paragraph{Finding 2: Integration complexity concentrates in five bank-owned capability clusters.}
Although the scheme interface reduces the need for bespoke settlement
mechanisms, bank work is not trivial. Evidence in the scheme-aligned model
indicates that the most impacted bank modules are: (1) identity/KYC and
customer master data mapping (e.g., wallet/account identifiers and alias
linkage), (2) liquidity/limits orchestration (including waterfall and
reverse-waterfall flows where applicable), (3) payment orchestration and
exception handling (including reservations/refunds/disputes), (4) offline
enablement (device security, provisioning, reconciliation after re-connection),
and (5) risk/compliance processes that must operate under privacy constraints
(especially offline, where transaction-level visibility is
limited).\footnote{ECB design principles highlight privacy-preserving
      processing and a multi-region resilient back-end. See ECB preparation-phase
      reporting and public materials:
      \url{https://www.ecb.europa.eu/euro/digital_euro/html/index.en.html}}

\paragraph{Finding 3: Non-functional requirements (latency, resilience, and certification) are first-order design drivers.}
The draft digital euro scheme rulebook (v0.9) places explicit requirements on
PSP processing latency for online transactions and expects PSPs to implement
detailed front-end and back-end implementation specifications (with further
annexes), including constraints that depend on transaction type (POS vs.\
e-commerce vs.\ P2P) and interactions with scheme services (e.g., fraud-score
signals).\footnote{ECB. \emph{Draft digital euro scheme rulebook v0.9}
      (Rulebook Development Group documents).
      \url{https://www.ecb.europa.eu/paym/intro/activities/html/digital_euro_rdg.en.html}}
A notable implication is architectural: banks should implement a low-latency,
idempotent, observable scheme adapter and orchestration layer, and treat
certification/testing artefacts as deliverables, not as ``end-of-project''
steps.

\paragraph{Finding 4: Offline is the main architectural discontinuity and dictates early design decisions.}
Offline digital euro functionality introduces requirements that differ from
online account-based payments: device security (secure element/TEE
assumptions), offline balance management, offline-to-online reconciliation, and
recovery workflows for lost devices. This is emphasised in your architecture as
a dedicated Offline Management capability and by the rulebook’s explicit note
that later versions will expand offline SDK and offline issuance integration
specifications.\footnote{ECB. \emph{Draft digital euro scheme rulebook v0.9},
      technical requirements section (noting future offline SDK/offline distribution
      specifications).
      \url{https://www.ecb.europa.eu/paym/intro/activities/html/digital_euro_rdg.en.html}}
Consequently, banks that postpone offline considerations risk rework across
channels, security/key management, customer support, and reconciliation.

\paragraph{Finding 5: ``In-house vs.\ outsourced vs.\ hybrid'' is primarily a governance and risk decision---not only a technology decision.}
The three implementation models differ less by the existence of microservices
or APIs (all need them) and more by (i) who owns the scheme adapter and
compliance evidence, (ii) how quickly the bank can iterate and certify changes,
(iii) how operational resilience and third-party risk are managed, and (iv)
whether the bank seeks strategic differentiation beyond baseline scheme
services.\footnote{EU operational resilience expectations are reinforced by
      DORA’s application since 17 January 2025 (third-party ICT risk, incident
      management, resilience testing). See EIOPA DORA overview:
      \url{https://www.eiopa.europa.eu/digital-operational-resilience-act-dora_en}}

\paragraph{Finding 6: Shared infrastructure is credible and economically necessary, but must be limited to ``shareable'' layers.}
The thesis showed that the most feasible shared elements are those that (a) do
not require access to sensitive customer data, or (b) can be shared under
strict governance: scheme connectivity components (adapter frameworks),
test/certification tooling, device security procurement patterns (e.g., secure
element provisioning processes), shared fraud/risk intelligence for
non-identifying signals, and common merchant acceptance enablement. By
contrast, onboarding/KYC decisions, customer support, dispute handling, and
product differentiation should remain bank-specific to preserve accountability
and competitive autonomy.

\subsection{Recommendations}
\label{subsec:recommendations}

Recommendations are organised by bank tier and ecosystem actor, reflecting the
scheme-centric structure of the digital euro programme and the rulebook
development approach.

\subsubsection{Recommendations for Banks (All Tiers)}
\label{subsubsec:rec_all_banks}

\paragraph{R1: Build a ``scheme adapter'' layer as the architectural pivot.}
Banks should implement an explicit Digital Euro Integration Layer (adapter +
orchestration), isolating core systems from DESP interface changes. This layer
should provide: API mediation, security and authentication,
mapping/transformation (identifiers and reference data), eventing/monitoring,
idempotency, and end-to-end traceability for certification evidence.

\paragraph{R2: Treat certification-readiness as a continuous workstream.}
Given the rulebook’s detailed implementation specifications and non-functional
requirements, banks should establish a certification pipeline with automated
conformance tests, performance tests (including latency budgets), security
testing, and auditable change control.\footnote{ECB. \emph{Draft digital euro
            scheme rulebook v0.9}.
      \url{https://www.ecb.europa.eu/paym/intro/activities/html/digital_euro_rdg.en.html}}

\paragraph{R3: Engineer for privacy-by-design and offline constraints early.}
Banks should implement privacy controls consistent with GDPR minimisation and
the scheme’s privacy-preserving design. For offline, banks must adopt risk
controls that do not assume transaction-level visibility and must instead
emphasise device integrity, limits/holdings governance, anomaly detection at
re-connection, and recovery workflows.\footnote{ECB digital euro information
      highlights privacy and offline as core objectives.
      \url{https://www.ecb.europa.eu/euro/digital_euro/html/index.en.html}}

\paragraph{R4: Align operational resilience and vendor governance with DORA.}
Where vendors are used, banks should implement DORA-consistent third-party ICT
governance: contractual SLAs, exit strategies, concentration-risk monitoring,
incident reporting processes, and resilience testing that covers the
vendor-delivered parts of the digital euro stack.\footnote{EIOPA DORA overview
      (application since 17 January 2025).
      \url{https://www.eiopa.europa.eu/digital-operational-resilience-act-dora_en}}

\paragraph{R5: Use phased delivery aligned to service bundles and channels.}
Banks should prioritise a phased rollout that maps directly to service bundles
and channel readiness: (i) online P2P/e-commerce + basic wallet lifecycle, (ii)
merchant/POS enablement, (iii) offline enablement and reconciliation, (iv)
advanced features (e.g., conditional payments) in the market-participant layer,
while continuously validating performance and supportability.

\subsubsection{Recommendations by Bank Tier}
\label{subsubsec:rec_by_tier}

\paragraph{High-tier banks (large, internationally active): In-house or ``hybrid-plus''.}
High-tier banks should pursue in-house development for the scheme adapter,
orchestration, and channel integration while selectively partnering for
specialised components (e.g., device security supply chain, secure element
provisioning services, or tooling accelerators). This approach maximises
roadmap control, supports multi-jurisdictional integration, and enables
differentiation through value-added services (e.g., advanced treasury/corporate
features, analytics, conditionality engines).\footnote{Policy direction for
      digital euro distribution and its complementarity to existing payments is
      described by ECB public materials.
      \url{https://www.ecb.europa.eu/euro/digital_euro/html/index.en.html}}

\paragraph{Mid-tier banks (regional to multi-market): Hybrid as the default strategy.}
Mid-tier banks should outsource commodity capabilities (baseline wallet
services, standard compliance tooling, generic reporting) while retaining
in-house ownership of: the integration layer, customer experience, and at least
one strategic differentiator (e.g., SME cash management, loyalty integration,
sector-focused conditional payments). Vendor selection should emphasise
portability and auditability, not only license costs.

\paragraph{Low-tier banks (small/community-focused): Vendor/consortium-first with strict exit clauses.}
Low-tier banks should adopt a vendor platform or a consortium model where the
scheme adapter, testing, and operational tooling are mutualised, while the bank
keeps customer relationship ownership, local compliance, and support. Banks
should contractually require data portability, documented APIs, and operational
continuity mechanisms to mitigate vendor lock-in.

\subsubsection{Recommendations for Shared Infrastructure and Industry Collaboration}
\label{subsubsec:rec_shared}

\paragraph{R6: Mutualise testing/certification tooling and environments.}
Industry associations or banking groups should invest in shared test harnesses
aligned to rulebook requirements (functional + non-functional). Shared labs
reduce duplication, accelerate learning, and improve standardisation of
evidence artefacts for compliance and audits.

\paragraph{R7: Mutualise device-security procurement patterns (where feasible).}
Given offline requirements, banks should consider shared frameworks for device
certification, secure element/TEE integration patterns, and operational
playbooks for loss/theft recovery. Shared procurement can reduce unit costs and
reduce security fragmentation without centralising customer data.

\paragraph{R8: Create shared ``non-identifying'' fraud/risk signal services.}
Where legally and operationally feasible, banks can pool fraud typologies and
non-identifying risk signals to improve collective detection while preserving
the privacy constraints of the scheme, especially given the limited transaction
visibility inherent to offline payments.

\subsubsection{Recommendations for the Eurosystem, Regulators, and Policymakers}
\label{subsubsec:rec_ecosystem}

\paragraph{Eurosystem/ECB: Stabilise specifications early and publish conformance artefacts.}
The Eurosystem should continue prioritising stable interface specifications,
conformance tests, and certification pathways aligned with the rulebook
development process. Additionally, offline specifications should be advanced as
early as possible to reduce bank rework, consistent with the rulebook’s staged
approach to offline SDK and offline issuance integration.\footnote{ECB Rulebook
      Development Group documents (including v0.9 draft rulebook).
      \url{https://www.ecb.europa.eu/paym/intro/activities/html/digital_euro_rdg.en.html}}

\paragraph{National competent authorities: harmonise supervision expectations for PSP digital euro services.}
Supervisors should provide consistent guidance on how existing AML/CFT
obligations apply across online and offline flows, and clarify supervisory
expectations on operational resilience and outsourcing governance for digital
euro-related services.

\paragraph{EU legislators and policymakers: finalise the legal framework while preserving cash and market competition.}
As the legislative process advances, policymakers should ensure that (i)
distribution responsibilities and incentives are coherent for PSPs, (ii) user
protections and privacy are enforceable, and (iii) the digital euro complements
cash and existing European payment solutions.\footnote{Council of the EU press
      release on the Council position for the digital euro and strengthening the role
      of cash (19 Dec 2025).
      \url{https://www.consilium.europa.eu/en/press/press-releases/2025/12/19/single-currency-council-agrees-position-on-the-digital-euro-and-on-strengthening-the-role-of-cash/}}

\subsection{Limitations and Future Research}
\label{subsec:limitations_future}

\paragraph{Limitations.}
First, the scheme design and technical specifications are evolving; the
rulebook explicitly anticipates additional specifications (notably for offline
SDKs and acceptance solutions). Second, implementation cost estimates in the
thesis depend on assumptions about bank legacy complexity, vendor pricing
structures, and adoption pace, which can vary materially by country and
institution. Third, the analysis focuses on technical and operational
integration rather than macroeconomic or behavioural adoption impacts.

\paragraph{Future research directions.}
Future work should address: (i) empirical validation of implementation cost
drivers using anonymised bank delivery data, (ii) quantitative modelling of
shared-infrastructure governance models (including competition-law and
data-protection constraints), (iii) detailed offline threat modelling and
resilience testing methodologies, (iv) adoption and merchant acceptance
incentives under different fee and holding-limit regimes, and (v) cross-border
interoperability implications (e.g., interactions with other CBDCs, instant
payment schemes, and private-sector tokenised money initiatives).

\paragraph{Closing statement.}
Overall, the thesis concludes that successful digital euro integration is
achievable for banks of all tiers if implementation decisions are anchored to
the scheme boundary, certification and non-functional requirements are treated
as core engineering constraints, and shared infrastructure is pursued for
``shareable'' components while preserving bank accountability for
customer-facing and compliance-critical functions.

\subsubsection{Deployment Architecture: Multi-Region Kubernetes for Operational Resilience and Controlled Scheme Connectivity}
\label{subsubsec:deployment_architecture}

Figure~\ref{fig:deployment_architecture} presents a reference deployment architecture for a bank/PSP digital euro integration platform based on containerised microservices orchestrated by Kubernetes across multiple EU regions. The architecture is designed to support high availability, controlled external connectivity to the Eurosystem scheme interface, and auditable change operations. It separates (i) shared EU-resident platform services (security controls, CI/CD, and central observability) from (ii) regional production clusters that host the core digital euro microservices (Access, Liquidity, Transaction, Offline, Notifications) and their supporting data services.

To reduce correlated failure risk and support continuity objectives, the topology uses two active production regions and one warm standby region. Stateless services are deployed in active-active mode across the two active regions, while stateful components apply a tiered replication strategy: synchronous replication within a region (across availability zones) for high availability and asynchronous replication cross-region to support disaster recovery. Scheme connectivity is isolated into a dedicated “connectivity zone” with restricted egress and allow-listed outbound traffic, ensuring that only the DESP Connectivity Adapter can reach Eurosystem endpoints, and only via managed security and key custody controls (KMS/HSM).

A secure software delivery pipeline (CI/CD) with approval gates and progressive deployment (canary and automated rollback) is integrated as a first-class operational capability. Observability (metrics, logs, tracing) and security monitoring (SIEM) are treated as shared services to enable end-to-end auditability, incident response, and compliance evidence generation. Overall, the deployment view demonstrates how banks/PSPs can implement digital euro microservices in a cloud-native manner while meeting supervisory expectations for operational resilience, ICT risk management, and outsourcing governance.

\begin{figure}[H]
  \centering
  \includegraphics[width=\linewidth]{./assests/8.1.5_deployment_arch.png}
  \caption{Multi-region Kubernetes deployment reference architecture (two active regions plus one warm standby) for a bank/PSP digital euro integration platform with segregated scheme connectivity, HA data services, and central observability and CI/CD.}
  \label{fig:deployment_architecture}
\end{figure}


\textbf{explanation}

diagram is a Deployment View with three main layers:

EU Shared Services (global but EU-resident)

EU-Central (Active) and EU-North (Active) production regions

EU-South (Warm Standby / Recovery) region

A) EU Shared Services (global platform layer)

WAF / DDoS Protection

Shields all public entry points (web/mobile/POS ingress endpoints) from volumetric attacks and common L7 threats (rate-based attacks, injection patterns).

Reduces the chance that a traffic spike becomes a region-wide outage.

Private Container Registry

Stores signed container images and SBOM artifacts (software bills of materials).

Enables reproducible deployments and supply-chain control (deploy only verified artifacts).

Secrets Manager

Central vault for secrets (DB creds, API tokens, service identities) with rotation and access policies.

Supports least-privilege access across clusters.

CI/CD Platform

Builds, tests, scans, signs, and deploys releases (often via GitOps).

Includes approval gates and evidence logs for audit trails, which is consistent with supervisory expectations on controlled ICT change.

Key Management / HSM (KMS/HSM)

Custody for cryptographic keys used by services (especially DESP connector), certificate lifecycle, signing keys, and mTLS identities.

Centralizes rotation and access controls.

Observability & SIEM

Central collection of metrics/logs/traces + security monitoring.

Supports incident response, compliance evidence, and operational resilience testing (a key expectation under DORA-style resilience management).

B) EU-Central Region (Primary Active)

This region is shown as a Kubernetes Cluster (multi-AZ) with distinct node pools/zones.

1) Ingress Node Pool

Ingress Controller (Nginx/Envoy)

L7 entry routing into the cluster (TLS termination / re-encryption, routing rules).

Works with WAF edge policies.

API Gateway container instance

AuthN/AuthZ, request routing to microservices, rate limiting, correlation IDs.

Keeps channel access consistent across services.

2) Core Services Node Pool (stateless business logic)

Access Management Service

Wallet/account lifecycle, onboarding state, alias binding hooks.

Liquidity Management Service

DCA monitoring + waterfall/reverse-waterfall orchestration (i.e., liquidity operations needed to satisfy holding limits and payer funding conditions).

Transaction Management Service

Payment orchestration and deterministic transaction state machine.

Offline Management Service

SE provisioning orchestration, offline lifecycle state, reconciliation packaging and upload initiation.

Notification Service

Sends receipts/status messages (settled/rejected) to the user/merchant channels.

Why separate “core services pool”?
Because these are mostly stateless and scale horizontally; isolating them from data workloads reduces noisy-neighbor issues and simplifies autoscaling.

3) DESP Connectivity Zone (restricted egress)

Egress Gateway (Firewall/NAT)

Enforces allow-listed outbound connectivity only to scheme endpoints.

Central control point for network policy, IDS/IPS, and egress logging.

DESP Connectivity Adapter container instance

The only component that talks to Eurosystem endpoints.

Provides secure transport (mTLS), retries/backoff, correlation IDs, versioning safety, and keeps scheme integration away from business services.

This “connectivity zone” directly supports risk containment and strong boundary controls, consistent with ICT risk management expectations.

4) Data Services (EU-Central) — HA within region

PostgreSQL HA cluster (txDb + identityDb)

System-of-record data.

HA via synchronous replication across AZs (survive one-AZ failure without data loss).

Kafka cluster (eventBus)

Event streaming backbone (transaction events, reconciliation events, ops signals).

Decouples services and supports async processing.

Redis cluster (cache)

Hot cache (sessions, frequently read flags, rate limit state).

Treated as ephemeral; can be rebuilt on failover.

C) EU-North Region (Secondary Active)

This region mirrors the structure of EU-Central for stateless services (active-active), while the data tier is DR-ready:

Ingress + API Gateway: receives traffic either via global traffic steering or failover

Core services pool: runs the same microservices to take full load if EU-Central degrades

DESP connectivity zone: isolated outbound connectivity to scheme endpoints

Data services: PostgreSQL replica / failover-ready

Typically asynchronous replication from EU-Central to reduce cross-region latency costs while enabling disaster recovery.

Kafka and Redis: region-local, with replicated topics or mirrored streams for continuity.

This aligns with the “design for resilience” approach expected in EU operational resilience regimes (DORA).

D) EU-South Region (Warm Standby / Recovery)

This is the recovery region:

Runs minimal capacity (“scaled down”) but can autoscale during a major regional incident.

Holds DR replicas (PostgreSQL DR replica, Kafka DR replication/standby).

Provides another layer of geographic redundancy (especially for correlated failures).

This supports disaster recovery and continuity testing expectations.

3) How the deployment behaves in practice (short flows)
Normal operation (two active regions)

User traffic hits WAF → ingress → API gateway (EU-Central or EU-North)

API gateway routes to Transaction/Liquidity/Access/Offline services

Events published to Kafka; caches used for hot reads

DESP calls go only through DESP Connectivity Zone (egress allow-list + mTLS keys from KMS/HSM)

Regional degradation / failover

Traffic shifts to the other active region for stateless workloads.

Data layer switches to replica/failover depending on DR plan and RTO/RPO objectives.

Observability/SIEM continues centrally for consistent incident evidence.

Controlled change / release

CI/CD builds + scans + signs artifacts → approval gate → canary rollout → automated rollback if error rate/latency breaches thresholds.

Evidence is retained (deploy logs, approvals, artifact signatures), consistent with governance controls emphasized in EBA ICT and outsourcing guidance

\section{Regulatory Considerations and Compliance Framework}
\label{sec:regulatory-compliance}

\subsection{Digital Euro Regulatory Framework}
\label{subsec:digital-euro-regulatory}

\subsubsection{Rulebook Compliance Requirements}
\label{subsubsec:rulebook-compliance}

The digital euro scheme rulebook (currently published in draft form) defines a single set of rules, standards and procedures for scheme participants (including payment service providers, PSPs) and is expected to evolve as the Eurosystem finalises functional specifications, service levels and certification artefacts.\cite{ECB2025RulebookV09,ECB2025RDGProgressOct}
In the current draft, multiple non-functional and service-level requirements are expressed as placeholders (e.g., \emph{XAvailability\%}, \emph{XRTO}, and PSP-latency KPIs), which implies that PSP implementations should be engineered for strict operational performance, but should avoid hard-coding compliance claims to specific numeric thresholds until the scheme KPIs are final.\cite{ECB2025RulebookV09}

\paragraph{Access management and participant obligations.}
At minimum, PSPs must implement controlled customer and wallet lifecycle processes (onboarding, wallet provisioning, alias/identifier management where applicable, suspension/offboarding, and customer servicing), including auditability and evidencing of compliance activities.\cite{ECB2025RulebookV09}
From an implementation perspective, these processes must be integrated with (i) the bank's identity and access management (IAM), (ii) KYC and customer due diligence (CDD) tooling under the EU AML framework, and (iii) internal controls ensuring segregation of duties and traceability.\cite{AMLR2024,EBAGL_ICT_2019}

\paragraph{Transaction processing, risk controls and disputes.}
The draft rulebook separates responsibilities across payer-PSP and payee-PSP roles and introduces scheme-level mechanisms for risk and fraud management (e.g., exchanging risk-related outputs in relevant use cases) and for dispute management and exception handling between participants.\cite{ECB2025RulebookV09}
Performance expectations should be understood in an end-to-end sense: Eurosystem design material for the digital euro infrastructure targets low-latency processing at the core (to enable rapid user experiences), while PSP-facing KPIs are specified as placeholders in the rulebook draft.\cite{ECB2023MarketResearchAnnex,ECB2025RulebookV09}
Consequently, banks should implement: (i) real-time authorization and status feedback patterns, (ii) deterministic reconciliation and exception handling, and (iii) dispute workflows aligned with established payment services practice where applicable.\cite{ECB2025RulebookV09,EC_PSD_FAQ_2023}

\paragraph{Privacy and data protection.}
ECB public documentation states that the digital euro is designed for very high privacy standards and that the Eurosystem would not be able to identify users from the payment data it processes; additionally, offline digital euro payments are intended to provide cash-like privacy, where personal transaction details are only known to payer and payee.\cite{ECB_DE_PrivacyMain,ECB_DE_PrivacyOfflineFAQ}
For PSPs, this implies \emph{privacy-by-design} and \emph{data minimisation} in customer-facing and compliance systems, with clear separation of identity data, transaction processing data, and analytics/monitoring data.\cite{GDPR2016}
Data retention must follow the GDPR storage limitation principle and sectoral legal obligations; in particular, AML/CFT rules require retaining CDD-related information for legally defined periods (typically measured in years), after which deletion or anonymisation is required unless an extension is lawfully justified.\cite{GDPR2016,AMLR2024}

\paragraph{Sanctions screening and AML/CFT controls.}
For online use cases, PSPs remain responsible for AML/CFT controls proportionate to risk, including sanctions screening based on applicable EU restrictive measures and other relevant lists depending on the PSP's jurisdictions and exposure.\cite{AMLR2024,EU_SanctionsMap}
In practice, this requires integration of transaction monitoring, sanctions screening, case management, and regulatory reporting (e.g., suspicious transaction reporting) with robust model governance if advanced analytics or ML-based monitoring is used.\cite{EBAGL_ICT_2019}

\subsubsection{Regulatory Approval and Certification Process}
\label{subsubsec:certification}

While the final digital euro certification and conformance regime will depend on the mature scheme artefacts, a bank can structure readiness along an evidence-based assurance lifecycle consistent with EU operational resilience and ICT governance expectations.\cite{DORA2022,EBAGL_ICT_2019}
A pragmatic compliance pathway comprises:

\begin{itemize}
  \item \textbf{Design compliance mapping}: traceability matrix mapping rulebook obligations and implementing acts to business processes, applications, and controls.\cite{ECB2025RulebookV09}
  \item \textbf{Implementation verification}: functional conformance testing, security testing, and resilience testing aligned with the institution's ICT risk management framework.\cite{DORA2022,EBAGL_ICT_2019}
  \item \textbf{Pilot and operational validation}: controlled pilot execution with documented incident handling, reconciliation performance, and dispute operations.\cite{ECB2025RulebookV09}
  \item \textbf{Ongoing compliance}: monitoring, audit logging, periodic control testing, and vendor/outsourcing oversight for critical ICT services.\cite{DORA2022,EBA_GL_Outsourcing}
\end{itemize}

\subsection{Risk Management Framework}
\label{subsec:risk-management}

\subsubsection{Operational Risk Management}
\label{subsubsec:operational-risk}

The digital euro increases the criticality of retail-payment availability and introduces new dependencies (e.g., scheme connectivity, offline functionality enablement, and multi-channel distribution). PSPs should therefore treat digital euro processing as a critical service and implement operational resilience aligned with DORA requirements (governance, ICT risk management, incident handling, testing, and third-party risk oversight).\cite{DORA2022}
The draft rulebook expresses non-functional requirements using KPI placeholders (e.g., availability and recovery objectives), which strongly suggests that banks must engineer for high availability and rapid recovery, but should not publish numeric commitments until the scheme KPIs are final.\cite{ECB2025RulebookV09}
Eurosystem infrastructure design materials indicate very high availability and fast recovery targets at core components, which can be used as an upper benchmark when defining PSP internal targets and capacity planning.\cite{ECB2023MarketResearchAnnex}

\subsubsection{Cybersecurity and Privacy Risk Management}
\label{subsubsec:cyber-privacy-risk}

\paragraph{Information security framework.}
A defensible approach is to combine (i) a risk-based ICT security programme per EBA ICT and security risk management guidance, (ii) DORA-aligned governance, testing and incident reporting, and (iii) secure software development and cryptographic engineering for online and offline use cases.\cite{EBAGL_ICT_2019,DORA2022}

\paragraph{Privacy-by-design and DPIA.}
GDPR requires data protection principles such as minimisation and storage limitation, and commonly necessitates a Data Protection Impact Assessment (DPIA) for high-risk processing.\cite{GDPR2016}
For digital euro integrations, PSPs should document: data flows, lawful bases, purpose limitation, retention schedules, access controls, and measures preventing unnecessary linkage between identity and transaction data, consistent with ECB privacy objectives for the digital euro.\cite{ECB_DE_PrivacyMain,ECB_DE_PrivacyOfflineFAQ}

% =========================================================
% END
% =========================================================











































% ====================================================================================================
% \section{Shared Infrastructure, Synergies, and Cost
%   Mutualization}
% % ========================================================================

% \section{Technical Blueprints and Best Practices}
% % ========================================================================

% \section{Regulatory Considerations and Compliance Framework}
% % ========================================================================

% \section{Conclusion and Recommendations}
% % ========================================================================

% \section{ Implementation Models: Technical and Strategic Analysis}

% \subsection{ In-House Implementation Model: Architecture and Requirements}

% \subsubsection{ Model Characteristics and Applicability}

% \textbf{Ideal Bank Profile:}
% \begin{itemize}
%   \item Large, internationally active banks (typically \(>\)€300 billion assets)
%   \item Advanced IT infrastructure and development capabilities
%   \item Significant technical staff and specialized expertise
%   \item Decentralized operations requiring customization
%   \item Strategic need for competitive differentiation
%   \item Sufficient capital for substantial upfront investment
% \end{itemize}

% \textbf{Key Characteristics:}
% \begin{itemize}
%   \item Full proprietary development and maintenance responsibility
%   \item Complete control over feature development and timelines
%   \item Direct accountability for security and compliance
%   \item Maximum flexibility for customization and innovation
%   \item Highest development and operational complexity
% \end{itemize}

% \subsubsection{ Technical Architecture for In-House Implementation}

% \textbf{Microservices Architecture Approach}

% \begin{verbatim}
% ┌─────────────────────────────────────────────────────┐
% │              Bank Customer Interfaces                │
% │    (Mobile App, Web, ATM, POS, Branch Systems)     │
% └────────┬────────────────────────┬──────────────────┘
%          │                        │
%     ┌────▼────────────────────────▼────┐
%     │   API Gateway & Orchestration     │
%     │  (REST, Authentication, Routing)  │
%     └────┬────────┬──────────┬─────┬───┘
%          │        │          │     │
%     ┌────▼──┐ ┌──▼────┐ ┌──▼──┐ ┌▼──────┐
%     │Access │ │Liquidity│Transaction│Risk & │
%     │Mgmt   │ │Mgmt   │Mgmt    │Compliance│
%     │Service│ │Service│Service │Service  │
%     └───┬───┘ └──┬────┘ └───┬──┘ └────┬──┘
%         │        │          │         │
%     ┌───▼─────────▼──────────▼─────────▼───┐
%     │   DESP Connectivity & Integration    │
%     │   (REST API Client, Message Queues)  │
%     └───┬──────────────────────────────────┘
%         │
%     ┌───▼──────────────────────────┐
%     │  DESP (External)              │
%     │ (Settlement, Liquidity, etc)  │
%     └──────────────────────────────┘
% \end{verbatim}

% \textbf{Microservices Components:}
% \begin{enumerate}
%   \item \textbf{Access Management Service}
%   \begin{itemize}
%     \item Functions: Onboarding, wallet provisioning, alias management
%     \item Technology stack: Java/Spring Boot, PostgreSQL
%     \item API endpoints: User creation, verification, wallet activation
%     \item Dependencies: Core banking system, KYC/AML systems
%   \end{itemize}

%   \item \textbf{Liquidity Management Service}
%   \begin{itemize}
%     \item Functions: DCA monitoring, waterfall operations, funding triggers
%     \item Technology stack: Node.js, MongoDB, Redis caching
%     \item API endpoints: DCA balance inquiry, waterfall request, reverse waterfall
%     \item Dependencies: Treasury systems, settlement systems, DESP
%   \end{itemize}

%   \item \textbf{Transaction Management Service}
%   \begin{itemize}
%     \item Functions: Transaction processing, clearing, settlement coordination
%     \item Technology stack: Java, Kafka message queue, PostgreSQL
%     \item API endpoints: Payment instruction, authorization, clearing
%     \item Dependencies: Authorization systems, clearing houses, fraud detection
%   \end{itemize}

%   \item \textbf{Risk and Compliance Service}
%   \begin{itemize}
%     \item Functions: Fraud detection, AML screening, risk scoring
%     \item Technology stack: Python (AI/ML), Apache Spark, feature store
%     \item API endpoints: Risk scoring, transaction flagging, compliance reporting
%     \item Dependencies: Regulatory reporting systems, sanctions databases
%   \end{itemize}

%   \item \textbf{Offline Management Service}
%   \begin{itemize}
%     \item Functions: Secure element provisioning, offline wallet management
%     \item Technology stack: C++, hardware security module (HSM) integration
%     \item API endpoints: Secure element provisioning, offline wallet creation
%     \item Dependencies: Device manufacturers, secure element providers
%   \end{itemize}
% \end{enumerate}

% \textbf{Integration with Existing Systems:}
% \begin{verbatim}
% Core Banking System
%     ├── Account Master Data
%     ├── Customer Records
%     ├── General Ledger
%     └── Statement Engine
%          │
%          ▼
% Digital Euro Microservices
%     ├── Access Management
%     ├── Liquidity Management
%     ├── Transaction Management
%     ├── Risk/Compliance
%     └── Offline Management
%          │
%          ▼
% External Systems
%     ├── Authorization Systems
%     ├── Settlement Systems
%     ├── Fraud Detection (Third-Party)
%     ├── Treasury Systems
%     └── DESP APIs
% \end{verbatim}

% \subsubsection{ Development and Deployment Considerations}

% \textbf{Team Structure and Expertise Requirements:}

% \begin{tabular}{|p{4cm}|p{3cm}|p{7.5cm}|}
% \hline
% \textbf{Role} & \textbf{Required FTEs} & \textbf{Key Expertise} \\
% \hline
% Platform Architects & 2--3 & Cloud architecture, microservices, system design \\
% \hline
% Backend Developers & 15--20 & Java, Python, API development, database design \\
% \hline
% DevOps Engineers & 5--8 & Kubernetes, CI/CD, infrastructure automation, monitoring \\
% \hline
% QA/Testing Engineers & 8--12 & Automated testing, performance testing, security testing \\
% \hline
% Security Engineers & 3--5 & Cryptography, secure element integration, threat modeling \\
% \hline
% Product Managers & 2--3 & Digital Euro requirements, market understanding, roadmap \\
% \hline
% Project Manager & 1 & Program coordination, stakeholder management, timeline tracking \\
% \hline
% \textbf{Total} & \textbf{36--52} & \textbf{Full-time commitment for 3--4 years} \\
% \hline
% \end{tabular}

% \vspace{0.75em}
% \textbf{Development Timeline:}
% \begin{verbatim}
% Phase 1: Foundation (Months 1-6)
% ├── Architecture design and stakeholder review
% ├── Technology stack finalization
% ├── Core API framework development
% ├── Database schema and integration patterns
% └── DevOps infrastructure setup

% Phase 2: Core Services (Months 7-18)
% ├── Access Management Service
% ├── Liquidity Management Service
% ├── Transaction Management Service
% ├── Risk/Compliance framework
% └── DESP integration framework

% Phase 3: Enhancement & Integration (Months 19-30)
% ├── Offline Management Service
% ├── Advanced conditional payments
% ├── Full channel integration (POS, ATM, etc.)
% ├── Performance optimization
% └── Security hardening

% Phase 4: Testing & Readiness (Months 31-36)
% ├── Comprehensive testing (unit, integration, load)
% ├── Security penetration testing
% ├── Regulatory compliance validation
% ├── Go-live preparation
% └── Operational runbook development

% Phase 5: Pilot & Production (Months 37-48)
% ├── Limited pilot deployment
% ├── Performance monitoring and tuning
% ├── User feedback incorporation
% └── Full production rollout
% \end{verbatim}

% \subsubsection{ Cost and Resource Implications}

% \textbf{Development Costs (4-Year Period):}

% \begin{tabular}{|p{6.2cm}|p{4cm}|p{4cm}|}
% \hline
% \textbf{Cost Category} & \textbf{Low Estimate} & \textbf{High Estimate} \\
% \hline
% Personnel (36--52 FTEs @ €100--150k avg) & €14.4M & €31.2M \\
% \hline
% Infrastructure (cloud, HSM, hardware) & €2M & €5M \\
% \hline
% Third-party software/licenses & €1M & €3M \\
% \hline
% Training and professional development & €0.5M & €1.5M \\
% \hline
% Testing and quality assurance & €2M & €4M \\
% \hline
% Contingency (10--15\%) & €2M & €4.5M \\
% \hline
% \textbf{Total} & \textbf{€21.9M} & \textbf{€49.2M} \\
% \hline
% \end{tabular}

% \vspace{0.75em}
% \textbf{Operational Costs (Post-Launch):}
% \begin{itemize}
%   \item Infrastructure and hosting: €500k--1M annually
%   \item Personnel maintenance team: 8--12 FTEs (€1--1.8M annually)
%   \item Vendor licenses and support: €300--500k annually
%   \item \textbf{Total annual operating costs: €1.8--3.3M}
% \end{itemize}

% \textbf{Capital Requirements:}
% \begin{itemize}
%   \item Upfront development: €20--50M
%   \item Hardware and infrastructure: €5--10M
%   \item Working capital and contingency: €5--10M
%   \item \textbf{Total capital requirement: €30--70M}
% \end{itemize}

% \subsubsection{ Risk Profile and Mitigation Strategies}

% \textbf{Key Risks in In-House Implementation:}

% \begin{tabular}{|p{4.2cm}|p{2.4cm}|p{2.6cm}|p{5.8cm}|}
% \hline
% \textbf{Risk} & \textbf{Probability} & \textbf{Impact} & \textbf{Mitigation} \\
% \hline
% Development delays and overruns & HIGH & HIGH & Agile methodology, external architecture review, contingency timeline \\
% \hline
% Skills gaps in emerging technologies & MEDIUM & HIGH & External consulting, vendor partnerships, training programs \\
% \hline
% Integration complexity with legacy systems & HIGH & MEDIUM & Strangler pattern, phased integration, dedicated integration team \\
% \hline
% Security vulnerabilities & MEDIUM & CRITICAL & Security review process, bug bounty programs, third-party testing \\
% \hline
% Regulatory compliance gaps & MEDIUM & HIGH & Compliance officer engagement, regulatory review checkpoints \\
% \hline
% Operational readiness issues & MEDIUM & MEDIUM & Pilot phase, comprehensive testing, operational runbook development \\
% \hline
% Resource availability & HIGH & MEDIUM & Dedicated hiring, external contractors, phased team building \\
% \hline
% \end{tabular}

% \vspace{0.75em}
% \textbf{Mitigation Strategies:}
% \begin{enumerate}
%   \item \textbf{External Architect Review}: Engage independent architecture review firm (quarterly)
%   \item \textbf{Vendor Partnerships}: Establish strategic partnerships with technology providers for specialized components
%   \item \textbf{Pilot Program}: Develop limited pilot with subset of users before full rollout
%   \item \textbf{Security Reviews}: Third-party security assessments at key milestones
%   \item \textbf{Compliance Officer}: Dedicated regulatory liaison coordinating with competent authorities
%   \item \textbf{Contingency Planning}: 20--25\% schedule contingency and budget reserve
% \end{enumerate}

% \subsection{5.2 Vendor/Outsourced Implementation Model}

% \subsubsection{5.2.1 Model Characteristics and Applicability}

% \textbf{Ideal Bank Profile:}
% \begin{itemize}
%   \item Smaller to mid-sized banks (€10--150 billion assets)
%   \item Limited internal IT development capacity
%   \item Existing relationships with technology vendors
%   \item Focus on core banking rather than technology differentiation
%   \item Lower capital availability for major infrastructure investments
%   \item Preference for faster time-to-market
% \end{itemize}

% \textbf{Key Characteristics:}
% \begin{itemize}
%   \item Reliance on third-party vendor platforms and services
%   \item Vendor provides integration APIs, compliance frameworks, and operational support
%   \item Bank responsibility limited to configuration, testing, and distribution
%   \item Reduced internal complexity and resource requirements
%   \item Limited customization and feature differentiation capabilities
% \end{itemize}

% \subsubsection{5.2.2 Vendor Ecosystem and Service Models}

% \textbf{Vendor Categories and Examples:}

% \textbf{Pan-European Platform Providers:}
% \begin{enumerate}
%   \item \textbf{Worldline}
%   \begin{itemize}
%     \item Coverage: 19 euro area countries
%     \item Services: End-to-end Digital Euro platform, payment processing, fraud detection
%     \item Model: SaaS-based platform with integration APIs
%     \item Customers: Medium to large PSPs
%   \end{itemize}

%   \item \textbf{Nexi}
%   \begin{itemize}
%     \item Coverage: Italy, Spain, other southern European markets
%     \item Services: Card issuing, acquiring, Digital Euro integration
%     \item Model: Hosted platform with customizable components
%     \item Customers: Banks of various sizes in primary markets
%   \end{itemize}

%   \item \textbf{equensWorldline}
%   \begin{itemize}
%     \item Coverage: Selected euro area markets
%     \item Services: Payment processing, Digital Euro connectivity
%     \item Model: Outsourced processing with integration options
%     \item Customers: Smaller to medium-sized banks
%   \end{itemize}
% \end{enumerate}

% \textbf{National Champions:}
% \begin{enumerate}
%   \item \textbf{SIBS (Portugal)}
%   \begin{itemize}
%     \item Coverage: Portugal primarily
%     \item Services: Domestic payments, Digital Euro integration
%     \item Model: Monopoly provider for Portuguese domestic payments
%     \item Customers: All Portuguese banks (virtually mandatory)
%   \end{itemize}

%   \item \textbf{Redsys (Spain)}
%   \begin{itemize}
%     \item Coverage: Spain
%     \item Services: Card processing, authentication, Digital Euro integration
%     \item Model: Central provider model with mandatory participation
%     \item Customers: All Spanish PSPs
%   \end{itemize}

%   \item \textbf{CBI (Italy)}
%   \begin{itemize}
%     \item Coverage: Italy
%     \item Services: Interbank clearing, payments infrastructure
%     \item Model: Cooperative infrastructure provider
%     \item Customers: Italian banks and payment processors
%   \end{itemize}
% \end{enumerate}

% \textbf{Service Model Options:}

% \textbf{Full-Service Platform Model:}
% \begin{itemize}
%   \item Vendor provides complete Digital Euro integration solution
%   \item Bank configures platform for specific requirements
%   \item Vendor manages DESP connectivity, compliance, updates
%   \item Bank maintains customer relationship and distribution
% \end{itemize}

% \textbf{Components Outsourcing Model:}
% \begin{itemize}
%   \item Bank outsources specific components (e.g., liquidity management)
%   \item Vendor provides APIs for integration with bank's other systems
%   \item Bank maintains responsibility for overall integration
%   \item More flexibility but higher complexity
% \end{itemize}

% \textbf{API Gateway Outsourcing Model:}
% \begin{itemize}
%   \item Vendor manages DESP connectivity and API gateway
%   \item Bank develops specific services internally using vendor APIs
%   \item Hybrid approach balancing flexibility and simplicity
%   \item Requires internal development capability
% \end{itemize}

% \subsubsection{5.2.3 Vendor Selection and Evaluation Framework}

% \textbf{Critical Selection Criteria:}
% \begin{enumerate}
%   \item \textbf{Technical Capabilities}
%   \begin{itemize}
%     \item Digital Euro schema compatibility
%     \item API completeness and documentation
%     \item Performance and scalability metrics
%     \item Support for advanced features (offline, conditional payments)
%     \item Integration with bank's existing systems
%   \end{itemize}

%   \item \textbf{Financial Terms}
%   \begin{itemize}
%     \item Implementation fees
%     \item Licensing/SaaS costs
%     \item Per-transaction fees (if applicable)
%     \item Support and maintenance costs
%     \item Upgrade and enhancement costs
%   \end{itemize}

%   \item \textbf{Operational Support}
%   \begin{itemize}
%     \item 24/7 technical support
%     \item SLA guarantees and penalties
%     \item Response time commitments
%     \item Regular update cycles and security patches
%     \item Professional services for implementation
%   \end{itemize}

%   \item \textbf{Regulatory and Compliance}
%   \begin{itemize}
%     \item Regulatory approval and certifications
%     \item Compliance with Digital Euro rulebook
%     \item Data protection and GDPR compliance
%     \item Security certifications and assessments
%     \item Audit trail and reporting capabilities
%   \end{itemize}

%   \item \textbf{Strategic Fit}
%   \begin{itemize}
%     \item Vendor financial stability and roadmap
%     \item Market positioning and customer base
%     \item Innovation track record
%     \item Vertical expertise in banking/payments
%     \item Growth trajectory and future capability
%   \end{itemize}
% \end{enumerate}

% \textbf{Vendor Selection Process:}
% \begin{verbatim}
% Stage 1: Market Scan & Initial Screening (2-3 weeks)
% ├── Identify potential vendors (5-10 candidates)
% ├── Request preliminary information
% ├── Screen for basic capability fit
% └── Short-list 3-4 vendors

% Stage 2: Detailed Assessment (4-6 weeks)
% ├── Detailed capability presentations
% ├── Technical architecture reviews
% ├── Reference checks with existing customers
% ├── Financial proposal evaluation
% └── Security and compliance assessment

% Stage 3: Proof of Concept (4-8 weeks)
% ├── Technical prototype development
% ├── Integration with bank's test environment
% ├── Performance and scalability testing
% ├── Security assessment
% └── Evaluation and comparison

% Stage 4: Vendor Selection & Negotiation (2-4 weeks)
% ├── Final vendor selection
% ├── Contract negotiation
% ├── SLA definition
% ├── Support model finalization
% └── Implementation planning

% Stage 5: Implementation Planning (4-6 weeks)
% ├── Detailed project planning
% ├── Resource allocation
% ├── Timeline development
% ├── Risk assessment
% └── Go-live preparation
% \end{verbatim}

% \subsubsection{5.2.4 Implementation Timeline and Phases}

% \textbf{Typical Outsourced Implementation Timeline: 18--24 Months}

% \begin{verbatim}
% Phase 1: Platform Setup & Configuration (Months 1-4)
% ├── Platform provisioning and access
% ├── System configuration for bank requirements
% ├── Integration environment setup
% ├── Compliance rule configuration
% └── Initial testing

% Phase 2: Integration & Testing (Months 5-12)
% ├── Integration with bank core systems
% ├── API development and testing
% ├── Channel integration (mobile, web, ATM)
% ├── Regulatory testing and certification
% ├── Performance and load testing

% Phase 3: Pilot Deployment (Months 13-18)
% ├── Limited customer pilot (5,000-10,000 users)
% ├── Operational testing
% ├── Performance monitoring
% ├── User feedback collection
% └── Refinement and optimization

% Phase 4: Production Rollout (Months 19-24)
% ├── Gradual customer activation
% ├── Monitoring and support
% ├── Channel expansion (POS, e-commerce)
% ├── Feature enhancements
% └── Full production operations
% \end{verbatim}

% \subsubsection{5.2.5 Cost and Resource Implications}

% \textbf{Implementation Costs (Full Lifecycle):}

% \begin{tabular}{|p{6.6cm}|p{7.2cm}|}
% \hline
% \textbf{Cost Category} & \textbf{Typical Range} \\
% \hline
% Platform license/setup & €2M--5M \\
% \hline
% Implementation services & €1M--3M \\
% \hline
% Integration consulting & €500k--1.5M \\
% \hline
% Testing and quality assurance & €500k--1M \\
% \hline
% Training and change management & €300k--500k \\
% \hline
% Contingency (15\%) & €600k--1.5M \\
% \hline
% \textbf{Total Implementation Cost} & \textbf{€5M--13M} \\
% \hline
% \end{tabular}

% \vspace{0.75em}
% \textbf{Ongoing Costs (Annual):}
% \begin{itemize}
%   \item Platform licensing/SaaS: €1M--2M annually
%   \item Per-transaction fees (if applicable): Variable (€0.01--0.05 per transaction)
%   \item Support and maintenance: €300k--600k annually
%   \item Upgrades and enhancements: €200k--500k annually
%   \item \textbf{Total annual operating cost: €1.5M--3.5M}
% \end{itemize}

% \textbf{Resource Requirements:}
% \begin{itemize}
%   \item Project manager: 1 FTE (full implementation period)
%   \item Systems analyst: 2--3 FTEs (implementation phase only)
%   \item Business analyst: 2 FTEs (implementation phase)
%   \item Compliance officer: 0.5 FTE (ongoing)
%   \item Operations staff: 3--5 FTEs (ongoing)
%   \item \textbf{Total dedicated staff: 8--12 FTEs during implementation, 3--6 ongoing}
% \end{itemize}

% \subsubsection{5.2.6 Risk Profile and Mitigation Strategies}

% \textbf{Key Risks in Vendor/Outsourced Model:}

% \begin{tabular}{|p{4.2cm}|p{2.4cm}|p{2.6cm}|p{5.8cm}|}
% \hline
% \textbf{Risk} & \textbf{Probability} & \textbf{Impact} & \textbf{Mitigation} \\
% \hline
% Vendor lock-in & MEDIUM & MEDIUM & Clear exit terms, API documentation, data portability clauses \\
% \hline
% Vendor financial instability & LOW & CRITICAL & Financial stability review, escrow arrangements, backup vendor relationships \\
% \hline
% Service disruptions & LOW & HIGH & SLA guarantees, redundancy requirements, support escalation procedures \\
% \hline
% Limited customization & MEDIUM & MEDIUM & Flexible API design, professional services options, vendor roadmap alignment \\
% \hline
% Integration complexity & MEDIUM & MEDIUM & Clear integration specifications, proof of concept, dedicated integration support \\
% \hline
% Regulatory compliance gaps & LOW & HIGH & Vendor certifications, compliance review process, regulatory liaison \\
% \hline
% Feature limitations & MEDIUM & LOW & Roadmap alignment, feature request processes, upgrade planning \\
% \hline
% \end{tabular}

% \vspace{0.75em}
% \textbf{Mitigation Strategies:}
% \begin{enumerate}
%   \item \textbf{Vendor Diversification}: Maintain relationships with 2+ vendors, avoid complete dependency
%   \item \textbf{API Standardization}: Require vendor adherence to open standards enabling future transitions
%   \item \textbf{Escrow Arrangements}: Place vendor code and documentation in escrow for business continuity
%   \item \textbf{SLA Guarantees}: Establish clear performance and uptime SLAs with financial penalties
%   \item \textbf{Exit Clauses}: Define clear exit provisions and data transition procedures
%   \item \textbf{Regulatory Oversight}: Maintain regulatory authority oversight of vendor relationship
% \end{enumerate}

% \subsection{5.3 Hybrid Implementation Model: Balanced Approach}

% \subsubsection{5.3.1 Model Characteristics and Applicability}

% \textbf{Ideal Bank Profile:}
% \begin{itemize}
%   \item Mid-sized to large banks (€100--500 billion assets)
%   \item Moderate to advanced IT capabilities
%   \item Strategic need for differentiation in selected areas
%   \item Desire to balance cost efficiency with feature flexibility
%   \item Multi-market operations requiring selective customization
%   \item Interest in building Digital Euro as competitive advantage
% \end{itemize}

% \textbf{Key Characteristics:}
% \begin{itemize}
%   \item Outsourced core integration with vendor platform
%   \item Selective in-house development for high-value services
%   \item Leverages vendor capabilities for commodity functions
%   \item Develops proprietary value-added services
%   \item Balanced risk and resource allocation
% \end{itemize}

% \subsubsection{5.3.2 Hybrid Model Architecture}

% \textbf{Tiered Integration Approach:}
% \begin{verbatim}
% Tier 1: Outsourced Core Integration (Vendor-Managed)
% ├── Access Management
% ├── Liquidity Management
% ├── Basic Transaction Processing
% ├── Compliance Framework
% └── Standard Reporting

% Tier 2: Integrated Enhancements (Bank-Managed)
% ├── Advanced Fraud Detection
% ├── Conditional Payment Enhancements
% ├── Liquidity Forecasting
% ├── Customer Segmentation
% └── Advanced Analytics

% Tier 3: Proprietary Value-Added Services (Bank-Developed)
% ├── Merchant Loyalty Integration
% ├── Supply Chain Financing
% ├── Working Capital Solutions
% ├── Subscription/Recurring Payment Management
% └── Corporate Treasury Integration
% \end{verbatim}

% \textbf{Implementation Architecture:}
% \begin{verbatim}
% ┌──────────────────────────────────────────────────────┐
% │           Bank Customer Interfaces                    │
% │   (Mobile, Web, ATM, POS, Corporate Treasury)       │
% └────┬──────────────────────────┬───────────────────────┘
%      │                          │
%      ▼                          ▼
% ┌──────────────────────┐  ┌─────────────────────┐
% │  Bank-Developed      │  │  Vendor Platform    │
% │  Value-Added         │  │  Core Integration   │
% │  Services            │  │                     │
% │ ├─Advanced Fraud     │  │ ├─Access Mgmt       │
% │ ├─Conditional Pay    │  │ ├─Liquidity Mgmt   │
% │ ├─Cash Mgmt          │  │ ├─Transaction Mgmt │
% │ └─Treasury           │  │ └─Compliance       │
% └────────┬─────────────┘  └──────┬──────────────┘
%          │                       │
%          └───────────┬───────────┘
%                      │
%               ┌──────▼──────┐
%               │ Integration │
%               │ Layer/APIs  │
%               └──────┬──────┘
%                      │
%               ┌──────▼──────────┐
%               │  DESP (External)│
%               └─────────────────┘
% \end{verbatim}

% \subsubsection{5.3.3 Value-Added Service Examples}

% \textbf{Advanced Conditional Payments for B2B}
% \begin{itemize}
%   \item Bill pay integration with advanced scheduling
%   \item Escrow arrangements for high-value transactions
%   \item Installment payment management for equipment leasing
%   \item Dynamic discounting for early payment
%   \item Working capital optimization
% \end{itemize}

% \textbf{Technical Implementation:}
% \begin{itemize}
%   \item Build on vendor's conditional payment framework
%   \item Bank develops custom condition evaluation logic
%   \item Integration with corporate treasury systems
%   \item Sophisticated reporting and analytics
% \end{itemize}

% \textbf{Merchant Loyalty and Incentive Management}
% \begin{itemize}
%   \item Automatic loyalty point award with Digital Euro transactions
%   \item Targeted merchant promotions and cashback offers
%   \item Network effects: incentivize merchant acceptance
%   \item Consumer engagement through gamification
% \end{itemize}

% \textbf{Technical Implementation:}
% \begin{itemize}
%   \item APIs for loyalty program integration
%   \item Real-time transaction data for promotion triggering
%   \item Reconciliation with merchant accounting systems
% \end{itemize}

% \textbf{Liquidity and Cash Management Enhancements}
% \begin{itemize}
%   \item Predictive liquidity forecasting using ML
%   \item Automated funding optimization
%   \item Integration with FX and hedging strategies
%   \item Real-time liquidity dashboard for treasury teams
% \end{itemize}

% \textbf{Technical Implementation:}
% \begin{itemize}
%   \item Data pipeline from transaction systems to analytics platform
%   \item Machine learning model development
%   \item API for treasury system integration
% \end{itemize}

% \textbf{Supply Chain Financing}
% \begin{itemize}
%   \item Supplier financing on Digital Euro payments
%   \item Automated invoice discounting
%   \item Working capital optimization across supply chains
%   \item Integration with procurement systems
% \end{itemize}

% \textbf{Technical Implementation:}
% \begin{itemize}
%   \item Supply chain data integration
%   \item Financial modeling and pricing engines
%   \item Integration with supplier and buyer financial systems
% \end{itemize}

% \subsubsection{5.3.4 Development and Integration Approach}

% \textbf{Phased Implementation Strategy:}
% \begin{verbatim}
% Phase 1: Core Integration (Months 1-12)
% ├── Vendor platform setup and configuration
% ├── Integration with bank core systems
% ├── Basic compliance and reporting
% ├── Channel integration (mobile, web)
% └── Pilot with initial customer base

% Phase 2: Enhancement Development (Months 9-20)
% ├── Parallel development of value-added services
% ├── Integration design for enhancement components
% ├── Testing of enhancement features
% ├── Pilot integration with core platform
% └── Performance optimization

% Phase 3: Advanced Features (Months 18-30)
% ├── Launch first value-added service (e.g., advanced fraud)
% ├── Gather performance metrics and feedback
% ├── Develop subsequent value-added services
% ├── Market testing and refinement
% └── Feature differentiation demonstration

% Phase 4: Scaling and Optimization (Months 25-36)
% ├── Scale successful value-added services
% ├── B2B/Treasury channel expansion
% ├── Cross-sell and upsell program development
% ├── Competitive positioning and marketing
% └── Long-term roadmap development
% \end{verbatim}

% \textbf{Team Structure:}

% \begin{tabular}{|p{4.2cm}|p{5.2cm}|p{2cm}|p{2.5cm}|}
% \hline
% \textbf{Component} & \textbf{Team} & \textbf{Size} & \textbf{Reporting} \\
% \hline
% Core Integration & Vendor + Bank Integration Team & 3--5 bank staff & CIO \\
% \hline
% Advanced Fraud Development & Bank Security/Risk Team & 3--4 people & Chief Risk Officer \\
% \hline
% Conditional Payments Enhancement & Bank Product Team & 2--3 people & Head of Payments \\
% \hline
% Treasury Integration & Bank Treasury IT & 2--3 people & Treasurer \\
% \hline
% Data/Analytics & Bank Analytics Team & 2--3 people & Chief Data Officer \\
% \hline
% Overall Program & Program Manager & 1 FTE & CIO/CFO \\
% \hline
% \end{tabular}

% \subsubsection{5.3.5 Cost and Resource Implications}

% \textbf{Cost Profile (3-Year Period):}

% \begin{tabular}{|p{7cm}|p{6.8cm}|}
% \hline
% \textbf{Cost Category} & \textbf{Amount} \\
% \hline
% Vendor platform licensing \& implementation & €5M--8M \\
% \hline
% Bank development (value-added services) & €8M--15M \\
% \hline
% Integration and consulting services & €2M--4M \\
% \hline
% Testing, training, and change management & €2M--3M \\
% \hline
% Contingency (15\%) & €2.5M--4.5M \\
% \hline
% \textbf{Total Implementation Cost} & \textbf{€20M--35M} \\
% \hline
% \end{tabular}

% \vspace{0.75em}
% \textbf{Ongoing Operating Costs (Annual):}
% \begin{itemize}
%   \item Vendor licensing and support: €1.5M--2.5M
%   \item Bank development team (5--8 people): €600k--1.2M
%   \item Infrastructure and hosting: €300k--500k
%   \item \textbf{Total annual cost: €2.4M--4.2M}
% \end{itemize}

% \textbf{Resource Requirements:}
% \begin{itemize}
%   \item Implementation period: 20--30 FTEs for 18--24 months
%   \item Ongoing: 6--10 FTEs for continuous development and operations
%   \item External support: 3--5 FTEs from consulting firms for first 12 months
% \end{itemize}

% \subsubsection{5.3.6 Risk Profile and Mitigation Strategies}

% \textbf{Key Risks in Hybrid Model:}

% \begin{tabular}{|p{4.2cm}|p{2.4cm}|p{2.6cm}|p{5.8cm}|}
% \hline
% \textbf{Risk} & \textbf{Probability} & \textbf{Impact} & \textbf{Mitigation} \\
% \hline
% Integration complexity & MEDIUM & MEDIUM & Clear integration architecture, dedicated integration team \\
% \hline
% Development delays on enhancements & MEDIUM & MEDIUM & Agile methodology, experienced development leadership \\
% \hline
% Vendor/bank misalignment & MEDIUM & MEDIUM & Clear governance, steering committee, regular communication \\
% \hline
% Feature duplication and conflicts & MEDIUM & LOW & Architecture review, clear separation of concerns \\
% \hline
% Skill gaps in bank team & MEDIUM & MEDIUM & Training programs, external mentoring, phased development \\
% \hline
% Regulatory compliance complexity & LOW & MEDIUM & Compliance officer oversight, regulatory testing \\
% \hline
% \end{tabular}

% \vspace{0.75em}
% \textbf{Mitigation Strategies:}
% \begin{enumerate}
%   \item \textbf{Clear Separation of Concerns}: Well-defined boundaries between vendor platform and bank enhancements
%   \item \textbf{Integration Architecture Review}: Third-party review of integration approach
%   \item \textbf{Governance Structure}: Joint steering committee with vendor and bank leadership
%   \item \textbf{Agile Development}: Sprint-based development with regular demos and feedback
%   \item \textbf{Compliance Officer Oversight}: Dedicated compliance review at each development phase
%   \item \textbf{Performance Baseline}: Clear metrics for vendor platform and bank enhancements
% \end{enumerate}

% \bigskip\hrule\bigskip

% \section{ Implementation Models by Bank Tier: Tailored Strategies}

% \subsection{ High-Tier Banks (Large, Internationally Active)}

% \subsubsection{ Bank Profile and Strategic Context}

% \textbf{Typical Characteristics:}
% \begin{itemize}
%   \item Total assets: €300 billion to \(>\)€3 trillion
%   \item Geographic reach: Multiple countries, significant international presence
%   \item Customer base: Large retail, substantial corporate/wholesale operations
%   \item IT infrastructure: Advanced, decentralized across multiple jurisdictions
%   \item Competitive position: Market leaders with significant technical capabilities
%   \item Strategic objectives: Maintain market leadership, drive innovation, maximize shareholder value
% \end{itemize}

% \textbf{Digital Euro Strategic Imperatives:}
% \begin{enumerate}
%   \item \textbf{Market Leadership}: Be among first movers with sophisticated Digital Euro services
%   \item \textbf{Competitive Differentiation}: Leverage advanced capabilities for market advantage
%   \item \textbf{Operational Integration}: Minimize disruption to existing operations while adding new capabilities
%   \item \textbf{Global Coordination}: Manage implementation across multiple jurisdictions and banking entities
%   \item \textbf{Innovation Positioning}: Position as technology innovator, not follower
% \end{enumerate}

% \subsubsection{ Recommended Implementation Approach: In-House with Selective Partnerships}

% \textbf{Core Strategy:}
% \begin{itemize}
%   \item Develop comprehensive in-house Digital Euro platform
%   \item Utilize selective partnerships for specialized components (offline, fraud detection)
%   \item Create innovation center for next-generation features
%   \item Establish Digital Euro as competitive differentiator
% \end{itemize}

% \textbf{Detailed Implementation Architecture:}

% \textbf{Tier 1 Core Development (In-House)}
% \begin{verbatim}
% Core Platform Components:
% ├── Access Management Service
% ├── Liquidity Management Service
% ├── Transaction Management Service
% ├── Risk and Compliance Service
% ├── Offline Management Service
% └── Advanced Fraud Detection
% \end{verbatim}

% \textbf{Technical Approach:}
% \begin{itemize}
%   \item Microservices architecture enabling independent scaling and updates
%   \item Distributed systems design for resilience across geographies
%   \item Cloud-native deployment for flexibility and scalability
%   \item Event-driven architecture for real-time processing
% \end{itemize}

% \textbf{Tier 2 Channel Integration (In-House + Partnerships)}
% \begin{verbatim}
% Distribution Channels:
% ├── Mobile Banking
% │   ├── Native iOS/Android apps
% │   ├── Digital Euro wallet UI
% │   ├── Offline capability support
% │   └── Biometric authentication
% ├── Web Banking
% │   ├── Enhanced web interfaces
% │   ├── Corporate treasury portal
% │   └── Merchant acceptance tools
% ├── POS & Merchant
% │   ├── Terminal integration framework
% │   ├── Merchant onboarding
% │   └── Acceptance network development
% └── ATM Network
%     ├── NFC upgrade support
%     ├── QR code capability
%     └── Funding/defunding operations
% \end{verbatim}

% \textbf{Tier 3 Advanced Services (In-House Development)}
% \begin{verbatim}
% Proprietary Innovation Services:
% ├── Conditional Payments Engine
% │   ├── Advanced escrow capabilities
% │   ├── Supply chain financing
% │   └── Treasury payment orchestration
% ├── Loyalty & Rewards Integration
% │   ├── Automatic point allocation
% │   ├── Merchant incentive programs
% │   └── Consumer engagement features
% ├── Advanced Analytics
% │   ├── Real-time transaction analytics
% │   ├── Fraud pattern detection
% │   ├── Customer behavior insights
% │   └── Competitive intelligence
% ├── API Economy for Partners
% │   ├── Developer ecosystem
% │   ├── Third-party service integration
% │   └── Fintech partnership framework
% └── Blockchain Integration (Future)
%     ├── Smart contract compatibility
%     ├── Supply chain transparency
%     └── Advanced programmability
% \end{verbatim}

% \subsubsection{ Governance and Organizational Structure}

% \textbf{Organizational Design:}
% \begin{verbatim}
% Chief Technology Officer (CTO)
% │
% ├── Head, Digital Euro Platform
% │   ├── Platform Architecture Team (5 people)
% │   ├── Core Services Development (30 people)
% │   │   ├── Access & Onboarding (8)
% │   │   ├── Liquidity & Settlement (8)
% │   │   ├── Transaction Processing (8)
% │   │   └── Risk & Compliance (6)
% │   ├── DevOps & Infrastructure (8)
% │   ├── Security & Privacy (5)
% │   └── QA & Testing (10)
% │
% ├── Head, Digital Euro Channels
% │   ├── Mobile Banking Enhancement (12)
% │   ├── Web & Corporate (8)
% │   ├── POS & Merchant Integration (6)
% │   ├── ATM Integration (5)
% │   └── Quality Assurance (8)
% │
% └── Head, Digital Euro Innovation
%     ├── Advanced Payments Team (6)
%     ├── Analytics & Data Science (6)
%     ├── Partner Ecosystem (4)
%     └── Research & Development (4)

% Chief Risk Officer (CRO)
% │
% └── Head, Digital Euro Compliance
%     ├── Regulatory Affairs (3)
%     ├── AML/KYC Compliance (3)
%     ├── Risk Management (3)
%     └── Internal Audit (2)
% \end{verbatim}

% \textbf{Program Governance:}
% \begin{itemize}
%   \item Digital Euro Steering Committee (Executive Board, CTO, CFO, CRO, Chief Commercial Officer)
%   \item Technical Architecture Review Board (Monthly)
%   \item Regulatory \& Compliance Review (Bi-weekly)
%   \item Product \& Innovation Council (Monthly)
%   \item External Advisory Board (Quarterly) --- including ECB, industry experts, fintech partners
% \end{itemize}

% \subsubsection{ Cost and Timeline for High-Tier Banks}

% \textbf{Financial Investment (48-Month Period):}

% \begin{tabular}{|p{7cm}|p{6.8cm}|}
% \hline
% \textbf{Category} & \textbf{Amount} \\
% \hline
% Personnel (50+ FTEs @ €120k--150k avg) & €24M--30M \\
% \hline
% Infrastructure (cloud, data centers, security) & €8M--12M \\
% \hline
% Technology licenses and third-party services & €3M--5M \\
% \hline
% External consulting and specialized expertise & €4M--6M \\
% \hline
% Testing, quality assurance, security & €4M--6M \\
% \hline
% Contingency (15\%) & €5.1M--7.65M \\
% \hline
% \textbf{Total Development Cost} & \textbf{€48.1M--66.65M} \\
% \hline
% \end{tabular}

% \vspace{0.75em}
% \textbf{Ongoing Annual Operating Costs:}
% \begin{itemize}
%   \item Personnel (12--18 FTEs for maintenance/enhancement): €1.5M--2.7M
%   \item Infrastructure and hosting: €1M--2M
%   \item Third-party services and licenses: €500k--1M
%   \item \textbf{Total annual: €3M--5.7M}
% \end{itemize}

% \textbf{Timeline:}
% \begin{verbatim}
% Q1-Q4 Year 1: Architecture & Foundation
% ├── Technology selection and architecture design
% ├── DevOps infrastructure setup
% ├── Core framework development
% ├── Initial API design
% └── Development team recruitment (75% staffing)

% Q1-Q4 Year 2: Core Services Development
% ├── Access Management Service development
% ├── Liquidity Management Service development
% ├── Transaction Management Service development
% ├── DESP integration framework
% ├── Beta testing with select customers (internal)

% Q1-Q4 Year 3: Enhancement & Channel Integration
% ├── Risk & Compliance Service development
% ├── Offline Management Service development
% ├── Mobile, web, and POS channel integration
% ├── Pilot with 5,000-10,000 external users
% ├── Performance optimization and hardening

% Q1-Q4 Year 4: Advanced Features & Production
% ├── Advanced services (conditional payments, analytics)
% ├── Full channel rollout (POS, ATM, corporate)
% ├── Advanced fraud detection deployment
% ├── Production readiness and go-live preparation
% ├── Gradual customer activation and monitoring
% \end{verbatim}

% \textbf{Key Milestones:}
% \begin{itemize}
%   \item Month 6: Architecture approved, core development began
%   \item Month 12: Core platform functional, internal testing
%   \item Month 24: Beta pilot launched, 5{,}000 active users
%   \item Month 30: Production readiness assessment
%   \item Month 36: Production launch, full customer access
%   \item Month 48: Full feature deployment, ecosystem maturity
% \end{itemize}

% \subsection{6.2 Mid-Tier Banks (Regional, Moderate Complexity)}

% \subsubsection{6.2.1 Bank Profile and Strategic Context}

% \textbf{Typical Characteristics:}
% \begin{itemize}
%   \item Total assets: €50--300 billion
%   \item Geographic reach: Primary country plus selected neighboring markets
%   \item Customer base: Significant retail, regional corporate focus
%   \item IT infrastructure: Moderate maturity, some legacy systems
%   \item Competitive position: Regional leaders in key markets
%   \item Strategic objectives: Maintain competitive relevance, manage costs efficiently
% \end{itemize}

% \textbf{Digital Euro Strategic Imperatives:}
% \begin{enumerate}
%   \item \textbf{Compliance Requirement}: Meet ECB mandates without overinvestment
%   \item \textbf{Cost Efficiency}: Manage implementation costs while maintaining quality
%   \item \textbf{Time-to-Market}: Launch Digital Euro services quickly
%   \item \textbf{Operational Integration}: Minimize disruption to existing operations
%   \item \textbf{Selective Differentiation}: Focus innovation on high-value customer segments
% \end{enumerate}

% \subsubsection{6.2.2 Recommended Implementation Approach: Hybrid Model}

% \textbf{Core Strategy:}
% \begin{itemize}
%   \item Outsource core integration via vendor platform
%   \item Develop selective proprietary services for regional differentiation
%   \item Leverage vendor expertise while controlling costs
%   \item Achieve faster time-to-market than full in-house development
% \end{itemize}

% \textbf{Detailed Implementation Architecture:}

% \textbf{Tier 1: Vendor-Managed Core (60\% of effort)}
% \begin{verbatim}
% Platform Components (Vendor Responsibility):
% ├── Access Management (onboarding, KYC, wallet provisioning)
% ├── Liquidity Management (DCA operations, waterfall)
% ├── Basic Transaction Processing
% ├── Compliance Framework (standard rulebook requirements)
% └── Standard Reporting and Statement Engine
% \end{verbatim}

% \textbf{Vendor Selection Criteria:}
% \begin{itemize}
%   \item Strong presence in bank's primary geographic market
%   \item Proven track record with similar-sized banks
%   \item Comprehensive Digital Euro platform
%   \item Responsive support and customization capabilities
%   \item Reasonable pricing model aligned with bank scale
% \end{itemize}

% \textbf{Tier 2: Shared Infrastructure (25\% of effort)}
% \begin{verbatim}
% Market-Based Collaboration:
% ├── Shared Liquidity Management (cooperative funding)
% ├── Shared Fraud Detection Consortium
% ├── Shared ATM Network Operations
% ├── Shared Settlement Operations
% └── Industry Shared Testing Infrastructure
% \end{verbatim}

% \textbf{Collaboration Benefits:}
% \begin{itemize}
%   \item Reduce individual bank investment by 40--50\%
%   \item Share operational complexity across peers
%   \item Negotiate better vendor terms collectively
%   \item Common compliance and regulatory testing
%   \item Economies of scale for infrastructure
% \end{itemize}

% \textbf{Tier 3: Bank-Developed Differentiation (15\% of effort)}
% \begin{verbatim}
% Proprietary Services (Bank-Developed):
% ├── Regional Payment Integration
% │   ├── Local payment method integration
% │   ├── Regional merchant ecosystem
% │   └── Regional customer behavior analysis
% ├── SME/Corporate Offerings
% │   ├── Supply chain financing for regional suppliers
% │   ├── Working capital solutions
% │   └── Cash management tools
% ├── Enhanced Customer Experience
% │   ├── Personalized merchant offers
% │   ├── Regional promotions
% │   └── Community engagement features
% └── Data & Analytics
%     ├── Transaction analytics for customers
%     ├── Business intelligence dashboards
%     └── Competitive positioning insights
% \end{verbatim}

% \subsubsection{6.2.3 Governance and Organizational Structure}

% \textbf{Organizational Design:}
% \begin{verbatim}
% Chief Information Officer (CIO)
% │
% ├── Head, Digital Euro Program (1.0 FTE)
% │   ├── Program Manager (1.0 FTE)
% │   ├── Systems Integration Manager (1.0 FTE)
% │   ├── QA Lead (1.0 FTE)
% │   ├── Operations Manager (0.5 FTE)
% │   └── Vendor Relationship Manager (0.5 FTE)
% │
% └── Head, Digital Euro Innovation (0.5 FTE)
%     ├── Product Manager (0.5 FTE)
%     ├── Developer (2 FTEs shared with other projects)
%     └── Analyst (1 FTE shared)

% Chief Risk Officer (CRO)
% │
% └── Digital Euro Compliance Officer (1.0 FTE)
%     ├── Regulatory Liaison (0.5 FTE)
%     └── AML/KYC Manager (1.0 FTE)

% Chief Commercial Officer (CCO)
% │
% └── Digital Euro Product Manager (1.0 FTE)
%     ├── Marketing Manager (0.5 FTE shared)
%     └── Customer Success Manager (1.0 FTE)
% \end{verbatim}

% \textbf{Program Governance:}
% \begin{itemize}
%   \item Digital Euro Steering Committee (Quarterly) --- CIO, CFO, CRO, CCO
%   \item Vendor Management Review (Monthly) --- Program Manager, Vendor Representative
%   \item Compliance \& Regulatory Review (Monthly) --- Compliance Officer, Regulatory Liaison
%   \item Product \& Customer Review (Bi-monthly) --- Product Manager, Marketing, Customer Success
% \end{itemize}

% \subsubsection{6.2.4 Cost and Timeline for Mid-Tier Banks}

% \textbf{Financial Investment (30-Month Period):}

% \begin{tabular}{|p{7cm}|p{6.8cm}|}
% \hline
% \textbf{Category} & \textbf{Hybrid Model} \\
% \hline
% Vendor platform \& services & €5M--8M \\
% \hline
% Bank project management \& integration & €1.5M--2.5M \\
% \hline
% In-house development (proprietary services) & €3M--5M \\
% \hline
% Testing, training, change management & €1.5M--2.5M \\
% \hline
% Infrastructure and operational setup & €1M--1.5M \\
% \hline
% Contingency (15\%) & €1.95M--3.15M \\
% \hline
% \textbf{Total Development Cost} & \textbf{€13.95M--22.65M} \\
% \hline
% \end{tabular}

% \vspace{0.75em}
% \textbf{Ongoing Annual Operating Costs:}
% \begin{itemize}
%   \item Vendor licensing and support: €1M--1.5M
%   \item Bank operations team (3--4 FTEs): €400k--600k
%   \item Development and enhancement (part-time): €300k--500k
%   \item \textbf{Total annual: €1.7M--2.6M}
% \end{itemize}

% \textbf{Timeline:}
% \begin{verbatim}
% Months 1-4: Vendor Selection & Planning
% ├── Vendor evaluation and selection
% ├── Implementation planning
% ├── Resource recruitment
% └── Integration architecture design

% Months 5-12: Core Integration Phase
% ├── Vendor platform setup and configuration
% ├── Integration with bank core systems
% ├── Compliance and testing framework
% ├── Limited pilot (500-1,000 users)

% Months 13-18: Enhancement Development
% ├── Parallel development of proprietary services
% ├── Regional feature integration
% ├── SME/Corporate product development
% └── Enhanced pilot expansion (2,000-5,000 users)

% Months 19-24: Production Preparation
% ├── Full regulatory testing and certification
% ├── Production deployment planning
% ├── Training and documentation
% ├── Go-live readiness assessment

% Months 25-30: Production Launch & Scaling
% ├── Staged customer activation
% ├── Monitoring and support
% ├── Feature rollout to all customers
% └── Optimization and enhancement
% \end{verbatim}

% \textbf{Key Milestones:}
% \begin{itemize}
%   \item Month 3: Vendor selected and contracted
%   \item Month 6: Platform setup complete
%   \item Month 12: Pilot launched with 1{,}000 users
%   \item Month 18: Full regulatory testing complete
%   \item Month 24: Production ready
%   \item Month 30: Full customer activation
% \end{itemize}

% \subsection{6.3 Low-Tier Banks (Small, Community-Focused)}

% \subsubsection{6.3.1 Bank Profile and Strategic Context}

% \textbf{Typical Characteristics:}
% \begin{itemize}
%   \item Total assets: €5--50 billion
%   \item Geographic reach: Single country, often single region
%   \item Customer base: Retail-focused, limited corporate services
%   \item IT infrastructure: Basic systems, limited IT staff
%   \item Competitive position: Niche players in local markets
%   \item Strategic objectives: Remain compliant while managing tight budgets
% \end{itemize}

% \textbf{Digital Euro Strategic Imperatives:}
% \begin{enumerate}
%   \item \textbf{Cost Minimization}: Implement Digital Euro with minimal investment
%   \item \textbf{Regulatory Compliance}: Meet ECB requirements without differentiation
%   \item \textbf{Resource Constraints}: Manage with existing small IT team
%   \item \textbf{Time-to-Market}: Achieve acceptable timeline without overextension
%   \item \textbf{Stability}: Avoid operational disruption to core banking
% \end{enumerate}

% \subsubsection{6.3.2 Recommended Implementation Approach: Vendor/Outsourced Model}

% \textbf{Core Strategy:}
% \begin{itemize}
%   \item Engage established vendor for end-to-end Digital Euro platform
%   \item Minimize internal development and complexity
%   \item Rely on vendor expertise and support
%   \item Focus bank resources on core banking operations
% \end{itemize}

% \textbf{Detailed Implementation Architecture:}

% \textbf{Vendor Platform (90\% of services):}
% \begin{verbatim}
% Complete Vendor-Provided Services:
% ├── Access Management (full onboarding, KYC, wallet management)
% ├── Liquidity Management (DCA operations, waterfall, automation)
% ├── Transaction Processing (full transaction lifecycle)
% ├── Risk & Compliance (fraud detection, AML screening)
% ├── Channel Integration (mobile, web, ATM support)
% ├── Reporting and Reconciliation
% ├── Customer Support Framework
% └── Operational Monitoring
% \end{verbatim}

% \textbf{Vendor Selection Priorities:}
% \begin{enumerate}
%   \item Established provider with local market presence
%   \item Comprehensive, turnkey platform with minimal customization
%   \item Strong customer support and professional services
%   \item Reasonable pricing for smaller bank scale
%   \item Proven track record with similar-sized institutions
% \end{enumerate}

% \textbf{Bank-Specific Configuration (10\% effort):}
% \begin{verbatim}
% Bank Customization:
% ├── Brand integration (logo, colors, bank messaging)
% ├── Customer communication templates
% ├── Regulatory compliance documentation
% ├── Internal process documentation
% ├── Staff training materials
% └── Customer support scripts
% \end{verbatim}

% \textbf{Option 1: Single Vendor Relationship}
% \begin{itemize}
%   \item One vendor provides complete platform and support
%   \item Simplest approach with minimal complexity
%   \item Complete vendor dependency risk
%   \item Lowest total cost of ownership
%   \item Typical vendor: Temenos, SAP, Oracle
% \end{itemize}

% \textbf{Option 2: Cooperative/Consortium Model}
% \begin{itemize}
%   \item Multiple banks share vendor platform through cooperative
%   \item Common infrastructure reduces individual bank costs
%   \item Shared governance and decision-making
%   \item Better terms negotiated collectively
%   \item Examples: German Atruvia, Spanish Redsys
% \end{itemize}

% \subsubsection{6.3.3 Governance and Organizational Structure}

% \textbf{Streamlined Organization:}
% \begin{verbatim}
% Chief Information Officer (CIO)
% │
% └── Digital Euro Project Manager (1.0 FTE)
%     ├── Vendor Relationship Manager (0.5 FTE)
%     ├── Systems Administrator (0.5 FTE shared)
%     └── Compliance Liaison (0.25 FTE shared)

% Chief Risk Officer (CRO)
% │
% └── Compliance Officer (0.5 FTE shared with other compliance)
%     └── Regulatory Liaison (0.25 FTE)

% Chief Commercial Officer (CCO)
% │
% └── Product Manager (0.25 FTE shared)
% \end{verbatim}

% \textbf{Governance Structure:}
% \begin{itemize}
%   \item Steering Committee (Quarterly): CIO, CFO, CRO, CCO
%   \item Vendor Review Meeting (Monthly): Project Manager, Vendor Representative
%   \item Regulatory Check-in (Monthly): Compliance Officer
% \end{itemize}

% \subsubsection{6.3.4 Cost and Timeline for Low-Tier Banks}

% \textbf{Financial Investment (24-Month Period):}

% \begin{tabular}{|p{7cm}|p{6.8cm}|}
% \hline
% \textbf{Category} & \textbf{Amount} \\
% \hline
% Vendor platform implementation & €2.5M--4M \\
% \hline
% Project management and coordination & €400k--600k \\
% \hline
% Integration and configuration & €300k--500k \\
% \hline
% Testing and training & €300k--500k \\
% \hline
% Infrastructure (servers, security) & €300k--500k \\
% \hline
% Contingency (10\%) & €380k--610k \\
% \hline
% \textbf{Total Development Cost} & \textbf{€4.18M--6.71M} \\
% \hline
% \end{tabular}

% \vspace{0.75em}
% \textbf{Ongoing Annual Operating Costs:}
% \begin{itemize}
%   \item Vendor platform licensing: €600k--1M
%   \item Operational staff (1 FTE): €70k--100k
%   \item Support and maintenance: €100k--200k
%   \item \textbf{Total annual: €770k--1.3M}
% \end{itemize}

% \textbf{Timeline:}
% \begin{verbatim}
% Months 1-3: Vendor Selection & Planning
% ├── Vendor evaluation
% ├── Contract negotiation
% ├── Project planning
% └── Resource allocation

% Months 4-9: Implementation
% ├── Vendor platform setup
% ├── Bank configuration
% ├── Testing (vendor-supported)
% ├── Pilot with 1,000-2,000 users

% Months 10-18: Integration & Testing
% ├── Full regulatory testing
% ├── Production preparation
% ├── Staff training
% ├── Customer communication

% Months 19-24: Production Launch
% ├── Gradual customer activation
% ├── Ongoing monitoring
% ├── Vendor support
% └── Stable production operations
% \end{verbatim}

% \textbf{Key Milestones:}
% \begin{itemize}
%   \item Month 2: Vendor selected
%   \item Month 6: Platform configured and ready
%   \item Month 12: Pilot complete, ready for production
%   \item Month 18: Full regulatory compliance achieved
%   \item Month 24: All customers activated
% \end{itemize}

% \subsubsection{6.3.5 Risk Considerations and Mitigation}

% \textbf{Primary Risks for Low-Tier Banks:}

% \begin{tabular}{|p{4.2cm}|p{2.4cm}|p{2.6cm}|p{5.8cm}|}
% \hline
% \textbf{Risk} & \textbf{Probability} & \textbf{Impact} & \textbf{Mitigation} \\
% \hline
% Complete vendor dependency & MEDIUM & MEDIUM & Cooperative arrangement, exit clause documentation \\
% \hline
% Limited customization & LOW & LOW & Accept standard platform features \\
% \hline
% Operational disruption & LOW & MEDIUM & Thorough testing, vendor support, phased rollout \\
% \hline
% Data security/privacy & LOW & CRITICAL & Vendor security certifications, regular audits \\
% \hline
% Cost overruns & MEDIUM & MEDIUM & Fixed-price vendor contracts, scope control \\
% \hline
% \end{tabular}

% \vspace{0.75em}
% \textbf{Mitigation Strategies:}
% \begin{enumerate}
%   \item \textbf{Cooperative Arrangements}: Join broader consortium reducing individual bank dependency
%   \item \textbf{Clear SLAs}: Establish performance and support guarantees in vendor contracts
%   \item \textbf{Phased Approach}: Implement gradually, managing risk and cost
%   \item \textbf{Training and Support}: Ensure adequate staff training before go-live
%   \item \textbf{Regulatory Oversight}: Maintain competent authority oversight of vendor relationship
% \end{enumerate}

% \bigskip\hrule\bigskip

% \section{ Shared Infrastructure, Synergies, and Cost Mutualization}

% \subsection{ Synergy Framework and Mechanisms}

% \subsubsection{ Banking Group Synergies (IPS-Based)}

% \textbf{Institutional Protection Scheme (IPS) Definition:}
% IPS are contractual or statutory liability arrangements among banks designed to protect member institutions through:
% \begin{itemize}
%   \item Common governance structures
%   \item Shared IT platforms and infrastructure
%   \item Integrated operational procedures
%   \item Mutual financial support mechanisms
% \end{itemize}

% \textbf{Typical IPS Structures:}
% \begin{itemize}
%   \item German Savings Banks (DSGV): 350+ local savings banks
%   \item German Cooperative Banks (BVR): Multiple cooperative banking groups
%   \item Austrian Savings Banks: Consolidated under Erste Group
%   \item Spanish and Italian Cooperative Banks: Sector-specific structures
% \end{itemize}

% \textbf{Synergy Calculation Methodology:}

% For IPS banks, synergies are calculated by comparing:
% \begin{enumerate}
%   \item Stand-alone approach: Sum of individual bank costs (each bank separately implements)
%   \item Consolidated approach: Single cost for group as whole (coordinated implementation)
% \end{enumerate}

% \textbf{Example --- German Savings Banks:}
% \begin{verbatim}
% Stand-Alone Approach:
% ├── Bank 1 (€50M assets): €9M implementation cost
% ├── Bank 2 (€60M assets): €10.7M implementation cost
% ├── Bank 3 (€45M assets): €8M implementation cost
% ├── ... (347 more banks)
% └── Total: €3.5 billion (sum of all individual costs)

% Consolidated Approach:
% └── Atruvia (IT provider) develops solution once
%     ├── Platform development: €100M
%     ├── Integration framework: €50M
%     ├── Per-bank configuration: €5M × 350
%     └── Total: €2.85 billion (40-45% reduction)

% Synergy Factor: 90-95% (15-35% cost reduction)
% \end{verbatim}

% \textbf{Key Synergy Sources Within IPS:}
% \begin{enumerate}
%   \item \textbf{Platform Development Reuse}
%   \begin{itemize}
%     \item Single platform development used by all members
%     \item Eliminates redundant development effort
%     \item Shared infrastructure and operations
%     \item Example: Finanz Informatik serves German savings banks
%   \end{itemize}

%   \item \textbf{Shared Operations}
%   \begin{itemize}
%     \item Single operational team manages platform
%     \item Centralized fraud detection and compliance
%     \item Coordinated vendor management
%     \item Shared testing and quality assurance
%   \end{itemize}

%   \item \textbf{Economies of Scale}
%   \begin{itemize}
%     \item Vendor negotiations leverage combined volume
%     \item Infrastructure costs distributed across members
%     \item Joint security and compliance investments
%     \item Collective training and support programs
%   \end{itemize}
% \end{enumerate}

% \textbf{Organizational Structure for IPS Implementation:}
% \begin{verbatim}
% Central Service Provider
% ├── Platform Development Team (50-100 people)
% ├── Operations & Support (30-50 people)
% ├── Compliance & Risk (20-30 people)
% ├── Infrastructure & Security (15-25 people)
% └── Member Services (10-15 people)

% Member Bank Branch Integration
% ├── Local customer interfaces
% ├── Customer service and support
% ├── Local compliance and regulatory liaison
% └── Member-specific customization (minimal)
% \end{verbatim}

% \subsubsection{ Market Synergies (Non-IPS Banks)}

% \textbf{Market Synergy Definition:}
% Cost savings achieved when multiple independent banks within a market use shared infrastructure, common vendors, or collaborative service providers.

% \textbf{Market Synergy Drivers:}
% \begin{enumerate}
%   \item \textbf{Vendor Concentration}
%   \begin{itemize}
%     \item Small number of vendors provide services to most banks
%     \item Reduces need for custom development
%     \item Examples:
%     \begin{itemize}
%       \item Worldline (France, Benelux): 70\%+ market coverage
%       \item SIBS Multibanco (Portugal): Monopoly provider
%       \item Redsys (Spain): Centralized payment processing
%     \end{itemize}
%   \end{itemize}

%   \item \textbf{Existing Shared Infrastructure}
%   \begin{itemize}
%     \item Banks already use common clearing platforms
%     \item Shared fraud detection systems
%     \item Common compliance frameworks
%     \item Examples:
%     \begin{itemize}
%       \item CBI (Italy): Interbank clearing
%       \item STET (France): Shared clearing and settlement
%       \item Iberpay (Spain): Shared infrastructure
%     \end{itemize}
%   \end{itemize}

%   \item \textbf{Outsourcing Prevalence}
%   \begin{itemize}
%     \item High proportion of banks outsource core services
%     \item Facilitates vendor platform adoption
%     \item Reduces duplicative development
%     \item Market averages: 60--80\% outsourcing for payment services
%   \end{itemize}

%   \item \textbf{Collaboration History}
%   \begin{itemize}
%     \item Banks with successful past collaboration
%     \item Industry associations supporting joint efforts
%     \item Proven joint governance models
%     \item Examples: EPC standards, SEPA implementations
%   \end{itemize}
% \end{enumerate}

% \textbf{Market Synergy Assessment by Country:}

% \begin{tabular}{|p{3.2cm}|p{3cm}|p{7.8cm}|}
% \hline
% \textbf{Country} & \textbf{Synergy Factor} & \textbf{Key Factors} \\
% \hline
% \textbf{Germany} & 30\% & Moderate vendor concentration, established outsourcing \\
% \hline
% \textbf{France} & 30\% & Central clearing infrastructure (STET), mixed outsourcing \\
% \hline
% \textbf{Italy} & 35\% & CBI infrastructure, strong cooperative models, Nexi dominance \\
% \hline
% \textbf{Spain} & 25\% & Redsys centralization, fragmented banking structures \\
% \hline
% \textbf{Netherlands} & 30\% & Worldline dominance, coordinated payment infrastructure \\
% \hline
% \textbf{Portugal} & 40\% & SIBS monopoly, centralized structure \\
% \hline
% \textbf{Austria} & 35\% & Cooperative infrastructure (ARZ), vendor consolidation \\
% \hline
% \textbf{Belgium} & 30\% & Worldline presence, Batopin ATM cooperation \\
% \hline
% \textbf{Finland} & 40\% & High vendor consolidation, Tietoevry/Nets/Temenos \\
% \hline
% \textbf{Ireland} & 25\% & Fragmented vendor landscape, international bank presence \\
% \hline
% \end{tabular}

% \vspace{0.75em}
% \textbf{Euro Area Weighted Average Market Synergy Factor: 30\%}

% (Weighted by retail payment volumes across euro area markets)

% \subsubsection{ Cost Mutualization Opportunities}

% \textbf{Specific Mutualization Initiatives:}

% \textbf{1. Shared Testing and Certification Infrastructure}

% \textbf{Concept:}
% \begin{itemize}
%   \item Central facility for regulatory testing and device certification
%   \item Industry-standard test cases aligned with Digital Euro rulebook
%   \item Shared investment in test equipment and expertise
%   \item Common certification process recognized across euro area
% \end{itemize}

% \textbf{Participants:}
% \begin{itemize}
%   \item ECB (standards \& governance)
%   \item Banking associations (cost sharing, governance)
%   \item Major vendors (test infrastructure, expertise)
%   \item Participating banks (testing services)
% \end{itemize}

% \textbf{Cost Impact:}
% \begin{itemize}
%   \item Individual bank testing cost: €1--2M per year
%   \item Mutualized testing cost: €15--20M total (shared across 2{,}000+ banks)
%   \item Per-bank savings: €500k--1M annually
%   \item Implementation cost: €3--5M initial investment, recovered in 5--7 years
% \end{itemize}

% \textbf{2. Shared Fraud Detection and AML Services}

% \textbf{Concept:}
% \begin{itemize}
%   \item Pooled fraud detection data and AI models
%   \item Centralized transaction monitoring for AML/CFT
%   \item Shared investigation resources
%   \item Industry-standard risk scoring framework
% \end{itemize}

% \textbf{Participants:}
% \begin{itemize}
%   \item Banking associations (governance)
%   \item Specialized service providers (technology)
%   \item Banks (data sharing, funding)
% \end{itemize}

% \textbf{Cost Impact:}
% \begin{itemize}
%   \item Individual bank fraud detection: €2--5M annually
%   \item Shared service model: €1--2M annually
%   \item Savings: €1--3M per bank annually (50--60\% reduction)
%   \item Industry total: €2--6B annually across euro area
% \end{itemize}

% \textbf{3. Shared ATM Network Operations}

% \textbf{Concept:}
% \begin{itemize}
%   \item Consolidated ATM network management across regions
%   \item Shared maintenance and cash replenishment
%   \item Common ATM software and configuration
%   \item Cooperative terminal procurement
% \end{itemize}

% \textbf{Participants:}
% \begin{itemize}
%   \item Banks with ATM networks
%   \item Independent ATM deployers
%   \item Shared service providers (Geldmaat, Batopin)
% \end{itemize}

% \textbf{Cost Impact:}
% \begin{itemize}
%   \item Individual ATM operations: €5--10M annually (for 1{,}000 ATMs)
%   \item Shared operations: €2--4M annually (through consolidation)
%   \item Savings: €3--6M annually (40--50\% reduction)
%   \item Network resilience improvements
% \end{itemize}

% \textbf{4. Shared Digital Euro as a Service (DaaS) Platform}

% \textbf{Concept:}
% \begin{itemize}
%   \item Central vendor-provided platform serving multiple banks
%   \item Common API gateway and DESP connectivity
%   \item Standardized compliance and fraud detection
%   \item Banks focus on distribution and customer service
% \end{itemize}

% \textbf{Participants:}
% \begin{itemize}
%   \item Major vendor (technology provider)
%   \item Banking association (procurement, governance)
%   \item Participating banks (funding, usage)
% \end{itemize}

% \textbf{Cost Impact:}
% \begin{itemize}
%   \item Individual bank implementation: €10--20M
%   \item DaaS model: €3--5M per bank
%   \item Vendor builds once, deploys to multiple banks
%   \item Aggregate savings: €5--15B across euro area
%   \item Example vendors positioned: Worldline, equensWorldline, Sapient
% \end{itemize}

% \textbf{5. Shared Merchant Acceptance Network}

% \textbf{Concept:}
% \begin{itemize}
%   \item Coordinated merchant onboarding and acceptance
%   \item Shared merchant incentive programs
%   \item Common merchant technical support
%   \item Collaborative merchant acquisition
% \end{itemize}

% \textbf{Participants:}
% \begin{itemize}
%   \item Banks (funding, customer relationships)
%   \item Merchant associations
%   \item Payment processors
%   \item Fintech partners
% \end{itemize}

% \textbf{Cost Impact:}
% \begin{itemize}
%   \item Reduces individual bank marketing costs: 30--40\% savings
%   \item Increases merchant acceptance through scale
%   \item Collaborative advantages: better terms from terminal vendors
%   \item Industry coordination: higher acceptance rates
% \end{itemize}

% \bigskip\hrule\bigskip

% \subsection{7.2 Scenarios and Sensitivity Analysis}

% \subsubsection{7.2.1 Cost Scenarios by Implementation Model}

% \textbf{Scenario Parameters:}
% \begin{itemize}
%   \item Base case: Adjusted PwC cost estimates with moderate synergies
%   \item Low synergy case: Limited collaboration, vendor lock-in
%   \item High synergy case: Aggressive mutualization, cooperative models
% \end{itemize}

% \textbf{Total Euro Area Implementation Cost Projections:}
% \begin{verbatim}
% Base Scenario (30% market synergies, 90-95% IPS synergies):
% ├── PwC base estimates (adjusted): €7.9 billion
% ├── Synergy reduction: -€2.1 billion (26%)
% ├── Net total: €5.77 billion over 4 years
% ├── Average per bank: €3M-40M (depending on size)
% └── Timeframe: 4-year implementation period

% Low Synergy Scenario (Limited mutualization):
% ├── PwC base estimates (adjusted): €7.9 billion
% ├── Synergy reduction: -€1.6 billion (20%)
% ├── Net total: €6.3 billion over 4 years
% ├── Average per bank: €3.3M-50M
% └── Higher vendor dependency and duplication

% High Synergy Scenario (Aggressive collaboration):
% ├── PwC base estimates (adjusted): €7.9 billion
% ├── Synergy reduction: -€2.8 billion (35%)
% ├── Net total: €5.07 billion over 4 years
% ├── Average per bank: €2.7M-30M
% └── Requires coordinated implementation approach
% \end{verbatim}

% \subsubsection{7.2.2 Bank-Specific Cost Analysis by Tier and Model}

% \textbf{High-Tier Banks (€300B+ assets) --- In-House Model:}
% \begin{verbatim}
% Base Scenario: €40-60M per bank
% ├── Personnel (50-60 FTEs): €24-30M
% ├── Infrastructure: €8-12M
% ├── External services: €8-12M
% ├── Contingency: €5-8M
% └── 4-year amortization; €10-15M annually

% Sensitivity:
% ├── Low case (aggressive delivery): €30-40M
% ├── High case (comprehensive features): €60-80M
% ├── Timeline variation: ±6-12 months
% \end{verbatim}

% \textbf{Mid-Tier Banks (€100--300B assets) --- Hybrid Model:}
% \begin{verbatim}
% Base Scenario: €15-25M per bank
% ├── Vendor platform: €5-8M
% ├── Bank development: €5-10M
% ├── Integration/consulting: €3-5M
% ├── Contingency: €2-3M
% └── 30-month implementation; €6-10M annually

% Sensitivity:
% ├── Low case (vendor-dependent): €10-15M
% ├── High case (significant innovation): €25-35M
% ├── Timeline variation: ±3-6 months
% \end{verbatim}

% \textbf{Low-Tier Banks (€5--50B assets) --- Vendor Model:}
% \begin{verbatim}
% Base Scenario: €4-7M per bank
% ├── Vendor implementation: €2.5-4M
% ├── Configuration: €0.5-1M
% ├── Training and testing: €0.5-1M
% ├── Contingency: €0.5-1M
% └── 24-month implementation; €2-3M annually

% Sensitivity:
% ├── Low case (minimal customization): €3-4M
% ├── High case (extensive integration): €6-8M
% ├── Timeline variation: ±2-3 months
% \end{verbatim}

% \subsubsection{7.2.3 Cost Drivers and Sensitivity Analysis}

% \textbf{Key Cost Drivers:}
% \begin{enumerate}
%   \item \textbf{Technical Scope}
%   \begin{itemize}
%     \item Basic digital euro (payment only): Base cost
%     \item Advanced features (conditional payments, offline): +30--50\%
%     \item B2B functionality: +20--30\%
%     \item Advanced analytics/AI: +10--20\%
%   \end{itemize}

%   \item \textbf{Channel Integration}
%   \begin{itemize}
%     \item Digital channels (mobile, web): Base cost
%     \item Physical channels (ATM, POS, branch): +20--30\%
%     \item Merchant ecosystem: +10--20\%
%   \end{itemize}

%   \item \textbf{Regulatory Complexity}
%   \begin{itemize}
%     \item Basic compliance: Base cost
%     \item Enhanced AML/CFT: +10--15\%
%     \item Multiple jurisdiction handling: +15--30\%
%   \end{itemize}

%   \item \textbf{System Integration}
%   \begin{itemize}
%     \item Modern, API-first core: Base cost
%     \item Legacy system integration: +20--40\%
%     \item Multiple core systems (decentralized): +30--50\%
%   \end{itemize}

%   \item \textbf{Outsourcing Approach}
%   \begin{itemize}
%     \item Full in-house: Base cost
%     \item Hybrid (mix): -20--30\%
%     \item Full outsource: -40--50\%
%   \end{itemize}
% \end{enumerate}

% \textbf{Sensitivity Example:}
% \begin{verbatim}
% Base Case: €20M implementation cost
% ├── Scope expansion: ±10% (e.g., offline capability)
% │   └── Impact: ±€2M
% ├── Timeline extension: ±3 months
% │   └── Impact: ±€1.5M (labor cost)
% ├── Regulatory complexity: ±15%
% │   └── Impact: ±€3M
% ├── Integration challenges: ±20%
% │   └── Impact: ±€4M
% └── Worst case (all factors high): €30M (+50%)
%     Best case (all factors low): €12M (-40%)

% Risk-Adjusted Range: €12M-€30M (likely: €18-22M)
% \end{verbatim}

% \bigskip\hrule\bigskip

% \section{ Technical Blueprints and Best Practices}

% \subsection{ Architecture Blueprint for High-Tier Banks}

% \subsubsection{ Reference Implementation Architecture}

% \textbf{Core Infrastructure Stack:}
% \begin{verbatim}
% ┌─────────────────────────────────────────────────────────────┐
% │               Customer-Facing Channels                       │
% ├─────────────────────────────────────────────────────────────┤
% │ Mobile App    Web Portal    POS Integration    ATM Interface │
% └─────┬──────────────┬──────────────┬──────────────┬──────────┘
%       │              │              │              │
%       └──────────────┼──────────────┼──────────────┘
%                      │
%       ┌──────────────▼──────────────┐
%       │   API Gateway (Kong/AWS)    │
%       │ ├─ Authentication (OAuth2)  │
%       │ ├─ Rate Limiting            │
%       │ ├─ Request Routing          │
%       │ ├─ Monitoring               │
%       │ └─ API Documentation        │
%       └──────────────┬──────────────┘
%                      │
%       ┌──────────────▼──────────────────────────────────┐
%       │    Orchestration Layer (Apache Airflow)         │
%       │ ├─ Service Composition                         │
%       │ ├─ Transaction Flow Management                 │
%       │ ├─ Error Handling & Retries                    │
%       │ └─ Audit Trail                                 │
%       └──┬──────────────┬──────────┬─────────┬────────┘
%          │              │          │         │
%     ┌────▼─────┐  ┌─────▼──┐  ┌───▼──┐  ┌──▼───┐
%     │ Access   │  │Liquidity│  │Trans │  │ Risk │
%     │Management│  │Management│  │action│  │Mgmt  │
%     │Service   │  │Service  │  │Mgmt  │  │      │
%     │(Go)      │  │(Node.js)│  │(Java)│  │(ML)  │
%     └────┬─────┘  └─────┬──┘  └───┬──┘  └──┬───┘
%          │              │          │        │
%     ┌────▼──────────────▼──────────▼────────▼───┐
%     │        Data & Cache Layer                  │
%     │ ├─ PostgreSQL (transactional)              │
%     │ ├─ MongoDB (operational documents)         │
%     │ ├─ Redis (cache, session store)            │
%     │ ├─ Elasticsearch (transaction search)      │
%     │ └─ Kafka (event streaming)                 │
%     └────┬──────────────────────────────────────┘
%          │
%     ┌────▼────────────────────────────────┐
%     │   DESP Connectivity                  │
%     │ ├─ REST API Client                  │
%     │ ├─ SSL/TLS Security                 │
%     │ ├─ Request/Response Handling         │
%     │ └─ Reconciliation Engine             │
%     └────┬────────────────────────────────┘
%          │
%     ┌────▼────────────────────────────┐
%     │  DESP (External Infrastructure)  │
%     │ (Eurosystem-Managed)             │
%     └──────────────────────────────────┘
% \end{verbatim}

% \textbf{Microservices Design Specifications:}

% \textbf{1. Access Management Service (Go)}

% \textbf{Service Endpoints:}
% \begin{verbatim}
% ├── POST /users/onboard
% │   Input: KYC documents, customer data
% │   Output: User ID, wallet ID
% │   Processing:
% │   ├─ Validate KYC documents
% │   ├─ Perform identity verification
% │   ├─ Create user account
% │   ├─ Provision wallet
% │   └─ Return DEAN assignment
% ├── POST /users/offboard
% │   Input: User ID, reason
% │   Output: Confirmation, remaining balance status
% ├── POST /aliases/register
% │   Input: User ID, alias (phone/email/IBAN)
% │   Output: Alias registration status
% └── GET /users/{userId}/status
%     Output: User status, wallet balance, account limits
% \end{verbatim}

% \textbf{Data Model:}
% \begin{verbatim}
% User Record:
% ├── User ID (bank internal)
% ├── DEAN (digital euro account number)
% ├── Alias mappings (phone, email, IBAN)
% ├── KYC status and documentation
% ├── Wallet configuration
% ├── Onboarding timestamp
% ├── Last activity timestamp
% └── Account limits (holding limit, transaction limits)
% \end{verbatim}

% \textbf{Database:} PostgreSQL \\
% \textbf{Partitioning:} By user ID / geographic region \\
% \textbf{Replication:} Multi-region (3+ regions) \\
% \textbf{Backup:} Daily incremental, weekly full \\
% \textbf{Recovery RTO:} 4 hours, \textbf{RPO:} 1 hour

% \textbf{Redundancy \& Resilience:}
% \begin{itemize}
%   \item Multi-region deployment (EU central, EU south, EU north)
%   \item Database replication with automatic failover
%   \item Service restart policies on failure
%   \item Circuit breaker pattern for external calls
%   \item 99.95\% uptime SLA
% \end{itemize}

% \textbf{2. Liquidity Management Service (Node.js)}

% \textbf{Service Endpoints:}
% \begin{verbatim}
% ├── GET /dca/balance
% │   Returns: Current DCA balance, available liquidity
% ├── POST /waterfall/fund
% │   Input: User ID, amount, funding source
% │   Output: Transaction ID, confirmation
% │   Processing:
% │   ├─ Validate customer has sufficient commercial bank funds
% │   ├─ Debit commercial bank account
% │   ├─ Update DCA balance
% │   ├─ Initiate DESP waterfall transaction
% │   └─ Confirm Digital Euro receipt
% ├── POST /reverse-waterfall/withdraw
% │   Input: User ID, amount
% │   Output: Transaction ID, settlement details
% ├── GET /liquidity/forecast
% │   Output: 7-day liquidity forecast, automated funding schedule
% └── POST /liquidity/triggers/configure
%     Input: Threshold levels, automatic funding rules
% \end{verbatim}

% \textbf{Liquidity Management Algorithm:}
% \begin{verbatim}
% AutomaticWaterfallTrigger:
%   While bank is operational:
%     DigitalEuroCustomerDeposits = GetPendingDeposits()
%     AvailableLiquidity = GetDCABalance()

%     If AvailableLiquidity < MinimumReserve:
%       FundingAmount = TargetReserve - AvailableLiquidity
%       TransferFromCommercialBank(FundingAmount)
%       UpdateDCAwithNewBalance()

%     If DigitalEuroWithdrawals > ReserveThreshold:
%       ProcessReverseWaterfall(WithdrawalAmount)
%       SettleWithCommercialBankAccounts()

%     ReconcileDCAWithECBRecords()
%     UpdateLiquidityForecast()
%     Sleep(5 minutes)
% \end{verbatim}

% \textbf{3. Transaction Management Service (Java/Spring Boot)}

% \textbf{Service Endpoints:}
% \begin{verbatim}
% ├── POST /transactions/initiate
% │   Input: Payer DEAN, Payee DEAN/Alias, Amount, Type
% │   Output: Transaction ID, authorization status
% │   Processing:
% │   ├─ Validate payer balance and limits
% │   ├─ Validate payee account
% │   ├─ Score transaction for fraud (async)
% │   ├─ Request authorization (POS flow)
% │   ├─ Reserve funds
% │   ├─ Submit to DESP
% │   └─ Return transaction ID
% ├── GET /transactions/{transactionId}/status
% │   Output: Transaction status (pending, cleared, settled)
% ├── POST /transactions/{transactionId}/reconcile
% │   Input: DESP settlement confirmation
% │   Output: Reconciliation status
% └── GET /transactions/history
%     Output: Transaction list with filters (date, amount, counterparty)
% \end{verbatim}

% \textbf{Transaction State Machine:}
% \begin{verbatim}
% INITIATED
% ├─ Payer & payee verified
% ├─ Initial fraud scoring
% ├─ Funds reserved
% └─ Awaiting DESP response

% SUBMITTED_TO_DESP
% ├─ Submitted to settlement layer
% ├─ Awaiting clearing
% └─ Contingency evaluated (conditional payments)

% CLEARED
% ├─ Cleared at DESP
% ├─ Final fraud check performed
% └─ Awaiting settlement

% SETTLED
% ├─ Final settlement completed
% ├─ Funds transferred
% └─ Terminal state (cannot be modified)

% FAILED
% ├─ Transaction rejected
% ├─ Funds returned to payer
% └─ Reason recorded for audit
% \end{verbatim}

% \textbf{Performance Requirements:}
% \begin{itemize}
%   \item End-to-end latency: \(<\)3 seconds (authorization to clearing)
%   \item Settlement latency: \(<\)5 minutes average
%   \item Peak throughput: 100{,}000 transactions/second
%   \item Database query p99 latency: \(<\)100ms
%   \item API response p99 latency: \(<\)500ms
% \end{itemize}

% \textbf{4. Risk and Compliance Service (Python/FastAPI)}

% \textbf{ML-Based Fraud Detection Model:}
% \begin{verbatim}
% ├── Feature Engineering
% │   ├─ Payer transaction history (last 30 days)
% │   ├─ Payee account patterns
% │   ├─ Transaction amount vs. average
% │   ├─ Geographic velocity (transactions per day)
% │   ├─ Device fingerprinting
% │   ├─ Time-of-day patterns
% │   └─ New account flags
% ├── Model Inputs → Risk Scoring Engine
% │   ├─ Logistic regression baseline
% │   ├─ Gradient boosting for complex patterns
% │   ├─ Ensemble methods for robustness
% │   └─ Real-time model updating
% └── Risk Score Output (0-1000 scale)
%     ├─ <100: Low risk (approve)
%     ├─ 100-500: Medium risk (advanced checks)
%     ├─ 500-750: High risk (challenge/decline)
%     └─ >750: Very high risk (mandatory decline)
% \end{verbatim}

% \textbf{AML/CFT Screening:}
% \begin{verbatim}
% ├── Sanctions list checks (OFAC, EU)
% ├── Suspicious activity pattern detection
% ├── Beneficial ownership verification
% └── Jurisdiction-based restrictions
% \end{verbatim}

% \textbf{5. Offline Management Service (C++)}

% \textbf{Secure Element Provisioning:}
% \begin{verbatim}
% ├── Secure element availability check
% ├── Download secure element application
% ├── Load offline Digital Euro wallet application
% ├── Generate cryptographic keys (in secure element)
% ├── Initialize wallet with limited offline funds
% └── Return wallet ready status to user device
% \end{verbatim}

% \textbf{Offline Transaction Processing:}
% \begin{verbatim}
% User Device 1 ─────── NFC/Bluetooth ────── User Device 2
%   Offline Digital Euro Wallet              Offline Digital Euro Wallet
%   ├─ Generate transaction token            ├─ Validate transaction
%   ├─ Create zero-knowledge proof           ├─ Verify cryptographic proof
%   ├─ Sign with private key                 ├─ Update local balance
%   └─ Transmit over NFC                     └─ Store transaction record
% \end{verbatim}

% \textbf{Offline Reconciliation:}
% \begin{verbatim}
% Device Reconnection
%   └─ Upload offline transaction records to DESP
%      ├─ Verify transaction validity
%      ├─ Check for double-spending
%      ├─ Update central ledger
%      ├─ Return reconciliation status to user device
%      └─ Update local balance
% \end{verbatim}

% \textbf{Security Considerations:}
% \begin{verbatim}
% ├─ Tamper resistance: CC EAL4 or higher
% ├─ Cryptographic strength: AES-256 equivalent
% ├─ Transaction limit enforcement: Hardware-enforced
% ├─ Anomaly detection: Unusual transaction patterns
% └─ Recovery mechanism: Device loss procedures
% \end{verbatim}

% \subsubsection{ Deployment Architecture}

% \textbf{Kubernetes-Based Container Orchestration:}
% \begin{verbatim}
% Production Cluster (Multi-Region):
% │
% ├── Region 1 (EU-Central): 
% │   ├── AZ1: Master + Worker Nodes (3 replicas)
% │   ├── AZ2: Worker Nodes
% │   └── AZ3: Worker Nodes
% │
% ├── Region 2 (EU-South):
% │   ├── AZ1: Master + Worker Nodes
% │   ├── AZ2: Worker Nodes
% │   └── AZ3: Worker Nodes
% │
% └── Region 3 (EU-North):
%     ├── AZ1: Master + Worker Nodes
%     ├── AZ2: Worker Nodes
%     └── AZ3: Worker Nodes
% \end{verbatim}

% \textbf{Node Specifications (per region):}
% \begin{verbatim}
% ├── 30 worker nodes (m5.2xlarge equivalent)
% ├── 3 master nodes (c5.2xlarge equivalent)
% ├── Auto-scaling from 30-100 nodes based on load
% ├── Pod disruption budget ensuring service continuity
% └── Network policies restricting traffic between services
% \end{verbatim}

% \textbf{CI/CD Pipeline:}
% \begin{verbatim}
% Code Commit
%   └─→ GitHub webhook trigger
%       └─→ Jenkins pipeline (10-minute cycle)
%           ├─ Code checkout
%           ├─ Unit testing (50+ test suites)
%           ├─ Code analysis (SonarQube)
%           ├─ Container image build
%           ├─ Docker push to registry
%           ├─ Integration testing (dev environment)
%           ├─ Security scanning (Aqua/Twistlock)
%           ├─ Performance testing (if applicable)
%           └─ Approval gate (human review)
%               └─→ Canary deployment (5% of traffic)
%                   ├─ Monitor metrics (error rate, latency)
%                   ├─ Automated rollback if metrics degrade
%                   └─→ Full deployment (if healthy)
%                       └─→ Production monitoring & alerting
% \end{verbatim}

% \textbf{Monitoring and Observability:}

% \textbf{Metrics Collection (Prometheus):}
% \begin{verbatim}
% ├─ Application metrics
% │   ├─ Transaction throughput
% │   ├─ API latency (p50, p95, p99)
% │   ├─ Error rates by service
% │   ├─ Database query performance
% │   └─ Cache hit rates
% └─ Infrastructure metrics
%     ├─ CPU/memory utilization
%     ├─ Disk I/O
%     ├─ Network throughput
%     └─ Pod scheduling efficiency
% \end{verbatim}

% \textbf{Distributed Tracing (Jaeger):}
% \begin{itemize}
%   \item Request flow through microservices
%   \item Identify performance bottlenecks
%   \item Latency attribution by service
%   \item Error path analysis
% \end{itemize}

% \textbf{Centralized Logging (ELK Stack):}
% \begin{verbatim}
% ├─ Application logs (JSON structured format)
% ├─ Audit logs (compliance-required)
% ├─ Security event logs
% ├─ Debug logs (sampling for performance)
% └─ Long-term retention (7 years for compliance)
% \end{verbatim}

% \textbf{Alerting Framework:}
% \begin{verbatim}
% ├─ Real-time alerts (PagerDuty)
% │   ├─ Service down (error rate >1%)
% │   ├─ High latency (p99 >5 seconds)
% │   ├─ Data loss risk (failed reconciliation)
% │   └─ Security alerts (unauthorized access attempts)
% ├─ Escalation procedures
% │   ├─ L1: Automated remediation
% │   ├─ L2: On-call engineer (5-min SLA)
% │   ├─ L3: Engineering team lead (15-min SLA)
% │   └─ L4: VP Engineering (30-min SLA)
% └─ Incident response automation
%     ├─ Automatic service restart
%     ├─ Database failover
%     ├─ Traffic rerouting
%     └─ Notification to management
% \end{verbatim}

% --- BIBLIOGRAPHY ---
\newpage
\addcontentsline{toc}{section}{Bibliography}
\renewcommand\refname{Bibliography}
\printbibliography

% --- APPENDICES ---
\newpage
\appendix
\addsec{Appendix}

\end{document}
