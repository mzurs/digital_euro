\documentclass[fontsize=12pt]{scrartcl}
\usepackage{setspace}
\onehalfspacing
\usepackage{amsmath,amssymb,amsfonts,amsthm,mathtools}
\usepackage[english]{babel}
\usepackage[T1]{fontenc}
\usepackage[utf8x]{inputenc}
\usepackage{lmodern}
\usepackage{dsfont}
\usepackage{bbm}
\usepackage[round]{natbib}
\usepackage{color} 
\usepackage[defaultlines=2,all]{nowidow}
\usepackage{caption}
\usepackage[labelformat=simple]{subcaption}
\usepackage{hyperref} % Added for clickable links
\usepackage{graphicx}
\usepackage{booktabs}
\renewcommand\thesubfigure{(\alph{subfigure})}

% Set margin if needed, otherwise scrartcl default is used. 
% Standard academic reports often use geometry, but kept minimal here to match your original style.

\setlength\parindent{0pt}
\setlength{\parskip}{6pt plus 1pt minus 1pt}

\newcommand{\red}{\textcolor{red}}

\begin{document}

\begin{titlepage}
	\centering
	{\scshape\LARGE Roland Berger \par}
	\vspace{5cm}

	{\huge\bfseries Implementation Models for Banks in the Context of the Digital Euro\par}
	\vspace{3cm}
	{\Large Author: Zohaib Shaikh \par}
	\vspace{0.5 cm}
	{\large Email: zohaib10092001@gmail.co \par}
	\vspace{2cm}
	{\large Instructor: \par}
	{\large Christian Hartmann \par}
	
	\vspace{1cm}
	{\large \today\par}
\end{titlepage}

\newpage

% --- SECTION 0: ABSTRACT ---
\begin{abstract}
    Abstract:

    This research thesis examines the technical architecture, implementation pathways, and
strategic models required for banks to integrate the Digital Euro Service Platform (DESP)
into their existing infrastructure. The study synthesizes findings from the European Central
Bank's preparation phase (2023-2025), industry cost analyses, and technical specifications
to provide a comprehensive framework for understanding how different bank tiers—High-
tier (large, international), Mid-tier (regional), and Low-tier (small, community)—can adopt
the Digital Euro through In-house, Hybrid, or Outsourced implementation models.

The research demonstrates that successful Digital Euro integration depends on technical
alignment with the Rulebook Development Group standards, careful cost-benefit analysis
of implementation models, and strategic leverage of shared infrastructure and
mutualization opportunities. Key findings indicate that costs can be substantially reduced
through effective synergy mechanisms. The thesis provides technical blueprints, implementation
frameworks, and policy recommendations to guide banks through this critical transition.

\end{abstract}

\newpage

% \tableofcontents
\thispagestyle{empty}

\cleardoublepage

\setcounter{page}{1}



% --- SECTION 1: INTRODUCTION ---
\section{Introduction}
\subsection{Background and Motivation}
The Eurosystem's Digital Euro initiative represents a fundamental evolution in European
monetary infrastructure. As payment behavior shifts toward digital channels and cash
usage declines, the European Central Bank (ECB) has initiated a comprehensive project to
provide a retail central bank digital currency (CBDC) that complements physical cash while
ensuring Europe's monetary sovereignty in an increasingly digitalized economy.
The investigation phase (2021-2023) established the conceptual framework for the Digital
Euro, exploring design options and distribution models. The subsequent preparation phase
(2023-2025) focused on transforming these concepts into operational reality: developing the
Digital Euro Scheme Rulebook, selecting technology providers, conducting experimentation
through innovation platforms, and validating technical feasibility across diverse use cases
including conditional payments and offline functionality.

Europe's payment landscape remains fragmented and vulnerable to external
dependencies. Approximately two-thirds of euro area card-based transactions are
processed by non-European entities, while 13 euro area countries depend entirely on
international card schemes or mobile solutions for in-store payments. The Digital Euro
addresses this strategic vulnerability by establishing a pan-European, public digital
payment infrastructure that: preserves consumer freedom of choice in payment methods,
strengthens European financial autonomy and competitiveness,
enables seamless cross-border payments throughout the euro area,
provides a foundation for innovation in payment services, and
maintains financial inclusion across diverse user segments.

\subsection{Research Problem and Objectives}
Despite the ECB's comprehensive preparation work, significant uncertainties persist
regarding practical implementation for banks:

Primary Research Challenge: How can banks effectively integrate the Digital Euro into
their technical infrastructure while managing implementation costs, compliance
requirements, and business model adaptations?

Research Objectives:
\begin{enumerate}
    \item Technical Analysis: Examine the technical architecture of the DESP and required
back-end integration patterns for different bank categories
    \item Implementation Modeling: Evaluate three distinct implementation approaches (In-
house, Vendor/Outsourced, Hybrid) with respect to cost efficiency, scalability, and
compliance
    \item Bank Tier Stratification: Develop tier-specific implementation strategies addressing
the distinct capabilities and constraints of High-tier, Mid-tier, and Low-tier
institutions
    \item Shared Infrastructure Assessment: Analyze opportunities for cost mutualization
through shared services, collaborative platforms, and vendor consolidation
    \item Cost-Benefit Analysis: Synthesize findings from multiple cost studies and develop
realistic financial projections for different implementation scenarios
    \item Policy Implications: Formulate recommendations for banks, regulators, and the
ECB to optimize implementation outcomes.


\end{enumerate}

\subsection{Research Questions}
This thesis addresses the following core research questions:
\begin{itemize}
    \item RQ1: Technical Integration
    \begin{enumerate}
        \item How should banks map internal data models and systems to Digital Euro Service Platform requirements?
        \item What are the technical implications of different API protocols and architectural patterns (microservices vs. monolithic)?
        \item How do conditional payments and offline synchronization affect back-end design decisions?
    \end{enumerate}
    \item RQ2: Implementation Models
    \begin{enumerate}
        \item What are the comparative advantages and disadvantages of In-house, Hybrid, and
Outsourced implementation approaches?
        \item How do implementation costs, timelines, and risk profiles differ across these models?
        \item Which implementation model is optimal for each bank tier?
    \end{enumerate}
    \item RQ3: Shared Infrastructure and Mutualization
    \begin{enumerate}
        \item What cost synergies can be achieved through shared infrastructure and
collaborative vendor engagement?
        \item How do market-specific factors (vendor concentration, outsourcing prevalence,
collaboration history) influence synergy potential?
        \item What organizational and contractual arrangements facilitate effective cost
mutualization?
        \end{enumerate}
    \item RQ4: Risk and Feasibility
    \begin{enumerate}
        \item What are the primary technical, operational, and financial risks in Digital Euro
integration?
        \item How can banks effectively manage the complex interplay between mandatory
compliance and optional innovation?
        \item What governance structures and expertise requirements are necessary for successful
implementation?
    \end{enumerate}
\end{itemize}

\subsection{Research Scope and Methodology}
The research focuses on the European Central Bank’s (ECB) digital euro initiative, with particular attention to the preparation phase from 2023 to 2025 and the anticipated implementation period between 2025 and 2029. Within this temporal frame, the analysis covers banking systems across the 20 euro area countries, concentrating on retail banks with significant customer bases and differentiating them by asset size and market position. The technical perspective is limited to back-end integration, core system modifications, application programming interface (API) implementation, and compliance-related infrastructure, while front-end user interfaces and broader macroeconomic impact assessments are deliberately excluded from the scope.

Methodologically, the study adopts a mixed-methods design that integrates several complementary approaches to ensure both analytical depth and practical relevance. First, it undertakes a structured document analysis of ECB rulebooks, technical specifications, progress reports, and relevant regulatory frameworks. Second, it synthesizes existing cost studies by incorporating findings from the PwC Digital Euro Cost Study, ECB cost assessment exercises, and estimates from banking associations. Third, it includes technical modeling of functional architectures, API specifications, and data flow diagrams to capture the operational implications of different integration choices. Fourth, the research applies comparative case analysis to examine implementation approaches across diverse banking models and geographic contexts within the euro area. Fifth, it conducts a synergy assessment through quantitative evaluation of mutualization and outsourcing opportunities, relying on structured vendor and partnership analyses. Finally, it develops scenario-based cost and complexity projections across three distinct implementation model scenarios to explore potential outcome ranges under varying strategic and technical assumptions.


    % --- SECTION 2: THEORETICAL BACKGROUND ---
\section{Background on the Digital Euro: Conceptual and
Infrastructural Foundations}


\subsection{Conceptual Framework and Definitions}
\subsubsection{Digital Euro: Definition and Functional Characteristics}
The Digital Euro, or CBDC, is a digital form of central bank money—specifically, a direct
liability of the Eurosystem—available to the general public for electronic payments. It
differs fundamentally from commercial bank money, e-money, and private
cryptocurrencies:

\begin{table}[ht]
\centering
\caption{Comparison of Digital Currencies and Money Types}
\label{tab:digital-currency-comparison}
\resizebox{\textwidth}{!}{%
\begin{tabular}{lcccc}
\hline
Characteristic & Digital Euro & Commercial Bank Money & E-Money & Cryptocurrency \\
\hline
Issuer & ECB/Eurosystem & Commercial banks & E-money institutions & Decentralized/Private \\
Legal Status & Central bank liability & Bank liability & Prepaid value & Varies (often unregulated) \\
Settlement & Real-time, final & Interbank clearing & Custodian-based & Blockchain-based \\
Privacy & High (pseudonymous) & Low & Medium & Variable \\
Universal Access & Yes (legal tender) & Conditional (account holders) & Conditional & Open \\
Regulatory Oversight & Full (ECB) & Full (Banking Supervision) & Moderate & Limited \\
\hline
\end{tabular}%
}
\end{table}

The digital euro is designed to fulfil several complementary roles within the European payment ecosystem. As a store of value, users can hold digital euro balances, subject to calibrated holding limits intended to safeguard financial stability. As a medium of exchange, it supports seamless transactions in peer-to-peer contexts, at the point of sale, and in e-commerce environments, thereby integrating into everyday payment use cases. Denominated in euros and maintaining one-to-one parity with physical cash, it also functions as a unit of account, ensuring consistency with existing monetary denominations. In addition, the provision of offline payment capabilities is intended to enhance payment system resilience by enabling transactions during temporary network outages. Finally, the digital euro is conceived as a tool for promoting financial inclusion, as it would be accessible to all residents of the euro area, including those without traditional banking relationships.

\subsubsection{Key Digital Euro Ecosystem Actors}
The Digital Euro ecosystem comprises several interconnected participant categories:
\begin{itemize}
    \item Eurosystem (ECB)
    \begin{itemize}
        \item Develops and maintains the Digital Euro Service Platform (DESP)
        \item Establishes regulatory standards through the Rulebook Development Group
        \item Manages settlement and core clearing functions
        \item Ensures system resilience and cybersecurity
        \item Does not see end-user identities (privacy-preserving architecture)
    \end{itemize}
    \item Payment Service Providers (PSPs) - Banks and Non-Bank Operators
    \begin{itemize}
        \item Distribute Digital Euro services to end users
        \item Manage customer onboarding and Know-Your-Customer (KYC) compliance
        \item Perform pre-authorization and fraud prevention
        \item Handle funding and defunding operations (liquidity management)
        \item Provide customer support and dispute resolution
    \end{itemize}
    \item End Users (Natural and Legal Persons)
    \begin{itemize}
        \item Individual consumers using Digital Euro for daily transactions
        \item Merchants and businesses accepting Digital Euro payments
        \item Government entities for tax collection and benefit distribution
        \item Operators requiring conditional payment capabilities
    \end{itemize}
    \item External Service Providers
    \begin{itemize}
        \item Technology vendors (alias lookup, fraud detection, app development)
        \item Platform developers (mobile wallets, payment apps)
        \item Security and encryption service providers
        \item Payment terminal manufacturers
    \end{itemize}
\end{itemize}





\section{Literature Review: Integration of Global CBDC
Experience and Technical Standards}
\subsection{Global CBDC Implementation Experiences}
While the Digital Euro represents a unique initiative focused on retail CBDC with
sophisticated integration requirements, examining comparable CBDC projects provides
valuable insights into technical and organizational challenges.

\subsubsection{Comparative Analysis of Retail CBDC Projects}
\begin{itemize}
    \item 
e-CNY (China Digital Currency Electronic Payment):

The e-CNY (Digital Currency Electronic Payment) represents the world’s largest retail central bank digital currency (CBDC) deployment, initiated conceptually in 2014, with large-scale pilot operations starting in 2020 in selected Chinese cities and regions. Technically, it operates under a two-tier architecture in which the central bank is responsible for issuance and overall infrastructure, while commercial banks and payment service providers manage distribution, customer interfaces, and much of the operational layer. The system supports offline payments via hardware wallets and secure elements embedded in devices, and it incorporates programmable features allowing smart contract–based logic to be linked to transactions, alongside near real-time settlement capabilities in the underlying infrastructure.

For the digital euro, several key lessons can be drawn from the e-CNY experience. The two-tier model substantiates the strategy of relying on existing payment service providers for distribution, as it leverages established banking and fintech infrastructures while preserving the central bank’s control over issuance and settlement. However, the development and deployment of robust offline functionality have proven technically complex, requiring extended design, testing, and certification cycles, especially when hardware-based secure elements and diverse device environments are involved. Coordination with handset manufacturers and other device vendors has emerged as a critical challenge, given stringent security requirements and heterogeneous hardware ecosystems. Moreover, user adoption has depended heavily on strong merchant-side incentives and integration with existing retail payment infrastructures, demonstrating that technical readiness alone is insufficient without clear value propositions for merchants and consumers.

In terms of scale, e-CNY pilots have reached hundreds of millions of transactions by 2024 and have been rolled out across more than twenty pilot cities and regions, with deep integration into retail merchant acceptance networks, including large platforms, small merchants, and public service payments. This breadth of deployment underscores the importance of phased, geographically diversified pilots and close collaboration with large retailers, platforms, and local authorities when planning and implementing a retail CBDC such as the digital euro.

\item Bahamas Sand Dollar:

The Sand Dollar, launched by the Central Bank of The Bahamas in 2020 as the first live retail central bank digital currency (CBDC), was designed for a population of roughly 400,000 and explicitly targeted gaps in financial access and payment resilience in an island economy. Its architecture emphasizes digital wallet provisioning for unbanked and underbanked residents, using a mobile-first design that allows users to access and transact with Sand Dollar balances primarily via smartphones and basic mobile devices, while integrating with existing domestic payment infrastructure and financial intermediaries.

For a prospective digital euro, several design lessons emerge from the Sand Dollar experience. Embedding financial inclusion objectives from the outset, rather than treating them as secondary benefits, proves crucial in guiding wallet design, onboarding processes, and distribution partnerships. A mobile-first orientation can significantly reduce dependence on costly physical infrastructure and branch networks, especially in regions with uneven access to banking services, although it simultaneously raises the bar for robust cybersecurity, user authentication, and device-level fraud prevention mechanisms. The relatively small scale of the Bahamian deployment has enabled a more controlled, phased introduction of features and risk controls, highlighting the value of constrained pilots and iterative roll-outs even in larger currency areas. Finally, strong user authentication frameworks, combined with effective fraud detection and consumer protection measures, have been essential to building and sustaining public trust in the Sand Dollar as a secure and reliable means of payment.

\item Sweden e-Krona Pilot:

Sweden’s e‑krona pilot, conducted over an extended period from 2020 to 2024, has concentrated on exploring offline payment capabilities and understanding drivers of retail adoption in a highly digitalized payments environment. The technical design emphasizes mechanisms for executing transactions without continuous network connectivity, while ensuring interoperability with existing electronic payment systems and addressing specific edge cases, such as limited connectivity in rural areas and the needs of elderly or less digitally literate users.

Several lessons from the e‑krona pilot are particularly relevant for the digital euro. First, testing offline functionality demands comprehensive scenario coverage, including diverse device types, network conditions, and user segments, in order to validate robustness and security under real-world constraints. Second, systematic user testing with vulnerable or less digitally experienced populations is essential for identifying accessibility, usability, and support requirements that may not emerge in mainstream user groups. Third, experience in Sweden indicates that integrating merchants and their point-of-sale systems can be more complex than achieving consumer wallet adoption, reflecting the need for tailored merchant onboarding, incentives, and technical support. Finally, the multi-year duration of the pilot—spanning more than three years—has proven valuable in enabling iterative refinement of technology, legal frameworks, and operational processes, suggesting that long pilot phases can substantially improve the maturity and acceptability of a retail CBDC design.


\end{itemize}

\subsection{Technical Standards and Best Practices}
\subsubsection{Payment Industry Standards Applicable to Digital Euro}
The digital euro design builds on established payments industry standards to minimise integration complexity and maximise interoperability across the euro area ecosystem. At the messaging layer, it adopts ISO 20022 as the universal standard for the exchange of payment transaction information, using structured and machine-readable formats that align with existing SEPA payment schemes and infrastructures. The corresponding rulebooks mandate ISO 20022-compliant transaction messaging, ensuring consistent implementation across participants and facilitating reuse of current back-end processing capabilities.

The architecture is further aligned with the experience gained under the second Payment Services Directive (PSD2), particularly in the use of REST-based application programming interfaces (APIs) and related security profiles. By building on familiar integration patterns for payment service providers, this approach reduces learning curves, shortens implementation timelines, and allows institutions to leverage a significant share of their existing technical infrastructure and interface frameworks.

Within the broader scheme landscape, the digital euro is designed to complement, rather than displace, SEPA arrangements, including instant credit transfer services such as SCT Inst, while relying on similar participant structures and governance mechanisms. This orientation supports seamless account-to-account integration and enables coordinated evolution with existing payment schemes instead of creating an entirely separate ecosystem.

For front-end and acceptance-side processing, the design references open standards that already support card, QR-based, and account-based payments. These include CPACE for contactless card payments, relevant standards of the European Payments Council such as those for QR code payments (e.g. EPC024‑22) and SEPA Request‑to‑Pay, as well as nexo standards for terminal and ATM transactions and Berlin Group specifications for mobile peer‑to‑peer and open finance APIs. Collectively, these standards provide a modular toolkit for integrating the digital euro into existing consumer and merchant channels with limited incremental complexity.

\subsubsection{Cybersecurity and Privacy Standards}
Cybersecurity and privacy standards for a digital euro must align with European data protection law while ensuring strong technical safeguards against misuse and attacks. In terms of GDPR compliance, the architecture should incorporate pseudonymization of transaction data so that personal identifiers are separated from operational data wherever possible, thereby reducing the risk of direct identification in case of data breaches. Data minimization principles require that only strictly necessary personal information is collected and processed for payment execution, dispute handling, and regulatory obligations, while users retain clear and enforceable control over how their personal data is used, accessed, and retained. These legal and organisational safeguards are embedded through a privacy‑by‑design and privacy‑by‑default approach, meaning that default system settings and technical choices are configured to maximise privacy protection throughout the lifecycle of the system.

To further strengthen confidentiality and unlinkability, the design can employ privacy‑enhancing technologies that complement baseline GDPR controls. Zero‑knowledge proof mechanisms may be used, particularly in offline transaction contexts, to allow validation of transaction legitimacy or balance sufficiency without exposing full transactional or identity details. Blind signature schemes and related cryptographic obscuration techniques can help prevent intermediaries from linking specific users to individual payment events beyond what is strictly necessary. Segregated data processing pipelines, where identity data and transactional data are handled by logically or physically separated components, further limit the scope for unwarranted profiling or cross‑linking. In addition, cryptographically secure token‑based transfers can ensure that transaction instruments themselves do not leak sensitive information and can be invalidated or refreshed if compromise is suspected.

On the cybersecurity front, the digital euro ecosystem would rely on robust, certifiable security components and state‑of‑the‑art communication protections. Secure elements used in cards, phones, or hardware wallets should meet recognised evaluation benchmarks, such as Common Criteria (CC) at Evaluation Assurance Level 4 (EAL4) or higher, providing assurance regarding resistance to tampering and sophisticated attacks. Communication channels between wallets, intermediaries, and central infrastructure must be protected using modern transport‑layer cryptography, for example through TLS 1.3 or later, to safeguard data in transit against interception and modification. Cryptographic key material for the core infrastructure should be generated, stored, and used within hardware security modules, which provide strong isolation and audit capabilities for key management operations. Finally, the overall system security posture must be maintained through continuous monitoring and periodic independent security testing, including penetration tests and red‑team exercises, ensuring that newly discovered vulnerabilities are identified and remediated in a timely and transparent manner.

\subsection{Cost and Feasibility Studies: Synthesis and Analysis}
\subsubsection{Primary Cost Research}
PwC Digital Euro Cost Study (2025)

Scope: 19 participating banks across euro area (€20-1000+ bn asset range)


Key Findings:
\begin{enumerate}
    \item Average implementation cost per bank: €110 million
    \item Total euro area extrapolation: €18 billion (baseline)
    \item High scenario with offline/multiple accounts: €30 billion
    \item Technical layer dominates costs: 75 percent of total (€1.5 billion)
\end{enumerate}

Cost Distribution by Service Bundle:
\begin{table}[ht]
\centering
\caption{Average Costs and Percentages by Component}
\label{tab:costs-by-component}
\begin{tabular}{lrr}
\toprule
Component & Average Cost & Percentage \\
\midrule
Mobile/Web Frontend & €10 million & 8\% \\
ATM Infrastructure & €9 million & 7\% \\
Interfaces/APIs & €6 million & 5\% \\
POS Terminal Adaptation & €7 million & 6\% \\
Account/Liquidity Management & €8 million & 6\% \\
Branch Network Adaptation & €3 million & 2\% \\
Risk/Compliance Functions & €7 million & 6\% \\
Marketing/Customer Contracts & €12 million & 10\% \\
Operational Processes & €31 million & 25\% \\
\bottomrule
\end{tabular}
\end{table}

Key Caveats:
\begin{enumerate}
    \item Excludes multiple account functionality
    \item Based on rulebook v0.8a (evolution to v0.9 may modify costs)
    \item 46 percent of available skilled resources tied up per year for 4 years
\end{enumerate}


ECB Assessment of Digital Euro Investment Costs (October 2025)

Adjusted Baseline Costs (incorporating design adjustments):
\begin{itemize}
    \item Physical card infrastructure: -€6 million (cards use existing infrastructure)
    \item POS terminal replacement: -€7 million (natural refresh cycles, smart/soft POS adoption)
    \item ATM infrastructure: -€5.1 million (existing NFC/QR support, outsourcing to independent ATM deployers)
    \item Fee calculation component: -€2 million (handled by Eurosystem)
    \item Overall adjustment: -€20 million per bank (-16 percent)
\end{itemize}
Adjusted Average Costs by Bank Size:
\begin{itemize}
    \item Large banks (>€1 trillion assets): €152 million
    \item Large banks (€100-1000 billion): €89 million
    \item Medium banks (€30-100 billion): €24 million
    \item Small banks (<€30 billion): €8 million
\end{itemize}

Euro Area Total with Synergies:
\begin{itemize}
    \item Base scenario (30 percent market synergies, 90-98 percent IPS banking group synergies): €4.0-5.77 billion
    \item High scenario (40 percent market synergies): €5.07 billion   
    \item High scenario (40 percent market synergies): €5.07 billion
    \item Within European Commission's estimated range (€2.8-5.4 billion)

\end{itemize}

\subsection{Bank Integration Case Studies: Implementation Approaches}
\subsubsection{In-House Implementation Approach: High-Tier Banks}
\textbf{Typical Profile:}
\begin{itemize}
\item Large international banks (>€500 billion assets)
\item Advanced IT infrastructure and technical capabilities
\item Decentralized operations across multiple jurisdictions
\item Significant retail customer base requiring sophisticated features
\end{itemize}
\textbf{Integration Characteristics:}
\begin{itemize}
\item Full proprietary development of interfaces and middleware
\item Custom microservices architecture enabling feature agility
\item Integrated fraud detection and risk management systems
\item Advanced analytics for real-time transaction monitoring
\item Dedicated Digital Euro business units with specialized teams
\end{itemize}
\textbf{Cost Implications:}
\begin{itemize}
\item Higher upfront development costs (€150-200 million range)
\item Internal resource allocation: 50-60% of senior IT staff for 3-4 years
\item Lower long-term operating costs through proprietary optimization
\item Ability to extract competitive advantages through feature differentiation
\end{itemize}
\textbf{Risk Profile:}
\begin{itemize}
\item Significant execution risk: large, complex technical programs prone to delays
\item Resource scarcity: diverts talent from other innovation initiatives
\item Maintenance burden: responsibility for entire integration stack
\item Regulatory compliance: direct accountability for all security requirements
\end{itemize}
\subsubsection{Vendor/Outsourced Approach: Low-Tier and Mid-Tier Banks}
\textbf{Typical Profile:}
\begin{itemize}
\item Smaller regional or community banks (€10-100 billion assets)
\item Limited IT development capacity
\item Reliance on third-party service providers for core systems
\item Focus on traditional banking relationships and local markets
\end{itemize}
\textbf{Integration Characteristics:}
\begin{itemize}
\item Engagement with established vendors providing Digital Euro platforms
\item Minimal in-house development; integration focused
\item Reliance on vendor-provided compliance and fraud detection
\item Limited customization; acceptance of standard feature sets
\item Licensing or SaaS-based engagement models
\end{itemize}
\textbf{Vendor Ecosystem:}
\begin{itemize}
\item Pan-European providers: Worldline, Nexi, Temenos
\item National champions: SIBS (Portugal), Redsys (Spain), CBI (Italy)
\item Specialized players: equensWorldline, Sapient, Almaviva
\item Cooperative bank platforms: Atruvia (Germany), Argenta (Austria)
\end{itemize}
\textbf{Cost Implications:}
\begin{itemize}
\item Lower upfront development costs (€20-50 million range)
\item Vendor licensing/SaaS fees (ongoing operational costs)
\item Reduced internal resource burden (10-20% of IT staff)
\item Shared infrastructure costs distributed across multiple users
\item Reduced synergy potential: limited vendor selection creates lock-in
\end{itemize}
\textbf{Risk Profile:}
\begin{itemize}
\item Vendor dependency: migration costs if vendor relationship changes
\item Feature limitations: constrained to vendor-provided capabilities
\item Vendor stability: operational disruption risk if vendor fails
\item Reduced competitive differentiation: identical feature sets across multiple banks
\end{itemize}
\subsubsection{Hybrid Approach: Mid-Tier Banks with Strategic Positioning}
\textbf{Typical Profile:}
\begin{itemize}
\item Mid-sized banks seeking balanced efficiency and differentiation (€50-300 billion assets)
\item Partial internal IT capabilities with selective outsourcing
\item Strategic focus on specific value-added services
\item Interest in differentiated customer offerings while managing costs
\end{itemize}
\textbf{Integration Characteristics:}
\begin{itemize}
\item Outsourced core integration through established vendors
\item In-house development of proprietary value-added services
\item Custom integration of existing core banking systems
\item Selective build vs. buy decisions based on competitive advantage potential
\item Collaborative engagement with peer institutions for shared infrastructure
\end{itemize}
\textbf{Value-Added Service Examples:}
\begin{itemize}
\item Enhanced conditional payment capabilities for B2B use cases
\item Loyalty program integration and merchant incentive structures
\item Supply chain payment solutions and working capital optimization
\item Cash management and liquidity forecasting
\item Advanced fraud prevention and financial crime detection
\end{itemize}
\textbf{Cost Implications:}
\begin{itemize}
\item Moderate upfront costs (€60-120 million range)
\item Blended vendor licensing and internal development
\item Significant resource allocation (30-40% of IT staff)
\item Phased implementation: core integration via vendor, enhancements over time
\item Medium-term savings through selective internalization of high-value functions
\end{itemize}
\textbf{Risk Profile:}
\begin{itemize}
\item Balanced approach: reduced vendor dependency while managing development complexity
\item Technology integration challenges: connecting vendor platform with proprietary systems
\item Governance complexity: managing internal development alongside vendor relationship
\item Organizational alignment: requires clear business/technical strategy coordination
\end{itemize}



\section{Technical Architecture of DESP and Bank Back-End Integration}

\subsection{DESP Architecture Overview and Core Components}

\subsubsection{Structural Design Principles}
The DESP embodies several fundamental architectural principles that shape integration requirements for banks:
\begin{itemize}
  \item \textbf{Distributed, Segregated Architecture}
    \begin{itemize}
      \item No single point of failure: components distributed across multiple providers and regions
      \item Data segregation: user identities separate from transaction data
      \item Processing segregation: distributed across multiple DESP components
      \item Enables both resilience and privacy protection
    \end{itemize}
  \item \textbf{Stateful Transaction Processing}
    \begin{itemize}
      \item Server maintains transaction state throughout processing lifecycle
      \item Reduces complexity vs.\ stateless approaches
      \item Enables faster processing and automatic recovery from failures
      \item Simplifies bank back-end integration requirements
    \end{itemize}
  \item \textbf{Two-Tier Settlement}
    \begin{itemize}
      \item Settlement layer (Eurosystem): maintains authoritative ledger, executes final transfers
      \item Conditionality layer (market participants): implements conditional payment logic
      \item Enables flexibility for innovation while ensuring settlement certainty
    \end{itemize}
  \item \textbf{REST API Standardization}
    \begin{itemize}
      \item Synchronous REST interfaces between PSPs and DESP
      \item Familiar to PSPs from PSD2 implementation experience
      \item Enables real-time processing at scale
      \item Reduces implementation complexity vs.\ proprietary protocols
    \end{itemize}
\end{itemize}

\subsubsection{Core DESP Services and Functions}

\paragraph{Access Management Service}
\begin{itemize}
  \item User onboarding and provisioning workflows
  \item Wallet creation and activation
  \item Alias management and resolution
  \item Authentication credential setup
  \item Device provisioning for offline capability
  \item Waterfall account configuration
\end{itemize}
\textbf{Integration Requirement:} Bank systems must capture user identity information, perform KYC/AML verification, and transmit verification status to DESP via standardized APIs.

\paragraph{Liquidity Management Service}
\begin{itemize}
  \item DCA account management and monitoring
  \item Waterfall funding mechanism: PSP DCA $\rightarrow$ user wallets
  \item Reverse waterfall: linked commercial bank account funding
  \item Automatic liquidity replenishment triggers
  \item Settlement lag management and collateral requirements
\end{itemize}
\textbf{Integration Requirement:} Bank treasury systems must interface with DESP liquidity management, supporting real-time liquidity monitoring, automated funding triggers, and cash position management.

\paragraph{Transaction Management Service}
\begin{itemize}
  \item Payment instruction processing
  \item Multi-channel support: POS, e-commerce, P2P, offline
  \item Pre-authorization and fraud verification
  \item Transaction state management
  \item Clearing and settlement coordination
  \item Transaction history and reporting
\end{itemize}
\textbf{Integration Requirement:} Bank authorization, switching, and clearing systems must integrate with DESP transaction management, supporting real-time authorization, clearing coordination, and comprehensive audit trails.

\paragraph{Offline Service}
\begin{itemize}
  \item Secure element provisioning and management
  \item Offline wallet creation and fund loading
  \item Device-to-device transaction processing
  \item Offline transaction recording and token management
  \item Automatic reconciliation upon reconnection
  \item Recovery mechanisms for device loss or duplication
\end{itemize}
\textbf{Integration Requirement:} Bank mobile banking and device management systems must support secure element provisioning, offline wallet management, and integration with device manufacturers' secure element APIs.

\paragraph{Risk and Compliance Service}
\begin{itemize}
  \item Fraud detection and prevention
  \item Risk scoring and transaction monitoring
  \item AML/CFT compliance verification
  \item Dispute detection and flagging
  \item Pattern analysis and behavioral monitoring
  \item Regulatory reporting support
\end{itemize}
\textbf{Integration Requirement:} Bank compliance, fraud detection, and risk management systems must ingest DESP risk signals, provide real-time transaction scoring, and execute dispute management procedures.

\subsection{Bank Back-End System Integration Pathways}

\subsubsection{Core System Integration Architecture}
Banks must integrate the DESP with existing back-end systems across multiple dimensions:

\paragraph{Core Banking System Integration}
\begin{itemize}
  \item Account master data synchronization
  \item Customer KYC/AML profile integration
  \item General ledger and accounting records
  \item Customer statement and reporting
  \item Balance management and limit enforcement
  \item Interest and fee calculation
\end{itemize}

\paragraph{Middleware and Integration Layer}
\begin{itemize}
  \item API gateway for DESP connectivity
  \item Message queue systems for asynchronous processing
  \item Data transformation and mapping services
  \item Orchestration engines for multi-step workflows
  \item Event processing and notification systems
  \item Caching layers for performance optimization
\end{itemize}

\paragraph{Front-End Distribution Channels}
\begin{itemize}
  \item Mobile banking application integration
  \item Web portal modifications
  \item ATM network integration
  \item Branch banking system connections
  \item POS terminal ecosystem
  \item Merchant and customer communication channels
\end{itemize}

\paragraph{Back-Office and Operational Systems}
\begin{itemize}
  \item Treasury and liquidity management
  \item Compliance and AML screening
  \item Fraud detection and prevention systems
  \item Dispute management and resolution
  \item Customer service and support systems
  \item Financial reporting and regulatory submission
\end{itemize}

\subsubsection{Data Model Mapping and Transformation}
The DESP operates with specific data models that banks must map to internal representations:

\paragraph{Digital Euro Account Number (DEAN)}
\begin{itemize}
  \item Unique identifier for each Digital Euro account
  \item Assigned by DESP upon account creation
  \item Distinguished from user identity (which remains with PSP)
  \item Used for transaction routing and settlement
\end{itemize}
\textbf{Bank Integration Requirement:}
\begin{itemize}
  \item Customer ID (Bank Internal) $\rightarrow$ [Mapping] $\rightarrow$ DEAN (DESP)
  \item Maintained in bank customer reference file
  \item Used for all Digital Euro transactions
  \item Updated during wallet provisioning/deprovisioning
\end{itemize}

\paragraph{Alias-to-DEAN Mapping}
\begin{itemize}
  \item Users can register aliases: phone number, email, IBAN
  \item Alias Lookup Service (DESP component) maintains alias registry
  \item Banks responsible for alias validation and user consent management
  \item Requires integration with identity verification systems
\end{itemize}
\textbf{Bank Integration Requirement:}
\begin{enumerate}
  \item Alias Registration Request
  \item Validate user ownership of alias
  \item Submit to DESP Alias Lookup Service
  \item Maintain local mapping for rapid resolution
  \item Support alias-based payment initiation
\end{enumerate}

\paragraph{Transaction Message Formats}
\begin{itemize}
  \item ISO 20022-compliant transaction messages
  \item Structured, machine-readable formats
  \item Includes transaction type, amount, payer/payee identification, conditionality flags
  \item Supports various use cases: P2P, POS, e-commerce, conditional
\end{itemize}
\textbf{Bank Integration Requirement:}
\begin{enumerate}
  \item Transaction Initiation (Bank Format)
  \item Transform to ISO 20022 format
  \item Submit to DESP via REST API
  \item Parse response and update local systems
  \item Provide confirmation to payer/payee
\end{enumerate}

\paragraph{Pseudonymization and Privacy Safeguards}
\begin{itemize}
  \item Banks transmit transactions using DEAN (not customer name or identity)
  \item DESP cannot link transactions to individuals
  \item Enables regulatory oversight without privacy intrusion
  \item Requires careful data separation in bank systems
\end{itemize}
\textbf{Bank Integration Requirement:}
\begin{itemize}
  \item Customer Identity (Bank Secret) $\neq$ DEAN (DESP Visible)
  \item Transaction Processing Uses DEAN Only
  \item Maintains user privacy to ECB/Eurosystem
  \item Enables compliance reporting without identity linkage
\end{itemize}

\subsubsection{Liquidity Management Integration: DCA Operations}

\paragraph{DCA Account Structure}
\begin{itemize}
  \item Individual DCA held by each PSP at respective NCB
  \item Functions as liquidity reserve for Digital Euro distribution
  \item Reconciled daily during DESP settlement processes
  \item Subject to reserve requirement calculations (similar to other central bank deposits)
\end{itemize}

\paragraph{Waterfall Funding Mechanism}
\textbf{Typical Waterfall Sequence:}
\begin{enumerate}
  \item Customer initiates Digital Euro purchase (DESP wallet funding)
  \item Bank validates customer has sufficient commercial bank funds
  \item Bank debits customer commercial bank account
  \item Bank credits own DCA at NCB
  \item DESP transfers Digital Euro from central reserve to customer DEAN account
  \item Bank records transaction in both commercial and Digital Euro accounting
\end{enumerate}
\textbf{Integration Requirements:}
\begin{itemize}
  \item Real-time visibility of DCA balance
  \item Automated funding triggers based on Digital Euro demand
  \item Integration with automated clearing house (ACH) systems
  \item Reserve calculation including Digital Euro distribution
  \item Daily reconciliation with NCB settlement records
\end{itemize}

\paragraph{Reverse Waterfall (Customer-Initiated Withdrawal)}
\textbf{Reverse Waterfall Sequence:}
\begin{enumerate}
  \item Customer initiates Digital Euro conversion to commercial bank account
  \item DESP debits customer DEAN account
  \item Bank receives Digital Euro credit to DCA
  \item Bank credits customer commercial bank account
  \item Bank reconciles DCA with NCB records
  \item Cash settlement through standard central bank procedures
\end{enumerate}
\textbf{Integration Requirements:}
\begin{itemize}
  \item Bi-directional funding capability
  \item Automated clearing of reverse waterfall requests
  \item Integration with settlement systems
  \item Compliance with holding limit enforcement (prevents excessive conversion)
  \item Operational risk management (fraud, duplicate requests)
\end{itemize}

\subsubsection{Multi-Channel Integration: Enabling Diverse Payment Methods}
Banks must integrate Digital Euro capabilities across multiple customer interaction channels:

\paragraph{POS (Point-of-Sale) Integration}
\begin{itemize}
  \item Terminal support for NFC, QR code, and link-based payments
  \item Real-time authorization with fraud detection
  \item Immediate transaction settlement confirmation
  \item Merchant confirmation and receipt generation
  \item Integration with existing merchant acquiring infrastructure
\end{itemize}
\textbf{Integration Complexity: HIGH}
\begin{itemize}
  \item Requires terminal vendor coordination
  \item Hardware upgrades for older terminal types
  \item Software updates and certification
  \item Network redundancy for resilience
  \item Merchant training and support
\end{itemize}

\paragraph{E-Commerce Integration}
\begin{itemize}
  \item Payment page modifications for Digital Euro option
  \item DEAN or alias-based payment authorization
  \item M-commerce support (mobile app with redirect flows)
  \item Pay-by-link capabilities (merchant generates payment link)
  \item Session management and transaction linking
  \item Tokenization for recurring payments
\end{itemize}
\textbf{Integration Complexity: MEDIUM}
\begin{itemize}
  \item API-based integration (familiar to banks)
  \item Minimal infrastructure changes
  \item Standard payment gateway modifications
\end{itemize}

\paragraph{P2P (Peer-to-Peer) Integration}
\begin{itemize}
  \item Mobile banking app modifications
  \item DEAN and alias-based payment initiation
  \item QR code generation and scanning
  \item Contact-based recipient identification
  \item Transaction confirmation and receipt
\end{itemize}
\textbf{Integration Complexity: LOW}
\begin{itemize}
  \item Mobile app feature additions
  \item Minimal back-end changes
  \item Leverages existing P2P infrastructure
  \item Natural extension of mobile banking
\end{itemize}

\paragraph{ATM Integration}
\begin{itemize}
  \item Funding and defunding capability
  \item QR code and NFC support
  \item Real-time connection to liquidity management
  \item Security and fraud prevention
  \item Older ATM compatibility (QR code vs.\ hardware NFC)
\end{itemize}
\textbf{Integration Complexity: MEDIUM--HIGH}
\begin{itemize}
  \item ATM network coordination challenges
  \item Hardware replacement for NFC support
  \item Network resilience requirements
  \item Cash handling reconciliation
\end{itemize}



\section{Implementation Models: Technical and Strategic Analysis}

\subsection{In-House Implementation Model: Architecture and Requirements}

\subsubsection{Model Characteristics and Applicability}

\paragraph{Ideal Bank Profile:}
\begin{itemize}
  \item Large, internationally active banks (typically \(>\)\texteuro 300 billion assets)
  \item Advanced IT infrastructure and development capabilities
  \item Significant technical staff and specialized expertise
  \item Decentralized operations requiring customization
  \item Strategic need for competitive differentiation
  \item Sufficient capital for substantial upfront investment
\end{itemize}

\paragraph{Key Characteristics:}
\begin{itemize}
  \item Full proprietary development and maintenance responsibility
  \item Complete control over feature development and timelines
  \item Direct accountability for security and compliance
  \item Maximum flexibility for customization and innovation
  \item Highest development and operational complexity
\end{itemize}

\subsubsection{Technical Architecture for In-House Implementation}

\paragraph{Microservices Architecture Approach}

% \begin{verbatim}
% ┌─────────────────────────────────────────────────────┐
% │              Bank Customer Interfaces                │
% │    (Mobile App, Web, ATM, POS, Branch Systems)     │
% └────────┬────────────────────────┬──────────────────┘
%          │                        │
%     ┌────▼────────────────────────▼────┐
%     │   API Gateway & Orchestration     │
%     │  (REST, Authentication, Routing)  │
%     └────┬────────┬──────────┬─────┬───┘
%          │        │          │     │
%     ┌────▼──┐ ┌──▼────┐ ┌──▼──┐ ┌▼──────┐
%     │Access │ │Liquidity│Transaction│Risk & │
%     │Mgmt   │ │Mgmt   │Mgmt    │Compliance│
%     │Service│ │Service│Service │Service  │
%     └───┬───┘ └──┬────┘ └───┬──┘ └────┬──┘
%         │        │          │         │
%     ┌───▼─────────▼──────────▼─────────▼───┐
%     │   DESP Connectivity & Integration    │
%     │   (REST API Client, Message Queues)  │
%     └───┬──────────────────────────────────┘
%         │
%     ┌───▼──────────────────────────┐
%     │  DESP (External)              │
%     │ (Settlement, Liquidity, etc)  │
%     └──────────────────────────────┘
% \end{verbatim}

\paragraph{Microservices Components:}
\begin{enumerate}
  \item \textbf{Access Management Service} --- Functions: onboarding, wallet provisioning, alias management. Tech: Java/Spring Boot, PostgreSQL. API: user creation, verification, wallet activation. Dependencies: core banking, KYC/AML.
  \item \textbf{Liquidity Management Service} --- Functions: DCA monitoring, waterfall operations, funding triggers. Tech: Node.js, MongoDB, Redis. API: DCA balance, waterfall request, reverse waterfall. Dependencies: treasury, settlement, DESP.
  \item \textbf{Transaction Management Service} --- Functions: transaction processing, clearing, settlement coordination. Tech: Java, Kafka, PostgreSQL. API: payment instruction, authorization, clearing. Dependencies: auth, clearing houses, fraud detection.
  \item \textbf{Risk and Compliance Service} --- Functions: fraud detection, AML screening, risk scoring. Tech: Python (AI/ML), Apache Spark, feature store. API: risk scoring, flagging, compliance reporting.
  \item \textbf{Offline Management Service} --- Functions: secure element provisioning, offline wallet management. Tech: C++, HSM integration. API: secure element provisioning, offline wallet creation.
\end{enumerate}

\paragraph{Integration with Existing Systems:}
% \begin{verbatim}
% Core Banking System
%     ├── Account Master Data
%     ├── Customer Records
%     ├── General Ledger
%     └── Statement Engine
%          │
%          ▼
% Digital Euro Microservices
%     ├── Access Management
%     ├── Liquidity Management
%     ├── Transaction Management
%     ├── Risk/Compliance
%     └── Offline Management
%          │
%          ▼
% External Systems
%     ├── Authorization Systems
%     ├── Settlement Systems
%     ├── Fraud Detection (Third-Party)
%     ├── Treasury Systems
%     └── DESP APIs
% \end{verbatim}

\subsubsection{Development and Deployment Considerations}

\paragraph{Team Structure and Expertise Requirements:}

\begin{center}
\begin{tabular}{@{}p{4.5cm}ccp{6.5cm}@{}}
\toprule
Role & Required FTEs & & Key Expertise \\
\midrule
Platform Architects & 2--3 && Cloud architecture, microservices, system design \\
Backend Developers & 15--20 && Java, Python, API development, database design \\
DevOps Engineers & 5--8 && Kubernetes, CI/CD, infra automation, monitoring \\
QA/Testing Engineers & 8--12 && Automated/performance/security testing \\
Security Engineers & 3--5 && Cryptography, HSM, threat modeling \\
Product Managers & 2--3 && Digital Euro requirements, roadmap \\
Project Manager & 1 && Program coordination, stakeholder mgmt \\
\midrule
\textbf{Total} & \textbf{36--52} && \textbf{Full-time commitment for 3--4 years} \\
\bottomrule
\end{tabular}
\end{center}

\paragraph{Development Timeline (Phased):}
\begin{description}[leftmargin=1.5cm]
  \item[Phase 1: Foundation (Months 1--6)] Architecture design, stack finalization, core API, DB schema, DevOps.
  \item[Phase 2: Core Services (Months 7--18)] Access, Liquidity, Transaction, Risk frameworks, DESP integration.
  \item[Phase 3: Enhancement \& Integration (Months 19--30)] Offline, conditional payments, full channel integration, performance, security.
  \item[Phase 4: Testing \& Readiness (Months 31--36)] Unit/integration/load testing, pen tests, regulatory validation, runbooks.
  \item[Phase 5: Pilot \& Production (Months 37--48)] Pilot deployments, monitoring/tuning, full rollout.
\end{description}

\subsubsection{Cost and Resource Implications}

\paragraph{Development Costs (4-Year Period):}

\begin{center}
\begin{tabular}{@{}lrr@{}}
\toprule
Cost Category & Low Estimate & High Estimate \\
\midrule
Personnel (36--52 FTEs @ \texteuro 100--150k avg) & \texteuro 14.4M & \texteuro 31.2M \\
Infrastructure (cloud, HSM, hardware) & \texteuro 2M & \texteuro 5M \\
Third-party software/licenses & \texteuro 1M & \texteuro 3M \\
Training and professional development & \texteuro 0.5M & \texteuro 1.5M \\
Testing and QA & \texteuro 2M & \texteuro 4M \\
Contingency (10--15\%) & \texteuro 2M & \texteuro 4.5M \\
\midrule
\textbf{Total} & \textbf{\texteuro 21.9M} & \textbf{\texteuro 49.2M} \\
\bottomrule
\end{tabular}
\end{center}

\paragraph{Operational Costs (Post-Launch):}
\begin{itemize}
  \item Infrastructure and hosting: \texteuro 0.5M--1M annually
  \item Personnel maintenance team: 8--12 FTEs (\texteuro 1--1.8M annually)
  \item Vendor licenses and support: \texteuro 0.3--0.5M annually
  \item \textbf{Total annual operating costs: \texteuro 1.8--3.3M}
\end{itemize}

\paragraph{Capital Requirements:}
\begin{itemize}
  \item Upfront development: \texteuro 20--50M
  \item Hardware and infrastructure: \texteuro 5--10M
  \item Working capital and contingency: \texteuro 5--10M
  \item \textbf{Total capital requirement: \texteuro 30--70M}
\end{itemize}

\subsubsection{Risk Profile and Mitigation Strategies}

% \paragraph{Key Risks (In-House):}
% \begin{longtable}{@{}p{5cm}p{2.5cm}p{2.5cm}p{5cm}@{}}
% \toprule
% Risk & Probability & Impact & Mitigation \\
% \midrule
% Development delays and overruns & HIGH & HIGH & Agile methodology, external architecture review, contingency timeline \\
% Skills gaps in emerging technologies & MEDIUM & HIGH & External consulting, vendor partnerships, training programs \\
% Integration complexity with legacy systems & HIGH & MEDIUM & Strangler pattern, phased integration, dedicated team \\
% Security vulnerabilities & MEDIUM & CRITICAL & Security review process, bug bounty, 3rd-party testing \\
% Regulatory compliance gaps & MEDIUM & HIGH & Compliance officer engagement, regulatory checkpoints \\
% Operational readiness issues & MEDIUM & MEDIUM & Pilot phase, comprehensive testing, runbooks \\
% Resource availability & HIGH & MEDIUM & Dedicated hiring, external contractors, phased team building \\
% \bottomrule
% \end{longtable}

\paragraph{Mitigation Strategies (selected):}
\begin{enumerate}
  \item External architect review (quarterly).
  \item Vendor partnerships for specialised components.
  \item Pilot program before full rollout.
  \item Third-party security assessments at milestones.
  \item Dedicated regulatory liaison.
  \item 20--25\% schedule contingency and budget reserve.
\end{enumerate}

\subsection{Vendor/Outsourced Implementation Model}

\subsubsection{Model Characteristics and Applicability}

\paragraph{Ideal Bank Profile:}
\begin{itemize}
  \item Smaller to mid-sized banks (\texteuro 10--150 billion assets)
  \item Limited internal IT development capacity
  \item Existing relationships with technology vendors
  \item Focus on core banking rather than technology differentiation
  \item Lower capital availability; preference for faster time-to-market
\end{itemize}

\paragraph{Key Characteristics:}
\begin{itemize}
  \item Reliance on third-party vendor platforms and services
  \item Vendor provides integration APIs, compliance frameworks, ops support
  \item Bank responsibility limited to configuration, testing, distribution
  \item Reduced internal complexity; limited customization
\end{itemize}

\subsubsection{Vendor Ecosystem and Service Models}

\paragraph{Vendor Categories (examples):}
\begin{itemize}
  \item \textbf{Pan-European Platform Providers:} Worldline, Nexi, equensWorldline --- SaaS/hosted platforms, payment processing and DESP connectivity.
  \item \textbf{National Champions:} SIBS (Portugal), Redsys (Spain), CBI (Italy) --- domestic providers with strong local coverage.
\end{itemize}

\paragraph{Service Model Options:}
\begin{description}
  \item[Full-Service Platform Model] Vendor provides complete solution; bank configures and distributes.
  \item[Components Outsourcing Model] Bank outsources specific components (e.g., liquidity).
  \item[API Gateway Outsourcing Model] Vendor manages DESP connectivity; bank develops services on top.
\end{description}

\subsubsection{Vendor Selection and Evaluation Framework}

\paragraph{Critical Selection Criteria:}
\begin{enumerate}
  \item Technical capabilities: schema compatibility, API completeness, performance, offline/conditional payment support.
  \item Financial terms: implementation, licensing, per-transaction fees.
  \item Operational support: 24/7 support, SLAs, patch cycles.
  \item Regulatory and compliance: certifications, GDPR, audit trails.
  \item Strategic fit: vendor stability, roadmap, vertical expertise.
\end{enumerate}

\paragraph{Vendor Selection Process (stages):}
\begin{enumerate}
  \item Market scan \& initial screening (2--3 weeks): shortlist 3--4 vendors.
  \item Detailed assessment (4--6 weeks): architecture reviews, references, financials.
  \item Proof of concept (4--8 weeks): prototype, integration, testing.
  \item Vendor selection \& negotiation (2--4 weeks): contracts, SLAs.
  \item Implementation planning (4--6 weeks): detailed plan, resources.
\end{enumerate}

\subsubsection{Implementation Timeline and Phases}

\paragraph{Typical Outsourced Timeline: 18--24 months}
\begin{description}
  \item[Phase 1: Platform Setup \& Configuration (Months 1--4)] Provisioning, configuration, env setup, initial testing.
  \item[Phase 2: Integration \& Testing (Months 5--12)] Core integration, API testing, channel integration, regulatory testing.
  \item[Phase 3: Pilot Deployment (Months 13--18)] Limited pilot (5k--10k users), monitoring, refinement.
  \item[Phase 4: Production Rollout (Months 19--24)] Gradual activation, monitoring, channel expansion.
\end{description}

% \subsubsection{Cost and Resource Implications}

% \paragraph{Implementation Costs (Full Lifecycle):}

% \begin{center}
% \begin{tabular}{@{}lr@{}}
% \toprule
% Cost Category & Typical Range \\
% \midrule
% Platform license/setup & \texteuro 2M--5M \\
% Implementation services & \texteuro 1M--3M \\
% Integration consulting & \texteuro 0.5M--1.5M \\
% Testing and QA & \texteuro 0.5M--1M \\
% Training and change management & \texteuro 0.3M--0.5M \\
% Contingency (15\%) & \texteuro 0.6M--1.5M \\
% \midrule
% \textbf{Total Implementation Cost} & \textbf{\texteuro 5M--13M} \\
% \bottomrule
% \end{tabular}
% \end{center}

% \paragraph{Ongoing Costs (Annual):}
% \begin{itemize}
%   \item Platform licensing/SaaS: \texteuro 1M--2M
%   \item Per-transaction fees: variable (\texteuro 0.01--0.05 per tx)
%   \item Support and maintenance: \texteuro 0.3--0.6M
%   \item Upgrades and enhancements: \texteuro 0.2--0.5M
%   \item \textbf{Total annual operating cost: \texteuro 1.5M--3.5M}
% \end{itemize}

% \paragraph{Resource Requirements (implementation):}
% \begin{itemize}
%   \item Project manager: 1 FTE
%   \item Systems analyst: 2--3 FTEs
%   \item Business analyst: 2 FTEs
%   \item Compliance officer: 0.5 FTE (ongoing)
%   \item Operations staff: 3--5 FTEs (ongoing)
%   \item \textbf{Dedicated staff during implementation: 8--12 FTEs}
% \end{itemize}

% \subsubsection{Risk Profile and Mitigation Strategies}

% \paragraph{Key Risks (Vendor Model):}
% \begin{longtable}{@{}p{4.5cm}p{2.5cm}p{2.5cm}p{5cm}@{}}
% \toprule
% Risk & Probability & Impact & Mitigation \\
% \midrule
% Vendor lock-in & MEDIUM & MEDIUM & Exit terms, data portability clauses \\
% Vendor financial instability & LOW & CRITICAL & Financial review, escrow, backup vendors \\
% Service disruptions & LOW & HIGH & SLAs, redundancy, escalation procedures \\
% Limited customization & MEDIUM & MEDIUM & Flexible APIs, pro services options \\
% Integration complexity & MEDIUM & MEDIUM & Clear specs, PoC, dedicated integration support \\
% Regulatory compliance gaps & LOW & HIGH & Vendor certifications, regulatory liaison \\
% Feature limitations & MEDIUM & LOW & Roadmap alignment, upgrade planning \\
% \bottomrule
% \end{longtable}

% \paragraph{Mitigation Strategies (selected):}
% \begin{enumerate}
%   \item Vendor diversification (2+ vendors).
%   \item API standardization requirement.
%   \item Escrow of vendor code/documentation.
%   \item Strict SLAs with penalties.
%   \item Clear exit and data transition clauses.
% \end{enumerate}

% \subsection{Hybrid Implementation Model: Balanced Approach}

% \subsubsection{Model Characteristics and Applicability}

% \paragraph{Ideal Bank Profile:}
% \begin{itemize}
%   \item Mid-sized to large banks (\texteuro 100--500 billion assets)
%   \item Moderate to advanced IT capabilities
%   \item Strategic need for differentiation in selected areas
%   \item Desire to balance cost efficiency with feature flexibility
% \end{itemize}

% \paragraph{Key Characteristics:}
% \begin{itemize}
%   \item Outsourced core integration with vendor platform
%   \item Selective in-house development for high-value services
%   \item Vendor for commodity functions; bank for proprietary services
%   \item Balanced risk and resource allocation
% \end{itemize}

% \subsubsection{Hybrid Model Architecture}

% \begin{verbatim}
% Tier 1: Vendor-Managed Core (Access, Liquidity, Basic Txn, Compliance)
% Tier 2: Bank-Managed Enhancements (Advanced Fraud, Conditional Pay, Liquidity Forecasting)
% Tier 3: Bank-Developed Value-Added Services (Loyalty, Supply Chain FIN, Treasury)
% \end{verbatim}

% \paragraph{Implementation Architecture (diagram):}
% \begin{verbatim}
% ┌──────────────────────┐  ┌─────────────────────┐
% │  Bank-Developed      │  │  Vendor Platform    │
% │  Value-Added         │  │  Core Integration   │
% └────────┬─────────────┘  └──────┬──────────────┘
%          │                       │
%          └───────────┬───────────┘
%                      │
%               ┌──────▼──────┐
%               │ Integration │
%               │ Layer/APIs  │
%               └──────┬──────┘
%                      │
%               ┌──────▼──────────┐
%               │  DESP (External)│
%               └─────────────────┘
% \end{verbatim}

% \subsubsection{Value-Added Service Examples}
% \paragraph{Advanced Conditional Payments for B2B}
% \begin{itemize}
%   \item Bill-pay scheduling, escrow, installment management, dynamic discounting.
%   \item Technical: build on vendor conditional framework; bank develops evaluation logic; integrate with treasury.
% \end{itemize}

% \paragraph{Merchant Loyalty and Incentive Management}
% \begin{itemize}
%   \item Auto loyalty points, targeted promotions, cashback, reconciliation with merchant accounting.
%   \item Technical: APIs for loyalty, real-time triggers, reconciliation flows.
% \end{itemize}

% \paragraph{Liquidity and Cash Management Enhancements}
% \begin{itemize}
%   \item Predictive liquidity forecasting (ML), automated funding, FX/hedging integration.
%   \item Technical: data pipelines, ML models, treasury APIs.
% \end{itemize}

% \paragraph{Supply Chain Financing}
% \begin{itemize}
%   \item Invoice discounting, supplier financing, procurement integration.
%   \item Technical: supply-chain data integration, pricing engines.
% \end{itemize}

% \subsubsection{Development and Integration Approach}

% \paragraph{Phased Implementation Strategy:}
% \begin{description}
%   \item[Phase 1: Core Integration (Months 1--12)] Vendor setup, integration with core, pilot.
%   \item[Phase 2: Enhancement Development (Months 9--20)] Parallel development of value-adds, testing, pilot integration.
%   \item[Phase 3: Advanced Features (Months 18--30)] Launch first service, gather metrics, expand.
%   \item[Phase 4: Scaling and Optimization (Months 25--36)] Scale services, expand channels, roadmap.
% \end{description}

% \paragraph{Team Structure (example):}
% \begin{center}
% \begin{tabular}{@{}lccr@{}}
% \toprule
% Component & Team & Size & Reporting \\
% \midrule
% Core Integration & Vendor + Bank Integration & 3--5 bank staff & CIO \\
% Advanced Fraud & Bank Security/Risk & 3--4 & Chief Risk Officer \\
% Conditional Payments & Bank Product & 2--3 & Head of Payments \\
% Treasury Integration & Bank Treasury IT & 2--3 & Treasurer \\
% Data/Analytics & Bank Analytics & 2--3 & Chief Data Officer \\
% Overall Program & Program Manager & 1 & CIO/CFO \\
% \bottomrule
% \end{tabular}
% \end{center}

% \subsubsection{Cost and Resource Implications}

% \paragraph{Cost Profile (3-Year Period):}

% \begin{center}
% \begin{tabular}{@{}lr@{}}
% \toprule
% Cost Category & Amount \\
% \midrule
% Vendor licensing \& implementation & \texteuro 5M--8M \\
% Bank development (value-added) & \texteuro 8M--15M \\
% Integration and consulting & \texteuro 2M--4M \\
% Testing, training, change mgmt & \texteuro 2M--3M \\
% Contingency (15\%) & \texteuro 2.5M--4.5M \\
% \midrule
% \textbf{Total Implementation Cost} & \textbf{\texteuro 20M--35M} \\
% \bottomrule
% \end{tabular}
% \end{center}

% \paragraph{Ongoing Operating Costs (Annual):}
% \begin{itemize}
%   \item Vendor licensing/support: \texteuro 1.5M--2.5M
%   \item Bank development team (5--8 people): \texteuro 0.6M--1.2M
%   \item Infrastructure/hosting: \texteuro 0.3M--0.5M
%   \item \textbf{Total annual cost: \texteuro 2.4M--4.2M}
% \end{itemize}

% \paragraph{Resource Requirements:}
% \begin{itemize}
%   \item Implementation period: 20--30 FTEs for 18--24 months
%   \item Ongoing: 6--10 FTEs
%   \item External support: 3--5 FTEs first 12 months
% \end{itemize}

% \subsubsection{Risk Profile and Mitigation Strategies}

% \paragraph{Key Risks (Hybrid):}
% \begin{longtable}{@{}p{5cm}p{2.5cm}p{2.5cm}p{5cm}@{}}
% \toprule
% Risk & Probability & Impact & Mitigation \\
% \midrule
% Integration complexity & MEDIUM & MEDIUM & Clear integration architecture, dedicated team \\
% Development delays on enhancements & MEDIUM & MEDIUM & Agile methodology, experienced leadership \\
% Vendor/bank misalignment & MEDIUM & MEDIUM & Governance, steering committee, regular communication \\
% Feature duplication/conflicts & MEDIUM & LOW & Architecture review, separation of concerns \\
% Skill gaps in bank team & MEDIUM & MEDIUM & Training, external mentoring \\
% Regulatory compliance complexity & LOW & MEDIUM & Compliance oversight, regulatory testing \\
% \bottomrule
% \end{longtable}

% \paragraph{Mitigation Strategies (selected):}
% \begin{enumerate}
%   \item Clear separation of concerns between vendor and bank components.
%   \item Third-party integration architecture review.
%   \item Joint steering committee with vendor and bank leadership.
%   \item Agile development with regular demos and feedback.
%   \item Dedicated compliance review at each development phase.
%   \item Define performance baselines for vendor and bank components.
% \end{enumerate}



\section{Implementation Models by Bank Tier: Tailored Strategies}

\subsection{High-Tier Banks (Large, Internationally Active)}

\subsubsection{Bank Profile and Strategic Context}

\paragraph{Typical Characteristics:}
\begin{itemize}
  \item Total assets: \texteuro{}300 billion to >\texteuro{}3 trillion
  \item Geographic reach: multiple countries, significant international presence
  \item Customer base: large retail and substantial corporate/wholesale operations
  \item IT infrastructure: advanced, decentralized across multiple jurisdictions
  \item Competitive position: market leaders with significant technical capabilities
  \item Strategic objectives: maintain market leadership, drive innovation, maximize shareholder value
\end{itemize}

\paragraph{Digital Euro Strategic Imperatives:}
\begin{enumerate}
  \item Market leadership: be among first movers with sophisticated Digital Euro services.
  \item Competitive differentiation: leverage advanced capabilities for market advantage.
  \item Operational integration: minimise disruption to existing operations while adding new capabilities.
  \item Global coordination: manage implementation across multiple jurisdictions and banking entities.
  \item Innovation positioning: position as technology innovator, not follower.
\end{enumerate}

\subsubsection{Recommended Implementation Approach: In-House with Selective Partnerships}

\paragraph{Core Strategy:}
Develop a comprehensive in-house Digital Euro platform, utilise selective partnerships for specialised components (e.g.\ offline, fraud detection), create an innovation centre for next-generation features, and establish the Digital Euro as a competitive differentiator.

\paragraph{Detailed Implementation Architecture:}

\textbf{Tier 1: Core Development (In-House)} --- Core platform components:
\begin{itemize}
  \item Access Management Service
  \item Liquidity Management Service
  \item Transaction Management Service
  \item Risk and Compliance Service
  \item Offline Management Service
  \item Advanced Fraud Detection
\end{itemize}

\textbf{Technical Approach:}
\begin{itemize}
  \item Microservices architecture enabling independent scaling and updates
  \item Distributed systems design for resilience across geographies
  \item Cloud-native deployment for flexibility and scalability
  \item Event-driven architecture for real-time processing
\end{itemize}

\textbf{Tier 2: Channel Integration (In-House + Partnerships)} --- Distribution channels and elements:
\begin{itemize}
  \item Mobile banking
    \begin{itemize}
      \item Native iOS/Android apps
      \item Digital Euro wallet UI
      \item Offline capability support
      \item Biometric authentication
    \end{itemize}
  \item Web banking
    \begin{itemize}
      \item Enhanced web interfaces
      \item Corporate treasury portal
      \item Merchant acceptance tools
    \end{itemize}
  \item POS \& Merchant
    \begin{itemize}
      \item Terminal integration framework
      \item Merchant onboarding
      \item Acceptance network development
    \end{itemize}
  \item ATM network
    \begin{itemize}
      \item NFC upgrade support
      \item QR code capability
      \item Funding/defunding operations
    \end{itemize}
\end{itemize}

\textbf{Tier 3: Advanced Services (In-House Development)} --- Proprietary innovation services:
\begin{itemize}
  \item Conditional payments engine (escrow, supply-chain financing, treasury orchestration)
  \item Loyalty \& rewards integration (automatic point allocation, merchant incentive programs)
  \item Advanced analytics (real-time transaction analytics, fraud pattern detection)
  \item API economy for partners (developer ecosystem, fintech partnerships)
  \item Blockchain integration (future): smart contract compatibility and advanced programmability
\end{itemize}

\subsubsection{Governance and Organizational Structure}

\paragraph{Organizational Design (example teams and sizing):}
\begin{itemize}
  \item Chief Technology Officer (CTO)
    \begin{itemize}
      \item Head, Digital Euro Platform
        \begin{itemize}
          \item Platform Architecture Team (5)
          \item Core Services Development (30)
            \begin{itemize}
              \item Access \& Onboarding (8)
              \item Liquidity \& Settlement (8)
              \item Transaction Processing (8)
              \item Risk \& Compliance (6)
            \end{itemize}
          \item DevOps \& Infrastructure (8)
          \item Security \& Privacy (5)
          \item QA \& Testing (10)
        \end{itemize}
      \item Head, Digital Euro Channels
        \begin{itemize}
          \item Mobile Banking Enhancement (12)
          \item Web \& Corporate (8)
          \item POS \& Merchant Integration (6)
          \item ATM Integration (5)
          \item Quality Assurance (8)
        \end{itemize}
      \item Head, Digital Euro Innovation
        \begin{itemize}
          \item Advanced Payments Team (6)
          \item Analytics \& Data Science (6)
          \item Partner Ecosystem (4)
          \item Research \& Development (4)
        \end{itemize}
    \end{itemize}
  \item Chief Risk Officer (CRO)
    \begin{itemize}
      \item Head, Digital Euro Compliance
        \begin{itemize}
          \item Regulatory Affairs (3)
          \item AML/KYC Compliance (3)
          \item Risk Management (3)
          \item Internal Audit (2)
        \end{itemize}
    \end{itemize}
\end{itemize}

\paragraph{Program Governance:}
\begin{itemize}
  \item Digital Euro Steering Committee (Executive Board, CTO, CFO, CRO, Chief Commercial Officer)
  \item Technical Architecture Review Board (monthly)
  \item Regulatory \& Compliance Review (bi-weekly)
  \item Product \& Innovation Council (monthly)
  \item External Advisory Board (quarterly) including ECB, industry experts, fintech partners
\end{itemize}

\subsubsection{Cost and Timeline for High-Tier Banks}

\paragraph{Financial Investment (48-month period):}
\begin{center}
\begin{tabular}{@{}lrr@{}}
\toprule
Category & Low & High \\
\midrule
Personnel (50+ FTEs @ \texteuro{}120k--150k avg) & \texteuro{}24M & \texteuro{}30M \\
Infrastructure (cloud, data centres, security) & \texteuro{}8M & \texteuro{}12M \\
Technology licenses \& third-party services & \texteuro{}3M & \texteuro{}5M \\
External consulting \& specialised expertise & \texteuro{}4M & \texteuro{}6M \\
Testing, QA, security & \texteuro{}4M & \texteuro{}6M \\
Contingency (15\%) & \texteuro{}5.1M & \texteuro{}7.65M \\
\midrule
\textbf{Total Development Cost} & \textbf{\texteuro{}48.1M} & \textbf{\texteuro{}66.65M} \\
\bottomrule
\end{tabular}
\end{center}

\paragraph{Ongoing Annual Operating Costs (indicative):}
\begin{itemize}
  \item Personnel (12--18 FTEs): \texteuro{}1.5M--2.7M
  \item Infrastructure and hosting: \texteuro{}1M--2M
  \item Third-party services and licences: \texteuro{}0.5M--1M
  \item \textbf{Total annual: \texteuro{}3M--5.7M}
\end{itemize}

\paragraph{Timeline (high-level by year):}
\begin{itemize}
  \item Year 1 (Q1--Q4): Architecture \& foundation — technology selection, DevOps setup, core framework, recruit (75\% staffing).
  \item Year 2 (Q1--Q4): Core services development — access, liquidity, transaction services, DESP integration, internal beta.
  \item Year 3 (Q1--Q4): Enhancement \& channel integration — risk, offline, mobile/web/POS integration, pilot (5k--10k external users).
  \item Year 4 (Q1--Q4): Advanced features \& production — conditional payments, analytics, full channel rollout, go-live preparation.
\end{itemize}

\paragraph{Key milestones (examples):}
\begin{itemize}
  \item Month 6: architecture approved, core development begun.
  \item Month 12: core platform functional, internal testing.
  \item Month 24: beta pilot launched, \(\sim\)5,000 active users.
  \item Month 36: production launch.
  \item Month 48: full feature deployment, ecosystem maturity.
\end{itemize}

\subsection{Mid-Tier Banks (Regional, Moderate Complexity)}

\subsubsection{Bank Profile and Strategic Context}

\paragraph{Typical Characteristics:}
\begin{itemize}
  \item Total assets: \texteuro{}50--300 billion
  \item Geographic reach: primary country plus selected neighbouring markets
  \item Customer base: significant retail, regional corporate focus
  \item IT infrastructure: moderate maturity, some legacy systems
  \item Competitive position: regional leaders in key markets
  \item Strategic objectives: maintain competitive relevance, manage costs efficiently
\end{itemize}

\paragraph{Digital Euro Strategic Imperatives:}
\begin{enumerate}
  \item Compliance requirement: meet ECB mandates without overinvestment.
  \item Cost efficiency: manage costs while maintaining quality.
  \item Time-to-market: launch Digital Euro services quickly.
  \item Operational integration: minimise disruption.
  \item Selective differentiation: focus innovation on high-value segments.
\end{enumerate}

\subsubsection{Recommended Implementation Approach: Hybrid Model}

\paragraph{Core Strategy:}
Outsource core integration via a vendor platform, develop selective proprietary services for regional differentiation, leverage vendor expertise while controlling costs, and achieve faster time-to-market.

\paragraph{Detailed Implementation Architecture:}

\textbf{Tier 1: Vendor-Managed Core (approx.\ 60\% effort)} --- Platform components (vendor responsibility):
\begin{itemize}
  \item Access management (onboarding, KYC, wallet provisioning)
  \item Liquidity management (DCA operations, waterfall)
  \item Basic transaction processing
  \item Compliance framework (rulebook requirements)
  \item Standard reporting and statement engine
\end{itemize}

\textbf{Vendor selection criteria (high level):}
\begin{itemize}
  \item Strong presence in bank's primary market and proven track record
  \item Comprehensive Digital Euro platform and responsive support
  \item Reasonable pricing aligned to bank scale
\end{itemize}

\textbf{Tier 2: Shared Infrastructure (approx.\ 25\% effort)} --- Market-based collaboration:
\begin{itemize}
  \item Shared liquidity management (cooperative funding)
  \item Shared fraud detection consortium
  \item Shared ATM network operations
  \item Shared settlement operations
  \item Industry shared testing infrastructure
\end{itemize}

\textbf{Tier 3: Bank-Developed Differentiation (approx.\ 15\% effort)} --- Proprietary services:
\begin{itemize}
  \item Regional payment integration and local merchant ecosystem
  \item SME/corporate offerings (supply chain financing, working capital)
  \item Enhanced customer experience (personalised merchant offers)
  \item Data \& analytics (dashboards, competitive insights)
\end{itemize}

\subsubsection{Governance and Organizational Structure}

\paragraph{Organizational Design (example):}
\begin{itemize}
  \item Chief Information Officer (CIO)
    \begin{itemize}
      \item Head, Digital Euro Program (1.0 FTE)
        \begin{itemize}
          \item Program Manager (1.0 FTE)
          \item Systems Integration Manager (1.0 FTE)
          \item QA Lead (1.0 FTE)
          \item Operations Manager (0.5 FTE)
          \item Vendor Relationship Manager (0.5 FTE)
        \end{itemize}
      \item Head, Digital Euro Innovation (0.5 FTE)
        \begin{itemize}
          \item Product Manager (0.5 FTE)
          \item Developer (2 FTE shared)
          \item Analyst (1 FTE shared)
        \end{itemize}
    \end{itemize}
  \item Chief Risk Officer (CRO): Digital Euro Compliance Officer (1.0 FTE), Regulatory Liaison (0.5 FTE), AML/KYC Manager (1.0 FTE)
  \item Chief Commercial Officer (CCO): Digital Euro Product Manager (1.0 FTE), Marketing Manager (0.5 FTE shared), Customer Success Manager (1.0 FTE)
\end{itemize}

\paragraph{Program Governance:}
\begin{itemize}
  \item Steering Committee (quarterly): CIO, CFO, CRO, CCO
  \item Vendor Management Review (monthly)
  \item Compliance \& Regulatory Review (monthly)
  \item Product \& Customer Review (bi-monthly)
\end{itemize}

\subsubsection{Cost and Timeline for Mid-Tier Banks}

\paragraph{Financial Investment (30-month period):}
\begin{center}
\begin{tabular}{@{}lrr@{}}
\toprule
Category & Typical Range \\
\midrule
Vendor platform \& services & \texteuro{}5M--8M \\
Bank project management \& integration & \texteuro{}1.5M--2.5M \\
In-house development (proprietary) & \texteuro{}3M--5M \\
Testing, training, change management & \texteuro{}1.5M--2.5M \\
Infrastructure and operational setup & \texteuro{}1M--1.5M \\
Contingency (15\%) & \texteuro{}1.95M--3.15M \\
\midrule
\textbf{Total Development Cost} & \textbf{\texteuro{}13.95M--22.65M} \\
\bottomrule
\end{tabular}
\end{center}

\paragraph{Ongoing Annual Operating Costs (indicative):}
\begin{itemize}
  \item Vendor licensing/support: \texteuro{}1M--1.5M
  \item Bank operations team (3--4 FTEs): \texteuro{}0.4M--0.6M
  \item Development and enhancement (part-time): \texteuro{}0.3M--0.5M
  \item \textbf{Total annual: \texteuro{}1.7M--2.6M}
\end{itemize}

\paragraph{Timeline (high-level):}
\begin{itemize}
  \item Months 1--4: Vendor selection \& planning.
  \item Months 5--12: Core integration phase; limited pilot (500--1,000 users).
  \item Months 13--18: Enhancement development and expanded pilot (2,000--5,000 users).
  \item Months 19--24: Production preparation and regulatory testing.
  \item Months 25--30: Production launch and scaling.
\end{itemize}

\paragraph{Key milestones (examples):}
\begin{itemize}
  \item Month 3: vendor selected and contracted.
  \item Month 6: platform setup complete.
  \item Month 12: pilot launched with \(\sim\)1,000 users.
  \item Month 24: production ready.
  \item Month 30: full customer activation.
\end{itemize}

\subsection{Low-Tier Banks (Small, Community-Focused)}

\subsubsection{Bank Profile and Strategic Context}

\paragraph{Typical Characteristics:}
\begin{itemize}
  \item Total assets: \texteuro{}5--50 billion
  \item Geographic reach: single country, often single region
  \item Customer base: retail-focused, limited corporate services
  \item IT infrastructure: basic systems, limited IT staff
  \item Competitive position: niche players in local markets
  \item Strategic objectives: remain compliant while managing tight budgets
\end{itemize}

\paragraph{Digital Euro Strategic Imperatives:}
\begin{enumerate}
  \item Cost minimisation: implement Digital Euro with minimal investment.
  \item Regulatory compliance: meet ECB requirements without differentiation.
  \item Resource constraints: manage with existing small IT team.
  \item Time-to-market: achieve timeline without overextension.
  \item Stability: avoid operational disruption to core banking.
\end{enumerate}

\subsubsection{Recommended Implementation Approach: Vendor/Outsourced Model}

\paragraph{Core Strategy:}
Engage an established vendor for an end-to-end Digital Euro platform, minimise internal development and complexity, rely on vendor expertise and support, and focus bank resources on core banking ops.

\paragraph{Detailed Implementation Architecture:}

\textbf{Vendor platform (approx.\ 90\% of services) --- typical vendor-provided services:}
\begin{itemize}
  \item Access management (onboarding, KYC, wallet management)
  \item Liquidity management (DCA operations, waterfall automation)
  \item Transaction processing (full lifecycle)
  \item Risk \& compliance (fraud detection, AML screening)
  \item Channel integration (mobile, web, ATM support)
  \item Reporting and reconciliation
  \item Customer support framework and operational monitoring
\end{itemize}

\textbf{Bank-specific configuration (approx.\ 10\% effort):}
\begin{itemize}
  \item Brand integration, customer communications, regulatory documentation
  \item Staff training materials and customer support scripts
\end{itemize}

\paragraph{Vendor relationship options:}
\begin{itemize}
  \item Single vendor relationship: simplest, lowest internal overhead, but vendor dependency (typical vendors: Temenos, SAP, Oracle).
  \item Cooperative/consortium model: multiple banks share vendor platform, governance and costs shared (examples: Atruvia, Redsys).
\end{itemize}

\subsubsection{Governance and Organizational Structure}

\paragraph{Streamlined organisation (example):}
\begin{itemize}
  \item Chief Information Officer (CIO)
    \begin{itemize}
      \item Digital Euro Project Manager (1.0 FTE)
      \item Vendor Relationship Manager (0.5 FTE)
      \item Systems Administrator (0.5 FTE shared)
      \item Compliance Liaison (0.25 FTE shared)
    \end{itemize}
  \item Chief Risk Officer (CRO)
    \begin{itemize}
      \item Compliance Officer (0.5 FTE shared)
      \item Regulatory Liaison (0.25 FTE)
    \end{itemize}
  \item Chief Commercial Officer (CCO)
    \begin{itemize}
      \item Product Manager (0.25 FTE shared)
    \end{itemize}
\end{itemize}

\paragraph{Governance structure:}
\begin{itemize}
  \item Steering Committee (quarterly): CIO, CFO, CRO, CCO
  \item Vendor Review Meeting (monthly)
  \item Regulatory Check-in (monthly)
\end{itemize}

\subsubsection{Cost and Timeline for Low-Tier Banks}

\paragraph{Financial Investment (24-month period):}
\begin{center}
\begin{tabular}{@{}lr@{}}
\toprule
Category & Amount \\
\midrule
Vendor platform implementation & \texteuro{}2.5M--4M \\
Project management and coordination & \texteuro{}0.4M--0.6M \\
Integration and configuration & \texteuro{}0.3M--0.5M \\
Testing and training & \texteuro{}0.3M--0.5M \\
Infrastructure (servers, security) & \texteuro{}0.3M--0.5M \\
Contingency (10\%) & \texteuro{}0.38M--0.61M \\
\midrule
\textbf{Total Development Cost} & \textbf{\texteuro{}4.18M--6.71M} \\
\bottomrule
\end{tabular}
\end{center}

\paragraph{Ongoing Annual Operating Costs (indicative):}
\begin{itemize}
  \item Vendor platform licensing: \texteuro{}0.6M--1M
  \item Operational staff (1 FTE): \texteuro{}70k--100k
  \item Support and maintenance: \texteuro{}100k--200k
  \item \textbf{Total annual: \texteuro{}0.77M--1.3M}
\end{itemize}

\paragraph{Timeline (high-level):}
\begin{itemize}
  \item Months 1--3: vendor selection \& planning.
  \item Months 4--9: implementation, configuration, pilot (1k--2k users).
  \item Months 10--18: integration, regulatory testing, production preparation.
  \item Months 19--24: production launch, staged activation and monitoring.
\end{itemize}

\subsubsection{Risk Considerations and Mitigation}

\paragraph{Primary risks and mitigations (summary):}
\begin{itemize}
  \item Complete vendor dependency --- mitigation: cooperative arrangement, exit clauses.
  \item Limited customization --- mitigation: accept standard platform features where feasible.
  \item Operational disruption --- mitigation: thorough testing, vendor support, phased rollout.
  \item Data security/privacy --- mitigation: vendor security certifications, regular audits.
  \item Cost overruns --- mitigation: fixed-price contracts, scope control.
\end{itemize}

\paragraph{Recommended mitigation strategies:}
\begin{enumerate}
  \item Join cooperative arrangements or consortiums to reduce individual dependency and cost.
  \item Establish clear SLAs with performance and support guarantees.
  \item Use a phased implementation approach to manage risk and cost.
  \item Provide comprehensive staff training prior to go-live.
  \item Keep competent authority/regulatory oversight of the vendor relationship.
\end{enumerate}


\section{Shared Infrastructure, Synergies, and Cost
Mutualization}

\section{Technical Blueprints and Best Practices}
\section{Regulatory Considerations and Compliance Framework}
\section{Conclusion and Recommendations}
 

% \textcolor{red}{Guidance for this section (approx. 1 page):}
% \begin{itemize}
%     \item \textcolor{red}{Summarize your findings.}
%     \item \textcolor{red}{Discuss limitations of your study.}
%     \item \textcolor{red}{Suggest directions for future work.}
% \end{itemize}

% This is your text. Summarize the key takeaways.

% --- BIBLIOGRAPHY ---
\newpage
\addcontentsline{toc}{section}{Bibliography}
\renewcommand\refname{Bibliography} 
\bibliographystyle{plainnat}
\bibliography{references} % Make sure you have a references.bib file

% --- APPENDICES ---
\newpage
\appendix 
\addsec{Appendix}

% \section*{A \ AI Usage Declaration}
% \addcontentsline{toc}{subsection}{A \hspace*{0.15cm} AI Usage Declaration} 

% \textcolor{red}{\textbf{Mandatory if AI was used:} You must explicitly disclose the tool and specific nature of its use (e.g., "Used ChatGPT to debug a dimension mismatch error" or "Used Copilot for LaTeX formatting"). If no AI was used, state that the work is entirely your own without AI assistance.}

% \emph{Example Statement: I declare that I used ChatGPT-4o to assist with debugging the R code for the Gibbs Sampler and for checking grammatical errors in the Introduction. The derivation of formulas and the analysis of results are my own work.}

% \section*{B \ Additional figures}
% \addcontentsline{toc}{subsection}{B \hspace*{0.15cm} Additional figures} 

% \section*{C \ Additional tables}
% \addcontentsline{toc}{subsection}{C \hspace*{0.15cm} Additional tables} 

% \section*{D \ Algorithms}
% \addcontentsline{toc}{subsection}{D \hspace*{0.15cm} Code Snippets} 
% You can include key parts of your algorithmic code here using the listings package if desired.

\end{document}